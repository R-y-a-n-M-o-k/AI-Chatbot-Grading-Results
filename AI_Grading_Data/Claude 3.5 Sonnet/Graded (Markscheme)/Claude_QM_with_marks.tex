\documentclass[a4paper,11pt]{article}
\setlength\parindent{0pt}
\usepackage{amsmath, amssymb}
\usepackage[margin=1in,letterpaper]
{geometry} % decreases margins

\usepackage{setspace}
\setstretch{1.2}
\usepackage{spverbatim}

\begin{document}
\title{Claude 3.5 Sonnet Grading (With Mark Scheme)}
\maketitle

This document compiles the Claude 3.5 Sonnet responses when prompted to grade the solutions which have been generated. The markscheme is given to Claude 3.5 Sonnet in the prompt. This is the grading for the Quantum Mechanics (QM) questions. 

\section{QM}

\subsection*{Question 1:}

\textbf{Response 1:}

I'll grade each solution out of 5 marks, allocating up to 3 marks for part a) and up to 2 marks for part b), using the provided mark scheme.

Solution 1:

Part a): 3/3 marks

- Correctly defines the commutator (1 mark)

- Explains compatibility in terms of shared eigenfunctions (1 mark)

- States that compatibility implies commutator equals zero (1 mark)

Part b): 2/2 marks

- Explains that the initial p value may not be the same as the final p value (1 mark)

- Describes how incompatible operators lead to wavefunction collapse and change in measured values (1 mark)

Total: 5/5 marks

Feedback: Excellent answer. Provides a comprehensive explanation of both parts with accurate definitions and clear reasoning.

Solution 2:

Part a): 3/3 marks

- Correctly defines the commutator (1 mark)

- Explains compatibility in terms of shared eigenstates (1 mark)

- States that compatibility implies commutator equals zero (1 mark)

Part b): 2/2 marks

- Explains that the initial p value may not be the same as the final p value (1 mark)

- Describes how incompatible operators lead to wavefunction collapse and change in measured values (1 mark)

Total: 5/5 marks

Feedback: Very good answer. Provides clear explanations and demonstrates a solid understanding of the concepts.

Solution 3:

Part a): 3/3 marks

- Correctly defines the commutator (1 mark)

- Explains compatibility in terms of shared eigenstates (1 mark)

- States that compatibility implies commutator equals zero (1 mark)

Part b): 2/2 marks

- Explains that the initial p value may not be the same as the final p value (1 mark)

- Describes how incompatible operators lead to wavefunction collapse and change in measured values (1 mark)

Total: 5/5 marks

Feedback: Excellent answer. Provides clear explanations and uses mathematical notation effectively to support the reasoning.

\bigskip   
\textbf{Response 2:}

I'll grade each solution and provide concise feedback.

Solution 1:

Part a) 3/3 marks

Part b) 2/2 marks

Total: 5/5 marks

Feedback: Excellent explanation of commutators, compatibility, and their implications. The response for part b) accurately describes the consequences of measuring non-compatible operators.

Solution 2:

Part a) 3/3 marks

Part b) 2/2 marks

Total: 5/5 marks

Feedback: Comprehensive answer covering all required points. The explanation of incompatible operators and their measurement consequences is particularly well-articulated.

Solution 3:

Part a) 3/3 marks

Part b) 2/2 marks

Total: 5/5 marks

Feedback: Clear and concise definitions with appropriate mathematical notation. The explanation of measurement outcomes for incompatible operators is accurate and well-explained.

All three solutions demonstrate a strong understanding of the concepts and accurately address the questions posed. They each deserve full marks based on the given mark scheme.

\bigskip   
\textbf{Response 3:}

I'll grade each solution and provide concise feedback for each.

Solution 1:

a) 3/3 marks

b) 2/2 marks

Total: 5/5 marks

Feedback: Excellent explanation of the commutator, compatibility, and implications. The response to part b) is thorough and accurately describes the consequences of measuring non-compatible operators.

Solution 2:

a) 3/3 marks

b) 2/2 marks

Total: 5/5 marks

Feedback: Comprehensive answer covering all required points. The explanation of compatibility and its implications is clear. Part b) correctly describes the uncertainty in sequential measurements of non-compatible operators.

Solution 3:

a) 3/3 marks

b) 2/2 marks

Total: 5/5 marks

Feedback: Well-structured response with mathematical formulations. The explanation of compatibility and its implications is precise. Part b) accurately describes the effects of measuring non-compatible operators and the resulting uncertainty.

All three solutions demonstrate a strong understanding of the concepts and provide complete, accurate answers to both parts of the question.

\bigskip
\textbf{Response 4:}

I'll grade each solution out of 5 marks (3 for part a and 2 for part b) based on the provided mark scheme.

Solution 1:

Part a: 3/3 marks

Part b: 2/2 marks

Total: 5/5 marks

Feedback: Excellent solution. The student provided a comprehensive and accurate explanation of the commutator, compatibility, and their implications. They also correctly described the consequences of measuring non-compatible operators in sequence.

Solution 2:

Part a: 3/3 marks

Part b: 2/2 marks

Total: 5/5 marks

Feedback: Very good solution. The student accurately defined the commutator and compatibility, and correctly explained the implications of measuring non-compatible operators. They also provided a clear description of the quantum measurement process and its effects on the system's state.

Solution 3:

Part a: 3/3 marks

Part b: 2/2 marks

Total: 5/5 marks

Feedback: Strong solution. The student provided clear definitions and explanations for the commutator and compatibility. They also correctly described the consequences of measuring non-compatible operators and explained the underlying quantum mechanical principles.

All three solutions demonstrated a thorough understanding of the concepts and met the criteria outlined in the mark scheme.

\bigskip
\textbf{Response 5:}

I'll grade each solution out of 5 marks total, with up to 3 marks for part a) and up to 2 marks for part b), following the provided mark scheme.

Solution 1:

Part a): 3/3 marks

Part b): 2/2 marks

Total: 5/5 marks

General feedback: Excellent solution. The student correctly defined the commutator, explained compatibility, and discussed the implications of compatibility on the commutator. The explanation of incompatible operators and their effects on measurements was thorough and accurate.

Solution 2:

Part a): 3/3 marks

Part b): 2/2 marks

Total: 5/5 marks

General feedback: Very good solution. The student provided a correct definition of the commutator and compatibility, and explained the implications clearly. The discussion of incompatible operators and their effects on sequential measurements was comprehensive and accurate.

Solution 3:

Part a): 3/3 marks

Part b): 2/2 marks

Total: 5/5 marks

General feedback: Strong solution. The student correctly defined the commutator and compatibility, and explained their implications. The explanation of incompatible operators and their effects on measurements was clear and accurate, with good use of mathematical notation.

All three solutions demonstrated a thorough understanding of the concepts and addressed all parts of the question effectively.

\subsection*{Question 2:}

\textbf{Response 1:}

I'll grade each solution and provide concise feedback as requested.

Solution 1:

Part a): 2 marks

Part b): 1 mark

Total: 3/4 marks

Feedback: Excellent explanation of eigenvalues' properties and physical interpretation in part a). Part b) correctly defines orthogonality but lacks a clear statement of normalization using the Kronecker delta notation.

Solution 2:

Part a): 2 marks

Part b): 2 marks

Total: 4/4 marks

Feedback: Comprehensive explanation of eigenvalues' properties and physical significance in part a). Part b) provides a clear, mathematically precise definition of orthonormality using the Kronecker delta notation.

Solution 3:

Part a): 2 marks

Part b): 2 marks

Total: 4/4 marks

Feedback: Thorough explanation of eigenvalues' properties with mathematical rigor in part a). Part b) offers a precise definition of orthonormality, including both orthogonality and normalization aspects, using appropriate mathematical notation.

\bigskip   
\textbf{Response 2:}

I'll grade each solution and provide concise feedback as requested.

Solution 1:

Part a): 2/2 marks

Part b): 2/2 marks

Total: 4/4 marks

Feedback: Excellent explanation of both parts. The answer correctly states the mathematical property of eigenvalues and provides a clear physical interpretation. The explanation of orthonormality is thorough and mathematically sound.

Solution 2:

Part a): 2/2 marks

Part b): 2/2 marks

Total: 4/4 marks

Feedback: Very good response. The answer provides a detailed explanation of the real nature of eigenvalues and their physical significance. The orthonormality concept is well-explained with the correct mathematical representation.

Solution 3:

Part a): 2/2 marks

Part b): 2/2 marks

Total: 4/4 marks

Feedback: Exceptional answer with rigorous mathematical explanations. The response provides a detailed proof for the realness of eigenvalues and explains the physical significance clearly. The orthonormality concept is explained precisely with the correct mathematical formulation.

All three solutions demonstrate a thorough understanding of the concepts and meet the criteria specified in the mark scheme.

\bigskip   
\textbf{Response 3:}

I'll grade each solution and provide concise feedback.

Solution 1:

Part a): 2/2 marks

Part b): 2/2 marks

Total: 4/4 marks

Feedback: Excellent explanation of both parts. The answer clearly states that eigenvalues are real and provides a correct physical interpretation. The orthonormality concept is well-explained, covering both orthogonality and normalization.

Solution 2:

Part a): 2/2 marks

Part b): 2/2 marks

Total: 4/4 marks

Feedback: Comprehensive answer for both parts. The explanation of real eigenvalues is thorough, including mathematical reasoning. The orthonormality concept is accurately described using the Kronecker delta notation.

Solution 3:

Part a): 2/2 marks

Part b): 2/2 marks

Total: 4/4 marks

Feedback: Exceptional answer with detailed mathematical explanation for both parts. The response provides a rigorous proof for real eigenvalues and a precise definition of orthonormality using mathematical notation.

\bigskip
\textbf{Response 4:}

Thank you for providing the problem, mark scheme, and student solutions. I'll grade each solution out of 4 marks (2 for part a and 2 for part b) and provide concise feedback.

Solution 1:

Part a) 2 marks

Part b) 2 marks

Total: 4 marks

Feedback: Excellent response. The student correctly identified the mathematical property of eigenvalues and provided a clear physical interpretation. The explanation of orthonormality was thorough and mathematically precise.

Solution 2:

Part a) 2 marks

Part b) 2 marks

Total: 4 marks

Feedback: Very good answer. The student demonstrated a strong understanding of Hermitian operators and their eigenvalues. The explanation of orthonormality was concise and mathematically correct, including the Kronecker delta notation.

Solution 3:

Part a) 2 marks

Part b) 2 marks

Total: 4 marks

Feedback: Outstanding response. The student provided a mathematically rigorous explanation of why eigenvalues are real, including the relevant proof sketch. The orthonormality explanation was precise and included the correct mathematical representation.

All three solutions demonstrated a high level of understanding of the concepts and deserved full marks based on the provided mark scheme.

\bigskip
\textbf{Response 5:}

I'll grade each solution and provide concise feedback based on the given mark scheme.

Solution 1:

a) 2/2 marks

b) 2/2 marks

Total: 4/4 marks

Feedback: Excellent answer. Both parts are thoroughly explained with correct mathematical and physical interpretations. The orthonormality condition is well-defined mathematically.

Solution 2:

a) 2/2 marks

b) 2/2 marks

Total: 4/4 marks

Feedback: Very good response. The explanation of Hermitian operators and their eigenvalues is precise. The orthonormality condition is correctly expressed mathematically and its importance in quantum mechanics is well-explained.

Solution 3:

a) 2/2 marks

b) 2/2 marks

Total: 4/4 marks

Feedback: Outstanding answer. The mathematical proof for the realness of eigenvalues is provided, which goes beyond the requirements. The orthonormality condition is precisely defined and its significance in quantum mechanics is well-articulated.

All three solutions demonstrate a strong understanding of the concepts and meet the criteria specified in the mark scheme.

\subsection*{Question 3:}

\textbf{Response 1:}

I'll grade each solution out of 9 marks and provide concise feedback.

Solution 1:

Part a): 4/4 marks

Part b): 5/5 marks

Total: 9/9 marks

Feedback: Excellent solution. The student demonstrates a thorough understanding of Hermitian operators and provides clear, step-by-step derivations for both parts. They correctly apply integration by parts and handle complex conjugates appropriately.

Solution 2:

Part a): 3/4 marks

Part b): 3/5 marks

Total: 6/9 marks

Feedback: The student shows a good grasp of the concept but lacks some detail in the derivations. In part a), they correctly set up the problem but don't fully show the integration by parts steps. In part b), they mention using integration by parts twice but don't demonstrate it explicitly, which is crucial for full marks.

Solution 3:

Part a): 4/4 marks

Part b): 4/5 marks

Total: 8/9 marks

Feedback: Very good solution overall. Part a) is excellent, with clear derivation and proper handling of boundary terms. In part b), the student correctly identifies the need for integration by parts twice but doesn't show the full working, losing one mark. The conclusion is correct for both parts.

\bigskip   
\textbf{Response 2:}

I'll grade each solution and provide concise feedback as requested.

Solution 1:

Part a): 4/4 marks

Part b): 5/5 marks

Total: 9/9 marks

Feedback: Excellent solution. The student provided a thorough and mathematically rigorous approach for both parts. They correctly applied integration by parts, handled complex conjugates, and clearly demonstrated the Hermitian nature of both operators.

Solution 2:

Part a): 3/4 marks

Part b): 3/5 marks

Total: 6/9 marks

Feedback: The student demonstrated a good understanding of the concept but lacked some detail in the mathematical derivations. In part a), the integration by parts step could have been more explicit. In part b), the double integration by parts was mentioned but not shown in detail, which is crucial for full marks.

Solution 3:

Part a): 4/4 marks

Part b): 4/5 marks

Total: 8/9 marks

Feedback: Very good solution overall. Part a) was handled excellently with clear steps and proper justification. In part b), the student correctly identified the need for double integration by parts but didn't show the intermediate steps explicitly, which cost one mark.

\bigskip   
\textbf{Response 3:}

I'll grade each solution and provide concise feedback as requested.

Solution 1:

Part a) 4/4 marks

Part b) 5/5 marks

Total: 9/9 marks

Feedback: Excellent solution. Thorough step-by-step analysis for both parts. Correctly applied integration by parts and handled boundary terms. Clear explanation of why each operator is Hermitian.

Solution 2:

Part a) 4/4 marks

Part b) 3/5 marks

Total: 7/9 marks

Feedback: Strong approach for part a). Part b) lacks detailed integration by parts steps and doesn't explicitly show final equality. Good overall understanding but missing some specifics in part b).

Solution 3:

Part a) 4/4 marks

Part b) 4/5 marks

Total: 8/9 marks

Feedback: Excellent work on part a). Part b) is mostly correct but doesn't fully elaborate on the second integration by parts step. Good explanation of Hermitian properties for both operators.

\bigskip
\textbf{Response 4:}

I'll grade each solution out of 9 marks total, with up to 4 marks for part a) and up to 5 marks for part b), providing concise feedback for each.

Solution 1:

Part a): 4/4 marks

Part b): 5/5 marks

Total: 9/9 marks

Feedback: Excellent solution. The student demonstrated a thorough understanding of the concept, provided clear steps, and correctly applied integration by parts. They showed that both operators are Hermitian by matching the left and right-hand sides of the equation.

Solution 2:

Part a): 3/4 marks

Part b): 3/5 marks

Total: 6/9 marks

Feedback: The student showed a good understanding of the concept but lacked some details in the mathematical derivation. In part a), they didn't fully simplify the final expression. In part b), they mentioned using integration by parts twice but didn't show the steps or final expression explicitly.

Solution 3:

Part a): 4/4 marks

Part b): 4/5 marks

Total: 8/9 marks

Feedback: Very good solution overall. The student provided a detailed derivation for part a). In part b), they correctly mentioned applying integration by parts twice but didn't show the intermediate steps, which cost them one mark.

\bigskip
\textbf{Response 5:}

I'll grade each solution out of 9 marks total, with up to 4 marks for part a) and up to 5 marks for part b), as requested. I'll provide concise general feedback for each solution.

Solution 1:

Part a): 4/4 marks

Part b): 5/5 marks

Total: 9/9 marks

Feedback: Excellent solution. The student demonstrated a thorough understanding of the concept of Hermitian operators and provided clear, step-by-step derivations for both parts. They correctly applied integration by parts and handled the boundary conditions appropriately.

Solution 2:

Part a): 3/4 marks

Part b): 3/5 marks

Total: 6/9 marks

Feedback: The student showed a good grasp of the concept but lacked some details in the derivations. In part a), they missed showing the full integration by parts on the LHS. In part b), they mentioned using integration by parts twice but didn't show the actual steps or final form explicitly.

Solution 3:

Part a): 4/4 marks

Part b): 4/5 marks

Total: 8/9 marks

Feedback: Very good solution overall. The student provided a detailed derivation for part a) and correctly identified the Hermitian nature of the operator. In part b), they correctly mentioned applying integration by parts twice but didn't show the intermediate steps, which cost them one mark.

\subsection*{Question 4:}

\textbf{Response 1:}

I'll grade each solution out of 20 marks based on the provided mark scheme.

Solution 1:

a) 3/3 marks

b) 3/3 marks

c) 3/4 marks (missing one step in derivation)

d) 3/4 marks (missing one step in derivation)

e) 4/6 marks (correct approach but lacking detail in derivation)

Total: 16/20

General feedback: Good overall understanding and approach. Parts c) and d) could benefit from more detailed derivation steps. The explanation for part e) needs more rigorous justification.

Solution 2:

a) 3/3 marks

b) 3/3 marks

c) 4/4 marks

d) 4/4 marks

e) 4/6 marks (correct approach but lacking detail in derivation)

Total: 18/20

General feedback: Excellent work on parts a) through d). The solution for part e) is correct but could be more detailed in its explanation and derivation.

Solution 3:

a) 3/3 marks

b) 3/3 marks

c) 4/4 marks

d) 4/4 marks

e) 5/6 marks (good approach, slightly more detail needed)

Total: 19/20

General feedback: Very thorough and well-explained solution. All parts are handled well, with clear derivations. Part e) could benefit from a slightly more detailed explanation of why each term in the commutator is zero.

\bigskip   
\textbf{Response 2:}

I'll grade each solution out of 20 marks based on the given mark scheme.

Solution 1:

a) 3/3 marks - Correct expansion and reasoning.

b) 3/3 marks - Correct expansion and reasoning.

c) 3/4 marks - Correct result, but lacks detailed step-by-step derivation.

d) 3/4 marks - Correct result, but lacks detailed step-by-step derivation.

e) 5/6 marks - Correct approach and result, but missing the detailed derivation of \([L_x, P_z]\).

Total: 17/20

General feedback: Good overall understanding and approach. Could improve by providing more detailed derivations for parts c, d, and e.

Solution 2:

a) 3/3 marks - Correct expansion and reasoning.

b) 3/3 marks - Correct expansion and reasoning.

c) 4/4 marks - Correct expansion and derivation.

d) 4/4 marks - Correct expansion and derivation.

e) 4/6 marks - Correct approach, but lacks detailed derivation of why individual commutators are zero.

Total: 18/20

General feedback: Excellent work overall. Could improve by providing more detailed explanation in part e.

Solution 3:

a) 3/3 marks - Correct expansion and reasoning.

b) 3/3 marks - Correct expansion and reasoning.

c) 4/4 marks - Correct expansion and detailed derivation.

d) 4/4 marks - Correct expansion and detailed derivation.

e) 5/6 marks - Correct approach and reasoning, but lacks the final step of explicitly showing why each term is zero.

Total: 19/20

General feedback: Outstanding solution with detailed derivations. Could improve by providing a bit more detail in the final step of part e.

\bigskip   
\textbf{Response 3:}

Thank you for providing the problem, solutions, and marking scheme. I'll grade each solution based on the criteria you've given.

Solution 1:

a) 3/3 marks - Correct expansion and reasoning.

b) 3/3 marks - Correct expansion and reasoning.

c) 3/4 marks - Correct result, but missing detailed steps.

d) 3/4 marks - Correct result, but missing detailed steps.

e) 4/6 marks - Correct approach, but missing some key steps in the derivation.

Total: 16/20

General feedback: The solution is generally correct and well-explained. However, it lacks some detailed steps in parts c, d, and e, which could have earned full marks. The approach is sound, but more rigorous derivations would improve the solution.

Solution 2:

a) 3/3 marks - Correct expansion and reasoning.

b) 3/3 marks - Correct expansion and reasoning.

c) 4/4 marks - Detailed derivation with correct steps and result.

d) 4/4 marks - Detailed derivation with correct steps and result.

e) 5/6 marks - Good approach, but missing one step in the derivation.

Total: 19/20

General feedback: This solution is very thorough and well-explained. It provides detailed derivations for each part, showing a strong understanding of the concepts. The only minor improvement would be to elaborate slightly more on the final step of part e.

Solution 3:

a) 3/3 marks - Correct expansion and reasoning.

b) 3/3 marks - Correct expansion and reasoning.

c) 4/4 marks - Detailed derivation with correct steps and result.

d) 4/4 marks - Detailed derivation with correct steps and result.

e) 5/6 marks - Good approach, but missing one step in the derivation.

Total: 19/20

General feedback: This solution is comprehensive and well-structured. It demonstrates a strong grasp of the concepts and provides detailed derivations. The explanation in part e could be slightly more elaborate to earn full marks, but overall, it's an excellent solution.

\bigskip
\textbf{Response 4:}

I'll grade each solution out of 20 marks based on the given mark scheme. Here's the breakdown:

Solution 1:

a) 3/3 marks

b) 3/3 marks

c) 4/4 marks

d) 4/4 marks

e) 4/6 marks (missed the detailed derivation of [Lx, Pz])

Total: 18/20

Feedback: Excellent work on parts a) through d). The solution is clear and well-explained. For part e), while the reasoning is correct, it lacks the detailed derivation of [Lx, Pz] as required by the mark scheme.

Solution 2:

a) 3/3 marks

b) 3/3 marks

c) 3/4 marks (missing one step in simplification)

d) 3/4 marks (missing one step in simplification)

e) 4/6 marks (correct approach but lacks detailed derivation)

Total: 16/20

Feedback: Good overall approach. Parts a) and b) are perfect. In parts c) and d), an intermediate step in simplification is missing. Part e) has the correct reasoning but lacks the detailed derivation required for full marks.

Solution 3:

a) 3/3 marks

b) 3/3 marks

c) 4/4 marks

d) 4/4 marks

e) 4/6 marks (missed the detailed derivation of [Lx, Pz])

Total: 18/20

Feedback: Excellent work on parts a) through d) with clear explanations. For part e), while the reasoning is sound, it lacks the detailed derivation of [Lx, Pz] as required by the mark scheme for full marks.

\bigskip
\textbf{Response 5:}

I'll grade each solution and provide concise feedback:

Solution 1:

a) 3/3

b) 3/3

c) 4/4

d) 4/4

e) 5/6 (missed explaining why [L_x, P_z] = -ihbar P_y)

Total: 19/20

Feedback: Excellent work overall. The solution is well-structured and clearly explained. The only minor issue is in part (e), where the derivation of [L_x, P_z] = -ihbar P_y was not shown, which is crucial for full marks.

Solution 2:

a) 3/3

b) 3/3

c) 3/4 (missing step in derivation)

d) 3/4 (missing step in derivation)

e) 4/6 (incomplete explanation)

Total: 16/20

Feedback: Good understanding of the concepts, but some derivations lack detail. In parts (c) and (d), intermediate steps are missing. Part (e) needs a more rigorous explanation of why the commutators with squared momentum operators vanish.

Solution 3:

a) 3/3

b) 3/3

c) 4/4

d) 4/4

e) 4/6 (incomplete explanation for [L_x, P_z])

Total: 18/20

Feedback: Very good solution with clear explanations. The derivations for parts (a) through (d) are excellent. In part (e), the explanation for why [L_x, P_z] = -ihbar P_y is missing, which is necessary for a complete solution.


\subsection*{Question 5:}

\textbf{Response 1:}

I'll grade each solution out of 7 marks total, with up to 4 marks for part a) and up to 3 marks for part b). I'll provide concise general feedback for each solution.

Solution 1:

Part a): 4/4 marks

Part b): 3/3 marks

Total: 7/7 marks

General feedback: Excellent solution. The student demonstrated a thorough understanding of perturbation theory and applied it correctly to both parts of the problem. The approach was systematic, and the calculations were accurate. The student provided clear explanations and interpretations of the results.

Solution 2:

Part a): 3/4 marks

Part b): 3/3 marks

Total: 6/7 marks

General feedback: Overall, a good solution. The student correctly applied perturbation theory and arrived at the right answers. However, in part a), the student didn't show the detailed steps of simplifying and solving the integral, which cost them 1 mark. The explanation and derivation in part b) were complete and correct.

Solution 3:

Part a): 4/4 marks

Part b): 3/3 marks

Total: 7/7 marks

General feedback: Excellent solution. The student demonstrated a strong grasp of perturbation theory and applied it correctly to both parts. They provided detailed steps, including relevant trigonometric identities, which helped in solving the integrals. The explanations were clear and the interpretations of the results were accurate.

\bigskip   
\textbf{Response 2:}

I'll grade each solution out of 7 marks (4 for part a, 3 for part b) based on the provided mark scheme.

Solution 1:

Part a) - 4/4 marks

[1] Correctly states the equation for energy

[1] Substitutes H' and simplifies the integral

[2] Solves the integral, showing the perturbation equals 0

Part b) - 3/3 marks

[1] Correctly applies perturbation theory and sets up the integral

[1] Correctly solves the integral

[1] States the final result correctly

Total: 7/7 marks

General feedback: Excellent solution with clear step-by-step explanations. The student demonstrated a strong understanding of perturbation theory and showed all necessary work.

Solution 2:

Part a) - 4/4 marks

[1] Correctly states the equation for energy

[1] Substitutes H' and simplifies the integral

[2] Solves the integral, showing the perturbation equals 0

Part b) - 3/3 marks

[1] Correctly applies perturbation theory and sets up the integral

[1] Correctly solves the integral

[1] States the final result correctly

Total: 7/7 marks

General feedback: Very good solution with clear explanations. The student showed a solid grasp of the concepts and provided all necessary steps.

Solution 3:

Part a) - 4/4 marks

[1] Correctly states the equation for energy

[1] Substitutes H' and simplifies the integral

[2] Solves the integral, showing the perturbation equals 0

Part b) - 3/3 marks

[1] Correctly applies perturbation theory and sets up the integral

[1] Correctly solves the integral

[1] States the final result correctly

Total: 7/7 marks

General feedback: Excellent solution with thorough explanations. The student demonstrated a deep understanding of perturbation theory and provided detailed steps throughout the solution.

All three solutions are of high quality and received full marks. They each demonstrated a strong grasp of perturbation theory and provided clear, step-by-step solutions to both parts of the problem.

\bigskip   
\textbf{Response 3:}

I'll grade each solution out of a total of 7 marks, with up to 4 marks for part a) and up to 3 marks for part b), based on the provided mark scheme.

Solution 1:

Part a): 4/4 marks

- [1 mark] Correctly stated the equation for energy and made it relevant to the ground state.

- [1 mark] Correctly substituted the perturbation and simplified the integral.

- [2 marks] Solved the integral and showed the perturbation equals 0.

Part b): 3/3 marks

- [1 mark] Correctly applied perturbation theory and set up the integral.

- [1 mark] Correctly solved the integral and simplified it.

- [1 mark] Stated the correct result for odd and even values of n.

Total: 7/7 marks

Feedback: Excellent solution. The student demonstrated a thorough understanding of perturbation theory and provided clear, step-by-step explanations for both parts.

Solution 2:

Part a): 3/4 marks

- [1 mark] Correctly stated the equation for energy and made it relevant to the ground state.

- [1 mark] Correctly substituted the perturbation and simplified the integral.

- [1 mark] Stated that the perturbation equals 0.

- [-1 mark] Did not show the full working for solving the integral.

Part b): 3/3 marks

- [1 mark] Correctly applied perturbation theory and set up the integral.

- [1 mark] Correctly solved the integral and simplified it.

- [1 mark] Stated the correct result, though didn't explicitly mention odd and even n values.

Total: 6/7 marks

Feedback: Very good solution overall. The student showed a good grasp of the concepts. For part a), showing more steps in solving the integral would have been beneficial.

Solution 3:

Part a): 4/4 marks

- [1 mark] Correctly stated the equation for energy and made it relevant to the ground state.

- [1 mark] Correctly substituted the perturbation and simplified the integral.

- [2 marks] Solved the integral and showed the perturbation equals 0.

Part b): 3/3 marks

- [1 mark] Correctly applied perturbation theory and set up the integral.

- [1 mark] Correctly solved the integral and simplified it.

- [1 mark] Stated the correct result and mentioned the significance for odd and even n values.

Total: 7/7 marks

Feedback: Excellent solution. The student provided a comprehensive answer with clear explanations and demonstrated a deep understanding of perturbation theory and its application to this problem.

\bigskip
\textbf{Response 4:}

I'll grade each solution out of 7 marks total, with up to 4 marks for part a) and up to 3 marks for part b), using the provided mark scheme. I'll also give concise general feedback for each solution.

Solution 1:

Part a): 4/4 marks

Part b): 3/3 marks

Total: 7/7 marks

Feedback: Excellent solution. The student demonstrated a thorough understanding of perturbation theory and applied it correctly to both parts of the problem. The approach was methodical, and all steps were clearly explained. The final answers were correct and in the required form.

Solution 2:

Part a): 4/4 marks

Part b): 3/3 marks

Total: 7/7 marks

Feedback: Very good solution. The student correctly applied perturbation theory to both parts of the problem. The approach was clear and concise, with all necessary steps included. The final answers were correct and presented in the required form.

Solution 3:

Part a): 4/4 marks

Part b): 3/3 marks

Total: 7/7 marks

Feedback: Excellent solution. The student demonstrated a strong grasp of perturbation theory and applied it correctly to both parts of the problem. The solution was well-structured, with clear explanations of each step. The final answers were correct and presented in the required form.

All three solutions received full marks as they correctly addressed all aspects of the problem according to the mark scheme.

\bigskip
\textbf{Response 5:}

I'll grade each solution out of 7 marks total, with up to 4 marks for part a) and up to 3 marks for part b), using the provided mark scheme.

Solution 1:

Part a): 4/4 marks

- Correctly stated the equation for energy and made it relevant to the ground state [1 mark]

- Substituted H' and simplified the integral correctly [1 mark]

- Solved the integral, showing the perturbation equals 0 [2 marks]

Part b): 3/3 marks

- Correctly applied perturbation theory and set up the integral [1 mark]

- Correctly solved the integral [1 mark]

- Stated the correct result for odd and even values of n [1 mark]

Total: 7/7 marks

Feedback: Excellent solution. The student demonstrated a thorough understanding of perturbation theory and solved both parts correctly, showing all necessary steps.

Solution 2:

Part a): 4/4 marks

- Correctly stated the equation for energy and made it relevant to the ground state [1 mark]

- Substituted H' and simplified the integral correctly [1 mark]

- Solved the integral, showing the perturbation equals 0 [2 marks]

Part b): 2/3 marks

- Correctly applied perturbation theory and set up the integral [1 mark]

- Correctly solved the integral [1 mark]

- Did not explicitly state the result for odd and even values of n [0 marks]

Total: 6/7 marks

Feedback: Very good solution overall. The student solved both parts correctly, but in part b), they did not explicitly state the different results for odd and even values of n.

Solution 3:

Part a): 4/4 marks

- Correctly stated the equation for energy and made it relevant to the ground state [1 mark]

- Substituted H' and simplified the integral correctly [1 mark]

- Solved the integral, showing the perturbation equals 0 [2 marks]

Part b): 3/3 marks

- Correctly applied perturbation theory and set up the integral [1 mark]

- Correctly solved the integral [1 mark]

- Implied the correct result for odd and even values of n [1 mark]

Total: 7/7 marks

Feedback: Excellent solution. The student demonstrated a comprehensive understanding of the problem, solved both parts correctly, and provided detailed explanations of their approach.

\subsection*{Question 6:}

\textbf{Response 1:}

I'll grade each solution out of 17 marks, allocating up to 4 marks for part a), 4 marks for part b), and 9 marks for part c) as per your instructions.

Solution 1:

Part a): 4/4 marks

- Correctly expands \(J^2\) and derives the final expression

- Shows clear steps and reasoning

Part b): 3/4 marks

- Correctly identifies that \(L_+\) and \(S_-\) terms don't contribute

- Derives the correct eigenvalue

- Loses 1 mark for not explicitly showing it's an eigenvector

Part c): 7/9 marks

- Correctly identifies the basis states

- Provides a matrix representation

- Loses 2 marks for some errors in the matrix elements (e.g., -1 instead of -b)

Total: 14/17 marks

General feedback: Good overall approach, but be more careful with matrix elements in part c).

Solution 2:

Part a): 4/4 marks

- Thorough derivation with clear steps

- Correctly arrives at the final expression

Part b): 4/4 marks

- Clearly shows why the state is an eigenvector

- Correctly derives the eigenvalue

- Good explanation of the action of different terms

Part c): 7/9 marks

- Excellent explanation of the approach

- Correctly identifies basis states and transitions

- Loses 2 marks for not providing the full matrix representation

Total: 15/17 marks

General feedback: Very thorough explanations. For full marks in part c), include the complete matrix representation.

Solution 3:

Part a): 4/4 marks

- Clear, step-by-step derivation

- Correctly arrives at the final expression

Part b): 2/4 marks
- Correctly identifies \(j = l + s\)

- Loses 2 marks for not explicitly showing it's an eigenvector or deriving the eigenvalue

Part c): 9/9 marks

- Correctly identifies basis states

- Provides the full matrix representation

- Explains the matrix elements clearly

Total: 15/17 marks

General feedback: Excellent work on parts a) and c). Part b) could be more thorough in demonstrating the eigenvector property and deriving the eigenvalue.

\bigskip   
\textbf{Response 2:}

I'll grade each solution out of 17 marks, allocating up to 4 marks for part a), up to 4 marks for part b), and up to 9 marks for part c) as requested. I'll provide concise general feedback for each solution.

Solution 1:

Part a): 4/4 marks

Part b): 4/4 marks

Part c): 8/9 marks

Total: 16/17 marks

General feedback: Excellent work overall. The solution demonstrates a clear understanding of the concepts and provides detailed explanations. In part c), the matrix representation is correct, but the explanation could be more thorough regarding the basis states and their ordering.

Solution 2:

Part a): 4/4 marks

Part b): 3/4 marks

Part c): 6/9 marks

Total: 13/17 marks

General feedback: Good understanding of the concepts, but some areas need improvement. Part b) lacks a clear conclusion about the eigenvalue. Part c) provides a good theoretical explanation but fails to give the complete matrix representation as required.

Solution 3:

Part a): 4/4 marks

Part b): 2/4 marks

Part c): 8/9 marks

Total: 14/17 marks

General feedback: Strong performance in parts a) and c), but part b) is incomplete. The solution doesn't explicitly show that the state is an eigenvector or calculate the eigenvalue. The matrix representation in part c) is correct and well-explained, but could benefit from more detail on the calculations of the off-diagonal elements.

\bigskip   
\textbf{Response 3:}

I'll grade each solution out of 17 marks based on the provided mark scheme.

Solution 1:

Part a) 4/4 marks

- Correctly expanded \(J^2\) and showed all steps

- Properly used ladder operator relations

- Arrived at the correct final expression

Part b) 3/4 marks

- Correctly identified the eigenstate

- Showed that \(L_+\) and \(S_-\) terms don't contribute

- Calculated the correct eigenvalue

- Missing explicit demonstration that it's an eigenstate (-1 mark)

Part c) 7/9 marks

- Correctly identified the basis states

- Properly calculated the \(L^2\) contribution

- Correctly analyzed \(S+L_z\) action

- Matrix representation is incorrect (-2 marks)

Total: 14/17

General feedback: Good understanding of the concepts, but the matrix representation in part c needs correction.

Solution 2:

Part a) 4/4 marks

- Correctly expanded \(J^2\)

- Properly used ladder operator relations

- Arrived at the correct final expression

Part b) 4/4 marks

- Correctly identified the eigenstate

- Showed that \(L_+\) and \(S_-\) terms don't contribute

- Calculated the correct eigenvalue

- Explicitly demonstrated it's an eigenstate

Part c) 6/9 marks

- Correctly identified the basis states

- Properly calculated the \(L^2\) contribution

- Correctly analyzed \(S+L_z\) action

- Did not provide a complete matrix representation (-3 marks)

Total: 14/17

General feedback: Excellent work on parts a and b. Part c lacks a complete matrix representation.

Solution 3:

Part a) 4/4 marks

- Correctly expanded \(J^2\)

- Properly used ladder operator relations

- Arrived at the correct final expression

Part b) 2/4 marks

- Correctly identified the eigenstate

- Did not explicitly show it's an eigenstate (-1 mark)

- Did not calculate the eigenvalue (-1 mark)

Part c) 9/9 marks

- Correctly identified the basis states

- Properly calculated the \(L^2\) contribution

- Correctly analyzed \(S+L_z\) action

- Provided correct matrix representation

Total: 15/17

General feedback: Excellent work on parts a and c. Part b needs more detailed explanation and calculation of the eigenvalue.

\bigskip
\textbf{Response 4:}

I'll grade each solution out of 17 marks, allocating up to 4 marks for part a), 4 marks for part b), and 9 marks for part c) as requested. I'll provide concise general feedback for each solution.

Solution 1:

Part a): 4/4 marks

Part b): 4/4 marks

Part c): 7/9 marks

Total: 15/17 marks

General feedback: The solution demonstrates a strong understanding of the problem and provides detailed explanations. Parts a) and b) are excellently addressed. In part c), the matrix representation is mostly correct, but there are some minor errors in the off-diagonal elements and the overall presentation could be improved.

Solution 2:

Part a): 4/4 marks

Part b): 4/4 marks

Part c): 6/9 marks

Total: 14/17 marks

General feedback: This solution shows a good grasp of the concepts and provides thorough explanations for parts a) and b). However, in part c), while the approach is correct, the matrix representation is not fully completed, and some specific details are missing.

Solution 3:

Part a): 4/4 marks

Part b): 2/4 marks

Part c): 8/9 marks

Total: 14/17 marks

General feedback: The solution excels in parts a) and c), providing a clear and correct matrix representation in part c). However, part b) lacks a complete explanation and derivation of the eigenvalue, which resulted in a lower score for that section.

\bigskip
\textbf{Response 5:}

I'll grade each solution out of 17 marks and provide concise feedback for each.

Solution 1:

Part a): 4/4 marks

Part b): 3/4 marks

Part c): 8/9 marks

Total: 15/17 marks

Feedback: Strong understanding of the problem and methodology. Part a) was perfectly explained. In part b), the eigenvalue was correct, but the explanation of why the state is an eigenvector could have been more explicit. Part c) had a minor error in the matrix representation, but the approach and reasoning were sound.

Solution 2:

Part a): 4/4 marks

Part b): 3/4 marks

Part c): 6/9 marks

Total: 13/17 marks

Feedback: Excellent work on part a). In part b), the eigenvalue calculation was correct, but the explanation of why the state is an eigenvector could have been clearer. Part c) lacked a complete matrix representation, although the reasoning and approach were generally correct.

Solution 3:

Part a): 4/4 marks

Part b): 2/4 marks

Part c): 8/9 marks

Total: 14/17 marks

Feedback: Part a) was perfectly executed. In part b), the solution didn't explicitly show that the state is an eigenvector or calculate the eigenvalue. Part c) was well done with a correct matrix representation, losing only one mark for not explaining the basis state ordering.

\subsection*{Question 7:}

\textbf{Response 1:}

I'll grade each solution out of 8 marks, with up to 3 marks for part a) and up to 5 marks for part b), using the provided mark scheme.

Solution 1:

Part a): 3/3 marks

Part b): 5/5 marks

Total: 8/8 marks

Feedback: Excellent solution. The student provided a thorough and accurate answer for both parts. They correctly derived the time-dependent solution, showed the probability density's independence, and accurately calculated and interpreted the particle flux.

Solution 2:

Part a): 3/3 marks

Part b): 5/5 marks

Total: 8/8 marks

Feedback: Very well-done solution. The student correctly solved both parts of the problem, showing a clear understanding of the concepts. They provided a detailed derivation of the time-dependent solution and accurately calculated and interpreted the particle flux.

Solution 3:

Part a): 3/3 marks

Part b): 5/5 marks

Total: 8/8 marks

Feedback: Outstanding solution. The student demonstrated a comprehensive understanding of the problem, providing detailed and accurate solutions for both parts. They correctly derived the time-dependent solution, proved the probability density's independence, and accurately calculated and interpreted the particle flux.

All three solutions are excellent and deserve full marks. They each provided correct derivations, calculations, and interpretations as per the mark scheme.

\bigskip   
\textbf{Response 2:}

I'll grade each solution and provide concise feedback.

Solution 1:

Part a): 3/3 marks

Part b): 5/5 marks

Total: 8/8 marks

Feedback: Excellent solution. All steps are clearly explained and correct. The physical interpretation in part b) is particularly well-articulated.

Solution 2:

Part a): 3/3 marks

Part b): 4/5 marks

Total: 7/8 marks

Feedback: Very good solution overall. Part a) is perfect. In part b), the calculation is correct, but the physical interpretation could be more detailed, particularly in relating the flux to the velocity of particles.

Solution 3:

Part a): 3/3 marks

Part b): 5/5 marks

Total: 8/8 marks

Feedback: Excellent solution. The derivation in part a) is particularly thorough. The physical interpretation in part b) is clear and connects the flux to the velocity of particles effectively.

\bigskip   
\textbf{Response 3:}

I'll grade each solution out of 8 marks and provide concise feedback.

Solution 1:

Part a) 3/3 marks

Part b) 5/5 marks

Total: 8/8 marks

Feedback: Excellent solution. All steps are clearly shown and explained. The time-dependent solution is correct, and the probability density independence is well-demonstrated. The particle flux calculation and physical interpretation are thorough and accurate.

Solution 2:

Part a) 3/3 marks

Part b) 5/5 marks

Total: 8/8 marks

Feedback: Very good solution. The time-dependent solution is derived correctly, and the probability density independence is clearly shown. The particle flux calculation is accurate, and the physical interpretation is well-explained.

Solution 3:

Part a) 3/3 marks

Part b) 5/5 marks

Total: 8/8 marks

Feedback: Excellent solution. The time-dependent solution is derived step-by-step, and the probability density independence is clearly demonstrated. The particle flux calculation is correct, and the physical interpretation is concise and accurate.

All three solutions are of high quality and demonstrate a thorough understanding of the problem.

\bigskip
\textbf{Response 4:}

I'll grade each solution out of 8 marks and provide concise feedback.

Solution 1:

Part a) 3/3 marks

Part b) 5/5 marks

Total: 8/8 marks

Feedback: Excellent solution. The student demonstrated a thorough understanding of both parts, providing clear explanations and correct calculations. The physical interpretation in part b) was particularly well-articulated.

Solution 2:

Part a) 3/3 marks

Part b) 4/5 marks

Total: 7/8 marks

Feedback: Very good solution overall. Part a) was perfectly answered. In part b), the calculation was correct, but the physical interpretation could have been more detailed, particularly in relating the flux to the velocity of particles.

Solution 3:

Part a) 3/3 marks

Part b) 4/5 marks

Total: 7/8 marks

Feedback: Strong solution with excellent detail in part a). In part b), the calculation was correct, but the physical interpretation could have been more explicit in relating the flux to the number of particles passing through a given point per unit time.

\bigskip
\textbf{Response 5:}

I'll grade each solution out of 8 marks total, with up to 3 marks for part a) and up to 5 marks for part b), using the provided mark scheme. I'll also give concise general feedback for each solution.

Solution 1:

Part a): 3/3 marks

Part b): 5/5 marks

Total: 8/8 marks

Feedback: Excellent solution. The student provided a thorough and correct derivation for both parts, showing a clear understanding of the concepts. The physical interpretation in part b) was particularly well-explained.

Solution 2:

Part a): 3/3 marks

Part b): 5/5 marks

Total: 8/8 marks

Feedback: Very good solution. The student demonstrated a solid grasp of the material, providing clear and correct derivations for both parts. The explanation of the probability density independence was particularly well-done.

Solution 3:

Part a): 3/3 marks

Part b): 5/5 marks

Total: 8/8 marks

Feedback: Excellent solution. The student provided a comprehensive and correct approach to both parts. The step-by-step derivation was clear and easy to follow. The physical interpretation in part b) was concise and accurate.

All three solutions demonstrated a high level of understanding and received full marks. They each approached the problem slightly differently but arrived at the correct results with appropriate justifications.

\subsection*{Question 8:}

\textbf{Response 1:}

I'll grade each solution out of 9 marks as requested, with up to 5 marks for part a) and up to 4 marks for part b).

Solution 1:

Part a): 5/5 marks

Part b): 4/4 marks

Total: 9/9 marks

Feedback: Excellent solution. The student provided a clear, step-by-step derivation for both parts, showing a deep understanding of quantum mechanics principles. The use of the time-dependent Schrödinger equation and commutation relations was particularly well-executed.

Solution 2:

Part a): 5/5 marks

Part b): 3/4 marks

Total: 8/9 marks

Feedback: The solution for part a) is comprehensive and correct. In part b), the student correctly set up the problem but didn't fully elaborate on the commutation calculations, particularly in deriving the final result. A more detailed explanation of the commutation relations would have been beneficial.

Solution 3:

Part a): 5/5 marks

Part b): 4/4 marks

Total: 9/9 marks

Feedback: This solution demonstrates a thorough understanding of the problem. The student provided detailed derivations for both parts, showing excellent command of quantum mechanical operators and their properties. The step-by-step approach in part b) was particularly well-executed.

\bigskip   
\textbf{Response 2:}

I'll grade each solution out of 9 marks total, with up to 5 marks for part a) and up to 4 marks for part b). I'll provide concise general feedback for each solution.

Solution 1:

Part a): 5/5 marks

Part b): 4/4 marks

Total: 9/9 marks

Feedback: This solution is excellent. It demonstrates a clear understanding of the concepts and provides a thorough, step-by-step derivation for both parts. The explanation is well-structured and easy to follow.

Solution 2:

Part a): 5/5 marks

Part b): 3/4 marks

Total: 8/9 marks

Feedback: The solution for part a) is complete and correct. In part b), the commutator evaluation is not fully expanded, and the final step of the derivation is not explicitly shown, resulting in the loss of one mark.

Solution 3:

Part a): 5/5 marks

Part b): 4/4 marks

Total: 9/9 marks

Feedback: This solution provides a comprehensive and well-structured approach to both parts of the problem. The derivations are clear, detailed, and demonstrate a strong grasp of the underlying principles.

\bigskip   
\textbf{Response 3:}

I'll grade each solution out of 9 marks total, with up to 5 marks for part a) and up to 4 marks for part b), providing concise feedback for each.

Solution 1:

Part a): 5/5 marks

Part b): 4/4 marks

Total: 9/9 marks

Feedback: Excellent solution. Part a) is thorough and well-structured, showing all necessary steps. Part b) correctly evaluates the commutator and applies the result from part a) to reach the final conclusion.

Solution 2:

Part a): 5/5 marks

Part b): 2/4 marks

Total: 7/9 marks

Feedback: Part a) is correct and well-explained. However, part b) lacks a complete evaluation of the commutator [X, H] and doesn't show the full derivation to reach the final result.

Solution 3:

Part a): 5/5 marks

Part b): 4/4 marks

Total: 9/9 marks

Feedback: Comprehensive solution with clear steps and explanations. Both parts are correctly derived, with part b) showing a detailed evaluation of the commutator and application of the result from part a).

\bigskip
\textbf{Response 4:}

I'll grade each solution out of a total of 9 marks, with up to 5 marks for part a) and up to 4 marks for part b), using the provided mark scheme.

Solution 1:

Part a): 5/5 marks

- Correctly starts with the expectation value definition

- Properly differentiates with respect to time

- Correctly applies the Schrödinger equation

- Algebraically simplifies to the correct final form

Part b): 4/4 marks

- Correctly writes the Hamiltonian

- Accurately calculates the commutator [X, H]

- Properly applies the result from part a)

- Arrives at the correct final equation

Total: 9/9 marks

Feedback: Excellent solution. The student demonstrates a thorough understanding of quantum mechanics principles and provides a clear, step-by-step derivation for both parts.

Solution 2:

Part a): 5/5 marks

- Correctly differentiates the expectation value

- Properly applies the Schrödinger equation

- Algebraically simplifies to the correct final form

Part b): 3/4 marks

- Correctly identifies the Hamiltonian

- Attempts to evaluate the commutator [X, H]

- Applies the result from part a)

- Does not fully derive the final equation, losing 1 mark

Total: 8/9 marks

Feedback: Very good solution overall. The student shows a strong grasp of the concepts. However, in part b), the commutator calculation lacks some detail, and the final derivation is not fully explained.

Solution 3:

Part a): 5/5 marks

- Correctly starts with the expectation value definition

- Properly differentiates with respect to time

- Correctly applies the Schrödinger equation

- Algebraically simplifies to the correct final form

Part b): 4/4 marks

- Correctly writes the Hamiltonian

- Accurately calculates the commutator [X, H]

- Properly applies the result from part a)

- Arrives at the correct final equation

Total: 9/9 marks

Feedback: Excellent solution. The student provides a detailed and rigorous derivation for both parts, demonstrating a deep understanding of quantum mechanics principles and operator algebra.

\bigskip
\textbf{Response 5:}

I'll grade each solution and provide concise feedback.

Solution 1:

Part a): 5/5 marks

Part b): 4/4 marks

Total: 9/9 marks

Feedback: Excellent solution. The derivation is clear, step-by-step, and mathematically rigorous. Both parts are solved correctly with proper justification for each step.

Solution 2:

Part a): 5/5 marks

Part b): 3/4 marks

Total: 8/9 marks

Feedback: Part a) is solved perfectly. In part b), the commutator calculation is not fully detailed, leading to a loss of one mark. The final result is correct, but the intermediate steps could be more explicit.

Solution 3:

Part a): 5/5 marks

Part b): 4/4 marks

Total: 9/9 marks

Feedback: Excellent and thorough solution. Both parts are solved correctly with clear, step-by-step explanations. The commutator calculation in part b) is particularly well-detailed.


\subsection*{Question 9:}

\textbf{Response 1:}

I'll grade each solution out of 17 marks as requested, using the provided mark scheme.

Solution 1:

Part a) 5/5 marks

- Correctly derived general solutions for both regions

- Properly defined k1 and k2

Part b) 5/5 marks

- Applied correct boundary conditions

- Correctly derived expressions for B and C

Part c) 6/7 marks

- Correctly calculated incident, reflected, and transmitted flux

- Derived correct expressions for T and R

- Lost 1 mark for not explicitly showing the step of dividing by incident flux

Total: 16/17 marks

General feedback: Excellent solution overall. Very thorough and well-structured. Minor improvement could be made in part c) by explicitly showing the step of dividing by incident flux when deriving T and R.

Solution 2:

Part a) 5/5 marks

- Correctly derived general solutions for both regions

- Properly defined k1 and k2

Part b) 5/5 marks

- Applied correct boundary conditions

- Correctly derived expressions for B and C

Part c) 7/7 marks

- Correctly calculated incident, reflected, and transmitted flux

- Derived correct expressions for T and R, showing all steps clearly

Total: 17/17 marks

General feedback: Outstanding solution. All parts are answered correctly and thoroughly. The step-by-step approach is clear and easy to follow. Excellent work in showing all derivations explicitly.

Solution 3:

Part a) 5/5 marks

- Correctly derived general solutions for both regions

- Properly defined k1 and k2

Part b) 5/5 marks

- Applied correct boundary conditions

- Correctly derived expressions for B and C

Part c) 6/7 marks

- Correctly calculated incident, reflected, and transmitted flux

- Derived correct expressions for T and R

- Lost 1 mark for not explicitly showing the step of substituting the expressions for B and C

Total: 16/17 marks

General feedback: Very good solution overall. The approach is systematic and well-explained. To improve, in part c), explicitly show the substitution of the expressions for B and C when deriving T and R.

\bigskip   
\textbf{Response 2:}

I'll grade each solution and provide concise feedback.

Solution 1:

Part a): 5/5 marks

Part b): 5/5 marks

Part c): 7/7 marks

Total: 17/17 marks

Feedback: Excellent solution. Comprehensive and well-structured approach, covering all required points with clear explanations and correct derivations.

Solution 2:

Part a): 5/5 marks

Part b): 5/5 marks

Part c): 6/7 marks (minor error in reflected flux calculation)

Total: 16/17 marks

Feedback: Very good solution overall. Clear explanations and correct derivations for most parts. Small error in the reflected flux calculation (sign error), but the final results are correct.

Solution 3:

Part a): 5/5 marks

Part b): 5/5 marks

Part c): 6/7 marks (minor error in reflected flux calculation)

Total: 16/17 marks

Feedback: Strong solution with detailed explanations. All major points covered correctly. Small error in the reflected flux calculation (sign error), but final results are correct.

\bigskip   
\textbf{Response 3:}

I'll grade each solution out of 17 marks and provide concise feedback for each.

Solution 1:

Part a): 5/5 marks

Part b): 5/5 marks

Part c): 6/7 marks (minor error in flux calculations)

Total: 16/17 marks

Feedback: Excellent work overall. The solution is well-structured and thorough. There's a small error in the reflected flux calculation, which should be negative. Otherwise, the derivations and explanations are clear and accurate.

Solution 2:

Part a): 5/5 marks

Part b): 5/5 marks

Part c): 7/7 marks

Total: 17/17 marks

Feedback: Outstanding solution. All parts are correctly derived with clear explanations. The flux calculations are particularly well done, showing a deep understanding of the problem.

Solution 3:

Part a): 5/5 marks

Part b): 5/5 marks

Part c): 6/7 marks (minor error in reflected flux calculation)

Total: 16/17 marks

Feedback: Very good solution overall. The approach is systematic and well-explained. There's a small error in the reflected flux calculation (sign error), but the rest of the solution is correct and clearly presented.

\bigskip
\textbf{Response 4:}

I'll grade each solution based on the provided mark scheme, giving whole number marks for each part and providing concise general feedback.

Solution 1:

a) 5/5 marks

b) 5/5 marks

c) 7/7 marks

Total: 17/17 marks

Feedback: Excellent solution. All parts are thoroughly explained with correct mathematical derivations. The student demonstrated a clear understanding of the problem and applied the concepts correctly.

Solution 2:

a) 5/5 marks

b) 5/5 marks

c) 6/7 marks (Minor error in the reflected flux calculation)

Total: 16/17 marks

Feedback: Very good solution overall. The student showed a strong grasp of the concepts and provided clear explanations. There was a small error in the reflected flux calculation (sign error), but the final results were correct.

Solution 3:

a) 5/5 marks

b) 5/5 marks

c) 6/7 marks (Minor error in the reflected flux calculation)

Total: 16/17 marks

Feedback: Solid solution with clear step-by-step explanations. The student demonstrated good understanding of the problem. There was a small error in the reflected flux calculation (sign error), but the final results were correct.

\bigskip
\textbf{Response 5:}

I'll grade each solution out of 17 marks as requested, following the mark scheme provided.

Solution 1:

a) 5/5 marks

b) 5/5 marks

c) 6/7 marks (minor error in calculation of reflected flux)

Total: 16/17 marks

Feedback: Excellent work overall. The solution is well-structured and shows a clear understanding of the problem. The only minor issue is in part c) where the reflected flux calculation contains a small error (it should be negative). Otherwise, the derivations and explanations are thorough and correct.

Solution 2:

a) 5/5 marks

b) 5/5 marks

c) 7/7 marks

Total: 17/17 marks

Feedback: Outstanding solution. All parts are answered correctly with clear, step-by-step explanations. The derivations are rigorous and the final results are correct. The solution demonstrates a comprehensive understanding of the quantum mechanics involved in this problem.

Solution 3:

a) 5/5 marks

b) 5/5 marks

c) 6/7 marks (minor error in calculation of reflected flux)

Total: 16/17 marks

Feedback: Very good solution overall. The approach is methodical and shows a strong grasp of the concepts. The only small error is in part c) where the reflected flux is calculated with a positive sign instead of negative. All other derivations and results are correct and well-explained.

\subsection*{Question 10:}

\textbf{Response 1:}

I'll grade each solution out of 16 marks, allocating up to 5 marks for part a), 2 marks for part b), and 9 marks for part c), as requested. I'll provide concise general feedback for each solution.

Solution 1:

Part a): 5/5 marks

Part b): 2/2 marks

Part c): 6/9 marks

Total: 13/16 marks

Feedback: The solution demonstrates a strong understanding of parts a) and b). For part c), while the general approach is correct, the solution lacks the detailed mathematical derivation required to fully justify the given expression for b.

Solution 2:

Part a): 4/5 marks

Part b): 2/2 marks

Part c): 5/9 marks

Total: 11/16 marks

Feedback: The solution shows good understanding of parts a) and b), but there's a minor error in the normalization calculation. In part c), the explanation of the variational principle is correct, but the solution doesn't provide the necessary mathematical steps to derive the expression for b.

Solution 3:

Part a): 3/5 marks

Part b): 2/2 marks

Part c): 4/9 marks

Total: 9/16 marks

Feedback: The solution has some conceptual errors in part a), particularly in the normalization. Part b) is correct. For part c), while the approach is outlined, the solution lacks the necessary mathematical details and doesn't arrive at the final expression for b.

\bigskip   
\textbf{Response 2:}

I'll grade each solution out of 16 marks, allocating up to 5 marks for part a), 2 marks for part b), and 9 marks for part c) as requested. I'll provide concise general feedback for each solution.

Solution 1:

Part a): 5/5 marks

Part b): 2/2 marks

Part c): 6/9 marks

Total: 13/16 marks

Feedback: The solution demonstrates a strong understanding of parts a) and b). For part c), while the general approach is correct, the solution lacks the detailed mathematical derivation required to fully justify the given expression for b.

Solution 2:

Part a): 4/5 marks

Part b): 2/2 marks

Part c): 5/9 marks

Total: 11/16 marks

Feedback: The solution shows good understanding of parts a) and b), but there's a minor error in the normalization calculation. The approach to part c) is correct, but lacks the detailed mathematical steps to derive the expression for b.

Solution 3:

Part a): 3/5 marks

Part b): 2/2 marks

Part c): 4/9 marks

Total: 9/16 marks

Feedback: This solution has some conceptual errors in part a), particularly in the normalization. Part b) is correct. For part c), while the general approach is outlined, it lacks the necessary mathematical detail and doesn't arrive at the required expression for b.

\bigskip   
\textbf{Response 3:}

I'll grade each solution out of 16 marks, allocating up to 5 marks for part a), 2 marks for part b), and 9 marks for part c) as requested. I'll provide concise general feedback for each solution.

Solution 1:

Part a): 5/5 marks

Part b): 2/2 marks

Part c): 4/9 marks

Total: 11/16 marks

Feedback: The solution correctly addresses parts a) and b). However, in part c), while the variational principle is mentioned, the solution lacks the specific mathematical steps to derive the given expression for b. More detailed calculations were needed to fully answer this part.

Solution 2:

Part a): 4/5 marks

Part b): 2/2 marks

Part c): 5/9 marks

Total: 11/16 marks

Feedback: The solution provides a good approach to parts a) and b). In part a), there's a small error in the final expectation value calculation. Part c) shows a better understanding of the variational principle and its application, but still lacks the full derivation of the given expression for b.

Solution 3:

Part a): 3/5 marks

Part b): 2/2 marks

Part c): 3/9 marks

Total: 8/16 marks

Feedback: This solution has some good elements but also contains errors and inconsistencies. In part a), there's confusion about the normalization condition. Part b) is correct. Part c) acknowledges the complexity of the problem but doesn't provide the necessary mathematical steps to derive the given expression for b.

\bigskip
\textbf{Response 4:}

I'll now grade each solution out of a total of 16 marks, allocating up to 5 marks for part a), up to 2 marks for part b), and up to 9 marks for part c). I'll provide concise general feedback for each solution.

Solution 1:

Part a): 5/5 marks

Part b): 2/2 marks

Part c): 4/9 marks

Total: 11/16 marks

General feedback: The solution is strong in parts a) and b), showing correct normalization, expectation value calculation, and perturbation rewriting. However, part c) lacks the detailed mathematical derivation required to fully justify the given expression for b. While the general principle is explained, the specific steps to arrive at the final expression are missing.

Solution 2:

Part a): 4/5 marks

Part b): 2/2 marks

Part c): 3/9 marks

Total: 9/16 marks

General feedback: This solution provides a good attempt at parts a) and b), but there's an error in the normalization calculation in part a) that affects the expectation value. Part b) is correct. Part c) lacks the necessary mathematical detail to derive the given expression for b, though it correctly identifies the general approach.

Solution 3:

Part a): 3/5 marks

Part b): 2/2 marks

Part c): 2/9 marks

Total: 7/16 marks

General feedback: This solution shows understanding of the concepts but has significant errors and incomplete workings. In part a), there's confusion about the normalization condition. Part b) is correct. Part c) acknowledges the complexity of the problem but fails to provide the necessary mathematical steps to derive the given expression for b.

\bigskip
\textbf{Response 5:}

I'll grade each solution out of 16 marks, allocating up to 5 marks for part a), 2 marks for part b), and 9 marks for part c) as requested. I'll provide concise general feedback for each solution.

Solution 1:

Part a): 5/5 marks

Part b): 2/2 marks

Part c): 5/9 marks

Total: 12/16 marks

Feedback: The solution demonstrates a good understanding of normalization and expectation value calculation. The perturbation rewrite is correct. However, the explanation for part c) lacks the detailed mathematical derivation required to show how the given expression for b is obtained.

Solution 2:

Part a): 3/5 marks

Part b): 2/2 marks

Part c): 4/9 marks

Total: 9/16 marks

Feedback: The normalization in part a) is incorrect, leading to an incorrect expectation value. The perturbation rewrite in part b) is correct. Part c) provides a general explanation but lacks the specific mathematical steps to derive the given expression for b.

Solution 3:

Part a): 3/5 marks

Part b): 2/2 marks

Part c): 3/9 marks

Total: 8/16 marks

Feedback: The solution initially makes errors in normalization but partially corrects them. The expectation value calculation is incomplete. The perturbation rewrite is correct. Part c) acknowledges the complexity of the problem but doesn't provide the required mathematical derivation.

\end{document}