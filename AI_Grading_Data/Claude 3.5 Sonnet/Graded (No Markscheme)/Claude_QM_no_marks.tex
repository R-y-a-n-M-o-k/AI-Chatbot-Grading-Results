\documentclass[a4paper,11pt]{article}
\setlength\parindent{0pt}
\usepackage{amsmath, amssymb}
\usepackage[margin=1in,letterpaper]
{geometry} % decreases margins

\usepackage{setspace}
\setstretch{1.2}
\usepackage{spverbatim}

\title{Claude 3.5 Sonnet Grading (No Mark Scheme)}

\begin{document}

\maketitle
This document compiles the Claude 3.5 Sonnet responses when prompted to grade the solutions which have been generated. No markscheme is given to Claude 3.5 Sonnet in the prompt. This is the grading for the Quantum Mechanics (QM) questions. 

\section{QM}
\subsection*{Question 1:}

\textbf{Response 1:}

Here are the grades and feedback for each solution:

Solution 1:

Grade: 5/5 (3/3 for part a, 2/2 for part b)

Feedback: Excellent response. The student provided a comprehensive definition of the commutator, clearly explained compatibility, and correctly described the implications of non-compatibility on measurement outcomes. The answer demonstrates a solid understanding of quantum mechanical principles.

Solution 2:

Grade: 5/5 (3/3 for part a, 2/2 for part b)

Feedback: Very good answer. The student accurately defined the commutator and compatibility, and correctly explained the consequences of non-compatibility. The response shows a clear grasp of the concepts and their implications in quantum measurements.

Solution 3:

Grade: 5/5 (3/3 for part a, 2/2 for part b)

Feedback: Strong response. The student provided clear definitions and explanations, demonstrating a good understanding of the commutator, compatibility, and their implications in quantum mechanics. The answer effectively addressed both parts of the question.

All three solutions are excellent and demonstrate a thorough understanding of the concepts. They each correctly defined the commutator, explained compatibility, and described the implications of non-compatibility on measurements in quantum mechanics.

\bigskip
\textbf{Response 2:}

I'll grade each solution out of 5 marks (3 for part a, 2 for part b) and provide concise feedback.

Solution 1:

Part a: 3/3

Part b: 2/2

Total: 5/5

Feedback: Excellent understanding of commutators, compatibility, and their implications. Clear explanation of non-compatible operators and their effects on measurements. Thorough and accurate response to both parts.

Solution 2:

Part a: 3/3

Part b: 2/2

Total: 5/5

Feedback: Comprehensive explanation of commutators and compatibility. Strong understanding of measurement effects for non-compatible operators. Well-articulated response with precise details for both parts.

Solution 3:

Part a: 3/3

Part b: 1/2

Total: 4/5

Feedback: Excellent definition of commutator and explanation of compatibility in part a. Part b lacks detail on the probabilities of different outcomes in the final measurement of P. Otherwise, a solid understanding of the concepts.

\bigskip
\textbf{Response 3:}

I'll grade each solution and provide concise feedback:

Solution 1:

Part a): 3/3 marks

Part b): 2/2 marks

Total: 5/5 marks

Feedback: Excellent explanation of commutators, compatibility, and their implications. The answer for part b) is comprehensive and accurately describes the consequences of non-compatibility.

Solution 2:

Part a): 3/3 marks

Part b): 2/2 marks

Total: 5/5 marks

Feedback: Clear and concise definition of commutators and compatibility. The explanation of non-compatibility in part b) is accurate and well-articulated, including the probabilistic nature of subsequent measurements.

Solution 3:

Part a): 3/3 marks

Part b): 1/2 marks

Total: 4/5 marks

Feedback: Part a) is well-explained with proper mathematical notation. However, part b) lacks a discussion of the probabilistic nature of the subsequent measurement and the dependence on the overlap between eigenstates of P and Q.

\bigskip   
\textbf{Response 4:}

I'll grade each solution out of 5 marks total, with 3 marks for part a) and 2 marks for part b), using only whole numbers. Here's the evaluation:

Solution 1:

Part a): 3/3

Part b): 2/2

Total: 5/5

Feedback: Excellent answer. The solution provides a comprehensive explanation of the commutator, compatibility, and their implications. It correctly addresses the non-commutative nature of incompatible operators and its consequences in quantum measurements.

Solution 2:

Part a): 3/3

Part b): 2/2

Total: 5/5

Feedback: Very well-written response. The solution accurately defines the commutator and compatibility, and explains their implications. The discussion on incompatible operators and the uncertainty in sequential measurements is thorough and correct.

Solution 3:

Part a): 3/3

Part b): 1/2

Total: 4/5

Feedback: Good explanation of the commutator and compatibility in part a). However, in part b), while the answer correctly states that the second measurement of P is not guaranteed to yield p again, it doesn't fully explain the range of possible outcomes or the probabilistic nature of the results.

\bigskip   
\textbf{Response 5:}

I'll grade each solution out of 5 marks (3 for part a and 2 for part b) and provide concise feedback.

Solution 1:

Part a: 3/3

Part b: 2/2

Total: 5/5

Feedback: Excellent explanation of commutators and compatibility. The answer thoroughly covers the implications of compatibility and non-compatibility. The discussion of measurement outcomes in part b is particularly well-articulated.

Solution 2:

Part a: 3/3

Part b: 2/2

Total: 5/5

Feedback: Very clear and precise definition of commutators and compatibility. The explanation of measurement outcomes for non-compatible operators is accurate and well-explained. The answer effectively illustrates the probabilistic nature of subsequent measurements.

Solution 3:

Part a: 3/3

Part b: 1/2

Total: 4/5

Feedback: Part a is correct and well-explained. In part b, while the explanation is generally correct, it lacks depth in discussing the probabilistic nature of the subsequent measurement of P. The answer could be improved by mentioning that the probabilities of different outcomes depend on the overlap between eigenstates of P and Q.

\subsection*{Question 2:}

\textbf{Response 1:}

I'll grade each solution out of 4 marks total, with 2 marks for part a) and 2 marks for part b).

Solution 1:

Total score: 4/4 (a: 2/2, b: 2/2)

Feedback: Excellent explanation of both parts. The answer correctly identifies the real nature of eigenvalues and provides a clear physical interpretation. The explanation of orthonormality is comprehensive and mathematically precise.

Solution 2:

Total score: 4/4 (a: 2/2, b: 2/2)

Feedback: Very good answer for both parts. The explanation of why eigenvalues are real is mathematically sound. The orthonormality concept is well-explained with appropriate mathematical notation.

Solution 3:

Total score: 4/4 (a: 2/2, b: 2/2)

Feedback: Excellent response with detailed mathematical explanations. The answer provides a rigorous proof for the real nature of eigenvalues and a clear explanation of orthonormality with proper mathematical notation.

All three solutions demonstrate a high level of understanding of the concepts and deserve full marks.

\bigskip
\textbf{Response 2:}

I'll grade each solution out of 4 marks total, with 2 marks for part a) and 2 marks for part b).

Solution 1:

Part a): 2/2 marks

Part b): 2/2 marks

Total: 4/4 marks

Feedback: This solution provides a comprehensive and accurate explanation of both parts. It correctly identifies the mathematical property of eigenvalues and their physical interpretation, and gives a clear explanation of orthonormality.

Solution 2:

Part a): 2/2 marks

Part b): 1/2 marks

Total: 3/4 marks

Feedback: The explanation for part a) is excellent. For part b), while the solution correctly defines orthonormality, it lacks a clear explanation of normalization and doesn't explicitly mention the integral form of the inner product.

Solution 3:

Part a): 2/2 marks

Part b): 2/2 marks

Total: 4/4 marks

Feedback: This solution provides a thorough and mathematically rigorous explanation for both parts. It includes a mathematical proof for the realness of eigenvalues in part a) and clearly defines orthonormality in part b) with proper mathematical notation.

\bigskip
\textbf{Response 3:}

I'll grade each solution out of 4 marks total, with 2 marks for part a) and 2 marks for part b).

Solution 1:

Part a): 2/2

Part b): 2/2

Total: 4/4

Feedback: Excellent explanation of both parts. The student correctly identified the mathematical property of eigenvalues and provided a clear physical interpretation. The explanation of orthonormality was thorough and mathematically precise.

Solution 2:

Part a): 2/2

Part b): 1/2

Total: 3/4

Feedback: Strong explanation of part a). For part b), the student correctly defined orthonormality but didn't elaborate on the normalization aspect as clearly as they could have. The physical significance was well-explained.

Solution 3:

Part a): 2/2

Part b): 2/2

Total: 4/4

Feedback: Excellent, mathematically rigorous explanation for both parts. The student provided a detailed proof for the realness of eigenvalues in part a) and gave a clear, concise definition of orthonormality in part b), including its mathematical representation and physical significance.

\bigskip   
\textbf{Response 4:}

I'll grade each solution out of 4 marks total, with 2 marks for part a) and 2 marks for part b).

Solution 1:

Part a): 2/2 marks

Part b): 2/2 marks

Total: 4/4 marks

Feedback: This solution provides a comprehensive and accurate explanation of both parts. It correctly identifies the real nature of eigenvalues and their physical interpretation, and gives a clear explanation of orthonormality with mathematical detail.

Solution 2:

Part a): 2/2 marks

Part b): 1/2 marks

Total: 3/4 marks

Feedback: The explanation for part a) is thorough and correct. However, in part b), while the concept of orthonormality is explained, the solution lacks a clear distinction between orthogonality and normalization, which are two separate conditions.

Solution 3:

Part a): 2/2 marks

Part b): 2/2 marks

Total: 4/4 marks

Feedback: This solution provides a mathematically rigorous explanation for both parts. It includes a proof sketch for the realness of eigenvalues in part a) and gives a clear, mathematically precise definition of orthonormality in part b).

\bigskip   
\textbf{Response 5:}

I'll grade each solution and provide concise feedback as requested.

Solution 1:

Part a): 2/2 marks

Part b): 2/2 marks

Total: 4/4 marks

Feedback: Excellent explanation of both parts. The answer provides a clear mathematical property of eigenvalues and its physical interpretation. The explanation of orthonormality is thorough and mathematically precise.

Solution 2:

Part a): 2/2 marks

Part b): 2/2 marks

Total: 4/4 marks

Feedback: Comprehensive answer for both parts. The explanation of the real nature of eigenvalues includes mathematical reasoning. The orthonormality concept is well-explained with appropriate mathematical notation.

Solution 3:

Part a): 2/2 marks

Part b): 2/2 marks

Total: 4/4 marks

Feedback: Rigorous mathematical explanation for both parts. The answer provides a detailed proof for the real nature of eigenvalues and a precise definition of orthonormality. The physical interpretations are also well-explained.

All three solutions demonstrate a high level of understanding of the concepts and provide correct, detailed explanations.


\subsection*{Question 3:}

\textbf{Response 1:}

I'll grade each solution out of 9 marks total, with 4 marks for part a) and 5 marks for part b). Here's the breakdown:

Solution 1:

Part a): 4/4 marks

Part b): 5/5 marks

Total: 9/9 marks

Feedback: Excellent solution. The student provides a thorough and mathematically rigorous proof for both parts, showing a clear understanding of Hermitian operators and integration by parts. The explanations are detailed and well-structured.

Solution 2:

Part a): 3/4 marks

Part b): 3/5 marks

Total: 6/9 marks

Feedback: The solution demonstrates a good understanding of the concept, but lacks some mathematical rigor. In part a), the integration by parts is not fully shown. In part b), the proof is incomplete, as the student doesn't explicitly show the equality after integration by parts.

Solution 3:

Part a): 4/4 marks

Part b): 4/5 marks

Total: 8/9 marks

Feedback: Very good solution overall. Part a) is excellent, with a clear and rigorous proof. In part b), the student correctly applies integration by parts twice, but doesn't explicitly show all steps, which costs one mark. The conclusion is correct for both parts.

\bigskip
\textbf{Response 2:}

I'll grade each solution out of 9 marks total, with 4 marks for part a) and 5 marks for part b).

Solution 1:

Part a): 4/4

Part b): 5/5

Total: 9/9

Feedback: Excellent solution. The student provided a thorough and mathematically rigorous approach for both parts. They correctly applied integration by parts and handled complex conjugates properly. The explanation is clear and well-structured.

Solution 2:

Part a): 3/4

Part b): 3/5

Total: 6/9

Feedback: The student demonstrates a good understanding of the concept but lacks some mathematical rigor. In part a), the integration by parts is not fully shown. In part b), the double integration by parts is mentioned but not explicitly carried out. More detailed mathematical steps would improve this solution.

Solution 3:

Part a): 4/4

Part b): 4/5

Total: 8/9

Feedback: This solution is very good overall. Part a) is handled excellently with clear steps and proper mathematical reasoning. In part b), the student correctly identifies the need for double integration by parts but doesn't show the full working, which costs them one mark. The conclusion is correct, but more detail in the mathematical steps would make this solution perfect.

\bigskip
\textbf{Response 3:}

I'll grade each solution out of 9 marks total, with 4 marks for part a) and 5 marks for part b).

Solution 1:

Part a): 4/4 marks

Part b): 5/5 marks

Total: 9/9 marks

Feedback: Excellent solution. The student provided a thorough and mathematically rigorous approach to both parts. They correctly applied integration by parts and handled complex conjugates. The explanation is clear and well-structured.

Solution 2:

Part a): 3/4 marks

Part b): 3/5 marks

Total: 6/9 marks

Feedback: The student demonstrated a good understanding of the concept but lacked some detail in the mathematical derivations, especially in part b). The approach is correct, but more explicit steps would have improved the solution.

Solution 3:

Part a): 4/4 marks

Part b): 4/5 marks

Total: 8/9 marks

Feedback: Very good solution overall. Part a) is excellent, with clear steps and proper handling of complex conjugates. In part b), the student correctly identified the need for integration by parts twice but didn't show the full working, which would have been beneficial.

\bigskip   
\textbf{Response 4:}

I'll grade each solution out of 9 marks total, with 4 marks for part a) and 5 marks for part b), providing concise feedback for each.

Solution 1:

Part a): 4/4 marks

Part b): 5/5 marks

Total: 9/9 marks

Feedback: Excellent solution. Both parts are thoroughly explained with clear mathematical steps. The integration by parts is correctly applied, and the vanishing boundary conditions are properly considered.

Solution 2:

Part a): 4/4 marks

Part b): 3/5 marks

Total: 7/9 marks

Feedback: Part a) is well-explained and correctly solved. In part b), the general approach is correct, but the solution lacks explicit mathematical steps for the integration by parts, which are crucial for a complete proof.

Solution 3:

Part a): 4/4 marks

Part b): 4/5 marks

Total: 8/9 marks

Feedback: Part a) is solved correctly with clear steps. In part b), the solution is mostly correct, but it could benefit from more explicit intermediate steps in the integration by parts process.

\bigskip   
\textbf{Response 5:}

I'll grade each solution and provide concise feedback as requested.

Solution 1:

Part a): 4/4 marks

Part b): 5/5 marks

Total: 9/9 marks

Feedback: Excellent solution. Both parts are thoroughly explained with clear mathematical steps. The integration by parts is well-executed, and the vanishing of boundary terms is properly addressed.

Solution 2:

Part a): 3/4 marks

Part b): 3/5 marks

Total: 6/9 marks

Feedback: Good attempt, but lacks some detail. In part a), the integration by parts is not fully shown. Part b) is correct in principle but doesn't show the complete mathematical working. More explicit steps would improve the solution.

Solution 3:

Part a): 4/4 marks

Part b): 4/5 marks

Total: 8/9 marks

Feedback: Very good solution. Part a) is perfect. In part b), the integration by parts is mentioned but not fully shown, which would have made the solution complete. Otherwise, the reasoning and conclusions are correct.


\subsection*{Question 4:}

\textbf{Response 1:}

I'll grade each solution out of 20 marks as requested:

Solution 1:

a) 3/3

b) 3/3

c) 4/4

d) 4/4

e) 6/6

Total: 20/20

Feedback: Excellent solution. All parts are correctly derived with clear explanations. The student demonstrates a thorough understanding of commutator relations and angular momentum operators.

Solution 2:

a) 3/3

b) 3/3

c) 4/4

d) 4/4

e) 5/6

Total: 19/20

Feedback: Very good solution overall. Parts a) through d) are derived correctly with proper explanations. In part e), while the final answer is correct, the explanation could be more detailed on why the commutators of \( \hat{L_{x}} \) with \( \hat{P_{y}}^2 \) and \( \hat{P_{z}}^2 \) are zero.

Solution 3:

a) 3/3

b) 3/3

c) 4/4

d) 4/4

e) 5/6

Total: 19/20

Feedback: Excellent work on parts a) through d) with clear derivations. In part e), while the final answer is correct, the explanation for why each term in the expansion is zero could be more detailed, particularly for \( \hat{P_{y}}^2 \) and \( \hat{P_{z}}^2 \).

\bigskip
\textbf{Response 2:}

I'll grade each solution out of 20 marks as requested:

Solution 1:

a) 3/3

b) 3/3

c) 4/4

d) 4/4

e) 6/6

Total: 20/20

Feedback: Excellent solution. All parts are correctly derived with clear explanations and proper use of commutator properties. The step-by-step approach is thorough and well-structured.

Solution 2:

a) 3/3

b) 3/3

c) 4/4

d) 4/4

e) 5/6

Total: 19/20

Feedback: Very good solution overall. Parts a) through d) are correctly derived. In part e), while the final answer is correct, the explanation could be more detailed on why the commutator of Lx with the square of momentum components vanishes.

Solution 3:

a) 3/3

b) 3/3

c) 4/4

d) 4/4

e) 5/6

Total: 19/20

Feedback: Excellent work on parts a) through d) with clear explanations. For part e), the reasoning is correct, but it would be beneficial to show explicitly why the commutator of Lx with each squared momentum term is zero, rather than just stating it.

\bigskip
\textbf{Response 3:}

I'll grade each solution out of 20 marks as requested.

Solution 1:

a) 3/3 marks

b) 3/3 marks

c) 4/4 marks

d) 4/4 marks

e) 5/6 marks (minor explanation lacking)

Total: 19/20

Feedback: Excellent solution with clear explanations and correct derivations. The approach is systematic and shows a good understanding of quantum mechanics principles. The only minor improvement could be in part e), where a bit more explanation on why the individual terms commute would be beneficial.

Solution 2:

a) 3/3 marks

b) 3/3 marks

c) 3/4 marks (missing detailed steps)

d) 3/4 marks (missing detailed steps)

e) 5/6 marks (good reasoning, but lacks detailed derivation)

Total: 17/20

Feedback: Good overall solution with correct results. The explanations for parts c) and d) could be more detailed, showing intermediate steps. The reasoning in part e) is sound, but a more explicit derivation would strengthen the solution.

Solution 3:

a) 3/3 marks

b) 3/3 marks

c) 4/4 marks

d) 4/4 marks

e) 5/6 marks (good approach, but could elaborate on why each term is zero)

Total: 19/20

Feedback: Excellent solution with clear, step-by-step derivations. The explanations are thorough and show a deep understanding of the concepts. In part e), a brief explanation of why each term in the expansion is zero would make the solution perfect.

\bigskip   
\textbf{Response 4:}

I'll grade each solution out of 20 marks, allocating marks as specified. Then I'll provide concise general feedback for each solution.

Solution 1:

a) 3/3

b) 3/3

c) 4/4

d) 4/4

e) 5/6

Total: 19/20

General feedback: Excellent solution with clear explanations and correct derivations. The only minor issue is in part e), where the explanation could be more detailed on why the commutator of \(L_x\) with each squared momentum component is zero.

Solution 2:

a) 3/3

b) 3/3

c) 3/4

d) 3/4

e) 4/6

Total: 16/20

General feedback: Good overall approach, but lacks detail in some derivations. Parts c) and d) need more explicit steps. The explanation for part e) is incomplete and doesn't fully justify why each term is zero.

Solution 3:

a) 3/3

b) 3/3

c) 4/4

d) 4/4

e) 5/6

Total: 19/20

General feedback: Very thorough solution with clear explanations and correct derivations. The only minor issue is in part e), where the explanation for why each term is zero could be slightly more explicit.

\bigskip   
\textbf{Response 5:}

I'll grade each solution out of 20 marks and provide concise feedback for each.

Solution 1:

a) 3/3

b) 3/3

c) 4/4

d) 4/4

e) 5/6

Total: 19/20

Feedback: Excellent work overall. The solution demonstrates a clear understanding of the angular momentum operator and commutation relations. The step-by-step approach is thorough and well-explained. The only minor issue is in part e), where the explanation could be more explicit about why the individual terms commute.

Solution 2:

a) 3/3

b) 3/3

c) 3/4

d) 3/4

e) 5/6

Total: 17/20

Feedback: Good overall understanding of the concepts. The approach is correct, but the explanations for parts c) and d) lack some detail in the intermediate steps. The solution for part e) is correct but could benefit from a more rigorous explanation of why the commutators with squared momentum operators vanish.

Solution 3:

a) 3/3

b) 3/3

c) 4/4

d) 4/4

e) 6/6

Total: 20/20

Feedback: Exceptional work. This solution provides clear, step-by-step derivations with appropriate explanations for each part. The use of vector notation at the beginning adds clarity. The explanation for part e) is particularly well-done, showing a deep understanding of the relationship between angular momentum and the squared momentum operator.

\subsection*{Question 5:}

\textbf{Response 1:}

I'll grade each solution out of 7 marks, with up to 4 marks for part a) and up to 3 marks for part b). I'll provide concise general feedback for each solution.

Solution 1:

Part a): 4/4 marks

Part b): 3/3 marks

Total: 7/7 marks

Feedback: This solution is comprehensive and well-structured. It correctly applies perturbation theory, shows clear steps in the calculations, and arrives at the correct results for both parts. The explanation of the trigonometric identities used is particularly thorough.

Solution 2:

Part a): 3/4 marks

Part b): 3/3 marks

Total: 6/7 marks

Feedback: This solution correctly applies perturbation theory and arrives at the right results. However, in part a), it lacks detailed explanation of the integration process and the trigonometric identities used, which would have made the solution more complete.

Solution 3:

Part a): 4/4 marks

Part b): 3/3 marks

Total: 7/7 marks

Feedback: This solution is well-organized and thorough. It correctly applies perturbation theory, clearly explains the steps, and provides the correct results. The use of trigonometric identities is well-explained, especially in part a).

\bigskip
\textbf{Response 2:}

I'll grade each solution out of 7 marks (4 for part a, 3 for part b) and provide concise feedback.

Solution 1:

Part a: 4/4

Part b: 3/3

Total: 7/7

Feedback: Excellent solution. Both parts are thoroughly explained with correct mathematical derivations. The student showed a clear understanding of perturbation theory and correctly applied it to both scenarios.

Solution 2:

Part a: 3/4

Part b: 3/3

Total: 6/7

Feedback: Good solution overall. Part a lacks detailed steps in the integration process, which would have been helpful. Part b is well-explained and correct. The student demonstrates a good grasp of the concepts but could improve on showing more detailed work.

Solution 3:

Part a: 4/4

Part b: 3/3

Total: 7/7

Feedback: Outstanding solution. Both parts are solved correctly with clear, step-by-step explanations. The student effectively used trigonometric identities and properties of the Dirac delta function. The solution demonstrates a deep understanding of the problem and perturbation theory.

\bigskip
\textbf{Response 3:}

I'll grade each solution out of 7 marks (4 for part a, 3 for part b) and provide concise feedback.

Solution 1:

Part a: 3/4

Part b: 3/3

Total: 6/7

Feedback: The approach and calculations are correct. In part a, the student correctly set up the integral but didn't show the detailed steps of simplification and integration. The final result is correct. Part b is well-explained and correctly derived.

Solution 2:

Part a: 3/4

Part b: 3/3

Total: 6/7

Feedback: The solution is mostly correct. In part a, the student correctly set up the problem but didn't show the detailed steps of simplification and integration. The final result is correct. Part b is well-explained and correctly derived.

Solution 3:

Part a: 4/4

Part b: 3/3

Total: 7/7

Feedback: This solution is comprehensive and correct. In part a, the student showed a detailed approach, including the use of trigonometric identities for simplification. The integration process was mentioned, leading to the correct result. Part b is also well-explained and correctly derived.

\bigskip   
\textbf{Response 4:}

I'll grade each solution out of 7 marks (4 for part a and 3 for part b) and provide concise feedback.

Solution 1:

Part a: 4/4

Part b: 3/3

Total: 7/7

Feedback: Excellent solution. Part a is solved thoroughly with clear steps and correct integration. Part b is also correct, with proper use of the Dirac delta function properties. The final expressions for both parts are accurate.

Solution 2:

Part a: 3/4

Part b: 3/3

Total: 6/7

Feedback: Good solution overall. Part a lacks detailed integration steps, though the final result is correct. Part b is solved correctly with appropriate use of the Dirac delta function. The final expressions for both parts are accurate.

Solution 3:

Part a: 4/4

Part b: 3/3

Total: 7/7

Feedback: Excellent solution. Part a is solved thoroughly with clear steps, including trigonometric identities, and correct integration. Part b is also correct, with proper use of the Dirac delta function properties. The final expressions for both parts are accurate.

\bigskip   
\textbf{Response 5:}

I'll grade each solution out of 7 marks, with up to 4 marks for part a) and up to 3 marks for part b). I'll provide concise feedback for each solution.

Solution 1:

Part a): 4/4 marks

Part b): 3/3 marks

Total: 7/7 marks

Feedback: Excellent solution. The student provided a thorough and correct approach to both parts. They showed a clear understanding of perturbation theory and applied it correctly. The calculations were detailed and accurate, and the final results were correct for both parts.

Solution 2:

Part a): 3/4 marks

Part b): 3/3 marks

Total: 6/7 marks

Feedback: Good solution overall. The approach and final results for both parts are correct. However, in part a), the student didn't show the detailed steps of simplifying and solving the integral, which would have been valuable to see. Part b) was handled well with correct reasoning and result.

Solution 3:

Part a): 4/4 marks

Part b): 3/3 marks

Total: 7/7 marks

Feedback: Excellent solution. The student demonstrated a strong understanding of perturbation theory and applied it correctly to both parts. They provided detailed steps, including relevant trigonometric identities in part a), and arrived at the correct results for both parts. The explanation was clear and thorough throughout.

\subsection*{Question 6:}

\textbf{Response 1:}

I'll grade each solution out of 17 marks and provide concise feedback for each.

Solution 1:

Part a): 4/4

Part b): 3/4

Part c): 8/9

Total: 15/17

Feedback: Excellent work on part a) with clear, step-by-step derivation. In part b), the eigenvalue was correctly derived, but the explanation of why the state is an eigenvector could be more explicit. Part c) was well-approached, but the final matrix representation had some minor errors in the off-diagonal elements.

Solution 2:

Part a): 4/4

Part b): 3/4

Part c): 6/9

Total: 13/17

Feedback: Part a) was perfectly executed. In part b), the reasoning was sound, but the final eigenvalue expression could be more concise. Part c) lacked a complete matrix representation, though the approach and reasoning were generally correct.

Solution 3:

Part a): 4/4

Part b): 2/4

Part c): 8/9

Total: 14/17

Feedback: Excellent derivation in part a). Part b) was incomplete, missing the explicit eigenvalue calculation and full explanation of why the state is an eigenvector. Part c) was very well done, with a correct matrix representation and clear explanation of the basis states used.

\bigskip
\textbf{Response 2:}

I'll grade each solution out of a total of 17 marks, allocating up to 4 marks for part a), 4 marks for part b), and 9 marks for part c). I'll provide concise general feedback for each solution.

Solution 1:

Part a): 4/4 marks

Part b): 3/4 marks

Part c): 7/9 marks

Total: 14/17 marks

General feedback: This solution demonstrates a strong understanding of the problem and provides detailed explanations. The approach to part a) is thorough and correct. In part b), the explanation is mostly correct, but it could be more explicit in deriving the final eigenvalue. Part c) shows good work, but the matrix representation contains some errors in the off-diagonal elements.

Solution 2:

Part a): 4/4 marks

Part b): 3/4 marks

Part c): 6/9 marks

Total: 13/17 marks

General feedback: This solution provides a comprehensive approach to the problem. Part a) is solved correctly with clear reasoning. In part b), the explanation is good but lacks a clear final statement of the eigenvalue. Part c) demonstrates understanding of the process, but falls short of providing a complete matrix representation, which is a significant part of the question.

Solution 3:

Part a): 4/4 marks

Part b): 2/4 marks

Part c): 8/9 marks

Total: 14/17 marks

General feedback: This solution shows strong mathematical skills. Part a) is solved correctly and concisely. Part b) lacks depth in the explanation and doesn't clearly state the eigenvalue, which is a key part of the question. Part c) is well-executed, providing a correct matrix representation with clear explanations, though it could benefit from more detailed derivation of the matrix elements.

\bigskip
\textbf{Response 3:}

I'll evaluate each solution and provide grades and feedback.

Solution 1:

Part a) 4/4 marks

Part b) 3/4 marks

Part c) 7/9 marks

Total: 14/17 marks

Feedback: This solution demonstrates a good understanding of the problem and provides detailed explanations. Part a) is solved correctly with clear steps. In part b), the eigenvalue is correctly derived, but there's no explicit statement that the state is an eigenvector. Part c) shows good work on setting up the problem, but the final matrix representation contains some errors in the off-diagonal elements.

Solution 2:

Part a) 4/4 marks

Part b) 3/4 marks

Part c) 6/9 marks

Total: 13/17 marks

Feedback: The solution for part a) is thorough and correct. In part b), the approach is good, but it lacks a clear statement that the state is an eigenvector. Part c) provides a good setup and explanation of the approach, but falls short of giving the complete matrix representation, which is a key part of the question.

Solution 3:

Part a) 4/4 marks

Part b) 2/4 marks

Part c) 8/9 marks

Total: 14/17 marks

Feedback: Part a) is solved correctly with clear steps. In part b), the solution correctly identifies the state as an eigenvector but doesn't derive the eigenvalue explicitly. Part c) provides an almost perfect matrix representation with correct explanations, losing one mark for not explicitly showing the calculations for the off-diagonal elements.

Overall, all solutions demonstrate a good understanding of the problem, with some minor weaknesses in different areas. Solution 1 and Solution 3 tie for the highest score, each excelling in different parts of the problem.

\bigskip   
\textbf{Response 4:}

I'll grade each solution out of 17 marks, providing concise feedback for each.

Solution 1:

Part a) 4/4 marks

Part b) 3/4 marks

Part c) 7/9 marks

Total: 14/17 marks

Feedback: Excellent work on part a), showing a clear and correct derivation. In part b), the reasoning is sound, but the final eigenvalue expression could be simplified further. Part c) is generally correct, but the matrix representation lacks clear explanation of how the elements were derived.

Solution 2:

Part a) 4/4 marks

Part b) 3/4 marks

Part c) 6/9 marks

Total: 13/17 marks

Feedback: Part a) is thoroughly explained and correct. Part b) provides good reasoning but doesn't explicitly state the final eigenvalue. Part c) offers a detailed explanation of the approach but falls short of providing the complete matrix representation.

Solution 3:

Part a) 4/4 marks

Part b) 2/4 marks

Part c) 8/9 marks

Total: 14/17 marks

Feedback: Part a) is correctly derived and well-explained. Part b) lacks a clear conclusion and doesn't provide the explicit eigenvalue. Part c) is very well done, providing a clear matrix representation with good explanation of its structure.

\bigskip   
\textbf{Response 5:}

I'll grade each solution out of 17 marks, with up to 4 marks for part a), 4 marks for part b), and 9 marks for part c). I'll provide concise feedback for each solution.

Solution 1:

Part a): 4/4 marks

Part b): 3/4 marks

Part c): 7/9 marks

Total: 14/17 marks

Feedback: The solution demonstrates a strong understanding of the problem and provides detailed explanations. Part a) is solved correctly with clear steps. In part b), the eigenvalue is correctly derived, but there's no explicit statement about why the state is an eigenvector. Part c) shows good understanding, but the matrix representation lacks clarity in its derivation and doesn't explicitly show the constants a and b.

Solution 2:

Part a): 4/4 marks

Part b): 3/4 marks

Part c): 6/9 marks

Total: 13/17 marks

Feedback: Part a) is solved comprehensively and correctly. In part b), the solution correctly identifies the eigenvalue but doesn't explicitly state why the given state is an eigenvector. Part c) provides a good explanation of the approach but falls short of presenting a complete matrix representation. The solution correctly identifies the diagonal elements but doesn't fully elaborate on all off-diagonal elements.

Solution 3:

Part a): 4/4 marks

Part b): 2/4 marks

Part c): 8/9 marks

Total: 14/17 marks

Feedback: Part a) is solved correctly with clear steps. Part b) lacks a complete explanation of why the state is an eigenvector and doesn't explicitly calculate the eigenvalue. Part c) is well-executed, providing a clear matrix representation with correct elements, including the constants a and b. However, it could benefit from a more detailed explanation of how each matrix element was derived.

\subsection*{Question 7:}

\textbf{Response 1:}

I'll grade each solution out of 8 marks and provide concise feedback.

Solution 1:

Part a): 3/3 marks

Part b): 5/5 marks

Total: 8/8 marks

Feedback: Excellent solution. It's thorough, well-structured, and demonstrates a clear understanding of quantum mechanics principles. The derivations are correct, and the physical interpretations are accurate.

Solution 2:

Part a): 3/3 marks

Part b): 4/5 marks

Total: 7/8 marks

Feedback: Very good solution overall. Part a) is perfect. In part b), the calculation is correct, but the physical interpretation could be more detailed, particularly in relating the flux to the velocity of the particles.

Solution 3:

Part a): 3/3 marks

Part b): 4/5 marks

Total: 7/8 marks

Feedback: Strong solution with clear step-by-step reasoning. Part a) is excellent. In part b), the calculation is correct, but the physical interpretation could be expanded slightly to fully connect the flux to the particle velocity.

\bigskip
\textbf{Response 2:}

I'll grade each solution out of 8 marks and provide concise feedback.

Solution 1:

Part a): 3/3 marks

Part b): 5/5 marks

Total: 8/8 marks

Feedback: Excellent solution. The student provided a comprehensive and accurate analysis of both parts, including correct derivations and clear physical interpretations.

Solution 2:

Part a): 3/3 marks

Part b): 4/5 marks

Total: 7/8 marks

Feedback: Very good solution overall. Part a) is perfect. In part b), the flux calculation is correct, but the physical interpretation could be more detailed, particularly in explaining the relationship between flux and velocity.

Solution 3:

Part a): 3/3 marks

Part b): 4/5 marks

Total: 7/8 marks

Feedback: Strong solution with clear steps and derivations. Part a) is excellent. In part b), the flux calculation is correct, but the physical interpretation could be more explicit about the relationship between flux and probability current density.

\bigskip
\textbf{Response 3:}

I'll grade each solution and provide concise feedback for each.

Solution 1:

Part a): 3/3 marks

Part b): 5/5 marks

Total: 8/8 marks

Feedback: Excellent solution. The student demonstrated a thorough understanding of both parts, providing clear derivations and explanations. The physical interpretation in part b) was particularly well-articulated.

Solution 2:

Part a): 3/3 marks

Part b): 4/5 marks

Total: 7/8 marks

Feedback: Very good solution overall. Part a) was solved perfectly. In part b), the derivation was correct, but the physical interpretation could have been more detailed, particularly in relating the flux to the velocity of particles.

Solution 3:

Part a): 3/3 marks

Part b): 4/5 marks

Total: 7/8 marks

Feedback: Strong solution with clear step-by-step reasoning. Part a) was solved comprehensively. In part b), while the calculation was correct, the physical interpretation could have been more explicitly linked to the particle velocity and probability density flow.

\bigskip   
\textbf{Response 4:}

I'll grade each solution and provide concise feedback for each.

Solution 1:

Part a): 3/3 marks

Part b): 5/5 marks

Total: 8/8 marks

Feedback: Excellent solution. Comprehensive explanation of both parts with correct mathematical derivations and physical interpretations. The student showed a clear understanding of the time-dependent Schrödinger equation, probability density, and particle flux.

Solution 2:

Part a): 3/3 marks

Part b): 4/5 marks

Total: 7/8 marks

Feedback: Very good solution overall. Part a) is correct and well-explained. In part b), the derivation is correct, but the physical interpretation could be more detailed, particularly in relating the flux to the velocity of the particles.

Solution 3:

Part a): 3/3 marks

Part b): 4/5 marks

Total: 7/8 marks

Feedback: Strong solution with good mathematical derivations. Part a) is thorough and correct. In part b), the flux calculation is correct, but the physical interpretation could be expanded to more clearly relate the flux to the particle velocity and probability density flow.

\bigskip   
\textbf{Response 5:}

I'll grade each solution out of 8 marks, with up to 3 marks for part a) and up to 5 marks for part b), and provide concise general feedback.

Solution 1:

Part a): 3/3 marks

Part b): 5/5 marks

Total: 8/8 marks

Feedback: Excellent solution. All steps are clearly explained and correct. The physical interpretation in part b) is particularly well-articulated.

Solution 2:

Part a): 3/3 marks

Part b): 4/5 marks

Total: 7/8 marks

Feedback: Very good solution overall. Part a) is perfect. In part b), the flux calculation is correct, but the physical interpretation could be more detailed, particularly in relating the flux to the velocity of particles.

Solution 3:

Part a): 3/3 marks

Part b): 4/5 marks

Total: 7/8 marks

Feedback: Strong solution with clear explanations. Part a) is excellent. In part b), the flux calculation is correct, but the physical interpretation could be more explicit in relating the flux to the probability current and particle velocity.


\subsection*{Question 8:}

\textbf{Response 1:}

I'll grade each solution and provide concise feedback:

Solution 1:

Part a): 5/5 marks

Part b): 4/4 marks

Total: 9/9 marks

Feedback: Excellent solution. Clear, step-by-step approach with proper mathematical reasoning. Both parts are thoroughly explained and correctly derived.

Solution 2:

Part a): 5/5 marks

Part b): 2/4 marks

Total: 7/9 marks

Feedback: Part a) is well-explained and correctly derived. However, part b) lacks a complete derivation of the commutator and doesn't show the full steps to reach the final result.

Solution 3:

Part a): 5/5 marks

Part b): 4/4 marks

Total: 9/9 marks

Feedback: Comprehensive solution with clear, logical steps. Both parts are correctly derived with proper mathematical reasoning and explanations.

\bigskip
\textbf{Response 2:}

I'll grade each solution out of 9 marks total, with up to 5 marks for part a) and up to 4 marks for part b). I'll provide concise general feedback for each solution.

Solution 1:

Part a): 5/5 marks

Part b): 4/4 marks

Total: 9/9 marks

Feedback: Excellent solution. Clear, step-by-step approach with proper explanations. Both parts are solved thoroughly and correctly, demonstrating a deep understanding of quantum mechanics principles and mathematical techniques.

Solution 2:

Part a): 5/5 marks

Part b): 2/4 marks

Total: 7/9 marks

Feedback: Part a) is solved correctly and thoroughly. However, part b) lacks detailed calculations, especially in evaluating the commutator \([X, H]\). The final result is stated without sufficient derivation, leading to a loss of marks in part b).

Solution 3:

Part a): 5/5 marks

Part b): 4/4 marks

Total: 9/9 marks

Feedback: Comprehensive and accurate solution. Both parts are solved correctly with detailed steps and clear explanations. The solution demonstrates a strong grasp of quantum mechanics concepts and mathematical rigor.

\bigskip
\textbf{Response 3:}

I'll grade each solution out of 9 marks total, with up to 5 marks for part a) and up to 4 marks for part b).

Solution 1:
Part a): 5/5 marks
Part b): 4/4 marks
Total: 9/9 marks

Feedback: Excellent solution. Part a) is thorough and clearly explained. Part b) correctly evaluates the commutator and applies the result from part a). The solution demonstrates a strong understanding of quantum mechanics principles.

Solution 2:
Part a): 5/5 marks
Part b): 2/4 marks
Total: 7/9 marks

Feedback: Part a) is well-explained and correct. However, part b) lacks a complete derivation of the commutator \([X,H]\). The final result is stated without sufficient justification, leading to a loss of marks in this section.

Solution 3:
Part a): 5/5 marks
Part b): 4/4 marks
Total: 9/9 marks

Feedback: Comprehensive and well-structured solution. Part a) is derived clearly and correctly. Part b) provides a detailed evaluation of the commutator \([X,H]\) and correctly applies the result from part a). The solution demonstrates excellent understanding and mathematical rigor.

\bigskip   
\textbf{Response 4:}

I'll grade each solution out of 9 marks total, with up to 5 marks for part a) and up to 4 marks for part b). I'll provide concise general feedback for each solution.

Solution 1:

Part a): 5/5

Part b): 4/4

Total: 9/9

General feedback: Excellent solution. The student demonstrated a thorough understanding of the problem, presented clear and logical steps, and arrived at the correct conclusions for both parts. The mathematical derivations were precise and well-explained.

Solution 2:

Part a): 5/5

Part b): 2/4

Total: 7/9

General feedback: The student showed a strong grasp of part a), providing a clear and correct derivation. However, in part b), while the approach was correct, the student didn't fully complete the derivation of the commutator [X, H]. The final step linking the commutator to the expectation value of momentum was not explicitly shown.

Solution 3:

Part a): 5/5

Part b): 4/4

Total: 9/9

General feedback: Outstanding solution. The student provided a comprehensive and mathematically rigorous approach to both parts. The derivations were clear, logical, and complete. The student demonstrated an excellent understanding of quantum mechanical operators and their properties.

\bigskip   
\textbf{Response 5:}

I'll grade each solution out of 9 marks total, with up to 5 marks for part a) and up to 4 marks for part b). I'll provide concise general feedback for each solution.

Solution 1:

Part a): 5/5 marks

Part b): 4/4 marks

Total: 9/9 marks

General feedback: This solution is excellent. It provides a clear, step-by-step approach for both parts, showing a deep understanding of quantum mechanics principles. The derivations are rigorous and well-explained, with appropriate use of mathematical notation and quantum mechanical concepts.

Solution 2:

Part a): 5/5 marks

Part b): 3/4 marks

Total: 8/9 marks

General feedback: Part a) is thoroughly explained and correctly derived. In part b), the solution correctly sets up the problem and identifies the key steps. However, it lacks the detailed derivation of the commutator [X, H], which is crucial for demonstrating the final result. The conclusion is correct, but the intermediate steps are not fully shown.

Solution 3:

Part a): 5/5 marks

Part b): 4/4 marks

Total: 9/9 marks

General feedback: This solution demonstrates a comprehensive understanding of the problem. Both parts are derived rigorously with clear, logical steps. The mathematical notation is precise, and the explanations are thorough. The solution shows excellent command of quantum mechanical operators and their properties.


\subsection*{Question 9:}

\textbf{Response 1:}

I'll grade each solution out of 17 marks and provide concise feedback for each.

Solution 1:

Part a): 5/5 marks

Part b): 5/5 marks

Part c): 6/7 marks

Total: 16/17 marks

Feedback: Excellent work overall. The solution is well-structured and thorough. Part a) and b) are perfect. In part c), the flux calculations are correct, but the solution could have been more explicit in showing how T and R were derived from the flux ratios.

Solution 2:

Part a): 5/5 marks

Part b): 5/5 marks

Part c): 7/7 marks

Total: 17/17 marks

Feedback: Outstanding solution. All parts are correctly addressed with clear explanations. The derivation of T and R from the flux ratios is explicitly shown, which is excellent. The step-by-step approach makes the solution easy to follow.

Solution 3:

Part a): 5/5 marks

Part b): 5/5 marks

Part c): 6/7 marks

Total: 16/17 marks

Feedback: Very good solution with clear explanations. Parts a) and b) are perfect. In part c), the flux calculations and final expressions for T and R are correct, but the solution could have been more explicit in showing the intermediate steps for deriving T and R from the flux ratios.

\bigskip
\textbf{Response 2:}

I'll grade each solution and provide concise feedback:

Solution 1:

Part a): 5/5

Part b): 5/5

Part c): 7/7

Total: 17/17

Feedback: Excellent solution. All parts are thoroughly explained with clear mathematical derivations. The student demonstrated a strong understanding of quantum mechanics principles and problem-solving skills.

Solution 2:

Part a): 5/5

Part b): 5/5

Part c): 6/7

Total: 16/17

Feedback: Very good solution overall. Parts a) and b) are perfectly answered. In part c), the calculation of reflected flux has a sign error, but the final expressions for T and R are correct.

Solution 3:

Part a): 5/5

Part b): 5/5

Part c): 5/7

Total: 15/17

Feedback: Strong solution for parts a) and b). In part c), the derivation of the reflected flux is incorrect, leading to an error in the expression for R. The expression for T is correct, but the method to obtain it is not clearly shown.

\bigskip
\textbf{Response 3:}

I'll grade each solution out of 17 marks and provide concise feedback for each.

Solution 1:

Part a): 5/5 marks

Part b): 5/5 marks

Part c): 6/7 marks

Total: 16/17 marks

Feedback: Excellent work overall. The solution is well-structured and thorough. Part a) and b) are perfect. In part c), the derivation is correct, but the solution could have been more concise in the flux calculations.

Solution 2:

Part a): 5/5 marks

Part b): 5/5 marks

Part c): 6/7 marks

Total: 16/17 marks

Feedback: Very good solution. Parts a) and b) are flawless. In part c), the flux calculations are correct, but the presentation could be more streamlined. The final derivations of T and R are well-explained.

Solution 3:

Part a): 5/5 marks

Part b): 5/5 marks

Part c): 5/7 marks

Total: 15/17 marks

Feedback: Strong solution overall. Parts a) and b) are perfect. In part c), while the flux calculations are correct, the derivation of T and R lacks some detail in the intermediate steps. The final expressions are correct, but more explanation of how they were obtained would improve the solution.

\bigskip   
\textbf{Response 4:}

I'll grade each solution out of 17 marks, with up to 5 marks for parts a) and b), and up to 7 marks for part c). I'll provide concise general feedback for each solution.

Solution 1:

Part a): 5/5

Part b): 5/5

Part c): 7/7

Total: 17/17

Feedback: Excellent solution. All parts are thoroughly explained with clear derivations. The student demonstrated a strong understanding of quantum mechanics principles and mathematical techniques.

Solution 2:

Part a): 5/5

Part b): 5/5

Part c): 6/7

Total: 16/17

Feedback: Very good solution overall. Parts a) and b) are perfectly answered. In part c), the calculation of the reflected flux has a minor sign error, but the final results are correct. The explanation is clear and well-structured.

Solution 3:

Part a): 5/5

Part b): 5/5

Part c): 6/7

Total: 16/17

Feedback: Excellent work on parts a) and b). In part c), the calculation of the reflected flux is incorrect (although it doesn't affect the final result). The derivation of T and R is correct but could be more clearly explained. Overall, a very good solution with strong mathematical reasoning.

\bigskip   
\textbf{Response 5:}

I'll grade each solution out of 17 marks, with up to 5 marks for parts a) and b), and up to 7 marks for part c). I'll provide concise general feedback for each solution.

Solution 1:

Part a): 5/5 marks

Part b): 5/5 marks

Part c): 7/7 marks

Total: 17/17 marks

General feedback: Excellent solution. The student provided a comprehensive and well-structured answer, clearly defining k1 and k2, correctly applying boundary conditions, and thoroughly deriving the transmission and reflection probabilities.

Solution 2:

Part a): 5/5 marks

Part b): 5/5 marks

Part c): 6/7 marks

Total: 16/17 marks

General feedback: Very good solution overall. The student correctly solved parts a) and b). In part c), the calculation of fluxes was correct, but the explanation of the relationship between fluxes and probabilities could have been more explicit.

Solution 3:

Part a): 5/5 marks

Part b): 5/5 marks

Part c): 6/7 marks

Total: 16/17 marks

General feedback: Strong solution with clear step-by-step explanations. The student excelled in parts a) and b). In part c), the flux calculations were correct, but the final step connecting the flux ratios to the probabilities could have been more clearly explained.

\subsection*{Question 10:}

\textbf{Response 1:}

I'll grade each solution out of 16 marks, providing concise feedback for each.

Solution 1:

Part a) 4/5 marks

Part b) 2/2 marks

Part c) 6/9 marks

Total: 12/16 marks

Feedback: Good normalization and expectation value calculation. Correct rewriting of perturbation. Explanation of variational principle is sound, but lacks specific derivation of the given expression for\(b\).

Solution 2:

Part a) 3/5 marks

Part b) 2/2 marks

Part c) 4/9 marks

Total: 9/16 marks

Feedback: Normalization error in part a). Correct perturbation rewriting. Variational principle explanation is brief and lacks specificity in relating to the given expression for \(b\).

Solution 3:

Part a) 2/5 marks

Part b) 2/2 marks

Part c) 5/9 marks

Total: 9/16 marks

Feedback: Significant errors in normalization. Correct perturbation rewriting. Better attempt at explaining the variational principle application, but still lacks complete derivation of \(b\).

\bigskip
\textbf{Response 2:}

I'll grade each solution out of 16 marks and provide concise feedback.

Solution 1:

Part a): 4/5 marks

Part b): 2/2 marks

Part c): 6/9 marks

Total: 12/16 marks

Feedback: Good normalization and expectation value calculation. Correct rewriting of perturbation. Explanation of variational principle is strong, but lacks detailed mathematical derivation of the final expression for b.

Solution 2:

Part a): 3/5 marks

Part b): 2/2 marks

Part c): 4/9 marks

Total: 9/16 marks

Feedback: Normalization calculation has an error (\(|b|^2 = 1\) is incorrect). Expectation value calculation is correct. Perturbation rewriting is good. Explanation of variational principle is present but lacks depth and mathematical derivation.

Solution 3:

Part a): 3/5 marks

Part b): 2/2 marks

Part c): 5/9 marks

Total: 10/16 marks

Feedback: Recognizes and attempts to correct normalization error, but still contains mistakes. Perturbation rewriting is correct. Provides a good conceptual framework for the variational principle but lacks the mathematical derivation of the final expression for \(b\).

\bigskip
\textbf{Response 3:}

I'll grade each solution out of 16 marks, with up to 5 marks for part a), 2 marks for part b), and 9 marks for part c). I'll provide concise general feedback for each solution.

Solution 1:

Part a): 5/5

Part b): 2/2

Part c): 7/9

Total: 14/16

General feedback: Excellent work on parts a) and b). The solution for part c) provides a good explanation of the variational principle but lacks the detailed mathematical derivation of the given expression for b. More rigorous mathematical steps would have earned full marks.

Solution 2:

Part a): 4/5

Part b): 2/2

Part c): 6/9

Total: 12/16

General feedback: Good approach to parts a) and b). There's a minor error in the normalization calculation in part a). Part c) provides a good conceptual explanation but lacks the mathematical derivation of the expression for b. More detailed calculations would improve the solution.

Solution 3:

Part a): 3/5

Part b): 2/2

Part c): 5/9

Total: 10/16

General feedback: The solution shows good understanding but contains errors in part a), particularly in the normalization. Part b) is correct. Part c) acknowledges the complexity of the problem but doesn't provide the necessary mathematical steps to derive the expression for b. More rigorous calculations and error correction would significantly improve this solution.

\bigskip   
\textbf{Response 4:}

I'll grade each solution out of 16 marks, with up to 5 marks for part a), 2 marks for part b), and 9 marks for part c). I'll provide concise general feedback for each solution.

Solution 1:

Part a): 5/5 marks

Part b): 2/2 marks

Part c): 7/9 marks

Total: 14/16 marks

General feedback: Excellent work on parts a) and b). The solution for part c) correctly outlines the variational principle but lacks the detailed mathematical derivation of the given expression for b. More explicit steps in the calculation would have been beneficial.

Solution 2:

Part a): 4/5 marks

Part b): 2/2 marks

Part c): 5/9 marks

Total: 11/16 marks

General feedback: Good attempt overall. There's a minor error in part a) where \(|b|^2\) is incorrectly set to 1. Part b) is correct. Part c) lacks the detailed mathematical derivation required to show how the given expression for b is obtained.

Solution 3:

Part a): 3/5 marks

Part b): 2/2 marks

Part c): 4/9 marks

Total: 9/16 marks

General feedback: This solution shows understanding of the concepts but contains errors and lacks completeness. There are mistakes in the normalization process in part a). Part b) is correct. Part c) acknowledges the complexity of the problem but doesn't provide the required mathematical derivation.

\bigskip   
\textbf{Response 5:}

I'll grade each solution out of 16 marks, allocating up to 5 marks for part a), 2 marks for part b), and 9 marks for part c). I'll provide concise general feedback for each solution.

Solution 1:

Part a): 5/5

Part b): 2/2

Part c): 7/9

Total: 14/16

General feedback: This solution demonstrates a strong understanding of the problem and provides clear, step-by-step explanations. The normalization and expectation value calculations in part a) are correct and well-explained. Part b) is solved correctly. In part c), while the general approach is correct, the solution lacks a detailed derivation of the final expression for b, which prevented it from receiving full marks.

Solution 2:

Part a): 4/5

Part b): 2/2

Part c): 6/9

Total: 12/16

General feedback: This solution shows good understanding of the concepts but has some errors. In part a), there's a mistake in the normalization calculation (|b|² = 1 instead of 0). Part b) is correct. Part c) provides a good explanation of the variational principle but lacks the detailed derivation of the expression for b, and doesn't fully connect the given expression to the problem setup.

Solution 3:

Part a): 3/5

Part b): 2/2

Part c): 5/9

Total: 10/16

General feedback: This solution starts strong but becomes confused and less coherent as it progresses. Part a) has correct calculations but confusion about normalization. Part b) is correct. Part c) acknowledges the complexity of the problem but doesn't provide a clear path to the solution or derive the given expression for b. The solution also contains some unnecessary self-corrections that disrupt the flow of the answer.


\end{document}