\documentclass[a4paper,11pt]{article}
\setlength\parindent{0pt}
\usepackage{amsmath, amssymb}
\usepackage[margin=1in,letterpaper]
{geometry} % decreases margins

\usepackage{setspace}
\setstretch{1.2}
\usepackage{spverbatim}

\title{Gemini 1.5 Pro Grading (No Mark Scheme)}


\begin{document}

\maketitle
This document compiles the Gemini 1.5 Pro responses when prompted to grade the solutions which have been generated. No markscheme is given to Gemini 1.5 Pro in the prompt. This is the grading for the Electromagnetism (EM) questions. 

\section{EM}
\subsection*{Question 1:}

\textbf{Response 1:}

**Solution 1: 8/8 marks**

*General Feedback:* This is an excellent and thorough answer. The student clearly understands the three types of magnetic materials, providing accurate definitions and detailed explanations of their responses to external magnetic fields. The discussion of the role of atomic structure and electron configuration is also very well done.

**Solution 2: 7/8 marks**

*General Feedback:* This is a very good answer that demonstrates a solid understanding of the three types of magnetic materials. The descriptions of their responses to magnetic fields are accurate, and the mention of Lenz's law in the context of diamagnetism is a nice addition. The explanation could be slightly more detailed regarding the atomic structure and electron configuration, but overall it is a well-written response.

**Solution 3: 6/8 marks**

*General Feedback:* This is a good answer that covers the basics of the three types of magnetic materials and their responses to magnetic fields. However, the explanations are somewhat simplistic and could be more detailed. The student mentions atomic dipoles but does not explicitly discuss the role of electron spins in magnetic behavior. Additionally, the explanation of diamagnetism could be more precise. Overall, the answer is accurate but lacks the depth and detail of the other two solutions.

\bigskip    
\textbf{Response 2:}

**Solution 1: 8/8 marks**
Excellent, comprehensive explanation covering all key points, accurate examples, and clear distinction between the behaviors of different types of materials. Well-structured response.

**Solution 2: 7/8 marks**
Very good explanation with accurate details and examples. Includes the impact of thermal motion on paramagnetic materials, which is a good addition. Could be slightly more concise in some parts.

**Solution 3: 6/8 marks**
Good overall explanation, but lacks some depth in certain areas. For example, the concept of magnetic domains could be elaborated further in the ferromagnetic materials section. The explanation for diamagnetic materials is also slightly simplified.

\bigskip    
\textbf{Response 3:}

**Solution 1: 8/8 marks**

Excellent explanation of the three types of magnetic materials and their responses to a bar magnet. The response clearly articulates the underlying physical principles determining the behavior of each type of material.

**Solution 2: 7/8 marks**

Good explanation of the three types of magnetic materials. The response demonstrates a solid understanding of the underlying physics, but could be improved by elaborating slightly on the role of electron arrangement in determining magnetic behavior.

**Solution 3: 6/8 marks**

A decent explanation of the three types of magnetic materials. The response shows a basic understanding of the underlying principles, but lacks some depth in explaining the atomic and electronic origins of the magnetic behavior. Some points, such as the discussion of atomic dipoles, could be refined for clarity.

\bigskip    
\textbf{Response 4:}

**Solution 1: 8/8 marks**

Excellent explanation of the three types of magnetic materials and their response to an external magnetic field. The answer clearly defines each material type, provides examples, and explains the underlying atomic mechanisms that determine their magnetic behavior. The explanation is concise and accurate, making it easy to understand for students.

**Solution 2: 7/8 marks**

The solution provides a clear and accurate explanation of the three types of magnetic materials and their response to an external magnetic field. The definition of each type is concise, and the explanation of the underlying atomic mechanisms is clear. However, the explanation of diamagnetism could be slightly more elaborate, mentioning the role of Lenz's Law in inducing an opposing magnetic field.

**Solution 3: 6/8 marks**

The solution correctly identifies the three types of magnetic materials and provides a basic explanation of their response to an external magnetic field. The use of terms like "atomic dipoles" and "Lenz's Law" demonstrates a good understanding of the underlying concepts. However, the explanation lacks the depth and detail seen in the other two solutions. The explanation of paramagnetism and diamagnetism, in particular, could be more comprehensive.

\bigskip    
\textbf{Response 5:}

**Solution 1: 8/8 marks**

Excellent explanation. The student has covered all the essential aspects: the three types of magnetic materials, their responses to a magnetic field, and the underlying atomic reasons for the responses. The answer is well-structured and uses precise terminology.

**Solution 2: 7/8 marks**

Good explanation. The student has correctly described the three types of materials and their responses. The answer lacks some of the depth of explanation seen in solution 1, particularly when discussing the atomic and electronic reasons for the behaviors. 

**Solution 3: 6/8 marks**

Satisfactory explanation. While the student has correctly identified the three types of materials, the explanation of the underlying reasons for their responses is less precise than in the other two solutions. The term "atomic dipole" is used instead of the more accurate term "magnetic moment." The discussion of Lenz's Law in the context of diamagnetism could be elaborated further.

\subsection*{Question 2:}

\textbf{Response 1:}

**Solution 1:**

Mark: 5/5

Feedback: This solution is excellent, demonstrating a deep understanding of the concepts. The definitions are precise, the explanation of the wave equation is clear, and the derivation of the speed of light is well-presented. The use of mathematical symbols and equations is appropriate and enhances the clarity of the response.

**Solution 2:**

Mark: 4/5

Feedback: This solution is well-written and accurate, but it lacks the same level of detail and clarity as Solution 1. The definition of free space is correct, and the explanation of the wave equation is adequate. However, the derivation of the speed of light could be more explicit, and the use of mathematical symbols and equations could be improved.

**Solution 3:**

Mark: 5/5

Feedback: This solution is also excellent. It provides a comprehensive and accurate explanation of the concepts, with clear definitions and a thorough discussion of the wave equation. The derivation of the speed of light is well-explained, and the use of mathematical symbols and equations is appropriate. The response demonstrates a strong understanding of the physics involved.

\bigskip    
\textbf{Response 2:}

**Solution 1:**

Mark: 5/5

Feedback: This is an excellent and thorough solution to the problem. It defines free space correctly, clearly explains the wave equation, and derives the speed of light correctly. It demonstrates a deep understanding of classical electromagnetism. 

**Solution 2:**

Mark: 4/5 

Feedback: A very good and mostly correct answer. The definition of free space and interpretation of the wave equation could be slightly more specific, but the overall explanation is clear and concise. The speed of light derivation is correct.

**Solution 3:** 

Mark: 5/5

Feedback: Another excellent and thorough solution. This answer goes into detail about the physical interpretation of the wave equation and provides a comprehensive explanation of how the speed of light can be derived. The definition of free space is accurate and well-explained.

\bigskip    
\textbf{Response 3:}

**Solution 1: (5/5)**

This is a perfect answer, clearly and concisely defining free space, interpreting the wave equation, and deriving the speed of light. It is comprehensive and shows a strong understanding of electromagnetism.

**Solution 2: (4/5)**

This solution is well-explained and covers all the key points, but the definition of free space lacks the nuance of it being an idealized medium. Otherwise, the interpretation of the equation and derivation of the speed of light are excellent.

**Solution 3: (5/5)**

This is another excellent answer. The definitions are clear and accurate, the interpretation of the equation is spot on, and the derivation of the speed of light is well-explained. This solution demonstrates a thorough understanding of the subject matter.

\bigskip    
\textbf{Response 4:}

**Solution 1: 5/5**

Excellent solution. This solution provides a comprehensive and accurate explanation of free space, the wave equation, and the derivation of the speed of light. The physical interpretation of the wave equation is clear and concise, and the mathematical steps are well-explained. The inclusion of the values of permittivity and permeability and the calculation of the speed of light is a nice addition.

**Solution 2: 4/5**

Good solution. This solution accurately defines free space and the wave equation. The physical interpretation of the wave equation is well-explained, and the derivation of the speed of light is correct. However, the solution could be improved by including the values of permittivity and permeability and the calculation of the speed of light.

**Solution 3: 5/5**

Excellent solution. This solution provides a clear and accurate definition of free space and the wave equation. The physical interpretation of the wave equation is well-explained, and the derivation of the speed of light is correct. The solution also includes the values of permittivity and permeability and the calculation of the speed of light. The explanation of the wave equation and its connection to the speed of light is particularly well-written.

\bigskip    
\textbf{Response 5:}

**Solution 1: 5/5**

Excellent solution. The explanation is clear, comprehensive, and accurate, covering all aspects of the question with precise use of terminology and notation. The mathematical derivations are correct, and the final answer is clearly stated.

**Solution 2: 4/5**

Very good solution. The explanation is mostly accurate and well-organized, but it lacks a bit of the depth seen in solution 1, particularly in the discussion of the physical interpretation of the wave equation.

**Solution 3: 4/5**

Very good solution. The explanation is clear and concise, but there are minor errors in the interpretation of the physical meaning of the wave equation. The student correctly identifies the speed of light formula but does not explicitly state the value of c.

\subsection*{Question 3:}

\textbf{Response 1:}

**Solution 1: (12/14)**
The solution is very well written, and the calculations are correct. There are a few minor issues. The numerical value for the electric field could be simplified further to \( \frac{\lambda}{2\pi\epsilon_0 d} \), rather than expressing it as a decimal. The physical interpretation provided is accurate, but it could be expanded to mention the concept of a multipole expansion, where the first term represents the monopole contribution (the total charge) and higher-order terms (dipole, quadrupole, etc.) vanish as the distance becomes much larger than the size of the charge distribution. 

**Solution 2: (13/14)**

This solution is clear, concise, and correctly solves the problem. The integration steps are well explained, and the physical interpretation at the end is thorough and insightful. The only minor improvement could be to explicitly state the assumption of \( dl = dx \), which holds for the given geometry where the line segment lies along the x-axis.

**Solution 3: (10/14)**

This solution demonstrates a good understanding of the problem and correctly applies Coulomb's law to find the electric field. The integration steps are correct, and the limit for \( d \gg L \) is well-explained. However, the numerical factor in the final answer is not simplified, and the physical interpretation is a bit lacking compared to the other two solutions. The student could have mentioned that the field reduces to that of a point charge when the distance is much larger than the length of the line segment and how the factor of \( 0.564 \sqrt{\pi} \) relates to the geometry of the problem.

\bigskip    
\textbf{Response 2:}

**Solution 1: 13/14**

Excellent work overall. The solution is well-organized, mathematically accurate, and clearly explains each step. However, the numerical constant in the final answer could be expressed more concisely as  \( \frac{1}{\sqrt{2}} \) instead of the decimal approximation.

**Solution 2: 12/14**

Very good solution. The approach is correct, and the explanation is clear. There are some minor notation inconsistencies (e.g., using both \( dl \) and \( dx \) for the infinitesimal line element), and the final approximation could be further simplified by replacing \( 0.141047395886939 \) with \( \frac{1}{\sqrt{2}} \).

**Solution 3: 11/14**

Good solution with a clear understanding of the concepts. The use of Coulomb's law and the integration process are well-explained. However, there are some numerical errors in the final answer (the constant factor should be \( \frac{1}{\sqrt{2}} \), not \( 0.564 \sqrt{\pi} \)). Also, the solution could benefit from explicitly mentioning the symmetry argument used to eliminate the horizontal components of the electric field.

\bigskip    
\textbf{Response 3:}

**Solution 1: (13/14)**

Excellent solution with a clear and detailed explanation of the integration process. The final expression for the electric field is correct, as is the simplification for \( d \gg L \) and the physical interpretation of the result.

**Solution 2: (12/14)**

The solution is well-organized and easy to follow. The integration process is clear and the final expression for the electric field is correct. The physical interpretation of the result in the limit \( d \gg L \) is also well-explained. However, the solution should have included the substitution of the total charge of the line segment, \( Q = \lambda L \), in the final expression for the electric field in this limit.

**Solution 3: (11/14)**

This solution provides a correct approach to the problem and the integration process is generally correct. However, there are a few errors and omissions:

1. The value for \( k_e \) is approximate and the exact value should be used for calculations.
2. The final numerical constant in the expression for \( E_y \) is incorrect, likely due to using the approximate value for \( k_e \) or an error in integration.
3. The solution should have used a vector notation for the electric field to emphasize its direction. 

Overall, the solution demonstrates a good understanding of the concepts involved, but the errors in calculation and lack of vector notation detract from its completeness.

\bigskip    
\textbf{Response 4:}

**Solution 1: (12/14)**
The solution is very thorough and well-explained. There's a minor error in the numerical constant in the final expression for the electric field; it should be approximately \( \frac{1}{2\sqrt{2}} \) instead of \( 0.141047395886939 \). However, the method and overall understanding are correct. The physical interpretation is also accurate.

**Solution 2: (13/14)**
This solution is excellent. It explains the concepts clearly, sets up the integral correctly, and arrives at the correct final expression for the electric field. The step-by-step approach makes it easy to follow. The limit of \( d \gg L \) is also handled correctly, and the physical interpretation is spot on.

**Solution 3: (10/14)**
This solution is mostly correct, but it uses Coulomb's constant \( k_e \) instead of the vacuum permittivity \( \epsilon_0 \), which are related by \( k_e = \frac{1}{4\pi\epsilon_0} \). This leads to an incorrect numerical factor in the final expression. The integration and the limiting case are handled correctly. The physical interpretation is also accurate.

\bigskip    
\textbf{Response 5:}

**Solution 1:**

* **Mark:** 12/14
* **General Feedback:** This solution is very well written and detailed, with a clear explanation of each step. The use of Coulomb's law and the integration process are correct. The derivation of the final expression for the electric field is also accurate. The physical interpretation of the result in the limit of \( d \gg L \) is explained correctly.  A minor improvement would be to simplify the constant factor in the final expression to the more standard form \( \frac{1}{2\sqrt{2}} \).

**Solution 2:**

* **Mark:** 13/14
* **General Feedback:** This solution is excellent, with a clear and concise explanation of the physics behind the problem. The use of symmetry to simplify the calculation is well explained. The integration process is carried out correctly, and the final expression for the electric field is accurate. The physical interpretation of the result in the limit of \( d \gg L \) is also well explained. The approximation in the limit \( d \gg L \) is justified and the final expression is presented in a clear and intuitive form. 

**Solution 3:**

* **Mark:** 11/14
* **General Feedback:** This solution is good, with a clear and organized presentation. The use of Coulomb's law is correct, and the integration process is carried out correctly. The final expression for the electric field is accurate, although it is not in the most simplified form. The physical interpretation of the result in the limit of \( d \gg L \) is correct, but could be explained in more detail. This solution could be improved by simplifying the constant factor in the final expression to the more standard form \( \frac{1}{2\sqrt{2}} \) and by providing a more detailed explanation of the physical interpretation in the limit \( d \gg L \). 

\subsection*{Question 4:}

\textbf{Response 1:}

Solution 1:
Part a) - 5/5
Part b) - 5/5
Part c) - 1/1

Overall feedback: This solution is well-presented and demonstrates a thorough understanding of the physics concepts involved. The calculations are accurate and the explanations are clear. The student correctly uses Ampère's law and Faraday's law to find the flux and emf. They also correctly apply Lenz's law to determine the direction of the induced current.

Solution 2:
Part a) - 2/5 
Part b) - 3/5
Part c) - 1/1

Overall feedback: This solution is generally correct, but there are some errors in the calculation of the magnetic flux in part a). In particular, the integral should have been done with respect to \( r \), not \( x \), since \( r \) is the distance from the wire. Also, the limits of integration should have been \( d \) to \( d + L \), not 0 to \( L \). The student also made an error in part b) by differentiating the flux and finding it to be 0. However, it is unclear how this student made this mistake as they did not show their working and the answer they got is incorrect. Part c) is correct.

Solution 3:
Part a) - 5/5
Part b) - 5/5
Part c) - 1/1

Overall feedback: This is a well-written and accurate solution. The student has a good understanding of the relevant physics concepts. The calculations are correct and the explanations are clear. The student correctly uses Ampère's law and Faraday's law to find the flux and emf. They also correctly apply Lenz's law to determine the direction of the induced current.

\bigskip    
\textbf{Response 2:}

Solution 1:

**Part a) - 5/5**
The student has correctly calculated the magnetic flux through the loop. The integration is done correctly, and the final expression is in terms of the given parameters.

**Part b) - 5/5**
The student has correctly used Faraday's law to find the induced emf and has used Lenz's law to determine the direction of the induced current.

**Part c) - 1/1**
The student has correctly explained why no emf is induced when the loop moves downwards.

Overall Feedback: Excellent work! The student has shown a strong understanding of the concepts of magnetic flux, Faraday's law, and Lenz's law. The solution is well-organized, and the steps are clearly explained.

Solution 2:

**Part a) - 2/5**
The student has started the problem correctly but made a mistake when integrating to find the flux. The flux should depend on the distance \(d\).

**Part b) - 1/5**
The student has made a mistake by differentiating the incorrect flux equation. The emf should not be zero.

**Part c) - 1/1**
The student has correctly explained why no emf is induced when the loop moves downwards.

Overall Feedback: The student understands the basic concepts but made errors in the calculation of the flux and emf. The student should be careful when integrating and differentiating.

Solution 3:

**Part a) - 5/5**
The student has correctly calculated the magnetic flux through the loop. The integration is done correctly, and the final expression is in terms of the given parameters.

**Part b) - 4/5**
The student has correctly used Faraday's law to find the induced emf and has used Lenz's law to determine the direction of the induced current. The student could have simplified the final expression for the emf.

**Part c) - 1/1**
The student has correctly explained why no emf is induced when the loop moves downwards.

Overall Feedback: The solution is well-organized and easy to follow. The student has shown a good understanding of the concepts and has applied them correctly. The only minor issue is the simplification of the emf expression.

\bigskip    
\textbf{Response 3:}

Solution 1:

* Part (a): 5/5
* Part (b): 5/5
* Part (c): 1/1
* General feedback: An excellent solution with a clear explanation and correct calculations.

Solution 2:

* Part (a): 3/5, the student made a mistake in their integration.
* Part (b): 0/5, the student made an error stating that the emf is zero when in fact, the emf is non-zero.
* Part (c): 1/1
* General feedback: The student had a good approach but made some key mistakes in their calculations and reasoning in parts (a) and (b). 

Solution 3:

* Part (a): 5/5
* Part (b): 5/5
* Part (c): 1/1
* General feedback: An excellent and very detailed solution with a clear explanation and correct calculations. 

\bigskip    
\textbf{Response 4:}

Solution 1:
a) 5/5
b) 5/5
c) 1/1

Overall, this solution is excellent and demonstrates a strong understanding of the problem. The student correctly applied Ampère's and Faraday's laws to solve the problem and provided a clear explanation of the direction of the induced current using Lenz's law.

Solution 2:
a) 2/5
b) 1/5
c) 1/1

The student correctly applied Ampère's law but made an error in computing the integral of \( d\Phi \). The student also made an error in differentiating \( \Phi \) with respect to \( d \) when trying to find the emf, resulting in an incorrect expression for \( \mathcal{E} \) and a misunderstanding of how the emf depends on \( d \). However, the student correctly stated what would happen in part c).

Solution 3:
a) 4/5
b) 4/5
c) 1/1

The student did an excellent job in solving the problem. They applied Ampère's and Faraday's laws correctly, and provided a clear explanation for the direction of the induced current. The student's solution is well-organized and easy to follow. However, the student did not show all working in part a), missing a constant of integration.

\bigskip    
\textbf{Response 5:}

Solution 1:
a) 5/5
b) 4/5 - The direction of the induced current is correct, but the sign of the emf is incorrect.
c) 1/1

Overall feedback: A well-structured solution with a clear understanding of the underlying concepts. However, there's a minor sign error in part b).

Solution 2:
a) 0/5 - The flux calculation is incorrect as it doesn't account for the varying magnetic field across the loop.
b) 0/5 - The differentiation is incorrect, and the explanation for zero emf is incorrect. The flux does change as the loop moves.
c) 1/1

Overall feedback: This solution demonstrates some misconceptions about the calculation of magnetic flux and induced emf. The flux depends on the distance d, and the emf is not zero when the loop moves to the right.

Solution 3:
a) 5/5
b) 5/5
c) 1/1

Overall feedback: An excellent solution with a clear and accurate presentation of the concepts and calculations involved.

\subsection*{Question 5:}

\textbf{Response 1:}

**Solution 1:**

*   **Part a):** 0/8 marks. The method is mostly correct, but there are several errors in the calculations. 
*   **Part b):** 3/3 marks. The method and answer are correct.
*   **Overall Feedback:** The solution shows a good understanding of circuit analysis. There are mistakes in the equivalent resistance calculation. The final answer is also incorrect due to the mistakes in the first part.

**Solution 2:**

*   **Part a):** 5/8 marks. The method is correct, but there is a minor error in finding the total resistance.
*   **Part b):** 3/3 marks. The method and answer are correct.
*   **Overall Feedback:** The solution demonstrates a good understanding of Ohm's law and circuit analysis. The only error is in calculating the parallel resistance, which leads to a slightly incorrect final answer for part a).

**Solution 3:**

*   **Part a):** 8/8 marks. The method and answer are correct.
*   **Part b):** 3/3 marks. The method and answer are correct.
*   **Overall Feedback:** This is an excellent solution. The student has a clear understanding of the problem and has applied the relevant concepts correctly to find the correct answers for both parts.

\bigskip    
\textbf{Response 2:}

**Solution 1: (8/8 for part a, 3/3 for part b)**

This solution demonstrates a clear understanding of the problem and its solution. The student has outlined the steps to be taken clearly and logically, and the calculations are accurate. The student has also provided clear explanations for each step, making it easy to follow their reasoning. The final answer is also correct and clearly stated.

**Solution 2: (6/8 for part a, 3/3 for part b)**

This solution is partially correct. The student has correctly identified the parallel and series connections and applied Ohm's law to find the total current. However, there are some errors in calculating the equivalent resistance of the circuit. The student did not account for the voltage drop across the internal resistance of the 2V battery when calculating the equivalent resistance of the series combination. The student correctly calculated the energy delivered by the 12V battery.

**Solution 3: (8/8 for part a, 3/3 for part b)**

This solution is correct and well-explained. The student has clearly outlined the steps involved in solving the problem, and the calculations are accurate. The student has also shown a good understanding of the concepts of equivalent resistance, Ohm's law, and energy. The final answer is also correct and clearly stated.

\bigskip    
\textbf{Response 3:}

**Solution 1: (8/8 for part a, 3/3 for part b)**

This solution demonstrates a thorough understanding of circuit analysis. The student correctly identifies the circuit elements, analyzes the parallel and series combinations, and applies Ohm's law to calculate the total current. They then accurately use the power and energy formulas to determine the energy delivered by the battery. The steps are clear and well-explained, and the final answer is correct.

**Solution 2: (8/8 for part a, 3/3 for part b)**

This solution also demonstrates a strong understanding of the problem. The student correctly calculates the total resistance and then applies Ohm's law to find the total current. They accurately use the power and energy formulas to calculate the energy delivered by the battery. While the approach is slightly different from Solution 1, it is equally valid and leads to the correct answer.

**Solution 3: (8/8 for part a, 3/3 for part b)**

This solution demonstrates a comprehensive understanding of the problem. The student uses a combination of equivalent resistance calculations and current division to correctly determine the total current. They then accurately apply the power and energy formulas to calculate the energy delivered by the battery. The steps are clear and logical, and the final answer is correct.

**General Feedback:**

All three solutions demonstrate a good understanding of circuit analysis, Ohm's law, power, and energy calculations. The students have approached the problem in slightly different ways, but all have arrived at the correct answers. The explanations are clear and easy to follow, making it easy to see how the students arrived at their solutions. Overall, these are well-written and accurate solutions to the given problem.

\bigskip    
\textbf{Response 4:}

**Solution 1:**
* **Part (a):** 0/8 marks. The solution is correct up until the point of stating "Once we calculate the total equivalent resistance, we can calculate the total current using the 12 V battery and Ohm's law." However, the calculation of the total equivalent resistance is not actually done. This leaves part (a) incomplete, as the current seen by the ammeter is not found.
* **Part (b):** 0/3 marks. Since the current from part (a) was never found, the answer for part (b) is incorrect, as the student uses an incorrect value for the current.
* **General Feedback:** The solution is very clear, detailed, and well-structured. The student demonstrates a good understanding of the relevant physics concepts. However, an error was made in the execution of the solution, leaving it incomplete.

**Solution 2:**
* **Part (a):** 8/8 marks. The total resistance and current are found correctly. The solution is also very clear and easy to follow. 
* **Part (b):** 0/3 marks. The power is calculated correctly, but the energy is incorrect, as the power is multiplied by 4, instead of the time of 4 seconds.
* **General Feedback:** The solution is well-presented and the student demonstrates a good understanding of how to solve the problem. However, the student made an error in the final calculation of part (b).

**Solution 3:**
* **Part (a):** 8/8 marks. The current seen by the ammeter is found correctly.
* **Part (b):** 3/3 marks. The energy is calculated correctly.
* **General Feedback:** The solution is excellent. It is clear, well-explained, and the answer is correct.

I hope this feedback is helpful. Please let me know if you have any other questions.

\bigskip    
\textbf{Response 5:}

**Solution 1: Grade: 8/11**

*   **Part a): 8/8**
*   **Part b): 0/3**

This solution demonstrates a clear understanding of circuit analysis and the relevant concepts. The steps are well-outlined, and the calculations are accurate for part a). However, the solution for part b) is incomplete. The student correctly identifies the formulas to use but doesn't carry out the final calculations to determine the energy delivered.

**Solution 2: Grade: 9/11**

*   **Part a): 8/8**
*   **Part b): 1/3**

The student demonstrates a good grasp of circuit analysis and correctly applies Ohm's Law to find the current. The calculations in part a) are accurate. However, there's a conceptual error in part b). The power calculation is correct, but the voltage used in the power formula is the total voltage of the circuit, not the voltage of the specific battery in question. This leads to an incorrect calculation for the energy delivered.

**Solution 3: Grade: 10/11**

*   **Part a): 8/8**
*   **Part b): 2/3**

This is the most comprehensive solution. The student explains their reasoning clearly and applies the relevant principles accurately in part a). They correctly calculate the equivalent resistance of the circuit and use Ohm's Law to find the current. There's a minor error in part b) where the student mislabels the units of power as joules (J) instead of watts (W), but the numerical value is correct, and the final energy calculation is accurate.

\subsection*{Question 6:}

\textbf{Response 1:}

Solution 1:

**Part a) - 3/3**

The student correctly identifies the relevant formulas and concepts and executes the calculation without errors.

**Part b) - 2/2**

The student demonstrates a good understanding of Newton's laws and how to apply them to this problem. The calculations are also correct.

**Overall feedback:** This is a well-written solution with a clear explanation of each step.

Solution 2:

**Part a) - 3/3**

The student demonstrates a good understanding of the physics of infinite planes of charge and how to apply the formula \( E = \sigma/\epsilon_0 \) to determine the total electric field.

**Part b) - 2/2**

The student correctly uses Coulomb's Law and Newton's 2nd Law to determine the acceleration, showing good understanding of these concepts.

**Overall feedback:** The solution is well-organized and the calculations are correct, but it could benefit from a more concise explanation in part a).

Solution 3:

**Part a) - 3/3**

The student correctly identifies the relevant equation and applies it accurately to determine the electric field.

**Part b) - 2/2**

The student demonstrates a good understanding of Coulomb's law and Newton's second law, applying them correctly to determine the force and acceleration of the object.

**Overall feedback:** The solution is clear and concise, demonstrating a good understanding of the problem and its solution.

\bigskip    
\textbf{Response 2:}

Solution 1:

**Part a) - 3/3**

The student clearly demonstrates their understanding of the physics involved, correctly using the relationship between electric field and potential difference to solve for the electric field. Their final answer is accurate.

**Part b) - 2/2**

The student accurately uses Newton's second law and Coulomb's law to solve for the acceleration of the object. Their final answer is correct.

**General feedback:** The solution is well-organized and easy to follow. The student clearly explains each step of their reasoning.

Solution 2:

**Part a) - 3/3**

The student demonstrates a good understanding of the concept of electric field due to infinite planes of charge. They correctly derive the total electric field and use the relationship between electric field and potential difference to solve for the electric field. Their final answer is accurate.

**Part b) - 2/2**

The student accurately uses Newton's second law and Coulomb's law to solve for the acceleration of the object. Their final answer is correct.

**General feedback:** The solution is well-explained and the steps are clearly presented. The student could potentially combine some of the steps to make the solution more concise.

Solution 3:

**Part a) - 3/3**

The student demonstrates a clear understanding of the relationship between electric field, potential difference, and distance. They correctly apply this relationship to solve for the electric field and arrive at the correct answer.

**Part b) - 2/2**

The student correctly applies Coulomb's law and Newton's second law to determine the force and acceleration of the object. Their calculations and final answer are accurate.

**General feedback:** The solution is well-structured and concise. The student could have included the values of the constants used in the calculations for completeness.

\bigskip    
\textbf{Response 3:}

Solution 1:

**Part a) - 3/3**

The student correctly identifies the relevant formulas and values and shows the calculations to determine the electric field. The final answer is also correct.

**Part b) - 2/2**

The student demonstrates a good understanding of how to apply Newton's second law to find the acceleration due to the electric field and provides the correct answer with the units.

**Overall feedback:**

The solution is well-organized, and the steps are clearly explained. The student demonstrates a strong grasp of the underlying concepts and correctly applies the relevant equations.

Solution 2:

**Part a) - 3/3**

This solution correctly identifies the formula for the electric field due to a charged plane and also correctly shows the total electric field will be double the value due to a single plate. This solution also arrives at the correct final answer. 

**Part b) - 2/2**

This solution correctly identifies Newton's second law to find the acceleration and provides the correct final answer with units.

**Overall feedback:**

This is a good solution with clearly explained steps. This solution also shows multiple ways to find the electric field. The student demonstrates they understand the physics concepts well.

Solution 3:

**Part a) - 3/3**

This solution correctly identifies the relationship between electric field, potential difference, and separation. This solution correctly applies this equation to find the electric field.

**Part b) - 2/2**

The student demonstrates a clear understanding of how to use Coulomb's law and Newton's second law to find the force and acceleration on the charged object. The final answer is also correct.

**Overall feedback:**

This is a well-structured solution. The student has shown a good understanding of the underlying principles and has presented the solution in a clear and concise manner. The calculations are accurate and well-explained.

\bigskip    
\textbf{Response 4:}

Solution 1: 
*   **Part a):** 3/3 marks
*   **Part b):** 2/2 marks
*   **General feedback:** This solution is correct and well-explained.

Solution 2: 
*   **Part a):** 3/3 marks
*   **Part b):** 2/2 marks
*   **General feedback:** This is a correct solution, however, part a) is overly complicated and does not need to involve the surface charge density. 

Solution 3: 
*   **Part a):** 3/3 marks
*   **Part b):** 2/2 marks
*   **General feedback:** This solution is correct and well-explained. However, it is not necessary to use Gauss's law to find the electric field.

\bigskip    
\textbf{Response 5:}

Solution 1: 
* **Part (a): 2/3** marks. This is mostly correct, but the electric field formula is not derived. Some explanation of the formula would be good to show understanding.
* **Part (b): 2/2** marks. This is correct and well explained.
* **Overall feedback:** This is a good solution. The explanation is clear and concise, and it is well written. However, a derivation of the electric field between two parallel plates would help to show a deeper understanding of the physics involved.

Solution 2:
* **Part (a): 3/3** marks. This is correct and well explained, with all necessary steps taken.
* **Part (b): 2/2** marks. This is correct and well explained.
* **Overall feedback:** This solution is very good. The explanation is clear, concise, and easy to follow. All steps are explained, and there is nothing that is left unexplained.

Solution 3:
* **Part (a): 2/3** marks. Although it mentions Gauss's law, it lacks explanation for not using Gauss's law and instead just jumps straight to using the formula, which is not ideal.
* **Part (b): 2/2** marks. This is correct and well explained.
* **Overall feedback:** The solution is mostly correct and the explanation is clear. However, it lacks some justification for why they chose to use the given formula for the electric field rather than using Gauss's law, and it would be better to state what equation they are using for the force before jumping to the calculation.

\subsection*{Question 7:}

\textbf{Response 1:}

Solution 1:

*   Part (a): 1/1. Correct answer.
*   Part (b): 3/3. Correct expression and explanation.
*   Part (c): 2/2. Correct description of the phase relationship in both limits.

Overall feedback: Excellent solution with clear explanations and correct calculations.

Solution 2:

*   Part (a): 1/1. Correct answer.
*   Part (b): 2/3. The expression for the voltage drop is correct but the explanation of the phase shift \( \phi \) could be more detailed.
*   Part (c): 2/2. Correct description of the phase relationship in both limits.

Overall feedback: Good solution, but could benefit from a more detailed explanation of the phase shift in part (b).

Solution 3:

*   Part (a): 1/1. Correct answer.
*   Part (b): 3/3. Correct expression and explanation.
*   Part (c): 2/2. Correct description of the phase relationship in both limits.

Overall feedback: Excellent solution with clear explanations and correct calculations. Some minor formatting issues, but these do not detract from the overall quality of the solution.

\bigskip    
\textbf{Response 2:}

Solution 1:

**Part a)** 1/1 mark. Correct answer with units.

**Part b)** 3/3 marks. Correct answer with clear working.

**Part c)** 2/2 marks. Correct explanation of phase relationship.

Overall feedback: A well-structured solution with correct answers and clear explanations. Demonstrates a good understanding of the physics involved.

Solution 2:

**Part a)** 1/1 mark. Correct answer with units.

**Part b)** 3/3 marks. Correct answer, although the impedance term should have been expanded.

**Part c)** 2/2 marks. Correct explanation of phase relationship.

Overall feedback: A good solution with minor improvements needed in part b) for clarity.

Solution 3:

**Part a)** 1/1 mark. Correct answer with units.

**Part b)** 2/3 marks. Correct method but there is a sign error in the phase of the current.

**Part c)** 2/2 marks. Correct explanation of phase relationship.

Overall feedback: A generally well-written solution, but attention should be paid to the sign of the phase in part b).

\bigskip    
\textbf{Response 3:}

Solution 1:

*   **Part (a): 1/1**

    The student correctly calculated the resonant frequency.

*   **Part (b): 3/3**

    The expression for the voltage drop across the inductor is correct.

*   **Part (c): 2/2**

    The phase relationship between the source e.m.f. and voltage across the inductor is accurately described for both low and high-frequency limits.

Overall, this is a well-written and comprehensive solution that demonstrates a clear understanding of the underlying concepts.

Solution 2:

*   **Part (a): 1/1**

    The student correctly calculated the resonant frequency.

*   **Part (b): 2/3**

    The expression for the voltage drop is mostly correct, but the student did not substitute the expression for impedance \(Z\) in terms of \(R\), \(X_L\), and \(X_C\), which was asked in the question. 

*   **Part (c): 2/2**

    The phase relationship is correctly described.

Overall, this solution is good, but it would be improved by substituting the expression for \(Z\) in part (b).

Solution 3:

*   **Part (a): 1/1**

    The student correctly calculated the resonant frequency.

*   **Part (b): 2/3**

    The expression for the voltage drop is mostly correct, but the student used \( \cos(\omega t - \phi) \) instead of \( \cos(\omega t + \phi) \).

*   **Part (c): 2/2**

    The phase relationship is correctly described.

Overall, this solution is good, but it would be improved by using the correct expression for the current, \( I = \frac{\mathcal{E}_{0}}{Z} \cos(\omega t + \phi) \), in part (b).

\bigskip    
\textbf{Response 4:}

Solution 1:

**Part a) (1/1):** Correct calculation of resonant frequency.

**Part b) (3/3):** Correct derivation of the expression for the voltage drop.

**Part c) (2/2):** Accurate description of the phase relationship in both limits.

**Overall feedback:** This solution is excellent, showing a strong understanding of the problem and its underlying concepts. The student provides clear explanations and correct derivations.

Solution 2:

**Part a) (1/1):** Correct calculation of resonant frequency.

**Part b) (3/3):** Correct derivation of the expression for the voltage drop.

**Part c) (2/2):** Correct description of the phase relationship, including the behavior at the limits of low and high frequencies.

**Overall feedback:** The solution is well-presented and demonstrates a good grasp of the topic. The explanations are concise yet informative.

Solution 3:

**Part a) (1/1):** Correct calculation of resonant frequency.

**Part b) (3/3):** Correct derivation of the expression for the voltage drop, although the phase shift is written as negative.

**Part c) (2/2):** Correct description of the phase relationship, including the behavior at the limits of low and high frequencies.

**Overall feedback:** The solution is mostly accurate, showing a good understanding of the concepts. The student has made a minor error in writing the phase shift as negative in part b. However, the explanation and reasoning are correct.

\bigskip    
\textbf{Response 5:}

Solution 1:

*   Part a) 1/1. Correct answer.
*   Part b) 3/3. Correct expression and explanation.
*   Part c) 2/2. Correct explanation of phase relationships.

Overall feedback: Excellent work! Your solution is comprehensive, well-explained, and demonstrates a thorough understanding of RLC circuits and their behaviour at different frequencies. 

Solution 2:

*   Part a) 1/1. Correct answer.
*   Part b) 2/3. The expression for the voltage drop across the inductor is correct, but you didn't substitute for $Z$ and $X_L$.
*   Part c) 2/2. Correct explanation of phase relationships.

Overall feedback: Good work! Your solution is mostly correct and shows a clear grasp of the concepts. However, you could improve by substituting for $Z$ and $X_L$ in part b) to get the final expression.

Solution 3:

*   Part a) 1/1. Correct answer.
*   Part b) 2/3. The expression for the voltage drop across the inductor is correct, but you used a negative sign in the phase of the current.
*   Part c) 2/2. Correct explanation of phase relationships.

Overall feedback: Good work! Your solution is generally correct and well-presented. However, be careful with the sign of the phase in the current expression for part b). The correct phase shift for an inductor is positive.

\subsection*{Question 8:}

\textbf{Response 1:}

**Solution 1: (5/6 marks)**

This is a well-structured solution with a clear understanding of boundary conditions and the relationship between \( \underline{B} \) and \( \underline{H} \). The trigonometric relationships are correctly applied. However, there's a minor calculation error in the final value of \( \alpha_{1} \).

**Solution 2: (3/6 marks)**

This solution demonstrates a good grasp of the problem's setup and the relevant boundary conditions. The explanation of the tangential and normal components is clear. However, there's a lack of connection between the permeabilities and the calculation of \( \alpha_{1} \). The solution also lacks a definitive numerical answer.

**Solution 3: (2/6 marks)**

While this solution starts correctly by stating the boundary conditions, there's a misapplication of them. The relationship between the tangential components of \( \underline{H} \) and \( \underline{B} \) is not used correctly, and the final calculation of \( \alpha_{1} \) is incorrect.

\bigskip    
\textbf{Response 2:}

**Solution 1: (6/6)**
This is a well-structured and correct solution. The student demonstrates a clear understanding of the boundary conditions for magnetic fields and applies them correctly to solve the problem. The calculations are accurate, and the final answer is presented clearly.
* General feedback: It would be better to state that we are only considering the magnitudes of the magnetic fields, and the directions are implicit in the use of the angle \( \alpha \). It is also important to state that the \(z\)-components of the magnetic field density are equal as well as continuous.


**Solution 2: (2/6)**
The student demonstrates a good understanding of the relevant boundary conditions and correctly identifies the tangential and normal components of the magnetic field. However, the solution does not progress beyond stating the boundary conditions and describing the magnetic fields in each material. 
* General feedback: The solution seems to get lost and the student should realize that the boundary condition for the normal component is not required to answer the question. There is no attempt to apply the boundary conditions to solve for the unknown angle. The solution ends prematurely without reaching a final answer. To improve, the student should focus on applying the boundary conditions to the specific problem and deriving an expression for the unknown angle.

**Solution 3: (4/6)**
The student demonstrates a good understanding of the boundary conditions for magnetic fields and correctly identifies the relevant equations. The student also correctly applies trigonometry to relate the tangential and normal components of the magnetic field. However, there is a calculation error in the final step, leading to an incorrect answer.
* General feedback: The solution is mostly correct, but the final calculation error needs to be corrected. Additionally, it is important to state that we are only considering the magnitudes of the magnetic fields, and the directions are implicit in the use of the angle \( \alpha \).

\bigskip    
\textbf{Response 3:}

**Solution 1:**
5/6 marks.
Good work. The boundary conditions are used appropriately to relate the components of the magnetic fields in both materials. The explanation is clear and concise, leading to the correct calculation of the angle  \(\alpha_{1}\). A small improvement can be made by including the reasoning behind why $B_{1z} = B_{2z}$, which would make the solution more complete.

**Solution 2:**
2/6 marks.
This solution demonstrates a basic understanding of the problem and the correct boundary conditions. However, it fails to use these conditions effectively to establish the necessary relationships between the components of the magnetic fields, and does not reach the correct answer. While recognizing that the direct calculation of $\alpha_{1}$ is impossible without more information, this solution would have been improved by explaining how the boundary conditions could be utilized if further information were given. 

**Solution 3:**
4/6 marks.
While the solution starts well by stating the relevant boundary conditions and using them to derive a correct relationship between the fields and the permeabilities, there's an error in the final calculation of \(\alpha_{1}\). The equation \( \sin(\alpha_{1}) = \sin(45^\circ) \times \frac{1.493}{3.012} \) should be \( \tan(\alpha_{1}) = \tan(45^\circ) \times \frac{1.493}{3.012} \). If this equation is used, the correct answer is obtained.

\bigskip    
\textbf{Response 4:}

**Solution 1: 5/6**

This solution demonstrates a good understanding of the boundary conditions for magnetic fields and correctly identifies the relevant equations. The student also uses the given information effectively to derive the correct relationship between the angles and the relative permeabilities. However, there are a few errors:
- The code seems to contain a minor error in the calculation of tan(alpha1) and should be corrected as follows:
```
tan_alpha1 = tan_alpha2 * (mu_r2 / mu_r1)
```
- The final answer is incorrect. It is approximately \(26.37^\circ\) rather than \(20.52^\circ\).

**General Feedback:**

- Overall, a well-structured and logical solution.
- The student could improve by double-checking the calculations and being more careful with the signs in the equations.

**Solution 2: 2/6**

This solution correctly identifies the relevant boundary conditions and equations. However, the student struggles to apply these concepts to the specific problem and makes several errors in reasoning. The student does not reach a correct final answer.

**General Feedback:**

- The student should review the concepts of tangential and normal components of magnetic fields and their behavior at boundaries.
- The student should focus on understanding how the relative permeabilities of the materials affect the relationship between the magnetic fields in each region.
- The student should also practice applying these concepts to solve similar problems.

**Solution 3: 6/6**

This solution is well-structured, and the student demonstrates a clear understanding of the problem and the relevant physics principles. The student correctly identifies the boundary conditions for magnetic fields and applies them appropriately to solve the problem. The student's reasoning is clear and concise, and the final answer is correct.

**General Feedback:**

- This is an excellent solution, and the student demonstrates a strong grasp of the material.
- There are no significant errors or areas for improvement.

\bigskip    
\textbf{Response 5:}

**Solution 1: Grade 6/6**

Excellent work! This solution is correct and demonstrates a thorough understanding of boundary conditions in electromagnetism. The steps are clearly explained, and the calculations are accurate. 

**Solution 2: Grade 3/6**

This solution starts well, demonstrating a good understanding of the boundary conditions for magnetic fields. The student has identified the correct relationships between the components of the magnetic fields and the angles. However, they seem to get stuck at the end, unable to deduce the final equation to find α1. The reasoning for why α1 cannot be calculated directly without more information is partially correct but could be better explained.

**Solution 3: Grade 5/6**

This solution is very good and almost completely correct. The student understands the boundary conditions, applies them correctly, and derives the necessary equations. The approach of using ratios to avoid needing the magnitudes of B1 and B2 is clever. The only error is in the final numerical calculation where 20.52° is obtained instead of 26.37°. This might be due to a minor computational mistake.

\subsection*{Question 9:}

\textbf{Response 1:}

Solution 1:
Part a): 6/7. There is an error when finding the time derivative of H.
Part b): 6/6
Part c): 2/2

General feedback: The solution shows a solid understanding of the physics concepts involved in the problem. The solution is well-structured and clearly presented, making it easy to follow. The calculations are correct, with only a minor error in finding the time derivative of H. This is a good solution that demonstrates a high level of proficiency.

Solution 2:
Part a): 7/7
Part b): 6/6
Part c): 1/2. This solution neglects to square cos(α).

General feedback: The solution demonstrates a strong grasp of the key concepts and equations relevant to the problem. Each step is logically presented, and the mathematical calculations are accurate. However, a minor mistake occurs when calculating the radiation pressure for a non-normal angle of incidence. Overall, the solution exhibits a good understanding of the material and a well-organized approach.

Solution 3:
Part a): 7/7
Part b): 6/6
Part c): 2/2

General feedback: The solution demonstrates a thorough understanding of electromagnetic waves and radiation pressure. The student systematically applies Faraday's law, calculates the Poynting vector, and derives the radiation pressure formula. The explanations are clear and concise, and the mathematical steps are well-organized. This is an excellent solution that reflects a high level of mastery of the subject matter.

\bigskip    
\textbf{Response 2:}

Solution 1:
* **Part a)** 7/7. A clear and well-explained derivation of the magnetic intensity magnitude using Faraday's Law. All steps are logically presented.
* **Part b)** 6/6. The Poynting flux calculation is accurate, and the steps are easy to follow. The relationship between E and H is well-utilized.
* **Part c)** 2/2. The explanation for the radiation pressure considering the angle of incidence is concise and correct. The reduction in momentum due to the angle is clearly addressed.
* **Overall feedback:** An excellent solution with clear explanations and accurate calculations. The student demonstrates a strong understanding of electromagnetic wave properties and their interactions.

Solution 2:
* **Part a)** 7/7. The derivation of the magnetic intensity is correct. The use of complex notation is appropriate and the steps are well-explained.
* **Part b)** 6/6. The Poynting flux calculation is correct. The student clearly explains the averaging process over one period and the role of the propagation direction.
* **Part c)** 2/2. The radiation pressure explanation is accurate. The relationship between intensity and Poynting vector is well-utilized, and the effect of the angle of incidence is correctly addressed.
* **Overall feedback:** A very good solution with accurate calculations and clear explanations. The student demonstrates a strong grasp of electromagnetic concepts and their applications.

Solution 3:
* **Part a)** 5/7. The derivation of the magnetic intensity has a minor error in the calculation of the curl of E, which affected the final result. While the approach is correct, the execution needs improvement.
* **Part b)** 6/6. The Poynting flux calculation is correct and well-explained. The student clearly understands the concept of energy flux and its calculation.
* **Part c)** 2/2. The explanation of radiation pressure is accurate. The student correctly considers the angle of incidence and its effect on the pressure.
* **Overall feedback:** A good solution with a minor error in part a. The student demonstrates a good understanding of the concepts but needs to be more careful in calculations. The explanations for parts b and c are clear and well-presented.

\bigskip    
\textbf{Response 3:}

Solution 1:
Part a): 6/7 marks
The solution is well written and easy to follow, however the reasoning behind setting \(\frac{\partial \underline{H}}{\partial t} = -i\omega \underline{H}\) is not given and the derivation of \(\omega = ck = \frac{ck}{\sqrt{\mu_{0}\epsilon_{0}}}\) is not provided. 
Part b): 6/6 marks 
The solution is very good and provides the correct answer with a detailed derivation.
Part c): 2/2 marks
The solution is perfect, providing the correct answer with a detailed solution. 
General feedback: This is a very good solution with some minor mistakes and omissions in part a). 

Solution 2:
Part a): 5/7 marks
This solution uses a very concise and efficient method to derive the relationship between the electric and magnetic field. However, it is not made clear why the curl of the electric field can be calculated as \(i\underline{k} \times \underline{E}_{0} \exp(i(\underline{k} \cdot \underline{r} - \omega t))\). 
Part b): 6/6 marks
The solution is very good and provides the correct answer with a detailed derivation.
Part c): 2/2 marks
The solution is perfect, providing the correct answer with a detailed solution. 
General feedback: This is a very good solution with some minor mistakes and omissions in part a). 

Solution 3:
Part a): 7/7 marks
The solution is very good and provides the correct answer with a detailed derivation.
Part b): 6/6 marks
The solution is very good and provides the correct answer with a detailed derivation.
Part c): 1/2 marks
The solution has a mistake at the end of the solution.  The radiation pressure \( P\) considering the angle of incidence \(\alpha\) becomes:
\[
P(\alpha) = \frac{2I\cos(\alpha)}{c}
\]
The factor of \( \cos(\alpha) \) in the numerator is not squared. 
General feedback: This is a very good solution with one minor mistake in part c). 

\bigskip    
\textbf{Response 4:}

Solution 1:

**Part a): 7/7 marks**

This solution correctly applies Faraday's law and demonstrates the relationship between the electric and magnetic fields. The steps are clear and well-explained.

**Part b): 6/6 marks**

The Poynting vector is calculated correctly, and the averaging over one period is done appropriately. The final result is derived with no errors.

**Part c): 2/2 marks**

The student has shown a good understanding of how radiation pressure works and how the angle of incidence affects it. The derivation is clear and concise.

**General Feedback:** This is an excellent solution that demonstrates a strong understanding of electromagnetism and wave propagation. The student has applied the relevant equations correctly and provided clear explanations for each step.

Solution 2:

**Part a): 7/7 marks**

This solution demonstrates a clear understanding of Faraday's Law and its application to electromagnetic waves. The derivation of the relationship between the electric and magnetic fields is correct and well-explained.

**Part b): 6/6 marks**

The Poynting vector is calculated accurately, and the averaging over one period is done correctly. The final result for the average Poynting flux is derived clearly.

**Part c): 2/2 marks**

The student has shown a good understanding of radiation pressure and how it is affected by the angle of incidence. The explanation of the cosine squared term is accurate and concise.

**General Feedback:** This solution is well-written and demonstrates a strong grasp of electromagnetic wave theory. The student has applied the appropriate equations and provided clear explanations for each step. The solution is complete and accurate.

Solution 3:

**Part a): 6/7 marks**

This solution demonstrates a good understanding of Faraday's Law and the relationship between the electric and magnetic fields. However, there is a small mistake in the final equation for H, where it should be \(H = \sqrt{\frac{\epsilon_{0}}{\mu_{0}}} E\).

**Part b): 6/6 marks**

The Poynting vector is calculated correctly, and the averaging over one period is done appropriately. The final result for the average Poynting flux is correct.

**Part c): 1/2 marks**

The student has a good understanding of radiation pressure but made a small error in the equation for P(a), where it should be \(P(\alpha) = \frac{2I\cos^{2}(\alpha)}{c}\).

**General Feedback:** This is a good solution that demonstrates a solid understanding of electromagnetic waves. The student has applied the relevant concepts and equations well, but there are minor errors in the final equations for part a and part c. With a bit more careful calculation, this could be an excellent solution.

\bigskip    
\textbf{Response 5:}

Solution 1:

**Part a):** 7/7 marks. This solution is comprehensive and correct. The student applies Faraday’s law, uses the correct relationship between ω, k and c, and successfully derives the magnitude of the magnetic intensity. 

**Part b):** 6/6 marks. The student correctly identifies the formula for the Poynting vector and uses the result from part a) to derive the time-averaged Poynting vector over one period. 

**Part c):** 2/2 marks. The student provides a clear and concise explanation for how the radiation pressure is modified when the angle of incidence is not normal.

**General feedback:** Overall, this is an excellent solution. The student demonstrates a strong understanding of electromagnetism and vector calculus. The explanations are clear and well-organized.

Solution 2:

**Part a):** 6/7 marks. The solution is mostly correct but contains a minor error. The student uses an incorrect formula to relate the magnetic field B to its time derivative. Instead of \(\frac{\partial \underline{B}}{\partial t} = -i\omega \underline{B}\), the correct formula is \(\frac{\partial \underline{B}}{\partial t} = -i\omega\mu_0 \underline{H}\). 

**Part b):** 6/6 marks. The student correctly uses the Poynting vector formula and the result from part a) to derive the time-averaged Poynting flux.

**Part c):** 2/2 marks. The student provides a clear and concise explanation for how the radiation pressure is modified when the angle of incidence is not normal.

**General feedback:** Overall, this solution is good. The student demonstrates a good understanding of electromagnetism. However, the minor error in part a) should be corrected.

Solution 3:

**Part a):** 7/7 marks. This solution is comprehensive and correct. The student correctly applies Faraday's Law, performs the necessary calculations, and arrives at the correct magnitude of the magnetic intensity. 

**Part b):** 5/6 marks. The student's explanation is mostly correct, but they miss a step in the derivation. They should have explicitly stated that since E and H are perpendicular and in phase in a plane wave, the magnitude of the Poynting vector becomes the product of the magnitudes of E and H.

**Part c):** 2/2 marks. The student provides a clear and concise explanation of how the radiation pressure is modified when the angle of incidence is not normal. 

**General feedback:** Overall, this solution is very good. The student demonstrates a clear understanding of the underlying physics concepts. However, the explanation for the Poynting vector in part b) could be more detailed.

\subsection*{Question 10:}

\textbf{Response 1:}

**Solution 1: Grade and Feedback**

* Part a): 0/4 marks
* Part b): 0/2 marks
* Part c): 0/1 mark

**Feedback:** This solution, while demonstrating understanding of the problem and its concepts, struggles with the algebraic manipulations. The invariance in part a) was not shown, and the phase transformation in part b) is incomplete. The discussion on the Doppler effect in part c) is conceptually sound, but the precise mathematical limit was not taken.

**Solution 2: Grade and Feedback**

* Part a): 3/4 marks
* Part b): 1/2 marks
* Part c): 1/1 mark

**Feedback:** The solution makes good progress but falls short in some areas. Part a) correctly sets up the problem and attempts simplification, but misses the crucial cancellation of terms. Part b) outlines the correct approach but struggles with direct algebraic manipulation. Part c) correctly identifies the Doppler effect and demonstrates understanding of the low-velocity limit.

**Solution 3: Grade and Feedback**

* Part a): 4/4 marks
* Part b): 2/2 marks
* Part c): 1/1 mark

**Feedback:** This solution provides a concise and accurate answer to all parts. Part a) correctly shows the invariance of the scalar product. Part b) outlines the correct approach, acknowledges the complexity of direct calculation, and provides a clear conceptual explanation. Part c) identifies the Doppler effect and explains the simplification to the classical limit. The only minor point is that the expressions for ω' and k' in the limit v/c → 0 could be further simplified.

\bigskip    
\textbf{Response 2:}

**Solution 1: (2/7)**

*   **Part a): 0/4** Incorrect approach to proving the invariance of the scalar product. Simplification errors and a lack of understanding of the key concepts are evident.
*   **Part b): 1/2** Correct identification of the need for Lorentz transformation but errors in the application and simplification process.
*   **Part c): 1/1** Correctly identifies the classical Doppler effect but lacks a clear connection to the limit \( v/c \rightarrow 0 \) in the given formulas.

**General Feedback:** The solution demonstrates some understanding of the problem's concepts, but it lacks precision in calculations and a clear connection between the mathematical steps and the underlying physical principles. A deeper understanding of Lorentz transformations and their impact on electromagnetic fields is needed.

**Solution 2: (4/7)**

*   **Part a): 3/4** Mostly correct approach, but minor errors in simplification prevent a complete proof of invariance. The explanation regarding the invariance is well articulated.
*   **Part b): 1/2** Correct identification of Lorentz transformation and substitution into the phase, but errors in simplification and arriving at the final form of \(\omega'\) and \(k'\).
*   **Part c): 0/1** While the limit \( v/c \rightarrow 0 \) is correctly identified, the explanation lacks a direct connection to the Doppler effect and the transition from relativistic to classical physics.

**General Feedback:** The solution demonstrates a good grasp of the problem's concepts and the application of Lorentz transformations. However, there are minor algebraic errors and a need for more precise explanations to fully connect the mathematical steps to the physical phenomena.

**Solution 3: (6/7)**

*   **Part a): 4/4** Correct calculation and clear demonstration of the invariance of the scalar product.
*   **Part b): 1/2** Correctly identifies the need for Lorentz transformation and its impact on the phase. However, the derivation of the exact forms of \(\omega'\) and \(k'\) lacks clarity.
*   **Part c): 1/1** Clear and concise explanation of the relationship between \( \omega \) and \( \omega' \) as \( v/c \rightarrow 0 \), correctly identifying the classical Doppler effect and its connection to relativistic physics.

**General Feedback:** The solution demonstrates a strong understanding of the problem's concepts, accurate calculations, and a clear explanation of the physical phenomena involved. The only minor issue is the lack of a detailed derivation for \(\omega'\) and \(k'\) in part b. However, the conceptual explanation provided compensates for this, demonstrating a solid grasp of the underlying physics.

\bigskip    
\textbf{Response 3:}

**Solution 1**

* a) 0/4. This solution does not provide the correct invariance property. The explanation suggests an attempt to substitute the transformation equations, but it fails to demonstrate how the terms cancel to show invariance. A proper approach would involve carefully simplifying the expression with the given transformations to arrive at the initial scalar product.
* b) 0/2. The substitution of the Lorentz transformations into the wave phase in frame \( S' \) does not lead to the given expressions for \( \omega' \) and \( k' \). There seems to be a misapplication of the Lorentz transformations, as the derived expression for the phase is complex and does not directly align with the expected form. A correct approach would involve isolating the \( x' \) and \( t' \) terms and comparing them to the given \( \omega' \) and \( k' \) expressions.
* c) 0/1. The limit case of \( v/c \rightarrow 0 \) is mentioned, and it's stated that \( \omega' \) approaches \( \omega \). However, there is no mention of the Doppler effect or any justification provided for why the frequencies converge in the classical limit. A proper explanation would involve highlighting the connection between the given expression and the classical Doppler effect when relativistic effects become negligible.

**General Feedback:** This solution demonstrates an understanding of the concepts and the necessary steps involved. However, it lacks the correct mathematical manipulation and the ability to connect the derived results with the expected forms and physical phenomena. The student needs to practice algebraic simplification and focus on the precise application of the Lorentz transformations to accurately derive the expected results and interpret their physical significance.

---

**Solution 2**

* a) 4/4. This solution correctly calculates the dot product \( \underline{E'} \cdot \underline{B'} \) using the given transformations and shows that it's equal to \( \underline{E} \cdot \underline{B} \), demonstrating invariance under the Lorentz transformation.
* b) 2/2. This solution provides a clear explanation of how the Lorentz transformation is applied to the phase of the electromagnetic wave and how it leads to the given expressions for \( \omega' \) and \( k' \), demonstrating the correct transformation of the wave's phase in the moving frame.
* c) 1/1. This solution accurately identifies the limit case of \( v/c \rightarrow 0 \) and explains that in this limit, the observed frequency \( \omega' \) approaches the source frequency \( \omega \), highlighting the transition from relativistic to classical Doppler effect as relativistic effects become negligible.

**General Feedback:** This solution demonstrates a strong understanding of the concepts and the ability to perform the necessary calculations correctly. The student has a clear grasp of the Lorentz transformation and its application to electromagnetic fields and wave propagation. The solution is well-structured and explains each step clearly, demonstrating a strong understanding of the physics involved.

---

**Solution 3**

* a) 4/4. This solution correctly calculates the scalar product \( \underline{E'} \cdot \underline{B'} \) and simplifies it to show that it's equal to \( \underline{E} \cdot \underline{B} \), demonstrating the invariance of the scalar product under the Lorentz transformation.
* b) 2/2. This solution provides a correct conceptual explanation of how the Lorentz transformation affects the phase of the electromagnetic wave and why the given expressions for \( \omega' \) and \( k' \) arise. It acknowledges the complexity of direct symbolic manipulation but still conveys the essential ideas behind the phase transformation.
* c) 1/1. This solution correctly identifies the limit case of \( v/c \rightarrow 0 \) and explains how the expressions for \( \omega' \) and \( k' \) simplify in this limit. It further mentions the connection to the classical Doppler effect and how the relativistic transformations reduce to their classical counterparts in this limit.

**General Feedback:** This solution is well-written and demonstrates a strong understanding of the concepts involved. The student's approach is systematic, and the explanations are clear. While the solution acknowledges the difficulty in directly deriving the expressions for \( \omega' \) and \( k' \), it provides a valid conceptual understanding of the phase transformation. Overall, the solution demonstrates a solid grasp of the physics and the ability to convey complex ideas effectively.

\bigskip    
\textbf{Response 4:}

**Solution 1: (2/7)**

*   **Part a): (0/4)** No clear attempt to calculate the dot product in both frames and show equivalence.
*   **Part b): (1/2)** Correctly identifies the need for Lorentz transformations, but the derivation of the phase is incorrect.
*   **Part c): (1/1)** Correctly identifies the Doppler effect and explains the limit $v/c \to 0$.

**Feedback:** This solution shows some understanding of the concepts but lacks the mathematical rigor to correctly execute the calculations. The student needs to carefully apply the given transformations and the Lorentz transformation to obtain the correct results.

**Solution 2: (4/7)**

*   **Part a): (2/4)** The dot product is computed in both frames, but there is no clear demonstration of invariance.
*   **Part b): (1/2)** Correctly applies the Lorentz transformation to the phase, but there are errors in the simplification.
*   **Part c): (1/1)** Correctly identifies the Doppler effect and explains the limit $v/c \to 0$.

**Feedback:** This solution is closer to the correct solution but contains algebraic errors in parts a) and b). The student should double-check their calculations and ensure they use the correct expressions for the Lorentz transformation and the given transformations for the electric and magnetic fields.

**Solution 3: (6/7)**

*   **Part a): (4/4)** Correctly demonstrates the invariance of the scalar product.
*   **Part b): (1/2)** Although the correct conceptual explanation is given, the derivation of the phase transformation is not shown.
*   **Part c): (1/1)** Correctly identifies the Doppler effect, explains the limit \( v/c \to 0 \), and relates it to the classical Doppler effect.

**Feedback:** This is a well-written solution, with a clear understanding of the underlying concepts and the correct application of the Lorentz transformation. The only minor issue is the lack of explicit derivation for the phase transformation in part b). Overall, this solution demonstrates a strong grasp of the material.

\bigskip    
\textbf{Response 5:}

**Solution 1**

*   Part a) - 2/4 marks
*   Part b) - 1/2 marks
*   Part c) - 0/1 marks

**Feedback:** The solution attempts to address all parts of the question but lacks clarity in execution. In part a), the approach is correct but needs simplification and a clear statement of the invariance. Part b) outlines the correct method but doesn't reach the desired form. Part c) needs a more concise and direct explanation of the classical Doppler effect. Overall, the solution demonstrates understanding but requires refinement for precision and clarity.

**Solution 2**

*   Part a) - 2/4 marks
*   Part b) - 1/2 marks
*   Part c) - 1/1 marks

**Feedback:** This solution demonstrates a good understanding of the problem. Part a) correctly addresses the invariance, though the simplification could be more concise. Part b) outlines the method correctly but doesn't arrive at the final expression. Part c) provides a clear explanation of the classical Doppler effect and its connection to relativistic physics. Overall, the solution shows strong potential, but the execution could be refined for parts a) and b).

**Solution 3**

*   Part a) - 4/4 marks
*   Part b) - 2/2 marks
*   Part c) - 1/1 marks

**Feedback:** This is an excellent solution. Part a) demonstrates a clear understanding of the invariance of the scalar product and provides a detailed derivation. Part b) correctly uses the Lorentz transformations to arrive at the desired form of the phase in the S' frame. Part c) provides a thorough explanation of the physical phenomenon, correctly identifying the classical Doppler effect and explaining its connection to the relativistic limit. Overall, the solution is well-structured, mathematically rigorous, and provides a comprehensive understanding of the problem.
\end{document}