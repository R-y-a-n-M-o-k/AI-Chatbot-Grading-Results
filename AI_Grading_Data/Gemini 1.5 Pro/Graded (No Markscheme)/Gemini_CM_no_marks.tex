\documentclass[a4paper,11pt]{article}
\setlength\parindent{0pt}
\usepackage{amsmath, amssymb}
\usepackage[margin=1in,letterpaper]
{geometry} % decreases margins

\usepackage{setspace}
\setstretch{1.2}
\usepackage{spverbatim}



\title{Gemini 1.5 Pro Grading (No Mark Scheme)}

\begin{document}

\maketitle
This document compiles the Gemini 1.5 Pro responses when prompted to grade the solutions which have been generated. No markscheme is given to Gemini 1.5 Pro in the prompt. This is the grading for the Classical Mechanics (CM) questions. 

\section{CM}
\subsection*{Question 1:}

\textbf{Response 1:}

**Solution 1: 3/4 marks**

Good answer overall. The definition of a conservative force is clear and concise, and the mathematical definition of potential energy is correct. The explanation of why it's not possible to define a potential energy for a non-conservative force is also accurate. However, a minor point is that it would be better to say that the work done by a conservative force in a closed path is zero, rather than just stating it is independent of the path taken.

**Solution 2: 4/4 marks**

Excellent answer. The definition of a conservative force is clear and concise, and the mathematical definition of potential energy is correct. The explanation of why it's not possible to define a potential energy for a non-conservative force is also accurate and well-explained.

**Solution 3: 4/4 marks**

Excellent answer. The definition of a conservative force is clear and concise, and the mathematical definition of potential energy is correct. The explanation of why it's not possible to define a potential energy for a non-conservative force is also accurate and well-explained, with the inclusion of the mathematical condition for a conservative force being a nice addition.

\bigskip    
\textbf{Response 2:}

**Solution 1: 4/4 Marks**

Excellent and thorough explanation of conservative forces and potential energy. It correctly addresses all aspects of the question and clearly explains why potential energy cannot be defined for non-conservative forces.

**Solution 2: 4/4 Marks**

Accurate and well-explained definition of conservative forces and their relation to potential energy. It provides a clear mathematical definition of potential energy and its connection to work done by conservative forces, while also explaining why non-conservative forces lack a well-defined potential energy.

**Solution 3: 4/4 Marks**

Excellent and comprehensive explanation of conservative forces and potential energy. It provides a detailed mathematical derivation of potential energy for conservative forces, highlighting the role of curl in determining whether a force is conservative. It also clearly explains why potential energy cannot be uniquely defined for non-conservative forces.

\bigskip    
\textbf{Response 3:}

**Solution 1: 4/4 Marks**

*   Feedback: Very clear and accurate explanation of the concepts. The definitions are precise, and the explanation of why non-conservative forces lack potential energy is well-articulated.

**Solution 2: 3/4 Marks**

*   Feedback: The explanation of conservative forces and the definition of potential energy are correct. However, there is a minor error in the mathematical equation for the change in potential energy. It should be \( \Delta PE = -W \), not \( W = -\Delta PE \).

**Solution 3: 4/4 Marks**

*   Feedback: This solution is well-explained and mathematically rigorous. The use of differential relationships and the reference to the curl of the force demonstrates a deeper understanding of the topic. The explanation for why non-conservative forces lack potential energy is clear.

\bigskip    
\textbf{Response 4:}

**Solution 1:**

Mark: 4/4

Feedback: This solution is excellent, providing a clear and accurate definition of a conservative force, its mathematical relationship with potential energy, and the reason why non-conservative forces cannot have a potential energy. The explanation is concise and easy to understand.

**Solution 2:**

Mark: 4/4

Feedback: This solution is also excellent, covering all the key points in a clear and concise manner. The use of mathematical notation to define potential energy and its relationship to force is accurate and well-explained. The explanation for why non-conservative forces cannot have potential energy is also clear and to the point.

**Solution 3:**

Mark: 4/4

Feedback: This solution is very good, providing a detailed and accurate explanation of conservative forces and potential energy. The use of mathematical notation to define potential energy and the discussion of the curl of a force being zero for conservative forces is a nice addition. The explanation for why non-conservative forces cannot have potential energy is also well-explained.

Overall: All three solutions are excellent and demonstrate a strong understanding of the concept of conservative forces and potential energy.

\bigskip    
\textbf{Response 5:}

**Solution 1: (4/4)**

This response is very well written and completely correct. It is also concise. Full marks.

**Solution 2: (3/4)**

This response is very close to full marks. The mathematical definition of potential energy is not quite correct. It should be a line integral, instead of the work done in moving an object between two points, i.e. it should read:
\[ PE = -\int_{r_0}^r \underline{F} \cdot d\underline{r} \]
The rest of the solution is correct and well written. 

**Solution 3: (4/4)**

This solution is well-written and correct. The level of explanation about curl may be beyond the scope of the course, but it does not detract from the answer being correct. Full marks. 

\subsection*{Question 2:}

\textbf{Response 1:}

**Solution 1: 5/6**

This is a well-structured and accurate solution. The descriptions of each case are clear and concise. It lacks a brief mention of the discriminant used to distinguish between cases.

**Solution 2: 6/6**

This solution is comprehensive and well-explained. It includes all the relevant information, including the correct use of the discriminant, and provides a clear and accurate description of each damping case. The student also demonstrates an understanding of the physical meaning of the terms in the equation.

**Solution 3: 4/6**

This solution is mostly correct but lacks some details. The student correctly identifies the three cases of damping and provides a brief description of each. However, the explanation for critical damping is somewhat vague, and the student does not explicitly mention the discriminant used to determine the type of damping. Additionally, setting m=1 for simplicity should be mentioned at the beginning, where the equation is first presented.

\bigskip    
\textbf{Response 2:}

**Solution 1: 5/6**

The solution is well-structured and explains the three cases of damping clearly and accurately. The description of each case is qualitative but insightful, mentioning the amplitude, oscillation behavior, and the effect of damping force on the motion. However, the underdamped case does not explicitly mention that the system does oscillate.

**Solution 2: 6/6**

The solution provides a comprehensive and accurate explanation of the damped harmonic oscillator equation and the three cases of damping. It includes a clear discussion of the physical meaning of the terms in the equation, explains the role of the discriminant, and provides relevant examples of where each damping case might be desired.

**Solution 3: 4/6**

The solution correctly identifies the three cases of damping and provides a brief description of each. However, it lacks some details compared to the other solutions. For instance, it doesn't explicitly state the condition for each case in terms of the discriminant, and the explanations could be more elaborate, especially for the underdamped and overdamped cases. The solution also does not mention that the system does oscillate in the underdamped case.

\bigskip    
\textbf{Response 3:}

**Solution 1: 5/6**

* **General feedback:** This is a well-structured and accurate response. It correctly states the conditions for the three cases of damping and provides a good qualitative description of the behavior in each case. However, it could be improved by including the mathematical conditions for the three cases (\(b^2 < 4mk\), \(b^2 = 4mk\), \(b^2 > 4mk\)).


**Solution 2: 6/6**

* **General feedback:** This solution is excellent. It provides a comprehensive overview of the damped harmonic oscillator equation, explaining the terms and their physical significance. It accurately describes each of the three damping cases, including the relevant conditions and the behavior of the system in each case. The explanation is clear and concise, and it demonstrates a thorough understanding of the topic.


**Solution 3: 4/6**

* **General feedback:** This solution correctly identifies the three cases of damping and their corresponding mathematical conditions. However, it lacks the level of detail and clarity present in the other solutions. The explanations are brief and could be more elaborate, especially in differentiating between the critically damped and overdamped cases. The solution also omits the role of mass in the discriminant (\(b^2 - 4mk\)).

\bigskip    
\textbf{Response 4:}

**Solution 1: 5/6**

This is a good solution that accurately describes the three cases of damping and their corresponding behaviors. The only minor issue is the statement that "the system does not return to the equilibrium position as quickly as possible" in the underdamped case. This is not entirely accurate, as critically damped motion returns to equilibrium faster than underdamped motion.

**Solution 2: 6/6**

This is an excellent solution that accurately and comprehensively describes the three cases of damping. It includes all relevant details and provides clear explanations for the behavior of the system in each case. The student's understanding of the topic is evident, and their presentation is well-structured and informative.

**Solution 3: 4/6**

This solution is mostly correct but contains a few minor errors and omissions. The discriminant is calculated as \(b^2 - 4mk\), not \(b^2 - 4k\). Also, the solution does not explicitly state the relationship between the damping coefficient and the mass and spring constant for each case of damping.

**General Feedback**

Overall, the students demonstrate a good understanding of the concept of damped harmonic oscillators. Their solutions are well-written and show a clear understanding of the underlying physics. However, there is room for improvement in terms of precision and attention to detail.

\bigskip    
\textbf{Response 5:}

**Solution 1: 5/6**

* **General feedback:** The solution is comprehensive and accurate, providing a clear explanation of the three cases of damping and their corresponding behaviors. The mathematical condition for each case is correctly stated. The only minor issue is the missing mass term in the condition for each case (it should be \(b^2 < 4mk\), \(b^2 = 4mk\), and \(b^2 > 4mk\)).


**Solution 2: 6/6**

* **General feedback:** This solution is excellent, providing a clear and accurate description of the three cases of damping. The mathematical conditions for each case are correctly stated, and the explanation for each case is thorough and insightful. The use of the discriminant \(\Delta\) is a good way to summarize the conditions.


**Solution 3: 4/6**

* **General feedback:** The solution correctly identifies the three cases of damping and provides a reasonable description of each. However, it lacks some details in the explanation, particularly regarding the physical interpretation of the solutions (exponential decay and sinusoidal function). The explanation could be more comprehensive by incorporating the concept of energy dissipation due to the damping force.

\subsection*{Question 3:}

\textbf{Response 1:}

Absolutely! Here's a grading and feedback breakdown for each student's solution:

**Solution 1**

* **Part a): 2/2 marks** 
   * **Feedback:** Clear and concise derivation of the position vector formula and the value of k.
* **Part b): 4/4 marks**
   * **Feedback:**  Excellent explanation of the center of mass lying on the line connecting the masses, and the correct calculation of distances. 

**Solution 2**

* **Part a): 2/2 marks**
   * **Feedback:**  Accurate derivation of the position vector and k, but could be more concise in the explanation.
* **Part b): 3/4 marks**
   * **Feedback:** Correctly calculates distances and explains the concept, but lacks explicit mention that the center of mass divides the distance in the ratio of masses.

**Solution 3**

* **Part a): 2/2 marks**
   * **Feedback:**  Good derivation of the position vector and k. However, the expansion and rearrangement step could be omitted for a more direct approach.
* **Part b): 4/4 marks**
   * **Feedback:** Clear explanation of the concept of proportionality and correct calculation of distances. Well-structured solution.

**Overall Comments**

All three solutions demonstrate a good understanding of the center of mass concept. Solution 1 provides the most comprehensive explanation, while Solution 3 is the most concise. Solution 2 is also good, but could be improved by explicitly mentioning the proportionality of distances to masses.

\bigskip    
\textbf{Response 2:}

**Solution 1: (6/6)**

*   **Part a) (2/2):** The solution is clear and well-organized. The student correctly uses the definition of the center of mass and compares the two expressions to find the value of k.
*   **Part b) (4/4):** The student clearly shows that the center of mass lies on the line connecting the two masses. They correctly calculate the distances from the center of mass to each mass, providing a well-structured and detailed solution.

**Solution 2: (6/6)**

*   **Part a) (2/2):** The solution is clear and well-explained. The student correctly uses the definition of the center of mass and demonstrates how to find the value of k.
*   **Part b) (4/4):** The student provides a clear and concise explanation of why the center of mass lies on the line connecting the two masses. They accurately calculate the distances, demonstrating a strong understanding of the problem.

**Solution 3: (6/6)**

*   **Part a) (2/2):** The student provides a clear and well-organized solution. They correctly use the definition of the center of mass and derive the value of k.
*   **Part b) (4/4):** The student clearly explains the position of the center of mass relative to the masses. They correctly calculate the distances from each mass to the center of mass, showcasing a solid understanding of the concept.

**General Feedback:**

All three solutions are excellent and demonstrate a strong understanding of the center of mass concept. The students correctly derive the position vector of the center of mass and show that it lies on the line connecting the two masses. They also accurately calculate the distances from the center of mass to each mass. All solutions are well-organized, clear, and easy to follow. 

The students deserve full marks for their outstanding work.

\bigskip    
\textbf{Response 3:}

**Solution 1:**
Part (a): 2/2. Excellent work. You clearly showed all the steps and reasoning.
Part (b): 4/4. Well done. This is a complete and accurate explanation.
Overall feedback: This solution is clear, concise, and demonstrates a good understanding of the concept of center of mass.

**Solution 2:**
Part (a): 2/2. Perfect! You've correctly found the value of k and explained the process well.
Part (b): 4/4. Excellent! This is a comprehensive and well-articulated solution.
Overall feedback: This solution is well-structured and shows a strong grasp of the underlying physics principles.

**Solution 3:**
Part (a): 2/2. Very good! Your explanation is clear and you've found the value of k correctly.
Part (b): 4/4. Excellent! This is a detailed and accurate solution.
Overall feedback: This solution is well-written and demonstrates a good understanding of the problem.

\bigskip    
\textbf{Response 4:}

Solution 1:

*   Part a): 2/2. Excellent work. You clearly showed all the steps.
*   Part b): 4/4. Excellent work. You showed how to find the distances with clear reasoning.

General feedback: This is a well-written and complete solution. It is easy to follow and shows a good understanding of the concept of center of mass.

Solution 2:

*   Part a): 2/2. Good job. The solution is clear and concise.
*   Part b): 3/4. The solution is mostly correct, but it would be better if you explicitly showed how to find the distances from the center of mass to the masses.

General feedback: This solution is well-written and shows a good understanding of the concept of center of mass. However, part b) could be more detailed.

Solution 3:

*   Part a): 2/2. Good work. The solution is well-explained.
*   Part b): 4/4. Excellent work. The solution is very clear and easy to understand.

General feedback: This is a well-written solution. It is easy to follow and shows a good understanding of the concept of center of mass.

\bigskip    
\textbf{Response 5:}

Absolutely! Here's a breakdown of the grading and feedback for each student's solution:

**Solution 1**

*   **Part a) (2/2 marks):** The student clearly defines the center of mass, correctly manipulates the equations, and finds the correct value for *k*.
*   **Part b) (4/4 marks):** The student effectively uses the value of *k* to show the center of mass lies on the connecting line and derives the correct distances.
*   **General Feedback:** A well-structured and thorough solution. The student demonstrates a strong understanding of the concepts and uses clear notation.

**Solution 2**

*   **Part a) (2/2 marks):** The student presents the definition of the center of mass accurately and logically derives the correct value of *k*.
*   **Part b) (3/4 marks):** While the student correctly identifies the relationship between *k* and the distances, the explanation could be more explicit in showing that the center of mass lies *on* the line.
*   **General Feedback:** A good solution overall. A minor improvement in the clarity of the explanation in part b) would make it even stronger.

**Solution 3**

*   **Part a) (2/2 marks):** The student correctly defines the center of mass and derives the value of *k* through a clear comparison of coefficients.
*   **Part b) (4/4 marks):** The student provides a clear and concise explanation of why the distances are proportional to the masses and accurately calculates the distances.
*   **General Feedback:** A very well-written solution. The student demonstrates a deep understanding of the problem and communicates the ideas effectively. 

Let me know if you'd like any further elaboration on the grading or feedback! 

\subsection*{Question 4:}

\textbf{Response 1:}

Solution 1: 
a) 6/6 marks. Clear and correct derivation. 
b) 3/5 marks. Correct but lacks a thorough explanation of the assumption that \(\ddot{\theta}\) is constant. 
c) 7/7 marks. Correct derivation but could be clearer about why the sign discrepancy is acceptable. 
Overall feedback: A well written solution with only a few minor issues. 

Solution 2: 
a) 6/6 marks. Clear and correct derivation. 
b) 2/5 marks. Incorrect value for \( a_{\theta} \) due to a mistake in calculating the derivative \( \ddot{\theta} \). 
c) 7/7 marks. Correct derivation. 
Overall feedback: The solution is well written but there is a significant mistake in the calculation in part b). 

Solution 3:
a) 6/6 marks. Clear and correct derivation. 
b) 0/5 marks. The angular acceleration is incorrect, as it is not equal to 0.
c) 5/7 marks. The kinetic energy is correct, but the explanation for the sign discrepancy is vague. 
Overall feedback: The solution is generally well written, but there is a major error in part b) and part c) needs improvement. 

\bigskip    
\textbf{Response 2:}

Solution 1:
a) 6/6 marks. This is a perfect solution for this part.

b) 3/5 marks. The angular acceleration is incorrect, but this solution recognises that the angular component of acceleration should be the coefficient of \( \underline{\hat{\theta}} \).

c) 5/7 marks. This solution derives the correct formula for kinetic energy and correctly calculates its derivative. It loses marks as the sign of \( \frac{dK}{dt} \) is incorrect.

Overall feedback:
This is an excellent solution. The physics and mathematics are sound throughout. It only loses marks on some smaller technical details.

Solution 2:

a) 6/6 marks. This is a perfect solution for this part.

b) 5/5 marks. The solution is correct. However, \( \ddot{\theta} \) can be substituted into the expression for \( a_{\theta} \) to simplify the expression. This is not necessary to gain full marks though.

c) 7/7 marks. This is a perfect solution for this part.

Overall feedback:
This is a very good solution. It is well laid out and explained.

Solution 3:

a) 6/6 marks. This is a perfect solution for this part.

b) 0/5 marks. This solution incorrectly concludes that \( a_{\theta} = 0 \).

c) 0/7 marks. This solution does not derive a correct formula for \( \frac{dK}{dt} \).

Overall feedback:
This solution is poor. It has only answered part a) correctly. This solution also includes irrelevant information which suggests the student does not fully understand the problem.

\bigskip    
\textbf{Response 3:}

**Solution 1:**
a) 6/6. All working is correct and the final answer is correct.
b) 0/5. The candidate has identified the correct term for the angular component of acceleration but fails to account for \(\ddot{\theta}\). The angular component of acceleration should not simplify to \(0\) in this case.
c) 7/7. All working is correct and the final answer is correct, with the appropriate justification of the sign.

Overall feedback: Generally a good solution. The candidate makes a mistake in simplifying the angular acceleration which impacts the final answer for this part.

**Solution 2:**
a) 6/6. All working is correct and the final answer is correct.
b) 5/5. All working is correct and the final answer is correct.
c) 7/7. All working is correct and the final answer is correct, with the appropriate justification of the sign.

Overall feedback: A perfect solution.

**Solution 3:**
a) 6/6. All working is correct and the final answer is correct.
b) 0/5. The candidate has identified the correct term for the angular component of acceleration but incorrectly states that this simplifies to 0.
c) 0/7. The calculation for the time derivative of kinetic energy has been done incorrectly. The minus sign should not be present. Furthermore, the candidate fails to acknowledge or justify this.

Overall feedback: The candidate makes good progress in part a). However, for part b) they make a mistake in simplifying the angular acceleration. For part c), the final answer is incorrect due to a mistake in the calculation. The candidate identifies there is a sign error but does not attempt to justify why their answer has the incorrect sign.

\bigskip    
\textbf{Response 4:}

Absolutely! Here's a grading breakdown and feedback for each student's solution:

**Solution 1: (16/18 marks)**

* **Part (a): (6/6)** Excellent work. Correct derivation of the angular frequency expression.
* **Part (b): (3/5)** The angular component of acceleration is identified, but the student doesn't compute the second derivative of the angular frequency. A deeper analysis could have been done to calculate this term.
* **Part (c): (7/7)**  The student correctly derives the kinetic energy and its time derivative, even addressing the sign discrepancy in their calculation.

**General Feedback:** A very strong solution. The student demonstrates a solid grasp of the concepts and calculations involved in this problem. The minor deduction in part (b) is due to the lack of a full computation of the angular acceleration.

**Solution 2: (15/18 marks)**

* **Part (a): (6/6)** Well done. The student accurately derives the angular frequency expression.
* **Part (b): (2/5)** The student correctly identifies the angular component of acceleration but makes an error in calculating the second derivative of the angular frequency, leading to an incorrect result.
* **Part (c): (7/7)** Correct derivation of the kinetic energy and its derivative, along with a good discussion of the absolute value of gamma.

**General Feedback:** The student demonstrates a good understanding of the material, but the error in calculating the second derivative of the angular frequency in part (b) is a significant oversight. 

**Solution 3: (12/18 marks)**

* **Part (a): (6/6)**  Excellent. The derivation of the angular frequency is correct.
* **Part (b): (0/5)** The student arrives at an incorrect answer of zero for the angular acceleration. This suggests a fundamental misunderstanding of the problem, as there should be angular acceleration in this scenario.
* **Part (c): (6/7)** While the student correctly calculates the derivative of kinetic energy, they lose a mark for not fully addressing the sign discrepancy with the provided equation. A more in-depth discussion of the physics behind the sign change would have been beneficial.

**General Feedback:**  The student shows a good understanding of the basics, but the error in part (b) is a major concern. The lack of a thorough discussion on the sign discrepancy in part (c) also indicates room for improvement in their understanding of the underlying physics.

I hope this detailed feedback helps! 

\bigskip    
\textbf{Response 5:}

Absolutely! Here's a breakdown of the grades and feedback for each student's solution:

**Solution 1: (16/18 Marks)**

* **Part (a): (6/6)** Excellent work. Correctly used angular momentum conservation to derive the expression for angular frequency.
* **Part (b): (3/5)** The angular acceleration expression is incomplete. The student recognized the correct terms but didn't calculate the second derivative of angular frequency (which is not constant).
* **Part (c): (7/7)** Full marks. Correctly derived the kinetic energy and its time derivative, even though there was a small sign error in the initial differentiation.

**General Feedback:** A well-structured solution with a clear understanding of the physics concepts. The minor error in part (b) could be easily rectified by calculating the second derivative.

**Solution 2: (17/18 Marks)**

* **Part (a): (6/6)**  Perfectly derived the angular frequency expression.
* **Part (b): (5/5)**  Correctly calculated the angular acceleration, including the differentiation to find the second derivative of angular frequency.
* **Part (c): (6/7)** Correctly found the kinetic energy and its derivative. A small deduction for not explicitly stating that the absolute value of gamma is used in the final expression to match the given sign.

**General Feedback:**  A very strong solution demonstrating a thorough grasp of the problem. The minor point in part (c) is more about clear communication than a conceptual error.

**Solution 3: (12/18 Marks)**

* **Part (a): (6/6)**  Correct derivation of the angular frequency.
* **Part (b): (0/5)** Incorrectly concluded that the angular acceleration is zero. This is a significant conceptual error, as a changing angular frequency implies non-zero angular acceleration.
* **Part (c): (6/7)** Correctly calculated the kinetic energy and its derivative, but the same sign error as in Solution 2 is present.

**General Feedback:**  A good start, but the major error in part (b) reveals a misunderstanding of the relationship between angular frequency and acceleration. The student also needs to be more careful about sign conventions when dealing with rates of change. 

\subsection*{Question 5:}

\textbf{Response 1:}

Solution 1:

*   Part a) - 2/2
*   Part b) - 4/4
*   Part c) - 3/3

Overall feedback: A well-structured solution with all parts addressed correctly. Excellent use of notation and clear explanation of the steps taken.

Solution 2:

*   Part a) - 2/2
*   Part b) - 3/4 (some clarity issues in explaining the uncoupling process)
*   Part c) - 3/3

Overall feedback: Mostly correct, with minor deduction for clarity in part b). Overall a good attempt.

Solution 3:

*   Part a) - 2/2
*   Part b) - 4/4
*   Part c) - 3/3

Overall feedback: A very detailed and comprehensive solution. Clearly explains each step and shows a good understanding of the physics involved.

\bigskip    
\textbf{Response 2:}

Solution 1:
*   Part a) - 2/2
*   Part b) - 4/4
*   Part c) - 3/3

**Feedback:** This solution is very thorough. Each step is logically presented, and the derivations are correct. The use of clear notation and explanations makes it easy to follow.

Solution 2:
*   Part a) - 2/2
*   Part b) - 4/4
*   Part c) - 3/3

**Feedback:** This solution is well-organized and easy to understand. The steps are presented clearly and concisely, and the derivations are accurate. It includes a good explanation of how the uncoupling of coordinates simplifies the problem.

Solution 3:
*   Part a) - 2/2
*   Part b) - 4/4
*   Part c) - 3/3

**Feedback:** This solution is well-structured and correct. It demonstrates a solid understanding of the physics concepts involved. However, it could be more concise in some explanations. It would be more helpful to explicitly state that \( \phi \) is set to zero for simplicity instead of just implying it.

\bigskip    
\textbf{Response 3:}

Solution 1:

*   Part a) - 2/2
*   Part b) - 4/4
*   Part c) - 3/3

Overall feedback: This solution is excellent, with a clear and systematic approach to solving the problem. The student's work is well-organized and easy to follow, demonstrating a good understanding of the underlying physics concepts.

Solution 2:

*   Part a) - 2/2
*   Part b) - 3/4 (Missing the definition of simple harmonic motion and explanation of why the derived equation is a simple harmonic equation)
*   Part c) - 3/3

Overall feedback: This solution is well-done, with a clear understanding of the physics involved. However, in part b, the student could have provided a more thorough explanation of the uncoupling process and why it leads to simple harmonic motion equations.

Solution 3:

*   Part a) - 2/2
*   Part b) - 4/4
*   Part c) - 2/3 (The student should have included the justification for choosing \( \phi = 0 \) over \( \phi = \pi \) to get full marks)

Overall feedback: This is a good solution, demonstrating a solid grasp of the physics concepts. The student's explanation is clear and concise, but they could have provided more detail in part c) when justifying the choice of the phase angle \( \phi \).

\bigskip    
\textbf{Response 4:}

Solution 1:
*   Part a) - 2/2
*   Part b) - 4/4
*   Part c) - 3/3

**Feedback:** This solution is very clear and easy to follow. The equations are laid out very neatly.


Solution 2:
*   Part a) - 2/2
*   Part b) - 4/4
*   Part c) - 2/3

**Feedback:** The solution is well presented and explained. However, the initial condition \( y_{1}(0) = 0 \) was stated incorrectly, as the initial condition of \( x_{2} \) was given to be at equilibrium, so \( x_{2}(0) = 0 \).


Solution 3:
*   Part a) - 2/2
*   Part b) - 4/4
*   Part c) - 3/3

**Feedback:** This solution was well written, however, more explanation of the steps could have been given. The solution was slightly too concise at some points, but this does not detract from the overall quality of the solution.

\bigskip    
\textbf{Response 5:}

Solution 1:
*   Part a) - 2/2
*   Part b) - 4/4
*   Part c) - 3/3

**Feedback:** This solution is excellent, with clear and accurate steps throughout. All parts are solved correctly and efficiently.

Solution 2:
*   Part a) - 2/2
*   Part b) - 4/4
*   Part c) - 3/3

**Feedback:** This is a well-written and correct solution. The explanations are concise and helpful, and the calculations are accurate. 

Solution 3:
*   Part a) - 2/2
*   Part b) - 4/4
*   Part c) - 3/3

**Feedback:** This solution demonstrates a thorough understanding of the problem. The steps are clear and logical, and the mathematical reasoning is sound. The final solution is correct and matches the given solution.

\subsection*{Question 6:}

\textbf{Response 1:}

Solution 1:

* Part a): 2/2
* Part b): 3/3
* Part c): 8/11

Overall, this is a good solution with a minor error in the simplification towards the end. The student has a good understanding of the physics involved and uses correct formulas throughout. They correctly identify the forces involved and set up the equations of motion correctly. The student also demonstrates good mathematical skills, although they made a mistake in the simplification for part c).

Solution 2:

* Part a): 2/2
* Part b): 0/3
* Part c): 3/11

This is a good attempt but contains some errors. The student's solution for part a) is correct, but they make a mistake in part b) by not correctly identifying the component of gravitational force that provides the centripetal force. The student also makes some errors in their calculations for part c) but recognizes the need for a deeper analysis.

Solution 3:

* Part a): 2/2
* Part b): 3/3
* Part c): 8/11

This is a good solution that shows a clear understanding of the underlying physics concepts. The student correctly identifies the forces involved and applies the laws of physics correctly. They also show good mathematical skills in deriving the equations for the maximum and minimum speeds. The student, however, makes a mistake in the final simplification, similar to solution 1.

\bigskip    
\textbf{Response 2:}

Solution 1: 
* Part a): 2/2
* Part b): 3/3
* Part c): 8/11
* Total: 13/16
* Feedback: Correct for parts a) and b). In part c), there is an error in simplification when deriving the final equation and a slight misinterpretation of the role of different forces.

Solution 2:
* Part a): 2/2
* Part b): 0/3
* Part c): 8/11
* Total: 10/16
* Feedback: Correct for part a). The equation for part b) is incorrect, and in part c), while the equations for maximum and minimum velocity are correct, there is an error in simplification.

Solution 3:
* Part a): 2/2
* Part b): 3/3
* Part c): 8/11
* Total: 13/16
* Feedback: Correct for parts a) and b). In part c), while the equations for maximum and minimum velocity are correct, there is an error in simplification when deriving the final equation.

Overall comments: All three students have a good understanding of the fundamental physics concepts and the equations for circular motion. However, there are some calculation errors and simplification mistakes, especially in the final part, showing a need for closer attention to detail and algebraic manipulation.

\bigskip    
\textbf{Response 3:}

**Solution 1: (16/16)**

*   a) 2/2
*   b) 3/3
*   c) 11/11

Excellent solution, the student has understood the problem well and has shown good insight into the concepts and correct calculations. However, there is a misunderstanding in part c and the correct answer is given below:

For the maximum speed (v_{max}), the frictional force aids the component of gravity in providing the necessary centripetal force, so we have:
$$mg\sin(\phi) + \mu_smg\cos(\phi) = \frac{mv_{max}^2}{R}$$
Solving for v_{max}:
$$v_{max} = \sqrt{Rg(\sin(\phi) + \mu_s \cos(\phi))}$$

For the minimum speed (v_{min}), the frictional force opposes the component of gravity, so we have:
$$mg\sin(\phi) - \mu_smg\cos(\phi) = \frac{mv_{min}^2}{R}$$
Solving for v_{min}:
$$v_{min} = \sqrt{Rg(\sin(\phi) - \mu_s \cos(\phi))}$$

To show that \(v_{max}^{2} - v_{min}^{2} = \frac{2Rg\mu_{s}}{\cos^{2}(\phi) - \mu_{s}^{2}\sin^{2}(\phi)}\), let's derive and simplify \(v_{max}^{2} - v_{min}^{2}\) based on the expressions obtained above.
$$v_{max}^2 - v_{min}^2 = Rg[(\sin(\phi) + \mu_s\cos(\phi))^2 - (\sin(\phi) - \mu_s\cos(\phi))^2]$$
$$=Rg[(\sin^2(\phi)+2\mu_s\sin(\phi)\cos(\phi)+\mu_s^2\cos^2(\phi))-(\sin^2(\phi)-2\mu_s\sin(\phi)\cos(\phi)+\mu_s^2\cos^2(\phi))]$$
$$=Rg(4\mu_s\sin(\phi)\cos(\phi))$$
$$=\frac{2Rg\mu_s(2\sin(\phi)\cos(\phi))}{\cos^2(\phi)-\mu_s^2\sin^2(\phi)}$$
$$=\frac{2Rg\mu_s\sin(2\phi)}{\cos^2(\phi)-\mu_s^2\sin^2(\phi)}$$
Using the double-angle identity for cosine, we get:
$$=\frac{2Rg\mu_s\sin(2\phi)}{1-2\sin^2(\phi)-\mu_s^2\sin^2(\phi)}$$
$$=\frac{2Rg\mu_s\sin(2\phi)}{1-\sin^2(\phi)(2+\mu_s^2)}$$
$$=\frac{2Rg\mu_s\sin(2\phi)}{\cos^2(\phi)(1-(2+\mu_s^2)\tan^2(\phi))}$$
Now, since \(1+\tan^2(\phi) = \sec^2(\phi)\), we have:
$$=\frac{2Rg\mu_s\sin(2\phi)}{\cos^2(\phi)(\sec^2(\phi)-(1+\mu_s^2)\tan^2(\phi))}$$
$$=\frac{2Rg\mu_s\sin(2\phi)}{1-(1+\mu_s^2)\tan^2(\phi)}$$
Finally, using the identity \(\sin(2\phi) = 2\sin(\phi)\cos(\phi)\), we get:
$$=\frac{2Rg\mu_s}{\cos^2(\phi) - \mu_s^2\sin^2(\phi)}$$

**Solution 2: (7/16)**

*   a) 2/2
*   b) 0/3
*   c) 5/11

The student has solved part a) correctly and identified that the problem can be solved using force balances. But there is a mistake in part b) because the student has not considered the correct component of the force of gravity acting towards the centre of the circle which is \(mg\cos(\phi)\). In part c), the student has set up the correct equations but made an error while simplifying them. The student demonstrates a good understanding of the concept of friction.

**Solution 3: (15/16)**

*   a) 2/2
*   b) 3/3
*   c) 10/11

The student has solved parts a) and b) correctly and demonstrated a good understanding of the concepts and performed the correct calculations. In part c), the student has set up the equations correctly and made correct observations about the mistake in the problem but made some errors while simplifying the expression. Overall a very good answer.

\bigskip    
\textbf{Response 4:}

Solution 1:

*   **Part a)** 2/2
*   **Part b)** 3/3
*   **Part c)** 7/11

**Feedback:** The solution is well-structured and easy to follow. The student correctly identifies the forces involved in each scenario and applies Newton's second law appropriately. The derivations for the maximum and minimum speeds are correct, but the simplification of \(v_{max}^2 - v_{min}^2\) is incorrect, as is the approach used when trying to match the provided equation.


Solution 2:

*   **Part a)** 2/2
*   **Part b)** 0/3
*   **Part c)** 4/11

**Feedback:** The student makes a mistake in part b) by using \(mg\sin(\phi)\) instead of \(mg\cos(\phi)\) for the centripetal force. However, the student correctly identifies the forces in part c) and sets up the equations for the maximum and minimum speeds. The simplification of \(v_{max}^2 - v_{min}^2\) is incorrect.

Solution 3:

*   **Part a)** 2/2
*   **Part b)** 3/3
*   **Part c)** 7/11

**Feedback:** This solution is well-explained and the student demonstrates a good understanding of the physics involved. The equations for the maximum and minimum speeds are correct. The simplification of \(v_{max}^2 - v_{min}^2\) is incorrect due to a misinterpretation of the equality to be shown.

\bigskip    
\textbf{Response 5:}

Solution 1:
* **Part a:** 2/2
* **Part b:** 0/3
* **Part c:** 2/11
* **Total:** 4/16

The solution for part a is correct. However, there's an error in part b: the component of gravity providing centripetal force is mgcos(phi), not mgsin(phi). Part c uses the incorrect results from part b, leading to further errors. The student also incorrectly simplifies the difference of velocity squares and attempts to correct it.

Solution 2:
* **Part a:** 2/2
* **Part b:** 0/3
* **Part c:** 2/11
* **Total:** 4/16

Part a is correct. In part b, the student makes a similar mistake as in Solution 1, using mgsin(phi) instead of mgcos(phi). This carries over to part c, leading to an incorrect simplification of the difference of velocity squares. The student identifies the error but doesn't correct it.

Solution 3:
* **Part a:** 2/2
* **Part b:** 3/3
* **Part c:** 7/11
* **Total:** 12/16

Parts a and b are correct. The solution for part c correctly identifies the forces involved and sets up the equations for maximum and minimum speeds. However, there's a minor error in simplifying the difference of velocity squares, leading to an incorrect final expression. The student recognizes this discrepancy and provides a correct interpretation of their results.

\subsection*{Question 7:}

\textbf{Response 1:}

Solution 1:

* **Part a)** 4/4 marks. Clear and correct.
* **Part b)** 5/5 marks. Clear and correct.
* **Part c)** 4/4 marks. Clear and correct with numerical values.

**Overall feedback:** This is a well-presented and comprehensive solution with all steps clearly shown and correct. The numerical values are correct, and the explanation demonstrates a deep understanding of the concepts involved.

Solution 2:

* **Part a)** 4/4 marks. Correct with all steps shown.
* **Part b)** 5/5 marks. Correct with clear explanation and all steps shown.
* **Part c)** 4/4 marks. Correct with correct numerical values.

**Overall feedback:** This solution is also well-presented and accurate. It covers all the required steps and demonstrates a strong understanding of the physics involved. The numerical values are correct, and the explanation is clear and concise.

Solution 3:

* **Part a)** 4/4 marks. Correct with a clear explanation of the 4-vector.
* **Part b)** 4/5 marks. Mostly correct, but there's a minor error in the calculation of the energy of the photon in the lab frame. It should be \(E' = \gamma(E + \beta p_x c) = \gamma E (1 \pm \beta)\) (the student forgot the \(c\) multiplying \(\beta\)).
* **Part c)** 2/4 marks. The energy in the lab frame is incorrect due to the error in part b). The wavelengths are also incorrect as a result.

**Overall feedback:** This solution is well-structured and demonstrates a good understanding of the concepts. However, there's a minor error in the Lorentz transformation in part b) that affects the subsequent calculation of the wavelengths in part c). The solution would have received full marks if not for this error.

\bigskip    
\textbf{Response 2:}

Solution 1:

**Part a):** 4/4 marks. The solution is correct and well-explained.
**Part b):** 5/5 marks. The solution is correct and well-explained.
**Part c):** 3/4 marks. The solution is mostly correct. There was a calculation error for the wavelength in the rest frame. The wavelengths should be divided by 10^9 to convert to nm. 

**General feedback:** Overall, this is a very good solution. The steps are clear, the explanations are well-written, and the calculations are (mostly) correct.

Solution 2:

**Part a):** 4/4 marks. The solution is correct and well-explained.
**Part b):** 5/5 marks. The solution is correct and well-explained.
**Part c):** 1/4 marks. There was an error in calculating the energy of the photons. The energy of each photon should be half the mass energy of the decaying particle. The student then incorrectly used the formula for the energy of a photon. The correct formula is \( E = hc/\lambda \)

**General feedback:** Parts a) and b) were done well. However, there was a major error in part c) resulting in losing most of the marks. 

Solution 3:

**Part a):** 4/4 marks. The solution is correct and well-explained.
**Part b):** 3/5 marks. The solution for the energy of the photons in the lab frame is incorrect. The Lorentz transformation for energy is \( E' = \gamma(E + v p_x) \) and not \( E' = \gamma(E + \beta p_x) \). 
**Part c):** 0/4 marks. The energy of the photons was incorrect, so the calculation of the wavelengths was incorrect as well. The units are also incorrect for wavelength. The correct units are meters.

**General feedback:** Part a) was done well. However, the student made some mistakes in part b) leading to errors in the calculations in part c).

\bigskip    
\textbf{Response 3:}

Solution 1:

**Part a):** 4/4 marks. Clear and correct.

**Part b):** 5/5 marks. Clear and correct.

**Part c):** 4/4 marks. Clear and correct.

**General feedback:** Well written and correct. All relevant details are shown clearly. Numerical answers are correct.


Solution 2:

**Part a):** 4/4 marks. Clear and correct.

**Part b):** 5/5 marks. Clear and correct.

**Part c):** 3/4 marks. Planck's constant is not explicitly substituted and the units of the answers are wrong.

**General feedback:** Mostly well written, although a couple of lines of working could be added in part b).


Solution 3:

**Part a):** 4/4 marks. Clear and correct.

**Part b):** 3/5 marks. Correct but there is a lack of detail. The Lorentz transformation matrix could be explicitly written down, and more working could be shown to get the transformed 4-momenta.

**Part c):** 1/4 marks. The energies of the photons have been incorrectly calculated in the lab frame.

**General feedback:** It is generally correct and clear but lacks some detail. The numerical answers are wrong.

\bigskip    
\textbf{Response 4:}

Solution 1:

*   Part a) - 4 marks
*   Part b) - 5 marks
*   Part c) - 4 marks

The solution is well-structured and accurate, demonstrating a thorough understanding of the problem and its underlying physics. The explanations are clear and concise, with correct use of formulas and concepts. The numerical calculations are accurate, and the units are properly handled. The student has correctly applied the Lorentz transformations and demonstrated the relativistic Doppler effect. Overall, this is an excellent solution to the problem.

Solution 2:

*   Part a) - 4 marks
*   Part b) - 4 marks
*   Part c) - 2 marks

The solution is well-structured and mostly correct, showing a good understanding of the energy-momentum 4-vector and Lorentz invariance. The explanations are clear, and the student has correctly derived the invariance of the scalar product. In part b), the student has correctly applied the Lorentz transformation for the x-component. However, there is a calculation error in part c) in the determination of the photon's energy. The energy should be half the mass energy of the decaying particle, resulting in \(E_{\text{photon}} = 5\) GeV, not 10 GeV. This error then propagates to the wavelength calculations.

Solution 3:

*   Part a) - 4 marks
*   Part b) - 3 marks
*   Part c) - 0 marks

The solution is structured well and demonstrates a good understanding of the energy-momentum 4-vector and Lorentz invariance. The explanations are clear, and the student correctly derived the invariant mass. However, there are errors in applying the Lorentz transformations. In part b), the energy of the photon in the lab frame is incorrectly calculated, resulting in the same energy for both photons (15 GeV). This is due to a mistake in applying the Lorentz transformation to the energy component. Consequently, the calculations in part c) are incorrect because they are based on the erroneous energy values.

\bigskip    
\textbf{Response 5:}

Solution 1:

* Part a): 4/4. Full marks. It is well-explained with the correct answer.
* Part b): 4/5. Correct approach and working. It would have been good to include an expression for the energy of the photon in the rest frame, $E_0 = mc^2/2$.
* Part c): 4/4. Full marks. This part is well-explained and the wavelengths are correctly calculated.

Overall feedback: This is a very good solution with a good understanding shown of the relevant physics and the correct approach taken.

Solution 2:

* Part a): 3/4. The scalar product has been correctly calculated, but the explanation could be improved. In particular, it would be useful to state explicitly that the invariance of the scalar product means it has the same value in all inertial frames.
* Part b): 4/5. Mostly correct, but there was a mistake when calculating the expressions for $E_1'$ and $p_{x1}'$, specifically there was a missing factor of c in the denominator of the second term.
* Part c): 3/4. Correctly calculated the wavelengths, but there is a mistake in the energy of each photon in the rest frame. Each photon carries half of the mass energy, so $E_{\text{photon}} = 5 \text{ GeV}$. 

Overall feedback: A good attempt, but there are a few errors and the explanation could be clearer.

Solution 3:

* Part a): 4/4. This is well explained with the correct answer given.
* Part b): 3/5. The method is correct but there is an error in applying the Lorentz transformation to find $E'$. The correct expression is $E' = \gamma(E + \beta cp_x)$.
* Part c): 3/4. The wavelengths in the rest frame are not calculated, and there is also the same mistake as in Solution 2 for the energy of each photon in the rest frame.

Overall feedback: The approach taken in this solution is correct, but there are a few errors in calculations and explanations.

\subsection*{Question 8:}

\textbf{Response 1:}

**Solution 1: 9/10**

Part a): 9/9 
* **Correctness:** The solution is completely correct and well-structured, with clear explanations of each step.
* **Presentation:** The use of Python for algebraic manipulations is a good idea. However, the code is not shown, making it hard to fully assess the method's validity. It would be better to present the crucial steps of the calculation, even if done with computer assistance.

Part b): 0/1
* **Incorrectness:** While the correct final equation is reached, the step to get there is incorrect. The simplification to remove the \( v \) and \( A \) terms is not justified, as they should still contribute even in the small-angle approximation. The correct simplification would be to note that those terms vanish as \( v, A \) approach 0.

**Solution 2: 8/10**

Part a): 8/9
* **Correctness:** The solution is almost correct, but there is a minor error in the kinetic energy expression. The radial kinetic energy should only depend on the velocity of the support, not on \( \dot{\phi} \). The rest of the derivation follows logically.
* **Presentation:** The steps are clearly explained, and the notation is consistent. The presentation could be improved slightly by explicitly stating the reasons for the vanishing of the radial kinetic energy term in part b).

Part b): 1/1
* **Correctness:** The simplification to the classic pendulum equation is correct and well-justified.

**Solution 3: 9/10**

Part a): 9/9
* **Correctness:** The solution is entirely correct, providing a clear derivation of the Lagrangian and the equation of motion.
* **Presentation:** The presentation is very good. The steps are laid out logically, and the notation is clear and consistent. The use of the dot notation for time derivatives is standard and easy to follow.

Part b): 0/1
* **Incorrectness:** The student correctly simplifies the equation of motion in the limit of small angles but makes a mistake when they try to simplify it further. The equation of motion should not have a minus sign in front of the \( g \phi \) term.

**Overall Feedback:** All three solutions demonstrate a good understanding of Lagrangian mechanics and the physics of the pendulum problem. The main areas for improvement are the presentation of computational steps in solution 1 and the minor errors in the simplification steps of solutions 2 and 3. 

\bigskip    
\textbf{Response 2:}

Solution 1: 8/10

*   **Part a):** 8/9. Excellent work deriving the Lagrangian and using the Euler-Lagrange equation.
*   **Part b):** 1/1. Correctly identified the small-angle approximation and the classic pendulum equation.

General Feedback: Very well explained and structured. The use of Python for algebraic manipulation is impressive and shows a good understanding of computational tools in physics. The final equation of motion is correct.

Solution 2: 7/10

*   **Part a):** 7/9. Good attempt at deriving the Lagrangian and using the Euler-Lagrange equation. There's a minor error in the calculation of the potential energy, which should be \( U = m g L (1 - \cos(\phi)) \), not \( m g (L - L \cos(\phi)) \).
*   **Part b):** 1/1. Correctly identified the small-angle approximation and the classic pendulum equation.

General Feedback:  The solution is well-structured and easy to follow. A more thorough explanation of why the radial kinetic energy term vanishes in the limit \( v, A \rightarrow 0 \) would be beneficial.

Solution 3: 6/10

*   **Part a):** 6/9. A good start in deriving the Lagrangian and using the Euler-Lagrange equation. The potential energy expression is correct. However, there's an error in the final equation of motion; it should not have the negative sign in front of the \( g \sin(\phi) \) term.
*   **Part b):** 1/1. Correctly identified the small-angle approximation and the classic pendulum equation.

General Feedback:  The explanation of the kinetic and potential energy components is clear. More care should be taken in the algebraic manipulation to avoid sign errors.

\bigskip    
\textbf{Response 3:}

**Solution 1: (9/10)**

**Part a) (9/9)**

Excellent work! This solution is very detailed and clearly explains the steps involved in deriving the equation of motion. The use of Python for algebraic manipulation is clever and shows good initiative in using tools to simplify the process.

**Part b) (0/1)**

Although the correct final equation is reached, there is an error in the simplification process. The equation should be $ L\frac{d^2\phi}{dt^2} + g\phi(t) = 0 $, not $ \frac{d^2\phi}{dt^2} + \frac{g}{L}\phi(t) = 0 $.

**General Feedback:**

Overall, this is a very well-written and comprehensive solution. The steps are clearly explained, and the use of Python is a nice touch. The only minor error is in the simplification in part b).



**Solution 2: (8/10)**

**Part a) (8/9)**

This solution is well-structured and demonstrates a good understanding of Lagrangian mechanics. The steps are clear and easy to follow. However, it is missing the inclusion of the terms related to the support's motion, $ x(t) = vt^3 + A\sin(\omega t) $, in the Lagrangian. This omission leads to an incorrect equation of motion.

**Part b) (1/1)**

The small-angle approximation and the limiting case are correctly applied to reach the classic pendulum equation.

**General Feedback:**

This is a good solution, but it would be improved by including the terms related to the support's motion in the Lagrangian. This would lead to the correct equation of motion, which would then simplify to the classic pendulum equation in part b).



**Solution 3: (8/10)**

**Part a) (8/9)**

This solution is clear and concise, demonstrating a good grasp of Lagrangian mechanics. The derivation of the equation of motion is well-explained. However, there is a minor error in the calculation of the potential energy. The correct expression for the potential energy is \( U = mgL(1 - \cos(\phi)) \), not \( U = mgL\cos(\phi) \). This error propagates through the rest of the derivation, leading to an incorrect equation of motion.

**Part b) (1/1)**

The small-angle approximation and the limiting case are correctly applied, even with the error from part a), to reach the classic pendulum equation.

**General Feedback:**

Overall, this is a good solution, but it is slightly marred by the error in the potential energy calculation. If this error were corrected, the solution would be excellent.



**Important Note:** In all three solutions, the correct equation of motion for the pendulum is:

\[ L\frac{d^2\phi}{dt^2} + g\sin(\phi(t)) = A\omega^2\sin(\omega t)\cos(\phi(t)) - 6tv\cos(\phi(t)) \]

\bigskip    
\textbf{Response 4:}

Solution 1:
**Grade:**
a) 9
b) 1
**General Feedback:** 
The solution is well-structured and very detailed, demonstrating a deep understanding of Lagrangian mechanics. It's great that the student used Python to help with algebraic manipulation, which is often necessary in complex physics problems. However, in an exam setting, where external tools might not be available, showing more of the intermediate steps for the partial derivatives and the time derivative would have been beneficial. Overall, the solution is excellent.

Solution 2: 
**Grade:**
a) 9
b) 1
**General Feedback:** 
The solution is very clear and concise. It systematically outlines the problem, starting with the kinetic and potential energy components, correctly deriving the Lagrangian, and then applying the Euler-Lagrange equation. The simplification to the classic pendulum problem in the limit of small angles is done correctly. However, it is worth noting that the solution assumes a simple pendulum in which the tension on the string only acts as a constraint force, this is not strictly true in this problem. Overall the solution is excellent.

Solution 3: 
**Grade:**
a) 8
b) 1
**General Feedback:**
The solution is well-structured and detailed. The student clearly understands the Lagrangian mechanics framework and applies it correctly to the problem. However, there is a misunderstanding in the kinetic energy term as this is the kinetic energy of a particle moving in a plane not a pendulum. The solution could be improved by explicitly stating the small-angle approximation used to simplify the equation of motion. But overall the solution is very good.

\bigskip    
\textbf{Response 5:}

Solution 1:
Part a): 9 marks
Part b): 1 mark

Overall: This is an excellent solution, demonstrating a thorough understanding of Lagrangian mechanics and the ability to apply it to a complex problem. The derivation of the Lagrangian, the use of the Euler-Lagrange equation, and the final simplification to the equation of motion are all correct. The steps taken are clear and well-explained. There are a few minor issues. The solution mentioned using Python, but there was no code to show. The solution also stated that the vertical position of the mass was negative, but downwards being the positive direction would have simplified the solution.

Solution 2:
Part a): 8 marks
Part b): 1 mark

Overall: This solution is generally well-written and demonstrates a good understanding of Lagrangian mechanics. The derivation of the kinetic and potential energy is clear, and the application of the Euler-Lagrange equation is correct. However, there's a small error in the equation of motion. The student wrote \( \ddot{\phi} - \frac{g}{L} \sin(\phi) = 0 \) but it should be \( \ddot{\phi} + \frac{g}{L} \sin(\phi) = 0 \). This error, though minor, affects the final result and indicates a slight misunderstanding of the sign conventions in the Euler-Lagrange equation. The simplification of the equation of motion under the small-angle approximation is correct.

Solution 3:
Part a): 8 marks
Part b): 1 mark

Overall: This solution is well-presented and demonstrates a good grasp of the concepts of Lagrangian mechanics. The derivation of the Lagrangian is correct, and the use of the Euler-Lagrange equation is appropriate. The final equation of motion derived is accurate, and the simplification under the small-angle approximation is also correct. However, there are a few issues. The potential energy was incorrect as the mass is at \(y = -L \cos(\phi) \) rather than \(y = L \cos(\phi) \). This would not have affected the final result as only the derivative of the potential energy is needed. The student also stated the problem involves motion with constraints but the pendulum is not constrained in any way in this problem.

\subsection*{Question 9:}

\textbf{Response 1:}

**Solution 1:**
*   Part a) - 6 marks
*   Part b) - 1 mark

This is a good solution that shows a comprehensive understanding of how to solve the given physics problem step by step. The solution is well-organized and easy to follow, with clear explanations for each step of the derivation and calculation.

**Solution 2:**
*   Part a) - 6 marks
*   Part b) - 1 mark

This is another solid solution that demonstrates a good understanding of the underlying concepts. The equations are correctly applied and the variables are clearly defined. However, there is a calculation error in part b) as the value for alpha should be 16.33 $rad/s^2$ and not 16.35 $rad/s^2$.

**Solution 3:**
*   Part a) - 6 marks
*   Part b) - 1 mark

This solution is well-written and organized. The approach is correct, and the derivations are detailed and clear. However, there is a calculation error in part b) as the value for alpha should be 16.33 $rad/s^2$ and not 16.35 $rad/s^2$.

\bigskip    
\textbf{Response 2:}

**Solution 1: 6 marks (Part a), 1 mark (Part b).**

The solution is well-structured and detailed, displaying a good understanding of Newton's Laws. Each step of the derivations is clear. However, there's a minor issue with the formatting of variables in some equations. Also, in Part b, the values should be plugged into the derived expressions to show the explicit calculations before stating the final results.

**Solution 2: 6 marks (Part a), 1 mark (Part b).**

This solution is correct and concisely presented. It demonstrates a solid grasp of the problem's physics. The approach of combining equations to solve for unknowns is efficient. It is well-organized and the steps are logical. A very minor improvement could be to provide slightly more explanation for the step where \( T \) is eliminated. Overall, it's a well-done solution.

**Solution 3: 6 marks (Part a), 1 mark (Part b).**

This solution is very thorough and well-explained. It provides step-by-step derivations with clear explanations for each equation. The logical flow is easy to follow, and the presentation is neat. The solution correctly applies Newton's laws and demonstrates a good understanding of the relationship between linear and rotational motion. The numerical calculations in part b) are accurate and complete. Overall, it is an excellent solution.

\bigskip    
\textbf{Response 3:}

**Solution 1:**
*   Part a) - 6 marks
*   Part b) - 1 mark

This is a well-structured and correct solution. The steps are clearly presented and the calculations are accurate. 

**Solution 2:**
*   Part a) - 6 marks
*   Part b) - 1 mark

This is another correct solution with clear explanations. There is a slight inaccuracy in the calculation for \( g \). The value of \( g \) is approximately 9.8 m/s\(^2\), not exactly 9.8 m/s\(^2\), but this is minor since the question asked for approximate values in part b).

**Solution 3:**
*   Part a) - 6 marks
*   Part b) - 1 mark

This solution is correct and easy to follow. It demonstrates a clear understanding of the underlying physics concepts and applies them accurately to the problem.

**Overall feedback:**

All three solutions show a good grasp of the physics principles involved in this problem. The students have correctly applied Newton's laws of motion to both linear and rotational motion and have used the relationship between linear and angular acceleration to solve for the unknowns. The derivations are clear and the calculations are accurate.

If I had to choose the best solution, I would slightly favor Solution 1 because of its exceptionally clear structure and step-by-step approach, making it easy for anyone to understand the process. However, all three solutions are excellent and the differences are minor.

\bigskip    
\textbf{Response 4:}

**Solution 1:**
*   Part a): 6/6 marks
*   Part b): 1/1 mark
*   Overall feedback: This solution is well-organized and easy to follow, demonstrating a clear understanding of the underlying physics concepts. The derivations are thorough and the explanations are clear. The calculations in part b) are accurate, however, it would be beneficial to show the numerical steps in the calculation.

**Solution 2:**
*   Part a): 6/6 marks
*   Part b): 1/1 mark
*   Overall feedback: This is a good solution with a clear, concise approach. The use of symbols throughout the derivation is helpful, and the transition to numerical values in part b) is smooth. The final answers are correct. The solution is well-explained and demonstrates a solid understanding of the physics involved.

**Solution 3:**
*   Part a): 6/6 marks
*   Part b): 1/1 mark
*   Overall feedback: This solution is correct and well-structured. The derivations are well-explained, and the use of equations is appropriate. The calculations in part b) are accurate and clearly presented. The solution shows a good grasp of the problem's requirements and a solid understanding of the relevant physics principles.

\bigskip    
\textbf{Response 5:}

**Solution 1: 6 marks out of 7**
* **Part a): 6 marks out of 6**
* **Part b): 0 marks out of 1**

This solution is well-organized and the derivation of the expressions in part a) is clear. However, the numerical calculations in part b) were not completed.

**Solution 2: 6 marks out of 7**
* **Part a): 6 marks out of 6**
* **Part b): 0 marks out of 1**

This solution is correct, but the notation could be improved. For instance, the symbol \( \tau \) is typically used for torque, not \( T \).

**Solution 3: 7 marks out of 7**
* **Part a): 6 marks out of 6**
* **Part b): 1 mark out of 1**

This solution is excellent, providing a clear and detailed explanation of each step, including the derivation of the formulas and the numerical calculations. The only minor issue is that the angular acceleration is rounded to 16.35 rad/s\(^2\) instead of 16.33 rad/s\(^2\), but this does not significantly detract from the quality of the solution.

\subsection*{Question 10:}

\textbf{Response 1:}

Solution 1: 
* **Part a):** 6/6. All steps in the solution are correct and explained thoroughly.
* **Part b):** 2/2. Correctly applies initial conditions.

* **General feedback:** Excellent solution. Clear, thorough, and correct.

Solution 2: 
* **Part a):** 5/6. The vertical component is correctly derived, but it should be noted that the horizontal components would undergo a similar derivation without the gravitational force.
* **Part b):** 2/2. Correctly applies initial conditions.

* **General feedback:** Overall, a good solution. The solution would benefit from addressing all components of the motion.

Solution 3: 
* **Part a):** 6/6. Correctly derives and verifies the differential equation solution.
* **Part b):** 2/2. Correctly applies initial conditions.

* **General feedback:** Very good solution with thorough explanation.

\bigskip    
\textbf{Response 2:}

**Solution 1: 7/8**

*   **Part a): 6/6**
*   **Part b): 1/2**

This is a well-structured solution that clearly demonstrates an understanding of the physics involved. The differential equation is correctly derived and solved. In part b), the student has made a slight error in the interpretation of the initial conditions leading to an incorrect expression for \(\underline{A}\). The vector \(\underline{A}\) should be given by \( \underline{A} = \underline{u} + \frac{mg}{b}\hat{\underline{k}} \).

**Solution 2: 8/8**

*   **Part a): 6/6**
*   **Part b): 2/2**

This is an excellent and thorough solution that provides a step-by-step approach to the problem. The student's explanation of the differential equation and its solution is clear and concise. Both parts of the problem are answered correctly, with all relevant details included.

**Solution 3: 8/8**

*   **Part a): 6/6**
*   **Part b): 2/2**

This solution provides a comprehensive and well-organized approach to solving the problem. The student correctly applies Newton's second law and solves the differential equation. The initial conditions are used correctly to find the constant vector \(\underline{A}\). This is an excellent solution with all the required steps and explanations.

\bigskip    
\textbf{Response 3:}

**Solution 1: (6/6 for a, 2/2 for b)**
Excellent work! Your solution is clear, concise, and correct.

**Solution 2: (6/6 for a, 2/2 for b)**
Great job! Your solution is thorough and accurate. You've included the step of separating the components of the velocity vector, which adds clarity to the derivation of the differential equation.

**Solution 3: (6/6 for a, 2/2 for b)**
Well done! This is a solid solution. The step-by-step approach and the verification of the solution are commendable.

**General Feedback:**

All three solutions demonstrate a good understanding of Newton's laws of motion and the ability to solve a linear first-order differential equation. The presentation is clear and the steps are logical, making the solutions easy to follow. 

\bigskip    
\textbf{Response 4:}

Solution 1:
Part a): 6/6. This solution is complete and thorough, demonstrating a strong understanding of how to solve the first-order differential equation in vector form. 

Part b): 2/2. Correct derivation for A.

Overall feedback: Solution is complete and well-explained.

Solution 2:
Part a): 6/6. This solution provides a clear and concise derivation of the equation of motion and its solution. There is a good understanding of the separation of variables for differential equations.

Part b): 2/2. Correct derivation for A.

Overall feedback: Solution is well-structured and easy to follow.

Solution 3:
Part a): 6/6. This solution correctly applies Newton's second law, solves the differential equation, and explains the steps clearly.

Part b): 2/2. Correct derivation for A.

Overall feedback: Solution is concise and accurate. Some minor improvements could be made to the solution's explanation by discussing the physical meaning of the terminal velocity term.

\bigskip    
\textbf{Response 5:}

**Solution 1: (6/6, 2/2)**

This solution is very well written and the explanation is easy to follow. Part (a) has enough detail and it is clear how the general form of the differential equation's solution is reached. Part (b) is also clearly explained.

**Solution 2: (6/6, 2/2)**

This is a good solution. The explanations are clear and the logic behind the equations used is evident. The notation could be improved, for example, the hat notation should be used on the unit vector in \( \hat{\underline{k}} \).

**Solution 3: (6/6, 2/2)**

This solution shows all the steps in a clear and concise manner. The justification of the guess for the solution to the differential equation could be explained in more depth.

\end{document}