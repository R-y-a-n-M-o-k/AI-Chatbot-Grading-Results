\documentclass[a4paper,11pt]{article}
\setlength\parindent{0pt}
\usepackage{amsmath, amssymb}
\usepackage[margin=1in,letterpaper]
{geometry} % decreases margins

\usepackage{setspace}
\setstretch{1.2}
\usepackage{spverbatim}



\title{Gemini 1.5 Pro Grading (With Mark Scheme)}

\begin{document}

\maketitle
This document compiles the Gemini 1.5 Pro responses when prompted to grade the solutions which have been generated. The markscheme is given to Gemini 1.5 Pro in the prompt. This is the grading for the Classical Mechanics (CM) questions. 

\section{CM}
\subsection*{Question 1:}

\textbf{Response 1:}

**Solution 1: 4 marks**

* This solution provides a clear and accurate definition of a conservative force, emphasizing the path independence of work done.
* It correctly defines potential energy using the gradient relationship with force.
* It explains why potential energy cannot be defined for non-conservative forces due to path dependence and energy dissipation.
* The explanation is concise and covers all the key points in the mark scheme.

**Solution 2: 4 marks**

* This solution also provides a clear and accurate definition of a conservative force, focusing on the conservation of mechanical energy.
* It correctly defines the change in potential energy as the negative of the work done by the force.
* It explains the relationship between force and potential energy using the gradient.
* It explains the path dependence of non-conservative forces and the resulting inability to define a unique potential energy function.
* While the solution is slightly longer than Solution 1, it still covers all the necessary points in the mark scheme.

**Solution 3: 4 marks**

* This solution provides a correct definition of a conservative force and its relation to mechanical energy conservation.
* It presents a clear mathematical derivation of potential energy based on work done and integration.
* It correctly states the condition of zero curl for a force to be conservative.
* It explains the path dependence of non-conservative forces and how it prevents the definition of a unique potential energy function.
* While the solution includes some additional details like the differential relationship and the curl condition, it still addresses all the key points in the mark scheme.

**General Feedback:**

All three solutions are well-written and demonstrate a good understanding of conservative forces and potential energy. They all accurately define the concepts, explain the mathematical relationship between force and potential energy, and address why potential energy cannot be defined for non-conservative forces. The differences between the solutions are mainly in the style of presentation and the level of detail provided. None of the solutions require any deduction of marks based on the given mark scheme.

\bigskip    
\textbf{Response 2:}

**Solution 1: 3 marks**

* This solution provides a clear and accurate definition of a conservative force, emphasizing the path independence of work done.
* It correctly defines potential energy using the gradient relationship with force.
* It explains why potential energy cannot be defined for non-conservative forces due to path dependence and energy dissipation.
* While the solution is correct, it does not explicitly state that the choice of origin point for the potential energy is arbitrary, as required by the mark scheme.

**Solution 2: 3 marks**

* This solution also provides a clear and accurate definition of a conservative force, focusing on the conservation of mechanical energy.
* It correctly defines the change in potential energy as the negative of the work done by the force.
* It explains the relationship between force and potential energy using the gradient.
* It explains the path dependence of non-conservative forces and the resulting inability to define a unique potential energy function.
* While the solution is correct, it also does not explicitly state that the choice of origin point for the potential energy is arbitrary, as required by the mark scheme.

**Solution 3: 4 marks**

* This solution provides a correct definition of a conservative force and its relation to mechanical energy conservation.
* It presents a clear mathematical derivation of potential energy based on work done and integration, including the arbitrary origin point.
* It correctly states the condition of zero curl for a force to be conservative.
* It explains the path dependence of non-conservative forces and how it prevents the definition of a unique potential energy function.
* This solution covers all the key points in the mark scheme, including the often-missed arbitrary origin point in the potential energy definition.

**General Feedback:**

All three solutions are well-written and demonstrate a good understanding of conservative forces and potential energy. They all accurately define the concepts, explain the mathematical relationship between force and potential energy, and address why potential energy cannot be defined for non-conservative forces. Solutions 1 and 2 lose a mark for not explicitly mentioning the arbitrary choice of origin point in the definition of potential energy. Solution 3 is the most complete and accurate answer, earning full marks.

\bigskip    
\textbf{Response 3:}

**Solution 1: 3 marks**

The answer is well-written and accurately defines a conservative force in terms of path independence. The definition of potential energy in terms of the gradient is also correct. However, it loses a mark due to not explicitly mentioning the arbitrary nature of the reference point in the potential energy definition, which is a key point in the mark scheme.

**Solution 2: 2 marks**

This response correctly identifies conservative forces as those with path-independent work and provides examples. It also explains the relationship between work and potential energy. However, it loses marks due to:

* **Incorrect mathematical definition of potential energy:**  The potential energy is not simply the negative of the work done; it's the negative of the work done by the conservative force in moving an object from an arbitrary reference point to a given position.
* **Not mentioning the arbitrary reference point:** The mark scheme explicitly requires mentioning that the choice of reference point is arbitrary.

**Solution 3: 4 marks**

This is the most comprehensive and accurate answer. It correctly defines conservative forces, provides examples, and explains the connection to mechanical energy conservation. It also presents a clear mathematical derivation of potential energy based on work done and integration, emphasizing the arbitrary choice of reference point. It further explains the condition of zero curl for a force to be conservative, and why this condition fails for non-conservative forces. This solution covers all the key points in the mark scheme and earns full marks.

\textbf{Response 4:}

**Solution 1: 4/4 Marks**

This is an excellent answer. It covers all the main points in a clear and concise manner, demonstrating a strong understanding of the topic. The mathematical definition of potential energy is accurate and relevant. The explanation of why a potential energy cannot be defined for a non-conservative force is also well-articulated.

**Solution 2: 4/4 Marks**

This solution is also very good. It provides a comprehensive explanation of conservative forces and their associated potential energies. The student has correctly explained the mathematical relationship between force and potential energy. The reasoning behind why non-conservative forces cannot have a potential energy is also well-stated.

**Solution 3: 4/4 Marks**

This solution is thorough and demonstrates a good grasp of the concept of conservative forces. The mathematical derivation of potential energy is accurate and the explanation is clear. The student correctly identifies the curl of the force being zero as a necessary condition for a force to be conservative. The explanation of why a potential energy cannot be defined for non-conservative forces is also well-explained.

\textbf{Response 5:}

**Solution 1: 3/4 Marks**

This solution demonstrates a good understanding of conservative forces and potential energy. The definitions are accurate and the explanation is clear. However, the mathematical definition of potential energy is missing a crucial detail: the arbitrary nature of the reference point. 

**Solution 2: 4/4 Marks**

This is an excellent solution. It provides a comprehensive and accurate explanation of conservative forces and their associated potential energies. The student has correctly explained the mathematical relationship between force and potential energy, including the arbitrary nature of the reference point. The reasoning behind why non-conservative forces cannot have a potential energy is also well-stated.

**Solution 3: 3/4 Marks**

This solution is thorough and demonstrates a good grasp of the concept of conservative forces. The explanation of why a potential energy cannot be defined for non-conservative forces is also well-explained. However, the mathematical definition of potential energy, while technically correct, would benefit from being more explicit about the arbitrary nature of the reference point. The discussion of the curl of the force could also be made more relevant to the definition of potential energy. 

\subsection*{Question 2:}

\textbf{Response 1:}

**Solution 1: 6 marks**

This solution is well-articulated and accurately covers all three cases of damping in the damped harmonic oscillator. It correctly states the conditions for each case (underdamped, critically damped, and overdamped), describes the behavior of the system in each case, and provides explanations for why the system behaves as it does. The solution also includes the relevant mathematical condition, \(b^2 < 4mk\), etc., for each case.

**Solution 2: 6 marks**

Similar to Solution 1, this solution correctly identifies the three cases of damping and provides accurate descriptions and explanations for each case. It also includes the relevant mathematical condition, \(b^2 - 4mk\), and explains the physical meaning of the damping term \(b\frac{dx}{dt}\). This solution also deserves full marks.

**Solution 3: 5 marks**

This solution also correctly identifies the three cases of damping and their corresponding mathematical conditions. It provides accurate descriptions of the behavior in each case. However, the explanation for the overdamped case could be more precise. It states that the system returns to equilibrium more slowly than in the critically damped case, which is correct. However, it doesn't explicitly state that the return is still exponential, just with a slower decay rate. This minor imprecision results in a deduction of one mark.

**General Feedback**

All three solutions demonstrate a good understanding of the damped harmonic oscillator and its different damping regimes. Solutions 1 and 2 are particularly well-written and comprehensive, deserving full marks. Solution 3 is also very good but could be slightly improved by clarifying the behavior in the overdamped case. Overall, all three students have shown a strong grasp of the topic.

\bigskip    
\textbf{Response 2:}

**Solution 1: 6 marks**
This solution provides a comprehensive and accurate description of the three cases of damping in a damped harmonic oscillator. It correctly states the conditions for each case in terms of the relationship between the damping coefficient, mass, and spring constant. The descriptions of the system's behavior in each case are also accurate and clear. The solution demonstrates a solid understanding of the concept and deserves full marks.

**Solution 2: 6 marks**
This solution is equally comprehensive and accurate as Solution 1. It correctly identifies the three cases of damping using the discriminant of the characteristic equation and provides clear explanations of the behavior in each case. The solution also includes relevant examples (automotive shock absorbers) and explains the physical meaning of the damping term, further demonstrating a thorough understanding of the topic. Full marks are deserved.

**Solution 3: 5 marks**
This solution also correctly identifies the three cases of damping and their corresponding mathematical conditions. The descriptions of the underdamped and critically damped cases are accurate. However, the description of the overdamped case could be more precise. While it correctly states that the system does not oscillate, it fails to mention that the return to equilibrium is still an exponential decay, albeit a slower one compared to the critically damped case. This minor imprecision results in a deduction of one mark.

**General Feedback**
All three solutions demonstrate a good understanding of the damped harmonic oscillator and its different damping regimes. Solutions 1 and 2 are particularly well-written and comprehensive, earning full marks. Solution 3 is also very good but could be slightly improved by clarifying the behavior in the overdamped case. Overall, all three students have shown a strong grasp of the topic.

\bigskip    
\textbf{Response 3:}

**Solution 1: 6 marks**

This solution is well-articulated and accurately covers all three cases of damping in the damped harmonic oscillator. It correctly states the conditions for each case in terms of the relationship between the damping coefficient, mass, and spring constant. The descriptions of the system's behavior in each case are also accurate and clear. The solution demonstrates a solid understanding of the concept and deserves full marks.

**Solution 2: 6 marks**

Similar to Solution 1, this solution correctly identifies the three cases of damping and provides accurate descriptions and explanations for each case. It also includes the relevant mathematical condition, \(b^2 - 4mk\), and explains the physical meaning of the damping term \(b\frac{dx}{dt}\). This solution also deserves full marks.

**Solution 3: 5 marks**

This solution also correctly identifies the three cases of damping and their corresponding mathematical conditions. The descriptions of the underdamped and critically damped cases are accurate. However, there's a minor imprecision in the overdamped case. While it's correct that the system returns to equilibrium more slowly than in the critically damped case, it fails to explicitly state that the return is still exponential, just with a slower decay rate. This minor imprecision results in a deduction of one mark.

**General Feedback**

All three solutions demonstrate a good understanding of the damped harmonic oscillator and its different damping regimes. Solutions 1 and 2 are particularly well-written and comprehensive, earning full marks. Solution 3 is also very good but could be slightly improved by clarifying the behavior in the overdamped case. Overall, all three students have shown a strong grasp of the topic.

\textbf{Response 4:}

**Solution 1: 6/6 Marks**

This is an excellent answer. It covers all the main points in a clear and concise manner, demonstrating a strong understanding of the topic. The three cases are correctly identified, and the qualitative descriptions of each case are accurate and detailed. 

**Solution 2: 6/6 Marks**

This solution is also very good. It provides a comprehensive explanation of the damped harmonic oscillator and the three cases of damping. The student has correctly identified the discriminant as the key factor in determining the type of damping and has provided accurate qualitative descriptions of each case.

**Solution 3: 6/6 Marks**

This solution is thorough and demonstrates a good grasp of the concept of damping in harmonic oscillators. The student correctly identifies the three cases of damping and provides accurate descriptions of each case. The explanation of the discriminant and its role in determining the type of damping is also well-presented.

\textbf{Response 5:}

**Solution 1: 5/6 Marks**

This is a good answer that demonstrates a solid understanding of the damped harmonic oscillator. The three cases are correctly identified, and the qualitative descriptions are accurate. However, one mark is deducted for the lack of explicit mention that the critically damped case represents the *fastest* return to equilibrium.

**Solution 2: 6/6 Marks**

This is an excellent solution. It provides a comprehensive and accurate explanation of the damped harmonic oscillator and the three cases of damping. The student has correctly identified the discriminant as the key factor in determining the type of damping and has provided accurate and detailed qualitative descriptions of each case.

**Solution 3: 5/6 Marks**

This solution is thorough and demonstrates a good grasp of the concept of damping in harmonic oscillators. The student correctly identifies the three cases of damping and provides accurate descriptions of each case. However, one mark is deducted for the omission of the fact that in the overdamped case, the system returns to equilibrium more slowly than in the critically damped case.

\subsection*{Question 3:}

\textbf{Response 1:}

**Solution 1: 6 marks**

This solution is well-organized and comprehensive. It correctly derives the position vector of the center of mass in part a) and clearly explains the relationship between the masses and the distances of the center of mass to each mass in part b). The calculations are accurate and well-presented. Full marks are deserved.

**Solution 2: 6 marks**

This solution is also very good and demonstrates a clear understanding of the concept. It correctly derives the value of  'k' in part a) and explains the proportional relationship between the distances of the center of mass to the masses in part b). The solution is well-structured and easy to follow, and the calculations are accurate. Full marks are deserved.

**Solution 3: 6 marks**

This solution is equally comprehensive and accurate as the previous two. It provides a clear derivation of the position vector of the center of mass and a detailed explanation of the relationship between the masses, distances, and the position of the center of mass. The calculations are precise and well-presented. This solution also deserves full marks.

**General Feedback**

All three solutions are excellent and demonstrate a thorough understanding of the concept of center of mass. They correctly derive the position vector of the center of mass and show that it lies on the line connecting the two masses with appropriate distances from each mass. The solutions are well-organized, easy to follow, and mathematically accurate. All three students have shown a strong grasp of the topic and deserve full marks.

\bigskip    
\textbf{Response 2:}

**Solution 1: (a) 2 marks, (b) 4 marks**

This solution is well-organized and comprehensive. In part (a), the student correctly starts with the general definition of the center of mass and then specializes it for the two-object system. The algebraic manipulation to obtain the desired form is clear and correct. In part (b), the student correctly calculates the distances from the center of mass to each object, demonstrating that the center of mass lies on the line connecting the two masses.

**Solution 2: (a) 2 marks, (b) 4 marks**

This solution is also well-organized and comprehensive. In part (a), the student starts with the correct definition of the center of mass for two objects and correctly manipulates it to the desired form. In part (b), the student provides a clear explanation of why the center of mass divides the distance between the two masses in the ratio of their masses, and correctly calculates the distances.

**Solution 3: (a) 2 marks, (b) 4 marks**

This solution is well-written and accurate. In part (a), the student correctly identifies the coefficients of the position vectors in the desired form and equates them to the corresponding coefficients in the definition of the center of mass. In part (b), the student correctly calculates the distances from the center of mass to each object, confirming the given result.

**General Feedback**

All three solutions are excellent and demonstrate a thorough understanding of the concept of center of mass. They correctly derive the position vector of the center of mass and show that it lies on the line connecting the two masses with the appropriate distances from each mass. The solutions are well-organized, easy to follow, and mathematically accurate. All three students deserve full marks.

\bigskip    
\textbf{Response 3:}

**Solution 1: (a) 2 marks, (b) 4 marks**

This solution is well-organized and comprehensive. It correctly derives the position vector of the center of mass in part (a), showing the steps clearly. In part (b), the student correctly uses the derived value of 'k' to calculate the distances from the center of mass to each mass, and the reasoning about why the center of mass lies on the line connecting the masses is sound. The presentation is clear and easy to follow.

**Solution 2: (a) 2 marks, (b) 4 marks**

This solution is also well-organized and comprehensive. It correctly derives the value of 'k' in part (a) by equating coefficients and provides a clear explanation of why the center of mass divides the distance between the two masses in the ratio of their masses. The calculations in part (b) are correct and well-presented. The student's explanation of the proportionality of distances to masses is accurate.

**Solution 3: (a) 2 marks, (b) 3 marks**

This solution is well-written and accurate in part (a), correctly deriving the value of 'k'. However, in part (b), the student makes an error in stating that the distances from the center of mass to the masses are proportional to the masses of the other object. This should be inversely proportional. This error leads to a deduction of 1 mark. The rest of the solution is correct and well-presented.

**General Feedback**

All three solutions demonstrate a good understanding of the concept of the center of mass. They correctly derive the position vector of the center of mass and show that it lies on the line connecting the two masses. Solutions 1 and 2 are particularly well-explained and provide a deeper understanding of the relationship between masses and distances. Solution 3 is also well-written, but it would benefit from correcting the minor error in the interpretation of proportionality in part (b).

\textbf{Response 4:}

**Solution 1: 6/6 Marks**

This is an excellent solution. The student has clearly and correctly derived the expression for the center of mass and shown that it lies on the line connecting the two masses. The distances from the center of mass to each mass have also been calculated correctly.

**Solution 2: 6/6 Marks**

This solution is also excellent. The student has demonstrated a good understanding of the concept of center of mass and has used it to solve the problem correctly. The explanation is clear and concise, and the mathematical derivations are accurate.

**Solution 3: 6/6 Marks**

This solution is thorough and demonstrates a strong grasp of the problem. The student has correctly derived the expression for the center of mass and shown that it lies on the line connecting the two masses. The distances from the center of mass to each mass have also been calculated correctly. The explanation is well-structured and easy to follow. 

\textbf{Response 5:}

**Solution 1: 6/6 Marks**

This solution is excellent. Both parts a) and b) are correctly solved, with clear explanations and accurate mathematical derivations. The student demonstrates a strong understanding of the concept of center of mass and its application to this problem.

**Solution 2: 6/6 Marks**

This solution is also excellent. The student has provided a comprehensive and accurate explanation of the problem, including the derivation of the position vector of the center of mass and the calculation of the distances from the center of mass to each mass. The reasoning is clear and the mathematical steps are well-explained.

**Solution 3: 6/6 Marks**

This solution is equally excellent. It is thorough and well-structured, with clear explanations and accurate mathematical derivations. The student has demonstrated a strong understanding of the problem and has presented a complete and accurate solution. The use of bullet points in part a) makes the steps easy to follow. 

**Overall Feedback:**
All three solutions are of high quality and demonstrate a good understanding of the concept of center of mass. They are well-written, with clear explanations and accurate mathematical derivations. There is no significant difference in quality among the three solutions, and all deserve full marks.

\subsection*{Question 4:}

\textbf{Response 1:}

**Solution 1: (a) 6 marks, (b) 1 mark, (c) 4 marks**

This solution is well-organized and comprehensive in its treatment of angular momentum conservation and kinetic energy. However, there's an error in part (b), where the angular acceleration component is incorrectly simplified to zero. Despite the minor sign discrepancy in part (c) that may be attributed to differences in interpretation, the overall solution is strong, demonstrating a solid understanding of the physics involved.

**Solution 2: (a) 6 marks, (b) 1 mark, (c) 4 marks**

This solution is also well-organized and demonstrates a clear understanding of angular momentum and kinetic energy. Similar to Solution 1, there's an error in part (b) where the angular acceleration component is incorrectly calculated. The calculation in part (c) accurately derives the expression for \( dK/dt \), though the sign difference should be acknowledged and attributed to the interpretation of energy changes in the system.

**Solution 3: (a) 6 marks, (b) 0 marks, (c) 0 marks**

This solution correctly derives the expression for angular frequency in part (a). However, there are significant errors in parts (b) and (c). The angular acceleration is incorrectly simplified to zero, and the calculation for the time derivative of kinetic energy is not presented. The solution shows potential, but the incomplete nature of parts (b) and (c) and the error in (b) prevent it from receiving full marks.

**General Feedback:**

All three solutions demonstrate a good understanding of angular momentum conservation and its application in polar coordinates. Solutions 1 and 2 are particularly strong, with accurate derivations and clear explanations. Solution 3 has a good start but needs more attention to detail and correct calculation of the angular acceleration component and the time derivative of kinetic energy.

\bigskip    
\textbf{Response 2:}

**Solution 1: (a) 6 marks, (b) 3 marks, (c) 6 marks**

This solution is well-organized and comprehensive in its treatment of angular momentum conservation and kinetic energy. In part (a), the student correctly derives the expression for angular frequency. Part (b) also demonstrates a correct approach, but the final simplification to zero is incorrect due to a missing negative sign. In part (c), the student correctly derives the expression for kinetic energy and its time derivative, although there's a minor sign discrepancy which can be attributed to interpretation. 

**Solution 2: (a) 6 marks, (b) 3 marks, (c) 6 marks**

Similar to Solution 1, this solution is well-organized and demonstrates a clear understanding of angular momentum and kinetic energy. The derivation in part (a) is accurate, and the expression for the angular frequency is correct. However, part (b) contains the same error in the simplification of angular acceleration. Part (c) accurately derives the expression for kinetic energy and its time derivative, with the same sign discrepancy as Solution 1.

**Solution 3: (a) 6 marks, (b) 0 marks, (c) 0 marks**

This solution correctly derives the expression for angular frequency in part (a), demonstrating a good understanding of angular momentum conservation. However, the angular acceleration calculation in part (b) is incorrect, and the student doesn't provide the derivation for the time derivative of kinetic energy in part (c). The solution shows potential, but the incomplete nature of parts (b) and (c) and the error in (b) prevent it from receiving full marks.

**General Feedback:**

All three solutions demonstrate a good understanding of angular momentum conservation and its application in polar coordinates. Solutions 1 and 2 are particularly strong, with accurate derivations and clear explanations. Solution 3 has a good start but needs more attention to detail and correct calculation of the angular acceleration component and the time derivative of kinetic energy.

\bigskip    
\textbf{Response 3:}

**Solution 1: (a) 6 marks, (b) 3 marks, (c) 6 marks**

This solution is well-structured and demonstrates a solid understanding of the physics involved. In part (a), the student correctly derives the expression for angular frequency using conservation of angular momentum. The derivation in part (c) is accurate and shows the correct steps to arrive at the expression for the rate of change of kinetic energy. However, there's an error in part (b), where the angular acceleration component is incorrectly simplified to zero, deducting 2 marks. The sign difference in the result of part (c) is likely due to a convention, and does not detract from the otherwise accurate solution.

**Solution 2: (a) 6 marks, (b) 3 marks, (c) 6 marks**

This solution is very similar to Solution 1 and also demonstrates a clear understanding of angular momentum and kinetic energy. The derivation in part (a) is accurate, and the expressions for angular frequency and kinetic energy are correct. However, part (b) contains the same error in the simplification of angular acceleration as Solution 1, resulting in a deduction of 2 marks. The sign difference in the result of part (c) is again attributed to a convention and does not affect the overall correctness of the solution.

**Solution 3: (a) 6 marks, (b) 0 marks, (c) 0 marks**

This solution correctly derives the expression for angular frequency in part (a), showing a good understanding of angular momentum conservation. However, there are significant errors in parts (b) and (c). The angular acceleration is incorrectly simplified to zero in part (b) and the derivation for the time derivative of kinetic energy in part (c) is not presented, deducting all marks for those sections. The solution shows potential, but the incomplete nature of parts (b) and (c) and the error in (b) prevent it from receiving full marks.

**General Feedback:**

All three solutions demonstrate a good understanding of angular momentum conservation and its application in polar coordinates. Solutions 1 and 2 are particularly strong, with accurate derivations and clear explanations, only requiring minor corrections. Solution 3 has a good start but needs more attention to detail and correct calculation of the angular acceleration component and the time derivative of kinetic energy.

\textbf{Response 4:}

**Solution 1: 16/18 Marks**

This is a very good solution.

*   **Part a):** 6/6. The student correctly derives the expression for the angular frequency.
*   **Part b):** 3/5. The student correctly identifies the angular component of acceleration but fails to calculate \(\ddot{\theta}\) and does not reach the correct answer (which is 0). 
*   **Part c):** 7/7. The student correctly derives the expression for the kinetic energy and its time derivative.

**Solution 2: 13/18 Marks**

This is a good solution, but with some errors.

*   **Part a):** 6/6. The student correctly derives the expression for the angular frequency.
*   **Part b):** 2/5. The student correctly identifies the angular component of acceleration, calculates \(\ddot{\theta}\) correctly, but makes a sign error in the final simplification.
*   **Part c):** 5/7. The student makes a sign error when calculating the time derivative of kinetic energy.

**Solution 3: 12/18 Marks**

This solution shows a good understanding of the problem but has some significant errors and omissions.

*   **Part a):** 6/6. The student correctly derives the expression for the angular frequency.
*   **Part b):** 0/5. The student does not correctly calculate the angular component of acceleration.
*   **Part c):** 6/7. The student correctly derives the expression for kinetic energy, but makes a sign error in the time derivative and does not address the sign discrepancy with the given expression. 

**Overall Feedback:**
All three students demonstrate a good understanding of polar coordinates and the concept of angular momentum. However, there are some calculation errors and oversights in parts b) and c) that need to be addressed.

\textbf{Response 5:}

**Solution 1: 16/18 Marks**

This is a very good solution. 

*   **Part a):** 6/6. The student correctly derives the expression for the angular frequency.
*   **Part b):** 3/5. The student correctly identifies the angular component of acceleration but fails to calculate \(\ddot{\theta}\) and does not reach the correct answer (which is 0). 
*   **Part c):** 7/7. The student correctly derives the expression for the kinetic energy and its time derivative, noting the sign discrepancy with the question. The justification of the correct sign is sound and the student demonstrates a deep understanding of the physics involved. 

**Solution 2: 15/18 Marks**

This is a good solution, but with some errors.

*   **Part a):** 6/6. The student correctly derives the expression for the angular frequency.
*   **Part b):** 4/5. The student correctly identifies the angular component of acceleration, calculates \(\ddot{\theta}\) correctly, but makes a sign error in the final simplification. The student is nearly there. 
*   **Part c):** 5/7. The student makes a sign error when calculating the time derivative of kinetic energy. The correct sign is justified but the student does not explicitly correct the prior calculation error.

**Solution 3: 12/18 Marks**

This solution shows a good understanding of the problem but has some significant errors and omissions.

*   **Part a):** 6/6. The student correctly derives the expression for the angular frequency.
*   **Part b):** 0/5. The student does not correctly calculate the angular component of acceleration, although they recognize their answer is likely incorrect.
*   **Part c):** 6/7. The student correctly derives the expression for kinetic energy, but makes a sign error in the time derivative, similar to the second student. However, they do not justify why the positive sign is physically correct. 

**Overall Feedback:**
All three students demonstrate a good understanding of polar coordinates and the concept of angular momentum. The first solution is the most complete and demonstrates the deepest understanding of the physics involved. The second solution has minor calculation errors, while the third solution has a more significant error in part b) and a similar sign error in part c) to the second solution. 

\subsection*{Question 5:}

\textbf{Response 1:}

**Solution 1: (a) 2 marks, (b) 4 marks, (c) 3 marks**
This solution is well-structured and provides a clear step-by-step approach. The student demonstrates a good understanding of Newton's second law, Hooke's law, and simple harmonic motion. All parts are answered correctly, and the derivations are presented clearly. 

**Solution 2: (a) 2 marks, (b) 4 marks, (c) 3 marks**
This solution is also well-explained and demonstrates a good understanding of the underlying concepts. The student correctly derives the equations of motion, uncouples them using new coordinates, and finds the solution for \(y_1\) by applying initial conditions. The steps are presented logically, and the final answer matches the given solution.

**Solution 3: (a) 2 marks, (b) 4 marks, (c) 2 marks**
This solution demonstrates a good understanding of Newton's second law and Hooke's law. The student correctly derives the equations of motion and uncouples them. However, in part (c), the student incorrectly states that \( \phi = \pi \) could be a solution, which is not consistent with the initial condition \(y_1(0) = 0\). This error results in a deduction of 1 mark.

**General Feedback**
All three solutions show a good grasp of the fundamental physics involved in the problem. Solutions 1 and 2 are particularly well-structured and provide clear explanations. Solution 3 is also well-written, but it needs a minor correction in part (c) to ensure consistency with the initial conditions. Overall, all three students have demonstrated a strong understanding of the topic.

\bigskip    
\textbf{Response 2:}

**Solution 1: (a) 2 marks, (b) 4 marks, (c) 3 marks**
This solution is well-structured and provides a clear step-by-step approach. The student demonstrates a good understanding of Newton's second law, Hooke's law, and simple harmonic motion. All parts are answered correctly, and the derivations are presented clearly. 

**Solution 2: (a) 2 marks, (b) 4 marks, (c) 3 marks**
This solution is also well-explained and demonstrates a good understanding of the underlying concepts. The student correctly derives the equations of motion, uncouples them using new coordinates, and finds the solution for \(y_1\) by applying initial conditions. The steps are presented logically, and the final answer matches the given solution.

**Solution 3: (a) 2 marks, (b) 4 marks, (c) 2 marks**
This solution demonstrates a good understanding of Newton's second law and Hooke's law. The student correctly derives the equations of motion and uncouples them. However, in part (c), the student incorrectly states that \( \phi = \pi \) could be a solution, which is not consistent with the initial condition \(y_1(0) = 0\). Additionally, the student seems to have overlooked the initial position of object Y, which is also 0. This results in a deduction of 1 mark.

**General Feedback**
All three solutions show a good grasp of the fundamental physics involved in the problem. Solutions 1 and 2 are particularly well-structured and provide clear explanations. Solution 3 is also well-written, but it needs a minor correction in part (c) to ensure consistency with the initial conditions. Overall, all three students have demonstrated a strong understanding of the topic.

\bigskip    
\textbf{Response 3:}

**Solution 1: (a) 2 marks, (b) 4 marks, (c) 3 marks**
This solution is well-structured and provides a clear step-by-step approach. The student demonstrates a good understanding of Newton's second law, Hooke's law, and simple harmonic motion. All parts are answered correctly, and the derivations are presented clearly. 

**Solution 2: (a) 2 marks, (b) 4 marks, (c) 3 marks**
This solution is also well-explained and demonstrates a good understanding of the underlying concepts. The student correctly derives the equations of motion, uncouples them using new coordinates, and finds the solution for \(y_1\) by applying initial conditions. The steps are presented logically, and the final answer matches the given solution.

**Solution 3: (a) 2 marks, (b) 4 marks, (c) 2 marks**
This solution demonstrates a good understanding of Newton's second law and Hooke's law. The student correctly derives the equations of motion and uncouples them using new coordinates. However, in part (c), when finding the solution for \( y_1 \), the student makes a mistake in the initial condition. The student claims that since both objects start from equilibrium, $y_1(0) = x_1(0) + x_2(0) = 0$. However, the problem statement said that object Y is moving with an initial velocity, so $x_2(0) \neq 0$. This mistake will change the value of A. So 1 mark was deducted. 

**General Feedback**
All three solutions show a good grasp of the fundamental physics involved in the problem. Solutions 1 and 2 are particularly well-structured and provide clear explanations. Solution 3 is also well-written, but it needs a minor correction in part (c) to ensure consistency with the initial conditions. Overall, all three students have demonstrated a strong understanding of the topic.

\textbf{Response 4:}

**Solution 1: 9/9 Marks**

This is an excellent solution. 

*   **Part a):** 2/2. The student correctly derives the equations of motion for both objects X and Y.
*   **Part b):** 4/4. The student correctly uncouples the equations of motion by defining new coordinates and shows that they obey simple harmonic motion.
*   **Part c):** 3/3. The student uses the initial conditions to find the specific solution for \(y_1\) correctly.

**Solution 2: 9/9 Marks**

This is also an excellent solution.

*   **Part a):** 2/2. The student correctly derives the equations of motion for both objects X and Y.
*   **Part b):** 4/4. The student correctly uncouples the equations of motion and explains that they represent simple harmonic motion.
*   **Part c):** 3/3. The student correctly derives the solution for \(y_1\) using the initial conditions.

**Solution 3: 8/9 Marks**

This solution is very good but has a minor error.

*   **Part a):** 2/2. The student correctly derives the equations of motion for both objects X and Y.
*   **Part b):** 4/4. The student correctly uncouples the equations of motion and explains that they represent simple harmonic motion.
*   **Part c):** 2/3. The student correctly derives the general solution for \(y_1\) but incorrectly assumes \(x_2(0) = 0\). However, the rest of the solution is correct based on this incorrect assumption.

**Overall Feedback:**

All three students demonstrate a strong understanding of classical mechanics and the concept of simple harmonic motion. They can correctly apply Newton's second law and Hooke's law to derive the equations of motion. They are also proficient in solving differential equations and applying initial conditions. The minor error in Solution 3 does not detract from the overall quality of the solution.

\textbf{Response 5:}

**Solution 1: 9/9 Marks**

This is an excellent solution. 

*   **Part a):** 2/2. The student correctly derives the equations of motion for both objects X and Y.
*   **Part b):** 4/4. The student correctly uncouples the equations of motion by defining new coordinates and shows that they obey simple harmonic motion.
*   **Part c):** 3/3. The student uses the initial conditions to find the specific solution for \(y_1\) correctly.

**Solution 2: 9/9 Marks**

This is also an excellent solution.

*   **Part a):** 2/2. The student correctly derives the equations of motion for both objects X and Y.
*   **Part b):** 4/4. The student correctly uncouples the equations of motion and explains that they represent simple harmonic motion.
*   **Part c):** 3/3. The student correctly derives the solution for \(y_1\) using the initial conditions.

**Solution 3: 8/9 Marks**

This solution is very good but has a minor error in part c.

*   **Part a):** 2/2. The student correctly derives the equations of motion for both objects X and Y.
*   **Part b):** 4/4. The student correctly uncouples the equations of motion and explains that they represent simple harmonic motion.
*   **Part c):** 2/3. The student makes a minor error in applying the initial conditions for y1. The initial condition for the velocity is correct, but the initial condition for the position should be \( y_1(0) = 0 \) since both masses start at their equilibrium positions. This leads to an incorrect phase angle in the final answer. 

**Overall Feedback:**

All three students demonstrate a strong understanding of classical mechanics and the concept of simple harmonic motion. They can correctly apply Newton's second law and Hooke's law to derive the equations of motion. They are also proficient in solving differential equations and applying initial conditions. The minor error in Solution 3 does not detract significantly from the overall quality of the solution.

\subsection*{Question 6:}

\textbf{Response 1:}

**Solution 1: (a) 2 marks, (b) 3 marks, (c) 6 marks**

This solution demonstrates a good understanding of the physics involved. The student correctly derives the maximum speed for a flat track and a frictionless inclined track in parts (a) and (b). In part (c), they correctly identify the forces contributing to the centripetal force and set up the equations for maximum and minimum speeds. They also attempt to simplify the difference of squares but arrive at an incorrect result due to a calculation error. However, the overall approach and understanding are sound.

**Solution 2: (a) 2 marks, (b) 1 mark, (c) 2 marks**

This solution also correctly derives the maximum speed for a flat track. However, in part (b), there's a conceptual error in identifying the component of gravity providing centripetal force, resulting in an incorrect expression for the maximum speed on the frictionless inclined track. In part (c), the student correctly identifies the forces involved but makes a similar error in setting up the equations, leading to an incorrect final result.

**Solution 3: (a) 2 marks, (b) 3 marks, (c) 7 marks**

This solution correctly derives the maximum speed for both the flat track and the frictionless inclined track in parts (a) and (b). In part (c), the student correctly identifies the forces, sets up the equations for maximum and minimum speeds, and simplifies the difference of squares, arriving at the correct result. The solution is well-organized and clearly demonstrates a thorough understanding of the problem.

**General Feedback**

All three solutions demonstrate a good understanding of the basic concepts of circular motion and friction. Solutions 1 and 3 show a stronger grasp of the forces involved in an inclined plane scenario. Solution 2 needs improvement in understanding the force components on an incline. Overall, the students have shown varying levels of proficiency in applying physics principles to a real-world problem.

\bigskip    
\textbf{Response 2:}

**Solution 1: (a) 2 marks, (b) 3 marks, (c) 4 marks**

The student demonstrates a good understanding of the physics involved in parts (a) and (b), correctly deriving the expressions for maximum speed on a flat and frictionless inclined track. In part (c), the student accurately sets up the equations for maximum and minimum speeds, however, there is a calculation error in simplifying the difference of squares, leading to an incorrect final result.

**Solution 2: (a) 2 marks, (b) 0 marks, (c) 0 marks**

The student correctly derives the maximum speed on a flat track in part (a). However, there is a fundamental error in part (b) in identifying the component of gravity providing centripetal force on an inclined frictionless track. This error carries over to part (c), leading to incorrect equations and an incorrect final result.

**Solution 3: (a) 2 marks, (b) 3 marks, (c) 4 marks**

This solution correctly derives the maximum speed for both the flat track and the frictionless inclined track in parts (a) and (b). In part (c), the student accurately identifies the forces involved and sets up the equations for maximum and minimum speeds. However, they make an error in simplifying the difference of squares, leading to an incorrect final result.

**General Feedback**

All three solutions demonstrate a good understanding of the basic concepts of circular motion and friction. Solutions 1 and 3 show a stronger grasp of the forces involved in an inclined plane scenario, but both make calculation errors in part (c). Solution 2 needs improvement in understanding the force components on an incline. Overall, the students have shown varying levels of proficiency in applying physics principles to a real-world problem.

\bigskip    
\textbf{Response 3:}

**Solution 1: (a) 2 marks, (b) 3 marks, (c) 6 marks**

This solution is well-structured and demonstrates a good understanding of the underlying physics concepts.

*   **Part (a)** is entirely correct.
*   **Part (b)** is also correctly done, deriving the maximum speed on a frictionless banked track.
*   **Part (c)** is where the solution falters. While the forces are correctly identified and the equations are set up properly, an error occurs during the algebraic manipulation to arrive at the final expression.

Overall, this solution shows a strong grasp of the problem's fundamentals but needs a minor correction in the algebraic simplification in part (c).

**Solution 2: (a) 2 marks, (b) 0 marks, (c) 0 marks**

This solution only successfully solves part (a).

*   **Part (a)** is correctly done, deriving the maximum speed for a flat track.
*   **Part (b)** contains a fundamental error. The student incorrectly identifies the component of gravity that provides the centripetal force on a frictionless banked track.
*   **Part (c)** is incorrect due to the error in part (b), and the subsequent steps are also flawed.

This solution shows some understanding of circular motion but needs a better grasp of forces on an inclined plane.

**Solution 3: (a) 2 marks, (b) 3 marks, (c) 6 marks**

This solution is well-presented and demonstrates a good understanding of the problem.

*   **Part (a)** and **(b)** are correctly solved, with accurate derivations of the maximum speeds for both flat and frictionless inclined tracks.
*   In **part (c)**, the student correctly identifies the forces and sets up the equations but makes a similar algebraic error to Solution 1 in the simplification step, leading to an incorrect final result.

Overall, this solution shows a good understanding of the physics involved but needs a minor correction in the algebraic simplification in part (c).

**General Feedback:**

All three solutions demonstrate a good understanding of the basic concepts of circular motion and friction. However, all three students struggle with the algebraic manipulation in part (c). This highlights the importance of careful calculation and double-checking results in physics problems. Additionally, Solution 2 needs to revisit the concept of force components on an inclined plane.

\textbf{Response 4:}

**Solution 1: 13/16 Marks**

This is a good solution, but it has some errors and oversights.

*   **Part a):** 2/2. The student correctly derives the expression for the maximum speed on a flat track.
*   **Part b):** 2/3. The student makes an error in identifying the component of gravitational force contributing to centripetal force. The correct component is \(mg \cos(\phi)\) as it acts perpendicular to the plane and directed towards the centre of the circular motion.
*   **Part c):** 9/11. The student correctly identifies the forces involved and sets up the equations for maximum and minimum speed. However, they make a mistake in the simplification of \(v^2_{max} - v^2_{min}\) and do not recognize the discrepancy with the given expression.

**Solution 2: 9/16 Marks**

This solution shows some understanding of the problem but has significant errors.

*   **Part a):** 2/2. The student correctly derives the expression for the maximum speed on a flat track.
*   **Part b):** 0/3. The student makes the same error as in Solution 1, identifying the wrong component of gravitational force contributing to centripetal force. They also make an additional error in equating this force directly to the centripetal force without accounting for the normal force.
*   **Part c):** 7/11. The student correctly identifies the forces involved and sets up the equations for maximum and minimum speed. They are able to simplify the expression for \(v^2_{max} - v^2_{min}\), but the result is incorrect due to the error in part b.

**Solution 3: 14/16 Marks**

This is a very good solution with minor mistakes.

*   **Part a):** 2/2. The student correctly derives the expression for the maximum speed on a flat track.
*   **Part b):** 3/3. The student correctly derives the expression for the maximum speed on a frictionless inclined track.
*   **Part c):** 9/11. The student correctly identifies the forces involved and sets up the equations for maximum and minimum speed. They are able to derive the correct expressions for \(v_{max}\) and \(v_{min}\). However, they incorrectly simplify the expression for \(v^2_{max} - v^2_{min}\), potentially due to an algebraic error.

**Overall Feedback:**
All three students demonstrate a good understanding of circular motion and the forces involved. However, all three students struggle with the transition from the flat track to the inclined track, making errors in identifying the correct component of gravitational force contributing to centripetal force. This leads to incorrect expressions for maximum speed on the inclined track and the difference between the squares of the maximum and minimum speeds. The third solution is the most complete and accurate, with only minor errors in the final simplification.

\textbf{Response 5:}

**Solution 1: 13/16 Marks**

This is a good solution, but it has some errors and oversights in parts b and c.

*   **Part a):** 2/2. The student correctly derives the expression for the maximum speed on a flat track.
*   **Part b):** 1/3. The student makes an error in identifying the component of gravitational force contributing to centripetal force. The correct component is  mgcos(ϕ)  as it acts perpendicular to the plane and directed towards the centre of the circular motion.
*   **Part c):** 10/11. The student correctly identifies the forces involved and sets up the equations for maximum and minimum speed.  The student has correctly derived the expressions for vmax and vmin. However, they were unable to arrive at the given equality.

**Solution 2: 9/16 Marks**

This solution shows some understanding of the problem but has significant errors in parts b and c.

*   **Part a):** 2/2. The student correctly derives the expression for the maximum speed on a flat track.
*   **Part b):** 0/3. The student makes the same error as in Solution 1, identifying the wrong component of gravitational force contributing to centripetal force. They also make an additional error in equating this force directly to the centripetal force without accounting for the normal force.
*   **Part c):** 7/11. The student correctly identifies the forces involved and sets up the equations for maximum and minimum speed. They are able to simplify the expression for \(v^2_{max} - v^2_{min}\), but the result is incorrect due to the error in part b.

**Solution 3: 14/16 Marks**

This is a very good solution with minor mistakes.

*   **Part a):** 2/2. The student correctly derives the expression for the maximum speed on a flat track.
*   **Part b):** 3/3. The student correctly derives the expression for the maximum speed on a frictionless inclined track.
*   **Part c):** 9/11. The student correctly identifies the forces involved and sets up the equations for maximum and minimum speed. They are able to derive the correct expressions for \(v_{max}\) and \(v_{min}\). However, they incorrectly simplify the expression for \(v^2_{max} - v^2_{min}\), potentially due to an algebraic error. The student was incredibly close and likely made a small mistake in the algebra.

**Overall Feedback:**
All three students demonstrate a good understanding of circular motion and the forces involved. However, all three students struggle with the transition from the flat track to the inclined track, making errors in identifying the correct component of gravitational force contributing to centripetal force. This leads to incorrect expressions for maximum speed on the inclined track and the difference between the squares of the maximum and minimum speeds. The third solution is the most complete and accurate, with only minor errors in the final simplification.

\subsection*{Question 7:}

\textbf{Response 1:}

**Solution 1: (a) 4 marks, (b) 5 marks, (c) 4 marks**

This solution is well-organized and comprehensive. In part (a), the student correctly writes down the energy-momentum 4-vector and shows its invariance under Lorentz transformation. Part (b) is also solved accurately, obtaining the correct expressions for the four-momenta of the photons in the lab frame. Part (c) demonstrates a good understanding of the relationship between energy and wavelength and correctly calculates the wavelengths in both frames. The student also mentions the Doppler effect, showing a good grasp of the physics involved.

**Solution 2: (a) 3 marks, (b) 4 marks, (c) 0 marks**

This solution also shows a good understanding of the concept of the energy-momentum 4-vector and its invariance in part (a). However, it loses a mark due to a minor error in the calculation of the scalar product. The student correctly obtains the four-momenta in the lab frame in part (b). However, part (c) is incorrect due to a misunderstanding of the energy distribution between the photons in the rest frame. The student also does not calculate the wavelengths correctly.

**Solution 3: (a) 4 marks, (b) 5 marks, (c) 4 marks**

This solution is well-presented and demonstrates a good understanding of the problem. The student correctly writes down the energy-momentum 4-vector and shows its invariance in part (a). Part (b) is also solved accurately, obtaining the correct expressions for the four-momenta in the lab frame. Part (c) shows a good understanding of the energy-wavelength relationship and correctly calculates the wavelengths in both frames. However, the student uses a different value for Planck's constant (eV-s), which is acceptable but leads to slightly different numerical results.

**General Feedback**

All three solutions demonstrate a good grasp of special relativity and the concept of four-momenta. Solutions 1 and 3 are particularly well-explained and show a good understanding of the physics involved. Solution 2 needs some minor corrections in part (a) and a better understanding of energy conservation in particle decays in part (c).

\bigskip    
\textbf{Response 2:}

**Solution 1: (a) 4 marks, (b) 5 marks, (c) 4 marks**

This solution is well-organized and comprehensive. In part (a), the student correctly writes down the energy-momentum 4-vector and shows its invariance under Lorentz transformation. Part (b) is also solved accurately, obtaining the correct expressions for the four-momenta of the photons in the lab frame. Part (c) demonstrates a good understanding of the relationship between energy and wavelength and correctly calculates the wavelengths in both frames. The student also mentions the Doppler effect, showing a good grasp of the physics involved.

**Solution 2: (a) 4 marks, (b) 5 marks, (c) 3 marks**

This solution also shows a good understanding of the concept of the energy-momentum 4-vector and its invariance in part (a). The student correctly obtains the four-momenta in the lab frame in part (b). However, part (c) is partially incorrect due to a misunderstanding of the energy distribution between the photons in the rest frame. Each photon should carry half the total energy of the decaying particle in the rest frame, resulting in a wavelength of approximately 2.48e-16 meters. The calculation for the lab frame wavelengths is correct, however.

**Solution 3: (a) 4 marks, (b) 5 marks, (c) 4 marks**

This solution is well-presented and demonstrates a good understanding of the problem. The student correctly writes down the energy-momentum 4-vector and shows its invariance in part (a). Part (b) is also solved accurately, obtaining the correct expressions for the four-momenta in the lab frame. Part (c) shows a good understanding of the energy-wavelength relationship and correctly calculates the wavelengths in both frames. However, the student uses a different value for Planck's constant (eV-s), which is acceptable but leads to slightly different numerical results.

**General Feedback**

All three solutions demonstrate a good grasp of special relativity and the concept of four-momenta. Solutions 1 and 3 are particularly well-explained and show a good understanding of the physics involved. Solution 2 needs a correction in the energy distribution in part (c) to calculate the rest frame wavelength correctly.

\bigskip    
\textbf{Response 3:}

**Solution 1: (a) 4 marks, (b) 5 marks, (c) 4 marks**

This solution is well-organized and comprehensive. In part (a), the student correctly writes down the energy-momentum 4-vector and shows its invariance under Lorentz transformation. Part (b) is also solved accurately, obtaining the correct expressions for the four-momenta of the photons in the lab frame. Part (c) demonstrates a good understanding of the relationship between energy and wavelength, and the wavelengths are correctly calculated for both frames. The student also mentions the Doppler effect, showing a good grasp of the physics involved.

**Solution 2: (a) 3 marks, (b) 5 marks, (c) 2 marks**

This solution also shows a good understanding of the concept of the energy-momentum 4-vector and its invariance in part (a). However, it loses a mark due to a minor error in the calculation of the scalar product (the final result should be -m^2c^2, not m^2c^2). The student correctly obtains the four-momenta in the lab frame in part (b). However, part (c) is incorrect due to a misunderstanding of the energy distribution between the photons in the rest frame. Each photon should carry half the total energy of the decaying particle in the rest frame.

**Solution 3: (a) 4 marks, (b) 5 marks, (c) 1 mark**

This solution is well-presented and demonstrates a good understanding of the problem. The student correctly writes down the energy-momentum 4-vector and shows its invariance in part (a). Part (b) is also solved accurately, obtaining the correct expressions for the four-momenta in the lab frame. However, in part (c), the student incorrectly calculates the wavelengths of the photons in both the rest frame and the lab frame by assuming the same energy for both. As the photons are massless, their energy and momentum are related by E=pc, which should be used to determine their wavelengths.


**General Feedback**

All three solutions demonstrate a good grasp of special relativity and the concept of four-momenta. Solutions 1 and 3 are particularly well-explained and show a good understanding of the physics involved. Solution 2 needs a minor correction in the scalar product calculation in part (a) and a better understanding of energy conservation in particle decays in part (c) to calculate the wavelengths correctly.

\textbf{Response 4:}

**Solution 1: 13/13 Marks**

This is an excellent solution. The student demonstrates a clear understanding of the energy-momentum 4-vector, Lorentz invariance, and Lorentz transformations. The explanations are clear and concise, and the calculations are accurate.

*   **Part a):** 4/4 marks. The student correctly states the energy-momentum 4-vector and demonstrates its Lorentz invariance.
*   **Part b):** 5/5 marks. The student correctly applies the Lorentz transformation to find the four-momenta of the decay products in the lab frame.
*   **Part c):** 4/4 marks. The student correctly calculates the wavelengths of the photons in both the rest frame and the lab frame.

**Solution 2: 12/13 Marks**

This solution is very good, with just a small error.

*   **Part a):** 4/4 marks. The student correctly states the energy-momentum 4-vector and demonstrates its Lorentz invariance.
*   **Part b):** 5/5 marks. The student correctly applies the Lorentz transformation to find the four-momenta of the decay products in the lab frame.
*   **Part c):** 3/4 marks. The student makes a mistake in calculating the energy of each photon in the rest frame, leading to an incorrect wavelength. However, the student correctly uses the Lorentz transformation to find the energies in the lab frame and calculates the corresponding wavelengths correctly given their previous mistake.

**Solution 3: 9/13 Marks**

This solution demonstrates a good understanding of the concepts but has some errors and omissions.

*   **Part a):** 4/4 marks. The student correctly states the energy-momentum 4-vector and demonstrates its Lorentz invariance.
*   **Part b):** 5/5 marks. The student correctly applies the Lorentz transformation to find the four-momenta of the decay products in the lab frame.
*   **Part c):** 0/4 marks. The student incorrectly calculates the wavelengths of the photons in both frames. The values should be 1-2 orders of magnitude smaller.

**Overall Feedback:**

All three students demonstrate a good understanding of special relativity and the energy-momentum 4-vector. However, Solutions 2 and 3 have errors in the calculation of the wavelengths of the photons. Solution 1 is the most complete and accurate solution.

\textbf{Response 5:}

**Solution 1: 13/13 Marks**

This is an excellent solution. The student demonstrates a clear understanding of the energy-momentum 4-vector, Lorentz invariance, and Lorentz transformations. The explanations are clear and concise, and the calculations are accurate.

*   **Part a):** 4/4 marks. The student correctly states the energy-momentum 4-vector and demonstrates its Lorentz invariance.
*   **Part b):** 5/5 marks. The student correctly applies the Lorentz transformation to find the four-momenta of the decay products in the lab frame.
*   **Part c):** 4/4 marks. The student correctly calculates the wavelengths of the photons in both the rest frame and the lab frame.

**Solution 2: 12/13 Marks**

This solution is very good, with a few minor oversights.

*   **Part a):** 4/4 marks. The student correctly states the energy-momentum 4-vector and demonstrates its Lorentz invariance.
*   **Part b):** 5/5 marks. The student correctly applies the Lorentz transformation to find the four-momenta of the decay products in the lab frame.
*   **Part c):** 3/4 marks. There is an oversight in part c when the student calculates the energy of the photons. Each photon only gets half of the total energy. The wavelengths in the lab frame are correct given the mistake with \(E_{photon}\). 


**Solution 3: 10/13 Marks**

This solution demonstrates a good understanding of the concepts but has some errors and omissions.

*   **Part a):** 4/4 marks. The student correctly states the energy-momentum 4-vector and demonstrates its Lorentz invariance.
*   **Part b):** 5/5 marks. The student correctly applies the Lorentz transformation to find the four-momenta of the decay products in the lab frame.
*   **Part c):** 1/4 marks. The student incorrectly calculates the wavelengths of the photons in both frames by several orders of magnitude. The energies of the photons in the lab frame are also not calculated.

**Overall Feedback:**
All three students demonstrate a good understanding of special relativity and the energy-momentum 4-vector. However, Solutions 2 and 3 have errors in the calculation of the wavelengths of the photons. Solution 1 is the most complete and accurate solution.

\subsection*{Question 8:}

\textbf{Response 1:}

**Solution 1: (a) 4 marks, (b) 1 mark, (c) 4 marks**

This solution provides a well-structured and detailed derivation of the Lagrangian and the equation of motion in part (a), showcasing a clear understanding of the concepts involved. The use of Python for algebraic manipulation demonstrates resourcefulness. However, the simplification in part (b) is incorrect, as the terms involving A and v should vanish in the limit. This indicates a misunderstanding of the small angle approximation.

**Solution 2: (a) 3 marks, (b) 1 mark, (c) 4 marks**

This solution correctly identifies the components of kinetic and potential energy. However, it loses a mark in part (a) due to an error in expressing the radial kinetic energy. The term \(\frac{1}{2} m (\dot{x}(t))^2 \) should include the horizontal component of the pendulum bob's motion, which depends on \(\phi\) and \(\dot{\phi}\). The simplification in part (b) is also incorrect due to the same misunderstanding of the small angle approximation as in Solution 1.

**Solution 3: (a) 4 marks, (b) 1 mark, (c) 4 marks**

This solution is well-presented and demonstrates a good understanding of the problem. The derivation of the Lagrangian is correct, and the student applies the Euler-Lagrange equation appropriately. However, there's an error in the final simplification of the equation of motion in part (a). The term with the second derivative of x should not be dropped, as it is still relevant even in the limit of small oscillations. The simplification in part (b) is also incorrect due to a misunderstanding of the small angle approximation.


**General Feedback:**

All three solutions demonstrate a good understanding of Lagrangian mechanics and its application to the pendulum problem. However, all solutions exhibit a common misunderstanding of the small angle approximation in part (b). It's important to note that the small angle approximation involves not only approximating sin(phi) to phi, but also cos(phi) to 1, and the first-order time derivatives of x and y to zero in the limit of small oscillations and v, A approaching zero.

\bigskip    
\textbf{Response 2:}

**Solution 1: (a) 9 marks, (b) 1 mark**

This solution demonstrates a strong understanding of Lagrangian mechanics and its application to the pendulum problem. The student correctly derives the Lagrangian, including both translational and rotational kinetic energy terms. The Euler-Lagrange equation is applied appropriately, leading to the correct equation of motion for the angle φ. The small-angle approximation in part (b) is also correctly applied, resulting in the classic pendulum equation. Overall, this is an excellent solution that deserves full marks.

**Solution 2: (a) 7 marks, (b) 1 mark**

This solution correctly identifies the kinetic and potential energy components and sets up the Lagrangian. However, the student makes a mistake in the expression for the radial kinetic energy, omitting the horizontal component of the pendulum bob's motion, which depends on $\varphi$ and \(\dot{\phi}\). This leads to an incorrect Lagrangian and, consequently, an incorrect equation of motion, thus deducting 2 marks from part (a). The small-angle approximation in part (b) is applied correctly, but due to the error in part (a), the final equation is not the correct one for the classic pendulum.

**Solution 3: (a) 8 marks, (b) 1 mark**

This solution is well-presented and demonstrates a good understanding of the problem. The derivation of the Lagrangian and the application of the Euler-Lagrange equation are correct. However, in the final simplification of the equation of motion in part (a), the student incorrectly drops the term with the second derivative of x. This term is still relevant even in the limit of small oscillations, resulting in a deduction of 1 mark. The small-angle approximation in part (b) is applied correctly, but due to the error in part (a), the final equation is not the correct one for the classic pendulum.

**General Feedback**

All three solutions show a good grasp of Lagrangian mechanics and its application to the pendulum problem. Solutions 1 and 3 are particularly well-structured and demonstrate a clear understanding of the underlying physics. However, all solutions exhibit a minor error in either the derivation of the Lagrangian or the simplification of the equation of motion. Solution 2 has a more significant error in the expression for the kinetic energy. Overall, the students have shown a good understanding of the topic but could benefit from paying closer attention to detail in their calculations.

\bigskip    
\textbf{Response 3:}

**Solution 1: (a) 9 marks, (b) 1 mark**

This solution demonstrates a strong understanding of Lagrangian mechanics and its application to the pendulum problem. The student correctly derives the Lagrangian, including both translational and rotational kinetic energy terms. The Euler-Lagrange equation is applied appropriately, leading to the correct equation of motion for the angle φ. The small-angle approximation in part (b) is also correctly applied, resulting in the classic pendulum equation. Overall, this is an excellent solution that deserves full marks.

**Solution 2: (a) 7 marks, (b) 1 mark**

This solution correctly identifies the components of kinetic and potential energy and sets up the Lagrangian. However, it loses two marks in part (a) due to an error in expressing the radial kinetic energy. The term \(\frac{1}{2} m (\dot{x}(t))^2 \) should include the horizontal component of the pendulum bob's motion, which depends on φ and \(\dot{\phi}\). The correct expression would be \(\frac{1}{2}m(\dot{x}(t) + L\dot{\phi}\cos(\phi))^2\). The small-angle approximation in part (b) is applied correctly, but due to the error in part (a), the final equation is not the correct one for the classic pendulum.

**Solution 3: (a) 9 marks, (b) 1 mark**

This solution is well-presented and demonstrates a good understanding of the problem. The derivation of the Lagrangian and the application of the Euler-Lagrange equation are correct. The equation of motion is derived accurately, and the small-angle approximation in part (b) is applied correctly, resulting in the classic pendulum equation. This solution deserves full marks.

**General Feedback**

All three solutions show a good grasp of Lagrangian mechanics and its application to the pendulum problem. Solutions 1 and 3 are particularly well-structured and demonstrate a clear understanding of the underlying physics. Solution 2 has a minor error in the expression for the radial kinetic energy. Overall, the students have shown a good understanding of the topic.

\textbf{Response 4:}

**Solution 1: 10/10 Marks**

This is an excellent solution. The student demonstrates a strong understanding of Lagrangian mechanics and has applied it correctly to derive the equation of motion for the pendulum with a moving support. The simplification to the classic pendulum equation in the limit of small oscillations is also well-explained.

*   **Part a):** 9/9 marks. The student correctly derives the Lagrangian, applies the Euler-Lagrange equation, and simplifies the resulting equation of motion. The use of Python for algebraic manipulation is efficient and shows a good grasp of computational tools.
*   **Part b):** 1/1 mark. The student correctly identifies the limits that lead to the classic pendulum equation and explains the simplifications involved.

**Solution 2: 7/10 Marks**

This solution demonstrates a good understanding of Lagrangian mechanics but has a few errors.

*   **Part a):** 6/9 marks. The student correctly derives the Lagrangian but makes a mistake in the Euler-Lagrange equation by failing to account for the \( \ddot{x} \) term resulting from the chain rule.
*   **Part b):** 1/1 mark. The student correctly identifies the limits that lead to the classic pendulum equation but does not explicitly mention the vanishing of the \( \dot{x}(t) \) term in the kinetic energy.

**Solution 3: 8/10 Marks**

This solution shows a good understanding of the problem and Lagrangian mechanics but has a few errors.

*   **Part a):** 6/9 marks. The student makes a mistake in the derivation of the kinetic energy by not considering the horizontal component of the velocity due to the support's motion. This leads to an incorrect Lagrangian and subsequent equation of motion.
*   **Part b):** 1/1 mark. The student correctly identifies the limits that lead to the classic pendulum equation and provides a clear explanation of the simplifications. However, they fail to explicitly state the vanishing of the terms related to the support's motion.

**Overall Feedback:** All three students demonstrate a good understanding of Lagrangian mechanics. Solution 1 is the most complete and accurate, while Solutions 2 and 3 have minor errors in the derivation of the equation of motion and in the explanation of the simplification to the classic pendulum equation.

\textbf{Response 5:}

**Solution 1: 10/10 Marks**

This is an excellent solution. The student demonstrates a strong understanding of Lagrangian mechanics and has applied it correctly to derive the equation of motion for the pendulum with a moving support. The simplification to the classic pendulum equation in the limit of small oscillations is also well-explained.

*   **Part a):** 9/9 marks. The student correctly derives the Lagrangian, applies the Euler-Lagrange equation, and simplifies the resulting equation of motion. The use of Python for algebraic manipulation is efficient and shows a good grasp of computational tools.
*   **Part b):** 1/1 mark. The student correctly identifies the limits that lead to the classic pendulum equation and explains the simplifications involved.

**Solution 2: 7/10 Marks**

This solution demonstrates a good understanding of Lagrangian mechanics but has a few errors in part a).

*   **Part a):** 6/9 marks. The student correctly derives the Lagrangian but makes a mistake in the Euler-Lagrange equation by failing to account for the cross-term in the kinetic energy involving \( \dot{x}\dot{\phi} \).
*   **Part b):** 1/1 mark. The student correctly identifies the limits that lead to the classic pendulum equation and explains the simplifications involved.

**Solution 3: 7/10 Marks**

This solution shows a good understanding of the problem and Lagrangian mechanics but has a few errors in part a).

*   **Part a):** 6/9 marks. The student makes a mistake in the derivation of the kinetic energy by incorrectly taking the derivative of \(x(t) + L\sin(\phi)\), specifically the term \(L\cos(\phi)\dot{\phi}\) should be added when differentiating with respect to time. This leads to an incorrect Lagrangian and subsequent equation of motion.
*   **Part b):** 1/1 mark. The student correctly identifies the limits that lead to the classic pendulum equation and provides a clear explanation of the simplifications.

**Overall Feedback:** All three students demonstrate a good understanding of Lagrangian mechanics. Solution 1 is the most complete and accurate, while Solutions 2 and 3 have errors in the derivation of the equation of motion (involving the kinetic energy term) that prevent them from obtaining full marks for part a).

\subsection*{Question 9:}

\textbf{Response 1:}

**Solution 1: (a) 5 marks, (b) 1 mark**

This solution demonstrates a good understanding of the basic principles and correctly applies Newton's second law for rotation and translation. However, there's an error in calculating the tension in the rope (T), as the mass M should be divided by 2 in the denominator, not multiplied by 2. The subsequent calculations for a and alpha are also incorrect due to this initial mistake.

**Solution 2: (a) 6 marks, (b) 1 mark**

This solution correctly applies Newton's laws and derives the correct expressions for acceleration, angular acceleration, and tension. The calculations in part (b) are also accurate and demonstrate a good understanding of the problem.

**Solution 3: (a) 6 marks, (b) 1 mark**

This solution is well-presented and demonstrates a clear understanding of the problem. The student correctly applies Newton's laws and derives the expressions for tension, acceleration, and angular acceleration. The calculations in part (b) are also correct and presented with appropriate units.

**General Feedback**

All three solutions demonstrate a good grasp of the fundamental physics concepts and the ability to apply Newton's laws to solve the problem. However, Solution 1 makes a calculation error in determining the tension, which affects the subsequent results. Solutions 2 and 3 are both well-explained and accurate, showing a strong understanding of rotational and translational dynamics.

\bigskip    
\textbf{Response 2:}

**Solution 1: (a) 5 marks, (b) 1 mark**

This solution demonstrates a good understanding of the basic principles and correctly applies Newton's second law for rotation and translation. However, there's an error in calculating the tension in the rope (T), as the mass M should be divided by 2 in the denominator, not multiplied by 2. The subsequent calculations for a and alpha are also incorrect due to this initial mistake.

**Solution 2: (a) 6 marks, (b) 1 mark**

This solution correctly applies Newton's laws and derives the correct expressions for acceleration, angular acceleration, and tension. The calculations in part (b) are also accurate and demonstrate a good understanding of the problem.

**Solution 3: (a) 6 marks, (b) 1 mark**

This solution is well-presented and demonstrates a clear understanding of the problem. The student correctly applies Newton's laws and derives the expressions for tension, acceleration, and angular acceleration. The calculations in part (b) are also correct and presented with appropriate units.

**General Feedback**

All three solutions demonstrate a good grasp of the fundamental physics concepts and the ability to apply Newton's laws to solve the problem. However, Solution 1 makes a calculation error in determining the tension, which affects the subsequent results. Solutions 2 and 3 are both well-explained and accurate, showing a strong understanding of rotational and translational dynamics.

\bigskip    
\textbf{Response 3:}

**Solution 1: (a) 5 marks, (b) 1 mark**

This solution demonstrates a good understanding of the basic principles and correctly applies Newton's second law for rotation and translation. However, there's an error in the derivation of the tension in the rope (T), as the mass M should be divided by 2 in the denominator, not multiplied by 2. The subsequent calculations for a and alpha are also incorrect due to this initial mistake.

**Solution 2: (a) 6 marks, (b) 1 mark**

This solution correctly applies Newton's laws and derives the correct expressions for acceleration, angular acceleration, and tension. The calculations in part (b) are also accurate and demonstrate a good understanding of the problem.

**Solution 3: (a) 6 marks, (b) 1 mark**

This solution is well-presented and demonstrates a clear understanding of the problem. The student correctly applies Newton's laws and derives the expressions for tension, acceleration, and angular acceleration. The calculations in part (b) are also correct and presented with appropriate units.

**General Feedback**

All three solutions demonstrate a good grasp of the fundamental physics concepts and the ability to apply Newton's laws to solve the problem. However, Solution 1 makes a calculation error in determining the tension, which affects the subsequent results. Solutions 2 and 3 are both well-explained and accurate, showing a strong understanding of rotational and translational dynamics.

\textbf{Response 4:}

**Solution 1: 7/7 Marks**

This is an excellent solution. The student has clearly and correctly derived the expressions for the angular acceleration, tension in the rope, and acceleration of the block. The numerical calculations are also accurate.

*   **Part a):** 6/6. The student correctly applies Newton's second law for rotation and linear motion to derive the required expressions.
*   **Part b):** 1/1. The student correctly substitutes the given values into the derived expressions and calculates the numerical values to appropriate precision.

**Solution 2: 7/7 Marks**

This solution is also excellent. The student has shown a good understanding of the problem and has correctly applied Newton's second law to both the block and the disk. The explanations are clear and the calculations are accurate.

*   **Part a):** 6/6. The student correctly derives the expressions for the angular acceleration, tension in the rope, and acceleration of the block.
*   **Part b):** 1/1. The student correctly substitutes the given values and calculates the numerical values to appropriate precision.

**Solution 3: 7/7 Marks**

This solution is well-presented and demonstrates a clear understanding of the problem. The student correctly derives the expressions for the angular acceleration, tension in the rope, and acceleration of the block. The numerical calculations are also accurate.

*   **Part a):** 6/6. The student clearly explains each step of the derivation, making it easy to follow.
*   **Part b):** 1/1. The student correctly substitutes the given values and calculates the numerical values to appropriate precision.

**Overall Feedback:**
All three solutions are of high quality and demonstrate a strong understanding of the concepts involved. The students have correctly applied Newton's second law and the concept of torque to solve the problem. They have also shown proficiency in algebraic manipulation and numerical calculation. There is no significant difference in the quality of the three solutions, and all deserve full marks.

\textbf{Response 5:}

**Solution 1: 7/7 Marks**

This is an excellent solution. The student demonstrates a strong understanding of the problem and uses clear and concise language to explain the steps involved in the derivation. The numerical calculations are accurate and presented with appropriate precision.

*   **Part a):** 6/6. The student correctly applies Newton's second law for rotation and linear motion to derive the required expressions.
*   **Part b):** 1/1. The student correctly substitutes the given values into the derived expressions and calculates the numerical values.

**Solution 2: 7/7 Marks**

This solution is also excellent. The student clearly understands the problem and uses appropriate equations and assumptions to arrive at the correct expressions and numerical values.

*   **Part a):** 6/6. The student correctly derives the expressions for the angular acceleration, tension in the rope, and acceleration of the block, showing a good understanding of the underlying physics principles.
*   **Part b):** 1/1. The student accurately substitutes the given values into the derived expressions and calculates the numerical values to appropriate precision.

**Solution 3: 7/7 Marks**

This is another excellent solution that demonstrates a clear understanding of the problem and the relevant physics principles. The student presents the solution in a logical and well-organized manner, making it easy to follow.

*   **Part a):** 6/6. The student correctly applies Newton's second law for rotation and linear motion to derive the expressions for angular acceleration, tension, and linear acceleration.
*   **Part b):** 1/1. The student accurately substitutes the given values into the derived expressions and calculates the numerical values to appropriate precision.

**Overall Feedback:**
All three solutions are well-written and demonstrate a strong grasp of the physics concepts involved. The students have used appropriate equations and assumptions to derive the correct expressions and calculate the numerical values accurately. There are no significant differences in quality among the three solutions, and all deserve full marks.

\subsection*{Question 10:}

\textbf{Response 1:}

**Solution 1: (a) 6 marks, (b) 1 mark**

This solution demonstrates a strong understanding of Newton's laws and differential equations. The student correctly derives the equation of motion and finds its solution. The explanation of the terminal velocity condition is clear and accurate. Additionally, the student correctly applies the initial condition to find the constant vector A.

**Solution 2: (a) 6 marks, (b) 1 mark**

This solution is also well-written and demonstrates a clear understanding of the problem. The student correctly derives the equation of motion and its solution, explaining the physical meaning of each term. The initial condition is correctly applied to find the constant vector A.

**Solution 3: (a) 6 marks, (b) 1 mark**

This solution is well-organized and easy to follow. The student correctly applies Newton's second law to both the horizontal and vertical components of motion, solves the resulting differential equations, and uses the initial condition to find the constant vector A.

**General Feedback:**

All three solutions are excellent and demonstrate a thorough understanding of the physics involved in this problem. They correctly derive the equation of motion, find its solution, and apply the initial conditions to determine the constant vector A. The students clearly explain their reasoning and use appropriate mathematical techniques. There is no need for any deductions based on the given mark scheme.

\bigskip    
\textbf{Response 2:}

**Solution 1: (a) 6 marks, (b) 2 marks**

This solution is well-organized and comprehensive. It correctly applies Newton's second law, solves the differential equation, and explains the physical meaning of each term in the solution. The student also demonstrates a clear understanding of how to apply the initial condition to find the constant vector A.

**Solution 2: (a) 6 marks, (b) 2 marks**

This solution is also well-written and demonstrates a clear understanding of the problem. The student correctly derives the equation of motion and its solution, explaining the significance of the terminal velocity term. The initial condition is correctly applied to find the constant vector A, and the student's explanation is concise and accurate.

**Solution 3: (a) 6 marks, (b) 2 marks**

This solution is well-organized and easy to follow. The student correctly applies Newton's second law, solves the differential equation by guessing the correct form, and verifies the solution through substitution. The initial condition is correctly applied to find the constant vector A.

**General Feedback:**

All three solutions are excellent and demonstrate a thorough understanding of the physics involved in this problem. They correctly derive the equation of motion, find its solution, and apply the initial conditions to determine the constant vector A. The students clearly explain their reasoning and use appropriate mathematical techniques. There is no need for any deductions based on the given mark scheme.

\bigskip    
\textbf{Response 3:}

**Solution 1: (a) 6 marks, (b) 2 marks**

This solution is well-organized and comprehensive. It correctly applies Newton's second law, solves the differential equation, and explains the physical meaning of each term in the solution. The student also demonstrates a clear understanding of how to apply the initial condition to find the constant vector A.

**Solution 2: (a) 6 marks, (b) 2 marks**

This solution is also well-written and demonstrates a clear understanding of the problem. The student correctly derives the equation of motion and its solution, explaining the significance of the terminal velocity term. The initial condition is correctly applied to find the constant vector A, and the student's explanation is concise and accurate.

**Solution 3: (a) 6 marks, (b) 2 marks**

This solution is well-organized and easy to follow. The student correctly applies Newton's second law, solves the differential equation by guessing the correct form, and verifies the solution through substitution. The initial condition is correctly applied to find the constant vector A. The solution also includes a clear explanation of the physical meaning of the constant vector A.

**General Feedback:**

All three solutions are excellent and demonstrate a thorough understanding of the physics involved in this problem. They correctly derive the equation of motion, find its solution, and apply the initial conditions to determine the constant vector A. The students clearly explain their reasoning and use appropriate mathematical techniques. There is no need for any deductions based on the given mark scheme.

\textbf{Response 4:}

**Solution 1: 8/8 marks**

Excellent solution. The student demonstrates a strong understanding of the concepts and applies Newton's second law correctly to derive the equation of motion. The solution of the differential equation is accurate, and the constant vector A is correctly determined using the initial conditions. The student also explains their steps in a concise and logical manner, making the solution easy to follow.

*   **Part a):** 6/6. All steps in the derivation are clear and correct.
*   **Part b):** 2/2. The student correctly applies the initial condition to find A.

**Solution 2: 8/8 marks**

Excellent solution. The student has a good grasp of the physics principles involved and correctly applies them to solve the problem. The solution is well-organized and easy to follow, with clear explanations and accurate calculations.

*   **Part a):** 6/6. The student accurately derives the equation of motion and finds the correct solution to the differential equation. The explanation of the terms in the solution is particularly well-done.
*   **Part b):** 2/2. The student correctly applies the initial condition to find the expression for A.

**Solution 3: 8/8 marks**

Excellent solution. The student demonstrates a solid understanding of the physics concepts and mathematical techniques required to solve this problem. The solution is clearly written and easy to follow, and the calculations are accurate.

*   **Part a):** 6/6. The student correctly derives the equation of motion and provides a clear and concise explanation of the solution to the differential equation.
*   **Part b):** 2/2. The student accurately determines the constant vector A using the initial condition.

**Overall feedback:** All three solutions are excellent and demonstrate a strong understanding of the problem. The students have used appropriate physics principles and mathematical techniques to arrive at the correct solutions. There are no significant differences in quality among the three solutions.

\textbf{Response 5:}

**Solution 1: 8/8 marks**

Excellent solution. The student demonstrates a strong understanding of the concepts and applies Newton's second law correctly to derive the equation of motion. The solution of the differential equation is accurate, and the constant vector A is correctly determined using the initial conditions. The student also explains their steps in a concise and logical manner, making the solution easy to follow.

*   **Part a):** 6/6. All steps in the derivation are clear and correct.
*   **Part b):** 2/2. The student correctly applies the initial condition to find A.

**Solution 2: 8/8 marks**

Excellent solution. The student has a good grasp of the physics principles involved and correctly applies them to solve the problem. The solution is well-organized and easy to follow, with clear explanations and accurate calculations.

*   **Part a):** 6/6. The student accurately derives the equation of motion and finds the correct solution to the differential equation. The explanation of the terms in the solution is particularly well-done.
*   **Part b):** 2/2. The student correctly applies the initial condition to find the expression for A.

**Solution 3: 8/8 marks**

Excellent solution. The student demonstrates a solid understanding of the physics concepts and mathematical techniques required to solve this problem. The solution is clearly written and easy to follow, and the calculations are accurate.

*   **Part a):** 6/6. The student correctly derives the equation of motion and provides a clear and concise explanation of the solution to the differential equation.
*   **Part b):** 2/2. The student accurately determines the constant vector A using the initial condition.

**Overall feedback:** All three solutions are excellent and demonstrate a strong understanding of the problem. The students have used appropriate physics principles and mathematical techniques to arrive at the correct solutions. There are no significant differences in quality among the three solutions.

\end{document}