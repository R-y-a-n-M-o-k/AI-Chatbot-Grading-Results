\documentclass[a4paper,11pt]{article}
\setlength\parindent{0pt}
\usepackage{amsmath, amssymb}
\usepackage[margin=1in,letterpaper]
{geometry} % decreases margins

\usepackage{setspace}
\setstretch{1.2}
\usepackage{spverbatim}

\begin{document}
    
\title{Gemini 1.5 Pro Grading (With Mark Scheme)}

\maketitle

This document compiles the Gemini responses when prompted to grade the solutions which have been generated by Gemini 1.5 Pro. The markscheme is given to Gemini in the prompt. This is the grading for the Electromagnetism (EM) questions. 

\section{EMT}

\subsection*{Question 1:}

\textbf{Response 1:}

**Solution 1: 7/8 marks**

Excellent explanation of the three types of magnetic materials and their responses to a bar magnet. The atomic structure explanation is clear and concise. To achieve full marks, explicitly mention the terms "magnetic susceptibility" or "magnetic dipole moment."

**Solution 2: 8/8 marks**

A comprehensive and accurate explanation of the three types of magnetic materials, their responses to a bar magnet, and the underlying atomic mechanisms. The mention of thermal motion disrupting the alignment of magnetic moments in paramagnetic materials is a noteworthy addition.

**Solution 3: 6/8 marks**

Good overall explanation of the three types of magnetic materials and their responses. However, the explanation lacks the explicit mention of "magnetic susceptibility" or "magnetic dipole moment," which are key concepts in determining the strength of the magnetic response. Additionally, while Lenz's Law is relevant for diamagnetism, it's not the sole factor determining the response of ferromagnetic and paramagnetic materials. 

\textbf{Response 2:}

**Solution 1: 7/8**

**General Feedback:** This solution is well-articulated and covers all the key points accurately. The student clearly understands the concepts of ferromagnetism, paramagnetism, and diamagnetism, and correctly explains their response to a bar magnet. The only minor issue is the lack of explicit mention of magnetic susceptibility.

**Solution 2: 6/8**

**General Feedback:** This solution is also well-written and accurately explains the three types of magnetic materials and their response to a bar magnet. However, it could be improved by providing more specific examples of paramagnetic and diamagnetic materials. The explanation of Lenz's Law in the context of diamagnetism could also be more detailed.

**Solution 3: 8/8**

**General Feedback:** This solution is excellent. It provides a clear and concise explanation of the three types of magnetic materials, their responses to a bar magnet, and the underlying atomic reasons for these responses. The student uses accurate terminology and demonstrates a strong understanding of the subject matter. The explanations are comprehensive, and the writing is well-organized and easy to follow. 


\textbf{Response 3:}

**Solution 1: 8/8 marks**

Excellent answer. It clearly names the three types of magnetic materials, explains the underlying mechanism for their magnetic behavior, and correctly states the response when brought near a bar magnet. The use of examples is also helpful.

**Solution 2: 8/8 marks**

Another excellent answer. Similar to Solution 1, it accurately identifies the three types of materials and explains the physical basis of their magnetic properties. The explanations are clear and concise, and the response to a bar magnet is correctly stated for each material type.

**Solution 3: 8/8 marks**

This solution also receives full marks. It correctly names the three types of materials and provides a detailed explanation of their magnetic behavior in response to an external magnetic field. The answer is well-organized and easy to understand.

**General Feedback:**

All three solutions demonstrate a strong understanding of the different types of magnetic materials and their properties. The explanations are clear, concise, and accurate. The students have used relevant terminology and concepts, such as magnetic susceptibility, magnetic dipole moment, and Lenz's law, to explain the different responses to a bar magnet. The use of examples further enhances the clarity of the explanations. Overall, all three solutions are excellent examples of how to answer this physics problem.

\textbf{Response 4:}

**Solution 1: Grade 8/8**

This solution is comprehensive and accurate. It clearly names the three types of magnetic materials, explains their response to a bar magnet, and correctly identifies the underlying atomic structure responsible for their behavior. The use of examples is also helpful in illustrating the concepts.

**Solution 2: Grade 8/8**

This solution is also comprehensive and accurate. It covers all the required points, including the names of the three types of magnetic materials, their response to a bar magnet, and the atomic-level explanations for their behavior. The mention of thermal motion's effect on paramagnetic materials is a noteworthy addition.

**Solution 3: Grade 7/8**

This solution is well-written and accurate in most aspects. It correctly identifies the three types of magnetic materials, their responses to a bar magnet, and the underlying electronic structure responsible for their behavior. However, it lacks the explicit mention of magnetic susceptibility or dipole moments, which are key factors in determining the strength of the magnetic response.

\textbf{Response 5:}

Solution 1: **7/8 marks**

* **General feedback:** This is a comprehensive and well-articulated explanation of the three types of magnetic materials. The student demonstrates a good understanding of the underlying concepts, including magnetic domains, the role of unpaired electrons, and the differing responses of the materials to an external magnetic field.
* **Mark deductions:** 1 mark deducted for not mentioning susceptibility.


Solution 2: **8/8 marks**

* **General feedback:** This is an excellent response that covers all the key points in a clear and concise manner. The student accurately explains the magnetic properties of ferromagnetic, paramagnetic, and diamagnetic materials, including their response to an external magnetic field. The student also demonstrates a good understanding of the role of atomic structure and electron behavior in determining these properties.
* **Mark deductions:** None


Solution 3: **6/8 marks**

* **General feedback:** The student demonstrates a good understanding of the basic concepts of ferromagnetism, paramagnetism, and diamagnetism, including the role of atomic dipoles and the behavior of electrons in an external magnetic field. However, the explanation lacks some details, such as the concept of magnetic susceptibility and its role in determining the strength of a material's response to a magnetic field.
* **Mark deductions:** 1 mark deducted for not mentioning magnetic susceptibility, 1 mark deducted for not mentioning differing susceptibility values determine response 



\subsection*{Question 2:}

\textbf{Response 1:}

**Solution 1: 5/5 marks**

Excellent answer. A clear and comprehensive explanation of the concept of free space, the wave equation, its physical interpretation, and the derivation of the speed of light.

**Solution 2: 5/5 marks**

A well-structured and accurate response. It covers all the key points concisely and effectively, including the definition of free space, the wave equation, its interpretation, and the derivation of the speed of light.

**Solution 3: 5/5 marks**

Another excellent answer. It provides a thorough explanation of the concept of free space, the wave equation, and its physical implications. The derivation of the speed of light is also clear and well-presented.

\textbf{Response 2:}

**Solution 1: 5/5**

**General Feedback:** This solution is excellent, demonstrating a strong understanding of the concept of free space and the wave equation. The explanation is clear, concise, and accurate, covering all the key points mentioned in the mark scheme.

**Solution 2: 4/5**

**General Feedback:** This solution is well-written and accurately explains the concept of free space and the wave equation. It could be improved by explicitly mentioning that light is an electromagnetic wave and by providing a more detailed explanation of how the equation leads to the speed of light.

**Solution 3: 5/5**

**General Feedback:** This solution is comprehensive and well-articulated. The student clearly understands the concept of free space, the wave equation, and the relationship between the equation and the speed of light. The explanation is clear, concise, and accurate, including all the key points from the mark scheme.

\textbf{Response 3:}

**Solution 1: 5/5 marks**

Excellent answer. It clearly defines free space, provides a concise and accurate interpretation of the wave equation, and correctly explains how to derive and calculate the speed of light.

**Solution 2: 5/5 marks**

Another excellent answer. Similar to Solution 1, it provides a clear definition of free space and correctly interprets the wave equation. The derivation and calculation of the speed of light are also accurate.

**Solution 3: 5/5 marks**

This solution also receives full marks. It accurately defines free space and interprets the wave equation. The derivation of the speed of light is clear and concise, and the final calculation is correct.

**General Feedback:**

All three solutions demonstrate a strong understanding of free space and its relationship to the wave equation for electric fields. The explanations are clear, concise, and accurate. The students have used relevant terminology and concepts to interpret the equation and derive the speed of light. Overall, all three solutions are excellent examples of how to answer this physics problem.

\textbf{Response 4:}

**Solution 1: Grade 5/5**
This solution is excellent. It provides a clear and detailed explanation of free space, the wave equation, its physical interpretation, and how the speed of light is derived from it. The numerical calculation of the speed of light is accurate.

**Solution 2: Grade 5/5**
This solution is also excellent. It covers all the required points in a concise and accurate manner. The explanation of the wave equation and its connection to the speed of light is well-articulated.

**Solution 3: Grade 5/5**
This solution is comprehensive and accurate. It explains the concept of free space, the wave equation, its physical interpretation, and the derivation of the speed of light in a clear and concise manner. The inclusion of the general form of the wave equation is a good addition.


\textbf{Response 5:}

Solution 1: 5/5 marks. 

This is a very well written and complete answer. It defines free space and explains how the equation given is a wave equation, which governs wave propagation. It then makes the clear link between the equation given and the speed of light, before calculating it. It also briefly mentions light is an EM wave.

Solution 2: 4/5 marks

The response provides a good definition of free space and accurate interpretation of the wave equation. It explains how the equation governs wave propagation and derives the speed of light from it, including calculation. However, there is no mention that light is an EM wave, which is the main reason why this equation is relevant in the context of electromagnetism.

Solution 3: 5/5 marks.

This is a very well written and complete answer. It defines free space and explains how the equation given is a wave equation, which governs wave propagation. It then makes the clear link between the equation given and the speed of light, before calculating it. It also briefly mentions light is an EM wave.

\subsection*{Question 3:}

\textbf{Response 1:}

**Solution 1: 13/14 marks**

Excellent solution with a clear and detailed explanation of the setup and integration process. The physical interpretation of the result in the limit \(d \gg L\) is also accurate. To achieve full marks, please correct the final expression for the electric field by converting numerical constant to the exact answer \(\frac{1}{2}\).

**Solution 2: 13/14 marks**

Overall, a very good solution with a well-explained approach. The steps are clear, and the integrals are set up correctly. The physical interpretation of the limit \(d \gg L\) is also explained well. To achieve full marks, please correct the final expression for the electric field by converting numerical constant to the exact answer \(\frac{1}{2}\). 

**Solution 3: 10/14 marks**

Good solution. However, the numerical calculation leads to an inexact result for \( E_y \). To get the exact answer, keep all the constants in terms of \( \pi \) and \( k_e \) while integrating.
Additionally, while stating Coulomb's law, the correct expression is \( d\mathbf{E} = k_e \frac{dq}{r^2} \hat{r} \), where \( \hat{r} \) is the unit vector pointing from the charge element to the point of interest. Also, the vertical component of the electric field is along \( d \), or the z-axis, not the y-axis.

To improve, work through the integral using the constants and unit vectors, which will lead to the exact answer. Furthermore, the correct physical interpretation should state that the line charge appears like a point charge with charge \( Q = \lambda L \), not just an unspecified point charge.


\textbf{Response 2:}

**Solution 1: 10/14**

**General Feedback:** The solution demonstrates a good understanding of the problem and the use of Coulomb's law to find the electric field. The setup of the integral is correct, and the integration is carried out correctly. However, the final expression for the electric field contains numerical approximations, which is not ideal for a physics problem. It would be better to leave the answer in exact form. Additionally, there's no explicit mention of the x-component of the electric field, which is zero due to symmetry. The limit as d >> L is calculated correctly, and the physical interpretation is accurate.

**Solution 2: 12/14**

**General Feedback:** The solution is well-structured and clearly presented. The explanation of the problem setup and the use of Coulomb's law is very clear. The integration is done correctly, and the final expression for the electric field is given in exact form. The limit as d >> L is calculated correctly, and the physical interpretation is accurate. However, like Solution 1, there's no explicit mention of the x-component of the electric field.

**Solution 3: 14/14**

**General Feedback:** This is an excellent solution. The explanation is clear and concise, and the mathematical steps are well-organized. The student correctly identifies the vertical component of the electric field as the only non-zero component due to symmetry. The integration is performed accurately, and the final expression is given in exact form. The limit as d >> L is calculated correctly, and the physical interpretation is clear and accurate. The student correctly identifies the factor of 0.564 * sqrt(pi) as arising from the integration and providing the exact proportionality in this specific geometry. This solution demonstrates a thorough understanding of the problem and its physical implications.


\textbf{Response 3:}

**Solution 1: 13/14 marks**

The solution is mostly correct and well-presented, with clear steps and explanations. However, there is a calculation error in the final evaluation of the integral. The correct final form for \( E \) should be:

```
E = \frac{1}{4 \pi \epsilon_0} \frac{\lambda}{d}  [  ( -1+\frac{d}{\sqrt{d^2+L^2}}) \hat{x} + ( \frac{L}{\sqrt{d^2+L^2}} ) \hat{z} ]
```
and in the limit as \( d \gg L \) it is:
```
E = \frac{1}{4 \pi \epsilon_0} \frac{\lambda L}{d^2} \hat{z}
```
The physical interpretation of the result is correct, but the precise numerical value is incorrect.

**Solution 2: 14/14 marks**
This solution is excellent. It uses Coulomb's law to integrate and find the electric field, and the integral setups for both x- and z-components are correct. The integrals are evaluated correctly, leading to the right expression for \underline{E}. The simplification in the limit \( d \gg L \) is also done correctly, and the physical interpretation is spot on.

**Solution 3: 13/14 marks**

This solution is well-organized and easy to follow, demonstrating a good understanding of the problem. However, there is a calculation error in the final evaluation of the integral. The correct final form for \( E_y \) should be:
```
(E_y) = \frac{1}{4 \pi \epsilon_0} \frac{\lambda}{d} \frac{L}{\sqrt{d^2+L^2}}
```
The solution is almost perfect, except for the numerical factor in the final result, which is a result of the calculation error.

**General Feedback:**

All three solutions demonstrate a good understanding of how to solve the problem of finding the electric field due to a line charge distribution. They all correctly use Coulomb's law, set up the integrals correctly, and provide a physical interpretation of the results. The main difference is in the accuracy of the final calculation. Solutions 1 and 3 both have calculation errors, while solution 2 is completely correct.

\textbf{Response 4:}

**Solution 1: 10/14 marks**

* **Overall Feedback:** The solution demonstrates a good understanding of the physics concepts and problem-solving approach. The integral setup is correct, and the final expression for the electric field is accurate. However, the solution lacks some details and uses an approximation for the final answer, rather than the exact solution.
* **Specific Feedback:** The solution would benefit from a clearer explanation of why the horizontal components of the electric field cancel out. Additionally, instead of using a numerical approximation, the final answer should be expressed in terms of fundamental constants and given parameters for better precision.

**Solution 2: 13/14 marks**

* **Overall Feedback:**  This solution is very well-structured and explained. It shows a clear understanding of the problem and the underlying physics principles. The integration is carried out correctly, and the final expression for the electric field is accurate. The simplification for the case \(d \gg L\) is also done correctly and well-explained.
* **Specific Feedback:** The only minor improvement could be to provide a more explicit connection between the final simplified expression and Coulomb's law for a point charge.

**Solution 3: 12/14 marks**

* **Overall Feedback:** This solution is also well-written and demonstrates a solid understanding of the physics involved. The integral setup and evaluation are correct, and the simplification for \( d \gg L \) is done properly. The physical interpretation of the result is also explained well.
* **Specific Feedback:** The solution could be improved by using the standard notation for Coulomb's constant, \( \frac{1}{4\pi\epsilon_0} \), instead of \( k_e \), and by providing the exact simplified expression rather than a numerical approximation. Additionally, a clearer explanation of why only the vertical component of the electric field matters would be beneficial.


\textbf{Response 5:}

Absolutely! Here's a breakdown of the grading and feedback for each student's solution:

**Solution 1:**

* **Mark:** 12/14
* **General Feedback:** The student demonstrates a strong understanding of the problem and its solution. The integral setup is correct, and the final expression for the electric field is well-derived. The simplification in the limit of \(d \gg L\) is accurate, and the physical interpretation is spot on. The only minor error is the use of a numerical approximation instead of the exact constant (which should be \(1/\sqrt{4\pi}\)). Additionally, the student did not split the electric field into x and z components as indicated in the markscheme. However, this is only a minor error because they are still able to correctly solve the problem. 

**Solution 2:**

* **Mark:** 13/14
* **General Feedback:** The student presents a very thorough solution with clear explanations of each step. The use of vector components and symmetry arguments is excellent. The integral setup and evaluation are correct. The limiting case and physical interpretation are well-articulated. The only minor issue is the same numerical approximation used in Solution 1.

**Solution 3:**

* **Mark:** 11/14
* **General Feedback:** This solution is also well-done, with a good grasp of the problem and solution strategy. The use of Coulomb's law and integration is appropriate. The final expression for the electric field is correct, and the simplification in the limit \(d \gg L\) is accurate. However, there are some minor deductions:
    * The constant \(k_e\) should be related back to \( \epsilon_0 \).
    * The use of a numerical approximation is not ideal.
    * The physical interpretation could be more detailed to explain why the line charge resembles a point charge at large distances.
    * The student did not split the electric field into x and z components as indicated in the markscheme. However, this is only a minor error because they are still able to correctly solve the problem. 

**Overall Comments:**

All three students demonstrate a solid understanding of electrostatics and the ability to solve a non-trivial problem. Their solutions are well-organized and clearly presented. The minor errors and omissions are relatively insignificant in comparison to the overall quality of their work.  


\subsection*{Question 4:}

\textbf{Response 1:}

**Solution 1: 11/11 marks**

Excellent solution. Clear and well-explained steps for all parts of the problem. Correct use of Ampere's Law, Faraday's Law, and Lenz's Law. The calculations are accurate and the final answer is in the correct form.

**Solution 2: 7/11 marks**

Part (a): 2/5 marks. Incorrect calculation of magnetic flux. The magnetic flux through the loop does depend on the distance \( d \).
Part (b): 2/5 marks. The calculation of emf is incorrect due to the error in part (a). Lenz's Law is applied correctly.
Part (c): 1/1 mark. Correctly states that there is no emf when the loop moves downward.

To improve, review the calculation of the magnetic flux through the loop. The integral setup is correct, but the evaluation is not.

**Solution 3: 9/11 marks**

Part (a): 4/5 marks. The expression for magnetic flux is correct up to the natural logarithm term. The simplified form is incorrect. 
Part (b): 4/5 marks. The expression for emf is correct, and Lenz's Law is applied correctly. However, the justification for the counterclockwise direction of the current using the right-hand rule is not clear.
Part (c): 1/1 mark. Correctly states that there is no emf when the loop moves downward.

To improve, review the simplification of the natural logarithm term in the magnetic flux expression. Additionally, provide a clearer explanation for the direction of the induced current using the right-hand rule.


\textbf{Response 2:}

**Solution 1: 10/11**

*   **Part (a): 5/5**  The student correctly sets up and evaluates the integral for the magnetic flux, showing a clear understanding of Ampère's law and the concept of flux.
*   **Part (b): 4/5** The student correctly uses Faraday's law and Lenz's law to find the emf and the direction of the induced current. However, they make a slight error in substituting \(r = d + vt\) in the final expression for the emf.
*   **Part (c): 1/1** The student correctly explains that there is no emf induced when the loop moves downwards because the flux through the loop does not change.

**General Feedback:** Overall, this is a very good solution with a minor error in the final expression of part (b). The student demonstrates a strong understanding of the relevant concepts and applies them correctly to solve the problem.

**Solution 2: 4/11**

*   **Part (a): 0/5** The integral is set up incorrectly and the final answer is wrong.
*   **Part (b): 2/5** Faraday's law was used but the differentiation was incorrect. Lenz's law was not applied to determine the direction of the current.
*   **Part (c): 2/1** While the conclusion is correct, the reasoning that the flux is independent of \(d\) is wrong.

**General Feedback:** This solution contains several errors and demonstrates a lack of understanding of how to apply Ampère's law and Faraday's law. The student should review these concepts and practice solving similar problems.

**Solution 3: 11/11**

*   **Part (a): 5/5** The student correctly sets up and evaluates the integral for the magnetic flux, showing a clear understanding of Ampère's law and the concept of flux. The student also uses the natural log notation as an alternative to ln, which is still correct.
*   **Part (b): 5/5** The student correctly uses Faraday's law and Lenz's law to find the emf and the direction of the induced current. The explanation is clear and concise.
*   **Part (c): 1/1** The student correctly explains that there is no emf induced when the loop moves downwards because the flux through the loop does not change.

**General Feedback:** This is an excellent solution that demonstrates a thorough understanding of the problem and the relevant concepts. The student's explanations are clear, concise, and accurate, and they have correctly applied the relevant laws and principles to arrive at the correct answers.


\textbf{Response 3:}

**Solution 1: 10/11 marks**

This solution is well-structured and demonstrates a good understanding of the problem. The calculations for the magnetic flux and emf are mostly correct. However, there's a minor error in the calculation of the emf. The correct expression for the emf is:

```
\mathcal{E} = \frac{\mu_0 I L^2 v}{2\pi r(r+L)}
```

The rest of the solution is correct, including the application of Lenz's law to determine the direction of the induced current.

**Solution 2: 2/11 marks**

This solution contains a major conceptual error. The magnetic flux through the loop does depend on the distance \( d \) from the wire. The correct expression for the flux is:

```
\ Phi = \frac{\mu_0 I L}{2\pi}\ln ( \frac{d+L}{d} )
```

Because of this error, the rest of the solution is also incorrect. The emf is not zero, and there would be an induced current if the loop moves to the right.

**Solution 3: 11/11 marks**

This solution is excellent. It correctly calculates the magnetic flux, emf, and the direction of the induced current. The explanation is clear, concise, and accurate. The student has demonstrated a strong understanding of electromagnetic induction and Faraday's law.

**General Feedback**

Solutions 1 and 3 demonstrate a good understanding of the problem, with only minor errors in solution 1. Solution 2 needs significant improvement in understanding the concept of magnetic flux and its dependence on the distance from the current-carrying wire.

\textbf{Response 4:}
**Solution 1: 10/11 marks**

* **Overall Feedback:** This is a well-structured solution that demonstrates a good understanding of the underlying concepts. The calculations are correct, and the explanations are clear. The student correctly applies Ampere's law, Faraday's law, and Lenz's law to solve the problem.

* **Specific Feedback:**
    * **Part a)** (5/5): The calculation of the flux is correct and well-explained.
    * **Part b)** (4/5): The calculation of the emf is correct. However, the expression for the emf could be simplified further by substituting  \(r = d + vt\).
    * **Part c)** (1/1): The explanation of why there is no emf when the loop moves downward is correct.

**Solution 2: 2/11 marks**

* **Overall Feedback:** This solution contains several errors and misunderstandings. The calculation of the flux is incorrect, and the student fails to apply Faraday's law correctly to find the emf. The explanation for part c) is also flawed.

* **Specific Feedback:**
    * **Part a)** (0/5): The calculation of the flux is incorrect. The student made an error in the integration process.
    * **Part b)** (0/5): The calculation of the emf is incorrect, and the reasoning based on the flux being independent of  \(d\) is flawed.
    * **Part c)** (2/1): While the conclusion that there is no emf is correct, the reasoning is incorrect as the flux does depend on d (as seen in the correct solutions). However, it does not change when moving the loop down.

**Solution 3: 11/11 marks**

* **Overall Feedback:** This is an excellent solution that is well-written and easy to follow. The student demonstrates a strong understanding of the relevant physics concepts and applies them correctly to solve the problem.

* **Specific Feedback:**
    * **Part a)** (5/5): The calculation of the flux is correct and clearly explained.
    * **Part b)** (5/5): The calculation of the emf is correct, and the use of Faraday's law is well-demonstrated. The direction of the induced current is also correctly determined using Lenz's law and the right-hand rule.
    * **Part c)** (1/1): The explanation for why there is no emf when the loop moves downward is correct and concise.

\textbf{Response 5:}

**Solution 1:**

* **Mark:** 9/11
    * Part a: 4/5. Correct setup and evaluation of the integral, but 1 mark deducted for not explicitly showing the vector nature of the magnetic field and area elements in the integral.
    * Part b: 5/5. Full marks for the correct derivation of the emf, clear explanation using Lenz's law, and determination of the current direction.
    * Part c: 1/1. Full marks for the correct explanation of why no emf is induced when the loop moves downward.
* **General Feedback:** The solution is well-organized and clearly written. The student demonstrates a strong understanding of electromagnetism. The minor deduction in part (a) is due to the lack of explicit vector notation in the integral.


**Solution 2:**

* **Mark:** 3/11
    * Part a: 0/5. The calculation of the flux is incorrect. The flux should depend on the distance \( d \) from the wire. 
    * Part b: 2/5. The student correctly states Faraday's law but fails to apply it correctly due to the error in part (a). The direction of the current is not discussed.
    * Part c: 1/1. Full marks for the correct explanation of why no emf is induced when the loop moves downward.
* **General Feedback:** The student demonstrates some understanding of the concepts but makes a significant error in calculating the magnetic flux. This error propagates to part (b), leading to an incorrect answer. The student should review the calculation of magnetic flux through a loop due to a long straight wire.


**Solution 3:**

* **Mark:** 11/11
    * Part a: 5/5. Full marks for the correct setup and evaluation of the integral, clear presentation of the final expression for the flux, and correct use of vector notation.
    * Part b: 5/5. Full marks for the correct derivation of the emf using Faraday's law, correct explanation of Lenz's Law, and accurate determination of the current direction.
    * Part c: 1/1. Full marks for the correct explanation of why no emf is induced when the loop moves downward.
* **General Feedback:** An excellent and well-explained solution. The student demonstrates a thorough understanding of electromagnetic induction and a clear grasp of the problem's concepts.

\subsection*{Question 5:}

\textbf{Response 1:}

**Solution 1: 3/11 marks**

Part (a): 0/8 marks. The solution incorrectly calculates the current through the ammeter. The analysis of the circuit is incorrect, and the equivalent resistance calculation is wrong. The method for calculating the total current is also incorrect.
Part (b): 3/3 marks. The calculation of the energy delivered is correct, based on the incorrect current found in part (a).

To improve, carefully analyze the circuit and apply Kirchhoff's laws correctly to find the currents in each branch. Then, use the correct current to calculate the energy delivered.

**Solution 2: 8/11 marks**

Part (a): 8/8 marks. The solution correctly calculates the current through the ammeter using Ohm's law and the concept of equivalent resistance.
Part (b): 0/3 marks. The calculation of energy delivered is incorrect. The power delivered by the battery is not equal to the total power dissipated by the resistors.

To improve, calculate the power delivered to each component separately and sum them up to find the total power delivered by the battery. Then, use this correct power value to calculate the energy delivered.

**Solution 3: 6/11 marks**

Part (a): 3/8 marks. While the solution shows the correct approach by applying the concept of equivalent resistance and Ohm's law, the calculation of the equivalent resistance of the two parallel branches is incorrect. Additionally, the voltage drop across the parallel branches is incorrect.
Part (b): 3/3 marks. The calculation of the energy delivered is correct, based on the incorrect current found in part (a).

To improve, revisit the calculation of the equivalent resistance of the two parallel branches, considering that one branch also includes a voltage source. Ensure the correct voltage drop across the parallel branches to accurately determine the current and subsequent energy calculations.


\textbf{Response 2:}

**Solution 1: 1/11**

*   **Part (a): 0/8** Incorrectly assumes that the \(2\,V\) battery has a zero internal resistance and misinterprets the equivalent resistance of the two \(2\,\Omega\) resistors in parallel.
*   **Part (b): 1/3** Incorrectly uses the total current to calculate the power delivered by the \(12\,V\) battery.

**General Feedback:** This solution demonstrates a lack of understanding of basic circuit analysis principles. The student should review Ohm's law, Kirchhoff's laws, and the concept of equivalent resistance.

**Solution 2: 4/11**

*   **Part (a): 2/8** Identifies parallel and series resistor combinations but incorrectly calculates the voltage across the series combination on the right.
*   **Part (b): 2/3** Incorrectly assumes that the total current is the current through the \(12\,V\) battery.

**General Feedback:** This solution shows some understanding of circuit analysis but makes errors in applying Ohm's law and Kirchhoff's voltage law. The student should practice solving similar problems to reinforce their understanding of these concepts.

**Solution 3: 11/11**

*   **Part (a): 8/8** Correctly analyzes the circuit, calculates equivalent resistances, and applies Ohm's law to find the current through the ammeter.
*   **Part (b): 3/3** Correctly calculates the power delivered by the \(12\,V\) battery and the energy delivered in \(4\) seconds.

**General Feedback:** This is an excellent solution that demonstrates a thorough understanding of circuit analysis and the relevant concepts. The student clearly explains each step of the calculation and arrives at the correct answers.


\textbf{Response 3:}

**Solution 1: 5/11 marks**

The analysis of the circuit is flawed in this solution. The 2V battery creates an additional loop that needs to be considered using Kirchhoff's Voltage Law. Also, the assumption of negligible internal resistance for the 2V battery is not stated explicitly. The subsequent calculations for equivalent resistance and current are incorrect due to these errors. As a result, the energy calculation in part b) is also incorrect.

**Solution 2: 8/11 marks**

This solution correctly identifies the need to calculate the total resistance and uses the correct formulas for series and parallel combinations. However, it overlooks the voltage drop across the internal resistance of the 2V battery when calculating the voltage across the series combination on the right. This leads to a slightly incorrect value for the total resistance and, consequently, the current. The energy calculation in part b) is also slightly off due to the error in current.

**Solution 3: 11/11 marks**

This solution is the most comprehensive and accurate of the three. It correctly identifies the parallel and series combinations in the circuit, accounts for the voltage drop across the internal resistance of the 2V battery, and uses the current division rule effectively. The calculation of total resistance and current is accurate. The power and energy calculations in part b) are also correct.

**General Feedback**

All three solutions attempt to solve the problem using Ohm's Law and the concept of power in electrical circuits. However, Solution 1 makes a significant error in circuit analysis, and Solution 2 makes a minor error in voltage calculation. Solution 3 stands out due to its correct analysis, accurate calculations, and clear explanations.

**To improve, students should focus on:**

* **Circuit Analysis:** Carefully identify all circuit elements and their connections before applying circuit laws.
* **Kirchhoff's Laws:**  Ensure correct application of Kirchhoff's Voltage Law (KVL) and Kirchhoff's Current Law (KCL) to all relevant loops and nodes in the circuit.
* **Internal Resistance:** Account for the internal resistance of batteries, especially when it is not explicitly stated to be negligible.
* **Double-Checking Calculations:** Always double-check calculations to avoid arithmetic errors that can propagate through the solution.

\textbf{Response 4:}

Solution 1:
a) 0/8 marks. The solution misinterprets the parallel combinations in the circuit, leading to an incorrect calculation of the equivalent resistance. Also, the solution did not make use of Kirchhoff Circuit Laws as required in the mark scheme.
b) 0/3 marks. The calculation of energy delivered by the 12V battery is incorrect due to the wrong value of the total current obtained in part a). 

Overall Feedback: Revise the concept of parallel and series connections and Kirchhoff Circuit Laws. Practice applying them to circuit analysis problems.

Solution 2:
a) 3/8 marks. The solution correctly identifies the parallel and series combinations of resistors and calculates the total resistance of the circuit. However, it neglects the voltage drop across the internal resistance of the 2V battery when calculating the voltage across the series combination on the right side. Also, the solution did not make use of Kirchhoff Circuit Laws as required in the mark scheme.
b) 0/3 marks. The calculation of energy delivered by the 12V battery is incorrect due to the wrong value of the total current obtained in part a).

Overall Feedback: Review the concept of internal resistance in batteries and how it affects the voltage across the circuit. Also, the solution did not make use of Kirchhoff Circuit Laws as required in the mark scheme.

Solution 3:
a) 7/8 marks. The solution is mostly correct and follows the correct approach using Ohm's law and equivalent resistance calculations. The only error is in calculating the voltage across the 2 Ohm resistor in series with the 2V battery. It should be 10V (the voltage of the 12V battery minus the 2V drop across the battery), not 12V.
b) 3/3 marks. This part is entirely correct and follows from the (slightly incorrect) answer in part a).

Overall Feedback: Excellent solution! The only minor correction is needed in the voltage calculation in part a). However, the solution did not make use of Kirchhoff Circuit Laws as required in the mark scheme.



\textbf{Response 5:}

**Solution 1:**
* **a)** 0/8 marks. The solution is completely incorrect. It is important to use Kirchhoff's laws and solve the equations to find the currents in the circuit.
* **b)** 0/3 marks. The calculation of the power delivered by the battery is incorrect, leading to an incorrect calculation of the energy delivered.
* **Overall feedback:** This solution demonstrates a lack of understanding of basic circuit analysis principles and the application of Kirchhoff's laws. The student should review these concepts and practice solving similar problems.

**Solution 2:**
* **a)** 3/8 marks. The student correctly identifies the parallel and series combinations of resistors. They also apply Ohm's law to find the total current. However, they incorrectly apply the voltage across the series combination in the middle branch.
* **b)** 2/3 marks. The student correctly uses the power formula and calculates the energy delivered by the battery. However, the power calculation is incorrect due to the error in part a).
* **Overall feedback:** This solution shows a good grasp of basic circuit analysis principles. The student should be careful when applying voltage drops across components and double-check their calculations.

**Solution 3:**
* **a)** 8/8 marks. The solution is correct. The student uses Kirchhoff's laws and Ohm's law to calculate the total current seen by the ammeter. They correctly identified the parallel and series combinations of resistors and solved the equations to find the answer.
* **b)** 3/3 marks. The solution is correct. The student correctly calculated the power delivered by the battery and the energy delivered in 4 seconds.
* **Overall feedback:** This solution is excellent. The student demonstrates a strong understanding of circuit analysis and applies the relevant principles correctly. The calculations are accurate, and the solution is well-presented.

**Note:** All solutions have ignored the internal resistance of the 12 V battery which is not a valid assumption in general. The question has stated to consider the batteries as ideal to simplify calculations, so this will not affect the marking.

\subsection*{Question 6:}

\textbf{Response 1:}

**Solution 1: 3/5 marks**

Part (a): 0/3 marks. The solution incorrectly doubles the electric field due to a single plate to find the field between the plates. The electric field between oppositely charged plates is simply the field due to one plate.
Part (b): 2/2 marks. The calculation of the acceleration is correct.

To improve, review the calculation of the electric field between two oppositely charged infinite planes.

**Solution 2: 5/5 marks**

Part (a): 3/3 marks. The solution correctly calculates the electric field between the plates using the relationship between electric field, potential difference, and separation.
Part (b): 2/2 marks. The calculation of the acceleration is correct, using the correct electric field value from part (a).

This is a well-presented solution.

**Solution 3: 5/5 marks**

Part (a): 3/3 marks. The solution correctly calculates the electric field between the plates using the relationship between electric field, potential difference, and separation.
Part (b): 2/2 marks. The calculation of the acceleration is correct, using the correct electric field value from part (a).

This is a well-presented solution.


\textbf{Response 2:}

**Solution 1: 1/5**

*   **Part (a): 0/3** While the equation for the electric field due to a single infinite plane is correct, the student incorrectly doubles it to find the electric field between two plates of opposite charge.
*   **Part (b): 1/2** Correctly uses Newton's second law and the formula for electric force to calculate the acceleration.

**General Feedback:** This solution demonstrates a misunderstanding of how electric fields combine between oppositely charged plates. The student should review the concept of superposition of electric fields.

**Solution 2: 3/5**

*   **Part (a): 2/3** Correctly calculates the magnitude of the electric field using the relationship between potential difference, electric field, and distance. However, the direction of the electric field is not mentioned.
*   **Part (b): 1/2** Correctly uses Newton's second law and the formula for electric force to calculate the acceleration.

**General Feedback:** This solution is mostly correct, but it lacks the direction of the electric field in part (a). The student should remember that electric fields are vector quantities and have both magnitude and direction.

**Solution 3: 5/5**

*   **Part (a): 3/3** Correctly calculates the magnitude of the electric field and specifies its direction from the positive to the negative plate.
*   **Part (b): 2/2** Correctly uses Newton's second law and the formula for electric force to calculate the acceleration.

**General Feedback:** This is an excellent solution that clearly and correctly addresses both parts of the problem. The student demonstrates a strong understanding of the relevant concepts and applies them accurately to arrive at the correct answers.


\textbf{Response 3:}

**Solution 1: 3/5 marks**

This solution correctly calculates the electric field in part a) using the relationship between potential difference, electric field, and distance. However, it incorrectly assumes that the electric field between two plates is double that of a single plate. This is only true for infinite sheets of charge. The student also does not specify the direction of the electric field. Part b) is correct, though.

**Solution 2: 5/5 marks**

This is an excellent solution. The student correctly derives the electric field between the plates using the formula for a single infinite plane and the fact that the fields add up between the plates. They also correctly calculate the acceleration of the object using Newton's second law and the electric force.

**Solution 3: 5/5 marks**

This solution is also correct and well-explained. The student uses the relationship between electric field, potential difference, and distance to find the electric field and then correctly applies Coulomb's law and Newton's second law to find the acceleration.

**General Feedback**

All three solutions show a good understanding of electrostatics and Newton's laws of motion. The main difference is in the approach to calculating the electric field. Solution 1 makes a conceptual error in assuming that the electric field between two plates is double that of a single plate. Solutions 2 and 3 are both correct and demonstrate different valid approaches to finding the electric field.

\textbf{Response 4:}

**Solution 1: Grade 3/5**

This solution starts off correctly by stating the formula for the electric field due to an infinite plane of charge, but it is not necessary to use this formula here as we are given the potential difference and the distance between the plates. The electric field between the plates is correctly calculated, however, no mention of the direction of the field is provided which should be included. In part b), the student correctly calculates the acceleration. 


**Solution 2: Grade 3/5**

This solution correctly identifies that the electric field between two infinite parallel planes is uniform and can be calculated by \( E = \frac{V}{d} \). It also correctly calculates the acceleration of the object using Newton's second law. However, the student should also mention that the direction of the field is from the positive plate to the negative plate. So 1 mark is deducted for part a).

**Solution 3: Grade 5/5**

This solution correctly calculates the electric field using the relationship between potential difference and separation, and it mentions the direction of the field from the positive plate to the negative plate. The acceleration of the object is also calculated correctly using Newton's second law and Coulomb's law.


\textbf{Response 5:}

Solution 1: 3/5 marks

General feedback: This solution demonstrates a good understanding of the concept of electric field between charged plates. However, the use of the formula \( E = \frac{\sigma}{\epsilon_0} \) is unnecessary and could lead to confusion since \( \sigma \) is not given. The calculation for the acceleration is correct.

Solution 2: 3/5 marks

General feedback: This solution correctly uses the relation between potential difference, electric field, and distance to find the electric field. However, the additional step of calculating the electric field due to a single plate and then doubling it is not needed, as the formula \( E = \frac{V}{d} \) directly gives the electric field between the plates. The calculation for the acceleration is correct.

Solution 3: 5/5 marks

General feedback: This solution is clear, concise, and correct. It directly uses the relevant formula \( E = \frac{V}{d} \) to find the electric field, and then correctly applies Newton's second law to calculate the acceleration of the object. The steps are well-explained, and the final answers are presented clearly with appropriate units.

\subsection*{Question 7:}

\textbf{Response 1:}

**Solution 1: 6/6 marks**

Excellent solution. Clear and well-explained steps for all parts of the problem. Correct use of the formulas for resonant frequency, impedance, inductive reactance, and voltage drop across the inductor. The phase relationship between the source e.m.f and voltage across the inductor is also well-explained.

**Solution 2: 5/6 marks**

Part (a): 1/1 mark. The resonant frequency is calculated correctly.
Part (b): 3/3 marks. The expression for the voltage drop across the inductor is correct and well-explained.
Part (c): 1/2 marks. The phase relationship at low frequencies is explained correctly, but the explanation for high frequencies lacks clarity. While it's correct that the circuit behaves more inductively at high frequencies, the reason for the voltage across the inductor leading the source e.m.f is not due to the decrease in capacitive reactance becoming negligible, but rather the increasing dominance of the inductive reactance.

To improve, clarify the explanation for the phase relationship at high frequencies, emphasizing the increasing dominance of inductive reactance over capacitive reactance.

**Solution 3: 6/6 marks**

Excellent solution. Clear and well-explained steps for all parts of the problem. Correct use of the formulas for resonant frequency, impedance, inductive reactance, and voltage drop across the inductor. The phase relationship between the source e.m.f and voltage across the inductor is also well-explained.


\textbf{Response 2:}

**Solution 1: 5/6**

*   **Part (a): 1/1** Correctly calculates the resonant frequency.
*   **Part (b): 3/3** Correctly derives the expression for the voltage drop across the inductor.
*   **Part (c): 1/2** Correctly identifies the phase relationship at low frequencies but does not explicitly state that the phase difference is approximately 90 degrees.

**General Feedback:** This is a well-written and accurate solution. The only minor issue is the lack of explicit mention of the approximate 90-degree phase shift at both high and low frequencies.

**Solution 2: 4/6**

*   **Part (a): 1/1** Correctly calculates the resonant frequency.
*   **Part (b): 2/3** Correctly identifies the necessary formulas but does not combine them to derive the final expression for the voltage drop across the inductor.
*   **Part (c): 1/2** Correctly identifies the phase relationship at low frequencies but incorrectly states that the voltage across the inductor lags the source e.m.f. at high frequencies.

**General Feedback:** This solution shows a good understanding of the concepts but needs improvement in combining formulas in part (b) and clarifying the phase relationship at high frequencies in part (c).

**Solution 3: 6/6**

*   **Part (a): 1/1** Correctly calculates the resonant frequency.
*   **Part (b): 3/3** Correctly derives the expression for the voltage drop across the inductor.
*   **Part (c): 2/2** Correctly describes the phase relationship between the source e.m.f and the voltage across the inductor in both limits, including the approximate 90-degree phase shift.

**General Feedback:** This is an excellent solution that is well-written, accurate, and comprehensive. The student demonstrates a strong understanding of RLC circuits and their behavior at different frequencies.


\textbf{Response 3:}

**Solution 1: 5/6 marks**

This solution is mostly correct and well-structured. The student correctly calculates the resonant frequency, impedance, and current. However, the phase shift in the current expression should be negative (\( \cos(\omega t - \phi) \)) since the current lags behind the voltage in an inductive circuit. The descriptions of the phase relationship are correct.

**Solution 2: 6/6 marks**

This is an excellent solution. The student correctly calculates the resonant frequency and derives the expression for the voltage drop across the inductor. The phase relationship descriptions are accurate and detailed, explaining the behavior of the circuit at high and low frequencies concerning the resonant frequency.

**Solution 3: 6/6 marks**

This solution is also correct and well-explained. The student correctly calculates the resonant frequency and derives the expression for the voltage drop across the inductor, including the correct phase shift term. The explanation of the phase relationship is clear and concise.

**General Feedback:**

All three solutions demonstrate a good understanding of RLC circuits and their behavior at different frequencies. The main difference lies in the accuracy of the phase shift term in the current expression in solution 1. Solutions 2 and 3 are both excellent and provide detailed explanations of the phase relationship.

\textbf{Response 4:}

**Solution 1: Grade 6/6**

This is an excellent solution. The resonant frequency is calculated correctly in part a). In part b), the voltage drop across the inductor is also expressed correctly, showing good understanding of impedance and phasors. The phase relationships are explained clearly for both low and high-frequency limits in part c).

**Solution 2: Grade 5/6**

This solution is well-presented and demonstrates a good understanding of the underlying concepts. The resonant frequency is calculated correctly. The expression for the voltage drop across the inductor is also correct, although it would be better to explicitly include the impedance \(Z\) in the formula. The phase relationships are accurately described for both low and high frequencies. However, a more detailed explanation of why the voltage leads or lags at certain frequencies would improve this section.

**Solution 3: Grade 5/6**

This solution is well-written and demonstrates a good understanding of the concepts involved. The resonant frequency is calculated correctly, and the voltage drop across the inductor is also derived correctly. The phase relationships are accurately described. However, the use of cosine in the voltage drop expression while the mark scheme mentions sine (or any other trigonometric function as long as the phase difference is accounted for) results in a deduction of 1 mark.

\textbf{Response 5:}

Solution 1: 6/6 marks

General feedback: This is a very good and comprehensive answer. All expressions are correct and well explained. There is a minor error in the minus sign for the emf in part b), however, since the question asks for the magnitude of the emf, this does not lead to any mark deductions.

Solution 2: 4/6 marks

General feedback: The solution provides a clear explanation of the concepts involved in the problem. The student correctly identifies the formula for the resonant frequency and the voltage drop across the inductor. However, the student made a calculation error when determining the flux through the loop, leading to an incorrect emf in part b). The phase relationships are correctly explained.

Solution 3: 6/6 marks

General feedback: This is a well-written and correct solution. The resonant frequency is calculated correctly, and the expression for the voltage drop across the inductor is derived accurately using Ohm's law and the concept of inductive reactance. The phase relationship between the source emf and the voltage across the inductor is also explained correctly for both low and high frequencies.

\subsection{Question 8:}

\textbf{Response 1:}

**Solution 1: 6/6 marks**

Excellent solution. The student correctly applies the boundary conditions and uses the relationship between B and H to derive the correct equation for alpha1. The calculation is accurate, and the final answer is clearly stated.

**Solution 2: 2/6 marks**

Part (a): 2/3 marks. The student correctly identifies the boundary conditions and expresses the magnetic fields in terms of their components and angles. However, the student does not proceed to use these conditions to solve for alpha1.
Part (b): 0/2 marks. The student does not attempt to solve for alpha1 or explain why it cannot be determined with the given information.

To improve, the student should use the boundary conditions to derive an equation relating alpha1 to the known quantities (mu_r1, mu_r2, and alpha2) and then solve for alpha1.

**Solution 3: 4/6 marks**

Part (a): 3/3 marks. The student correctly identifies the boundary conditions and uses them to derive the correct equation for sin(alpha1).
Part (b): 1/2 marks. The calculation of alpha1 is incorrect due to a mistake in evaluating the sine function.

To improve, the student should double-check the calculation of the arcsine to obtain the correct value for alpha1.


\textbf{Response 2:}

**Solution 1: 6/6**

*   **Part (a): 3/3** Correctly applies the boundary conditions to both the parallel and perpendicular components of \(H\) and \(B\).
*   **Part (b): 1/1** Correctly uses the relationship \(B = \mu_0 \mu_r H\).
*   **Part (c): 2/2** Correctly manipulates the equations to find the angle \(\alpha_1\).

**General Feedback:** This is an excellent solution. The student clearly understands the boundary conditions for magnetic fields and applies them correctly to solve the problem. The solution is well-organized and easy to follow.

**Solution 2: 2/6**

*   **Part (a): 2/3** Correctly states the boundary conditions but fails to apply them effectively to the problem.
*   **Part (b): 0/1** Does not use the relationship \(B = \mu_0 \mu_r H\).
*   **Part (c): 0/2** Does not derive the correct equation for \(\alpha_1\) and does not calculate its value.

**General Feedback:** This solution shows some understanding of the boundary conditions, but the student struggles to apply them to the specific problem. The solution also lacks the necessary steps to derive the equation for \(\alpha_1\).

**Solution 3: 4/6**

*   **Part (a): 3/3** Correctly applies the boundary conditions to both the parallel and perpendicular components of \(H\) and \(B\).
*   **Part (b): 1/1** Correctly uses the relationship \(B = \mu_0 \mu_r H\).
*   **Part (c): 0/2** Makes a calculation error when manipulating the equations, leading to an incorrect value for \(\alpha_1\).

**General Feedback:** This solution is mostly correct, but the student makes a calculation error in the final step. This could be due to a simple mistake or a misunderstanding of the trigonometric relationships involved. The student should double-check their calculations to avoid such errors in the future.


\textbf{Response 3:}

**Solution 1: 6/6 marks**

Excellent answer. It correctly applies the boundary conditions on the interface to parallel and perpendicular components of H and B, uses \( B = \mu_0 \mu_r H \), and manipulates equations to obtain the correct value of \( \alpha_1 \).

**Solution 2: 2/6 marks**

This solution correctly identifies the boundary conditions and the relationship between B and H. However, it incorrectly assumes that \( B_{1x} = B_{2x} \) based on the given information and fails to derive the correct relationship between \( \alpha_1 \) and \( \alpha_2 \). As a result, the calculation of \( \alpha_1 \) is not attempted.

**Solution 3: 4/6 marks**

This solution correctly applies the boundary conditions and uses the relationship between B and H. It also correctly sets up the equations to solve for \( \alpha_1 \). However, there is a calculation error in the final step, leading to an incorrect value for \( \alpha_1 \). The correct value should be \(26.366^\circ\).

**General Feedback:**

Solution 1 is the most comprehensive and accurate of the three. It clearly demonstrates an understanding of the boundary conditions for magnetic fields and how to apply them to solve for the unknown angle. Solution 2 needs to improve its understanding of the relationship between the components of the magnetic field and the angles involved. Solution 3 is almost correct, but the student should be careful to avoid calculation errors.

\textbf{Response 4:}

**Solution 1: Grade 6/6**

This is a very well-written and correct solution. The student correctly identifies the relevant boundary conditions, uses the correct relationship between the magnetic field and magnetic field intensity, and derives the correct equation for \(\alpha_{1}\). The calculation of \(\alpha_{1}\) is also correct.

**Solution 2: Grade 3/6**

This solution starts off well by identifying the correct boundary conditions and expressing the magnetic fields in terms of their components and the angles. However, the student gets sidetracked by focusing on the relationship between \(B_{1z}\) and \(B_{2z}\) through permeabilities, which is not directly needed to solve for \(\alpha_{1}\). The student also makes a conceptual error by claiming that the calculation of \(\alpha_{1}\) is impossible without knowing the magnitudes of the magnetic fields, which is incorrect. 

**Solution 3: Grade 5/6**

This solution demonstrates a good understanding of the boundary conditions and the relationship between the magnetic field and magnetic field intensity. The student derives the correct equation for \(\sin(\alpha_{1})\) and uses it to find \(\alpha_{1}\). However, they incorrectly use sine in the calculation. They should use the inverse tangent function to find the correct value for \(\alpha_{1}\).

\textbf{Response 5:}

Solution 1: 6/6 marks

General feedback: This is an excellent solution that clearly and concisely applies the boundary conditions for magnetic fields at the interface. The student correctly identifies the tangential and normal components of the fields and uses the relationship between B and H to derive the equation relating the angles and relative permeabilities. The final calculation is also correct.

Solution 2: 1/6 marks

General feedback: While the student correctly identifies the boundary conditions and expresses the magnetic fields in terms of their components, the solution does not progress beyond that. The student correctly states that the tangential components of the magnetic field are equal but fails to utilize the relationship between the tangential and normal components through the tangent function. The conclusion that the problem cannot be solved without more information is incorrect.

Solution 3: 5/6 marks

General feedback: This solution is mostly correct and demonstrates a good understanding of the boundary conditions for magnetic fields. The student correctly uses the boundary conditions and the relationship between B and H to derive the equation relating the angles and relative permeabilities. However, the student makes a minor error in the final calculation, likely due to a misunderstanding in applying the boundary condition for the normal components of the magnetic field.

\subsection*{Question 9:}

\textbf{Response 1:}

**Solution 1: 14/15 marks**

**Solution 1: 15/15 marks**

Excellent solution. Clear and well-explained steps for all parts of the problem. Correct use of Faraday's law, Poynting vector, and the concept of radiation pressure. The calculations are accurate, and the final answers are in the correct form.

**Solution 2: 14/15 marks**

Part (a): 7/7 marks. Correctly applies Faraday's Law and derives the relationship between the magnetic intensity and the electric field.
Part (b): 5/6 marks. The expression for the average Poynting flux is correct. However, the explanation for the time average of the cosine term is missing. It should be mentioned that the average of \( \cos^2(x) \) over one period is 1/2.
Part (c): 2/2 marks. Correctly explains the effect of the angle of incidence on radiation pressure.

To improve, provide the missing explanation for the time average of the cosine term in part (b).

**Solution 3: 14/15 marks**

Part (a): 6/7 marks. Correctly applies Faraday's Law to derive the relationship between the magnetic field and electric field. However, the relationship between \( \underline{B} \) and \( \underline{H} \) could be explicitly mentioned before substitution.
Part (b): 6/6 marks. Correctly derives the expression for the average Poynting flux and provides a clear explanation.
Part (c): 2/2 marks. Correctly explains the effect of the angle of incidence on radiation pressure.

To improve, explicitly mention the relationship \( \underline{B} = \mu_0 \underline{H} \) before using it in part (a).


\textbf{Response 2:}

**Solution 1: 13/15**

*   **Part (a): 6/7** The solution is mostly correct, but there's a minor error in the calculation of the curl of E. The curl of E should be \(i\underline{k} \times \underline{E}\), not \(i\underline{k} \times \underline{E}_0\). This error does not significantly affect the subsequent steps, as the magnitude is still correct.
*   **Part (b): 5/6** The solution is correct.
*   **Part (c): 2/2** The solution is correct.

**General Feedback:** This is a good solution demonstrating a solid understanding of the concepts involved. The minor error in part (a) does not detract from the overall quality of the solution.

**Solution 2: 13/15**

*   **Part (a): 6/7** Similar to solution 1, there is a minor error in the calculation of the curl of E. The curl of E should be \(i\underline{k} \times \underline{E}\), not \(i\underline{k} \times \underline{E}_0\).
*   **Part (b): 5/6** The solution is correct.
*   **Part (c): 2/2** The solution is correct.

**General Feedback:** This solution is also well-done, with a similar minor error in part (a) as in solution 1. The student demonstrates a good understanding of the concepts and applies them correctly to solve the problem.

**Solution 3: 15/15**

*   **Part (a): 7/7** The solution is correct and well-explained.
*   **Part (b): 6/6** The solution is correct and well-explained.
*   **Part (c): 2/2** The solution is correct and well-explained.

**General Feedback:** This is an excellent solution. The student provides a clear and concise explanation of each step, demonstrating a strong understanding of the underlying physics principles.


\textbf{Response 3:}

**Solution 1: 14/15 marks**

This solution is well-presented and demonstrates a good understanding of the problem. The calculations for the magnetic field intensity and Poynting flux are accurate. There is a slight error in part c) where the student only uses one factor of \( \cos(\alpha) \) instead of \( \cos^2(\alpha) \).

**Solution 2: 15/15 marks**

This solution is excellent. The student correctly applies Faraday's law, derives the expression for magnetic intensity, and calculates the Poynting flux. The explanation of the phase relationship is accurate and detailed. The derivation of the radiation pressure formula for a perfect planar reflector is also correct.

**Solution 3: 14/15 marks**

This solution is also well-structured and demonstrates a strong understanding of the concepts involved. The calculations for the magnetic field intensity and Poynting flux are accurate. However, like Solution 1, there's a slight error in part c) where the student only uses one factor of \( \cos(\alpha) \) instead of \( \cos^2(\alpha) \).

**General Feedback:**

All three solutions demonstrate a good understanding of electromagnetic waves and their properties. The main difference lies in the accuracy of the radiation pressure formula in part c). Solution 2 is the most comprehensive and accurate, while Solutions 1 and 3 both have minor errors in part c).

\textbf{Response 4:}

**Solution 1: Grade 15/15**

This solution is excellent, showing a clear and complete understanding of the problem. All steps are explained clearly and correctly, and the final answers are all correct.

**Solution 2: Grade 14/15**

This solution is also very good. The student correctly applies Faraday's law and the relationship between \(\underline{B}\) and \(\underline{H}\) in vacuum. The derivation of the Poynting vector is also correct. However, in part a), the student did not explicitly mention that the assumption \( \underline{H}(\underline{r}, t) = \underline{H}_0 \exp (i(\underline{k} \cdot \underline{r} - \omega t)) \) was made.

**Solution 3: Grade 15/15**

This solution is excellent and demonstrates a clear understanding of the underlying concepts. All steps are explained clearly and correctly, and the final answers are all correct.

\textbf{Response 5:}

Solution 1: 13/15 marks

General feedback: This is a very good solution that correctly uses Faraday's law to find the relationship between the electric and magnetic fields. The student also demonstrates a good understanding of the Poynting vector and radiation pressure. However, the student makes a minor error in part b) by not justifying the time average of the cosine term.

Solution 2: 14/15 marks

General feedback: This solution is very good and demonstrates a clear understanding of electromagnetic waves and the relevant physics principles. The calculations are accurate and the explanations are clear. The only point where it could be improved is by explicitly stating that the real part of the complex representation is the physical field in the Poynting vector calculation in part b).

Solution 3: 15/15 marks

General feedback: This is an excellent solution that is well-written and easy to follow. The student clearly explains each step of the calculation and provides clear physical interpretations. The solution is complete and correct, demonstrating a strong understanding of the concepts involved.

\subsection*{Question 10:}

\textbf{Response 1:}

**Solution 1: 5/7 marks**

Part (a): 2/4 marks. The student correctly applied Faraday's law and substituted the given transformation equations. However, the simplification process was not completed to show the invariance of the scalar product.
Part (b): 2/2 marks. The student correctly used the inverse Lorentz transformations and simplified the expression to obtain the desired relations for ω' and k'.
Part (c): 1/1 mark. The student correctly identified the Doppler effect and provided a reasonable explanation for the relationship between ω and ω' in the limit v/c approaches 0.

To improve, the student should focus on the algebraic simplification in part (a) to demonstrate the invariance of the scalar product.

**Solution 2: 6/7 marks**

Part (a): 4/4 marks. The student correctly applied Faraday's law and derived the relationship between the magnetic intensity and the electric field.
Part (b): 1/2 marks. The student correctly substituted the Lorentz transformation but did not simplify the expression to show the desired relations for ω' and k'.
Part (c): 1/1 mark. The student correctly identified the Doppler effect and provided a reasonable explanation.

To improve, the student should focus on the algebraic simplification in part (b) to reach the given expressions for ω' and k'.

**Solution 3: 7/7 marks**

Excellent solution. The student correctly applied Faraday's law, derived the relationship between magnetic intensity and electric field, and used the Lorentz transformation to show the phase transformation. The explanation for the Doppler effect in the limit v/c approaches 0 is also clear and concise.


\textbf{Response 2:}

**Solution 1: 1/7**

*   **Part (a): 0/4** Incorrectly applies the transformation equations and fails to demonstrate invariance.
*   **Part (b): 0/2** Incorrectly applies the Lorentz transformations to the phase.
*   **Part (c): 1/1** Correctly identifies the Doppler effect and provides some reasoning, despite the errors in the previous parts.

**General Feedback:** This solution demonstrates some familiarity with the concepts but lacks a clear understanding of how to apply the transformations correctly. The student should review the Lorentz transformations and their application to electromagnetic fields.

**Solution 2: 2/7**

*   **Part (a): 0/4**  Incorrectly applies the transformation equations and fails to demonstrate invariance.
*   **Part (b): 1/2** Recognizes the need to use the Lorentz transformation but fails to derive the correct expressions for ω' and k'.
*   **Part (c): 1/1** Correctly identifies the Doppler effect and provides some reasoning, despite the errors in the previous parts.

**General Feedback:** This solution shows some effort in applying the Lorentz transformations, but the execution is flawed. The student should focus on understanding the algebraic manipulations and the physical meaning of the transformations.

**Solution 3: 7/7**

*   **Part (a): 4/4** Correctly applies the transformation equations and demonstrates the invariance of the scalar product.
*   **Part (b): 2/2**  Correctly derives the expressions for ω' and k' using the Lorentz transformation.
*   **Part (c): 1/1**  Correctly identifies the Doppler effect and explains its relationship to the limit v/c → 0.

**General Feedback:** This is an excellent solution that clearly and correctly addresses all parts of the problem. The student demonstrates a strong understanding of the Lorentz transformations and their implications for electromagnetic waves. The explanations are concise and well-articulated.


\textbf{Response 3:}

**Solution 1: 3/7 marks**

This solution demonstrates a good understanding of the basic concepts but falls short in the execution of the mathematical derivations. The attempt to show the invariance of the scalar product is incorrect, leading to an incorrect conclusion. The approach to find the phase in frame S' is also flawed due to algebraic errors. The final answer in part c) is correct, but the explanation could be more precise.

**Solution 2: 7/7 marks**

This is an excellent solution that clearly demonstrates a strong understanding of special relativity and electromagnetic waves. The student correctly applies the Lorentz transformations, derives the invariance of the scalar product, and provides a clear explanation of the phase transformation and the physical implications of the limit as v/c approaches 0.

**Solution 3: 5/7 marks**

This solution correctly shows the invariance of the scalar product and provides a good conceptual explanation of the phase transformation. However, the mathematical derivation of the phase transformation is incomplete, and the final simplification to the classical Doppler effect in part c) is not explicitly shown.

**General Feedback:**

Overall, Solution 2 is the most comprehensive and accurate, earning full marks. Solution 3 is also well-done, but it lacks the complete mathematical derivation for the phase transformation. Solution 1 needs to improve its understanding of the Lorentz transformations and their application to electromagnetic fields.

\textbf{Response 4:}

**Solution 1: Grade 4/7**

The student demonstrates a good understanding of Faraday's Law and Poynting Flux, providing clear derivations and explanations. However, the student struggles with the algebraic simplification in part a) and the phase transformation in part b). The conceptual understanding is sound, but the precise mathematical manipulation is lacking. Additionally, the explanation in part c) could be more concise and directly address the Doppler effect.

**Solution 2: Grade 5/7**

This solution showcases a strong grasp of electromagnetic wave theory and special relativity. The student correctly applies Faraday's Law and derives the Poynting flux. The explanation of the phase transformation is comprehensive and accurate. However, the student could improve the presentation by providing more intermediate steps in the derivations, especially in part a).

**Solution 3: Grade 5/7**

This response is well-structured and demonstrates a good understanding of the key concepts. The student correctly derives the relationship between electric and magnetic field magnitudes using Faraday's law. The explanation of the Poynting vector and its time average is clear and concise. In part b), however, the student could show more working steps to derive the transformation of the phase. The discussion of the Doppler effect in part c) is accurate and insightful.

\textbf{Response 5:}

Solution 1: 2/7 marks

General feedback: The student demonstrates a good understanding of the concept of invariance and Faraday's law. However, the calculations are not carried out correctly in part a) and part b), leading to incorrect conclusions. The explanation in part c) is also not clear and concise.

Solution 2: 4/7 marks

General feedback: This solution shows a good understanding of the relevant concepts and attempts to use the correct equations. However, there are errors in the calculation of the scalar product in part a) and the phase transformation in part b). The explanation in part c) is correct, but it lacks clarity and conciseness.

Solution 3: 7/7 marks

General feedback: This is an excellent solution that clearly and concisely addresses all parts of the problem. The student correctly uses Faraday's law to derive the relationship between the electric and magnetic fields. The explanation of the phase transformation in part b) is clear and concise. The student also demonstrates a good understanding of the Doppler effect in the limit of \( v/c \rightarrow 0 \) in part c).

\end{document}