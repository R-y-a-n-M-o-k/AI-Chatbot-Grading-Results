\documentclass[a4paper,11pt]{article}
\setlength\parindent{0pt}
\usepackage{amsmath, amssymb}
\usepackage[margin=1in,letterpaper]
{geometry} % decreases margins

\usepackage{setspace}
\setstretch{1.2}
\usepackage{spverbatim}

\begin{document}
    
\title{Gemini 1.5 Pro Grading (With Mark Scheme)}

\maketitle

This document compiles the Gemini responses when prompted to grade the solutions which have been generated by Gemini 1.5 Pro. The markscheme is given to Gemini in the prompt. This is the grading for the Quantum Mechanics (QM) questions. 

\section{QM}

\subsection*{Question 1:}

\textbf{Response 1:}
Solution 1: (5/5)

Part a: (3/3)
The definition of the commutator is accurate and complete.
The explanation of compatible operators and their relation to the commutator is thorough and insightful.
Part b: (2/2)
The answer correctly identifies the uncertainty in the second measurement of \(\hat{P}\) .
The explanation clearly demonstrates an understanding of the implications of non-commutativity on measurement outcomes.
Feedback: This is an excellent response. The student clearly understands the concepts and provides detailed explanations. The answer goes beyond simply repeating the definitions and demonstrates a deep understanding of quantum mechanics principles.

Solution 2: (5/5)

Part a: (3/3)
The definition of the commutator and compatible operators is correct.
The explanation of the relationship between compatibility and common eigenstates is clear and accurate.
Part b: (2/2)
The student accurately explains the uncertainty in the second \(\hat{P}\) measurement.
The discussion of the collapse into eigenstates and the non-commutative nature is well-articulated.
Feedback: This is another excellent response. The student demonstrates a strong grasp of the material and provides clear and concise explanations. The answer is well-organized and easy to follow.

Solution 3: (5/5)

Part a: (3/3)
The definition of the commutator and compatible operators is correct.
The explanation of compatibility and its implications for simultaneous measurements is accurate.
Part b: (2/2)
The student correctly states that the second \(\hat{P}\) measurement is not guaranteed to yield the same value as the first.
The explanation of the disturbance caused by the \(\hat{Q}\) measurement and the non-commutativity principle is clear.
Feedback: This solution is also excellent. The student presents the information in a clear and concise manner. The answer demonstrates a solid understanding of the fundamental concepts and their application to quantum measurements.

Overall Feedback: All three students have provided outstanding solutions that demonstrate a thorough understanding of the physics concepts involved. They have clearly articulated the definitions, explanations, and implications of commutators and compatible operators. Their responses to part (b) show a deep understanding of the non-commutative nature of quantum mechanics and its impact on measurement outcomes.

\textbf{Response 2:}

Solution 1:

a) (3 marks): Full marks. The definition of the commutator is correct, the definition of compatible operators is correct, and the implication of compatibility on the commutator is also correctly stated.

b) (2 marks): Full marks. The explanation correctly identifies that the initial measurement of P will be disturbed by the measurement of Q, and that the second measurement of P can yield a different result due to the collapse of the wavefunction to a different eigenstate.

General feedback: Excellent solution. Clear and concise explanation with correct use of terminology.

Solution 2:

a) (3 marks): Full marks. The definition of the commutator is correct, the definition of compatible operators is correct, and the implication of compatibility on the commutator is also correctly stated.

b) (2 marks): Full marks. The explanation correctly identifies that the initial measurement of P will be disturbed by the measurement of Q, and that the second measurement of P can yield a different result due to the collapse of the wavefunction. The mention of probabilities dependent on eigenstate overlap is a nice addition.

General feedback: Excellent solution. Clear and concise explanation with correct use of terminology.

Solution 3:

a) (3 marks): Full marks. The definition of the commutator is correct, the definition of compatible operators is correct, and the implication of compatibility on the commutator is also correctly stated.

b) (2 marks): Full marks. The explanation correctly identifies that the initial measurement of P will be disturbed by the measurement of Q, and that the second measurement of P can yield a different result due to the collapse of the wavefunction to a different eigenstate.

General feedback: Excellent solution. Clear and concise explanation with correct use of terminology.

Overall Summary:

All three solutions are excellent and demonstrate a strong understanding of the concepts of commutators and compatible operators. Each solution correctly addresses both parts of the question, providing clear and concise explanations. The use of terminology is accurate, and the explanations are well-structured. There is no significant difference in the quality of the solutions, and all three students deserve full marks.

\textbf{Response 3:}

Solution 1:
a) 3/3 marks
The definition of the commutator and compatible operators is correct. The explanation of compatibility and its implication on the commutator is clear and accurate.

b) 2/2 marks
The answer correctly explains why the initial and final values of (p) may not be the same due to the incompatibility of the operators and the resulting collapse of the wavefunction to different eigenstates.

General Feedback: This solution is excellent, demonstrating a clear and thorough understanding of the concepts. The explanations are precise and well-articulated.

Solution 2:
a) 3/3 marks
The definition of the commutator and compatible operators is correct and the explanation of compatibility and its implication on the commutator is clear and accurate.

b) 2/2 marks
The answer accurately describes the uncertainty in sequential measurements of incompatible observables and the resulting change in the system's state.

General Feedback: This is another excellent solution that showcases a solid grasp of the subject matter. The explanations are concise and effectively convey the key ideas.

Solution 3:
a) 3/3 marks
The definition of the commutator and compatible operators is correct and the explanation of compatibility and its implication on the commutator is clear and accurate.

b) 2/2 marks
The answer correctly explains how measuring incompatible observables disturbs the system and alters the outcomes of subsequent measurements.

General Feedback: This solution is also well-written and demonstrates a good understanding of the concepts. The explanations are clear and to the point.

Overall Assessment: All three solutions are of high quality and demonstrate a strong understanding of the physics involved. There are no significant errors or omissions in any of the solutions. Congratulations to all students for their excellent work!

\textbf{Response 4:}

**Solution 1:**

a) **3/3 marks** - The definition of the commutator is correct, and the explanation of compatibility and its implications is clear and accurate.

b) **2/2 marks** - The answer correctly addresses the uncertainty of the second measurement of \( \hat{P} \) and provides a clear explanation based on the collapse of the wavefunction and the non-commutativity of the operators.

**Overall feedback:** This is an excellent solution with clear and concise explanations. The student demonstrates a strong understanding of the concepts of commutators and compatibility, as well as their implications in quantum mechanics.

**Solution 2:**

a) **3/3 marks** - The definitions and explanations are accurate and well-articulated.

b) **2/2 marks** - The answer correctly describes the uncertainty in the second \( \hat{P} \) measurement and provides a good explanation based on the collapse of the wavefunction and the non-commutativity of operators.

**Overall feedback:** This is a very good solution. The student demonstrates a solid grasp of the concepts and provides clear explanations.

**Solution 3:**

a) **3/3 marks** - The commutator and compatibility are defined correctly, and the implications of compatibility are explained accurately.

b) **2/2 marks** - The answer correctly identifies the uncertainty in the second \( \hat{P} \) measurement and provides a valid explanation.

**Overall feedback:** This is a good solution with correct explanations. However, the explanations could be more concise and direct. The student should focus on clarifying the key points and avoid unnecessary elaboration.


\textbf{Response 5:}

**Solution 1:**

* a) **3/3 marks:** The definition, meaning, and implication of compatibility are clearly and correctly explained.
* b) **2/2 marks:** The uncertainty in the second measurement of P is well-explained, referencing the collapse of the wavefunction and the non-commutativity of the operators.

**Overall feedback:** This is an excellent solution. It demonstrates a thorough understanding of the concepts and uses precise language to convey the ideas effectively.

**Solution 2:**

* a) **3/3 marks:** All key points are covered correctly and clearly.
* b) **2/2 marks:** The explanation is accurate, emphasizing the collapse into different eigenstates and the resulting uncertainty in the second P measurement.

**Overall feedback:** This is a very good solution. It is well-structured and presents the information in a logical and easy-to-follow manner.

**Solution 3:**

* a) **3/3 marks:** The definition and meaning of compatibility are correct. The explanation of the implication could be slightly more elaborate, but it is sufficient for full marks.
* b) **2/2 marks:** The answer correctly identifies the uncertainty in the second measurement and provides a reasonable explanation based on the disturbance of the system and the non-common eigenstates.

**Overall feedback:** This is a good solution. It demonstrates a solid understanding of the topic, although the explanations could be more detailed in some parts.


\subsection*{Question 2:}

\textbf{Response 1:}

Solution 1: (4/4)

Part a: (2/2) The student provides a clear and accurate explanation of the realness of eigenvalues and their connection to observable quantities. The physical interpretation is well-articulated.
Part b: (2/2) The definition of orthonormality is comprehensive, covering both orthogonality and normalization aspects. The explanation of their significance in quantum mechanics is accurate and insightful.
Feedback: This is an excellent response that demonstrates a thorough understanding of the concepts and their physical significance.

Solution 2: (4/4)

Part a: (2/2) The student correctly explains the mathematical origin of real eigenvalues from the Hermitian property. The connection to observable quantities is clear, and the example of the Hamiltonian operator is helpful.
Part b: (2/2) The mathematical expression of orthonormality is accurate, and the explanation of orthogonality and normalization is concise and to the point. The student correctly mentions the implication of orthonormality for forming a complete basis.
Feedback: Another excellent response with a well-structured explanation and clear understanding of the concepts.

Solution 3: (4/4)

Part a: (2/2) The student provides a detailed mathematical proof of the realness of eigenvalues, which is commendable. The physical interpretation in terms of observable quantities is accurate.
Part b: (2/2) The mathematical expression of orthonormality is correct. The explanation of orthogonality and normalization, while slightly less detailed than other solutions, is still accurate and covers the essential points.
Feedback: This solution demonstrates a strong mathematical grasp of the concepts and provides a clear explanation of their physical significance.

Overall Feedback: All three students have provided excellent solutions that demonstrate a deep understanding of Hermitian operators, their eigenvalues, and the concept of orthonormality. They have accurately explained the mathematical properties and their crucial implications in quantum mechanics.

\textbf{Response 2:}

Solution 1:
a) 2/2 marks
The answer correctly states that the eigenvalues of a Hermitian operator are real and provides a clear physical interpretation of eigenvalues as possible measurement outcomes.

b) 2/2 marks
The answer accurately explains the concept of orthonormality, including both orthogonality and normalization conditions, and highlights their significance in quantum mechanics.

General Feedback: This solution is excellent, demonstrating a clear and thorough understanding of the concepts. The explanations are precise and well-articulated.

Solution 2:
a) 2/2 marks
The answer correctly identifies that the eigenvalues of a Hermitian operator are real and relates this property to the definition of Hermitian operators. The physical interpretation of eigenvalues as observable quantities is also accurate.

b) 2/2 marks
The answer accurately describes the orthonormality condition using the Kronecker delta notation and explains the importance of orthogonality and normalization in quantum mechanics.

General Feedback: This is another excellent solution that showcases a solid grasp of the subject matter. The explanations are concise and effectively convey the key ideas.

Solution 3:
a) 2/2 marks
The answer correctly states that the eigenvalues of a Hermitian operator are real and provides a mathematical justification using the conjugate transpose property. The physical significance of real eigenvalues as observable quantities is also explained.

b) 2/2 marks
The answer accurately defines orthonormality using the Kronecker delta and explains the concepts of orthogonality and normalization in the context of quantum mechanics.

General Feedback: This solution is also well-written and demonstrates a good understanding of the concepts. The explanations are clear and to the point, although the mathematical justification in part a) could be more concise.

Overall Assessment: All three solutions are of high quality and demonstrate a strong understanding of the physics involved. There are no significant errors or omissions in any of the solutions. Congratulations to all students for their excellent work!

\textbf{Response 3:}

Solution 1:

a) (2 marks): Full marks. The explanation correctly identifies the real nature of eigenvalues and provides a clear physical interpretation in the context of quantum mechanics.

b) (2 marks): Full marks. The explanation clearly defines orthogonality and normalization, emphasizing their importance in quantum mechanics for forming a complete basis set.

General feedback: Excellent solution. Clear, concise, and accurate explanation with appropriate emphasis on physical interpretation.

Solution 2:

a) (2 marks): Full marks. The explanation correctly identifies the real nature of eigenvalues, relates it to the properties of Hermitian operators, and provides a clear physical interpretation.

b) (2 marks): Full marks. The explanation clearly defines orthonormality using the Kronecker delta notation and highlights its importance for forming a basis and ensuring mutually exclusive measurements.

General feedback: Excellent solution. Clear, concise, and accurate explanation with a good mathematical representation of orthonormality.

Solution 3:

a) (2 marks): Full marks. The explanation correctly identifies the real nature of eigenvalues and provides a concise mathematical derivation based on the Hermitian property. The physical interpretation is also well-articulated.

b) (2 marks): Full marks. The explanation clearly defines orthogonality and normalization using the Kronecker delta notation and provides a geometric interpretation in the function space. The importance for representing states in Hilbert space is also mentioned.

General feedback: Excellent solution. Clear, concise, and accurate explanation with a strong mathematical foundation and a good link to the Hilbert space concept.

Overall Summary:

All three solutions demonstrate an excellent understanding of the topic. They correctly identify the real nature of eigenvalues, provide clear physical interpretations, and accurately define orthonormality. While the explanations differ slightly in their approach and emphasis, all three are well-structured, mathematically sound, and convey the key concepts effectively. Therefore, all three students deserve full marks.

\textbf{Response 4:}

**Solution 1:**

a) **2/2 marks** - The answer correctly identifies the real nature of eigenvalues and provides a clear explanation of its physical significance in quantum mechanics.

b) **2/2 marks** - The answer accurately describes the orthonormality condition and explains its importance in forming a complete basis set.

**Overall feedback:** This is an excellent solution that clearly demonstrates a strong understanding of the concepts and their physical interpretations. The explanations are comprehensive and well-articulated.

**Solution 2:**

a) **2/2 marks** - The answer correctly identifies the real nature of eigenvalues and provides a good explanation based on the Hermitian property. The physical interpretation is also accurate.

b) **2/2 marks** - The answer accurately describes the orthonormality condition and explains its importance in forming a complete basis and representing any state.

**Overall feedback:** This is a very good solution with accurate explanations and a clear understanding of the concepts. The explanations are concise and well-structured.

**Solution 3:**

a) **2/2 marks** - The answer correctly identifies the real nature of eigenvalues and provides a good explanation based on the Hermitian property. The physical interpretation is accurate.

b) **2/2 marks** - The answer accurately describes the orthonormality condition and provides a good explanation of its significance in forming a complete basis set.

**Overall feedback:** This is a very good solution with accurate explanations and a clear understanding of the concepts. The use of mathematical notation adds clarity to the explanation.

\textbf{Response 5:}

**Solution 1:**
a) **2/2 marks:** The answer correctly identifies the reality of eigenvalues and provides a clear explanation of its physical interpretation. The explanation links the mathematical property to the real-world measurements of physical quantities, which is excellent.
b) **2/2 marks:** The explanation of orthonormality is clear and concise, correctly stating both orthogonality and normalization conditions. The use of mathematical notation adds clarity to the explanation.

**Overall feedback:** This is an excellent solution. The answer is well-structured, accurate, and demonstrates a thorough understanding of the topic.

**Solution 2:**
a) **2/2 marks:** The answer correctly states the reality of eigenvalues and provides a good explanation based on the definition of Hermitian operators. The connection to the physical interpretation is clear and concise.
b) **2/2 marks:** The explanation of orthonormality is accurate, providing both the mathematical expression and its interpretation in terms of orthogonality and normalization. The emphasis on the importance of orthonormality in forming a basis is well-stated.

**Overall feedback:** This is a very good solution. It demonstrates a solid understanding of the concept and presents the information clearly and concisely.

**Solution 3:**
a) **2/2 marks:** The answer correctly states the reality of eigenvalues and provides a good mathematical explanation using the Hermitian conjugate property. The physical interpretation is briefly mentioned but could be more elaborated.
b) **2/2 marks:** The explanation of orthonormality is correct, including both the mathematical expression and its meaning. The interpretation of orthogonality and normalization is clear.

**Overall feedback:** This is a good solution. It demonstrates a good understanding of the topic, although the physical interpretation of the eigenvalues could be more detailed.


\subsection*{Question 3:}

\textbf{Response 1:}

Solution 1: (9/9)

Part a: (4/4)

The substitution into the definition is correct.
Integration by parts is applied correctly, with clear reasoning for vanishing boundary terms.
The simplification and conclusion that \(\hat{p}\) is Hermitian are accurate and well-explained.
Part b: (5/5)

Correct substitution and application of integration by parts (twice).
Correct reasoning for vanishing boundary terms.
Clear arrangement into the correct form and the statement that \(\hat{Q}\) is Hermitian.
Feedback: This is an excellent solution. The steps are clear, the reasoning is sound, and the conclusion is correct. It demonstrates a thorough understanding of the Hermitian operator concept and its application.

Solution 2: (7/9)

Part a: (4/4)

The substitution and integration by parts are correct.
The explanation of the vanishing boundary terms is clear.
The conclusion that \(\hat{p}\) is Hermitian is correct.
Part b: (3/5)

Correct substitution and the first application of integration by parts.
However, the second integration by parts is not explicitly shown, leading to some loss of clarity.
The conclusion is correct, but the explanation could be more detailed.
Feedback: This is a good solution, but it lacks some detail in part (b), specifically in the application of the second integration by parts. However, the overall understanding of the concept is evident.

Solution 3: (8/9)

Part a: (4/4)

The substitution and integration by parts are correct.
The reasoning for vanishing boundary terms is clear.
The conclusion that \(\hat{p}\) is Hermitian is accurate.
Part b: (4/5)

Correct substitution and application of integration by parts (twice).
Correct reasoning for vanishing boundary terms.
The conclusion is correct, but the intermediate steps could be shown more explicitly for clarity.
Feedback: This is a very good solution. It demonstrates a solid understanding of the concept and the ability to apply it correctly. The only minor issue is the lack of explicit steps in part (b), but the overall reasoning is sound.

Overall Feedback: All three students have demonstrated a good grasp of the Hermitian operator concept and its application. Their solutions are generally clear, well-reasoned, and correct. Minor improvements in clarity and detail would elevate some of the solutions to the level of Solution 1.

\textbf{Response 2:}

**Solution 1:**
a) **4/4 marks**
The answer correctly substitutes the momentum operator into the definition, uses integration by parts, and simplifies the remaining integral. They correctly conclude that \(\hat{p}\) is Hermitian.

b) **5/5 marks**
The answer correctly substitutes the operator into the integral and applies integration by parts twice. They also correctly remove vanishing terms due to the limits and well-behaved functions. They arrange into the correct form and state that \(\hat{Q}\) is Hermitian.

**General Feedback:** This solution is excellent, demonstrating a clear and thorough understanding of the concepts. The explanations are precise and well-articulated.

**Solution 2:**
a) **4/4 marks**
The answer correctly substitutes the momentum operator into the definition, uses integration by parts, and simplifies the remaining integral. They correctly conclude that \(\hat{p}\) is Hermitian.

b) **3/5 marks**
The answer correctly substitutes the operator into the integral and identifies that the complex conjugate of a real derivative is unchanged. However, the explanation of the application of integration by parts twice is insufficient to obtain the full 2 marks in the mark scheme.

**General Feedback:** This solution is very good, demonstrating a good understanding of the concepts. The explanation of the integration by parts in part b) could be improved.

**Solution 3:**
a) **4/4 marks**
The answer correctly substitutes the momentum operator into the definition, uses integration by parts, and simplifies the remaining integral. They correctly conclude that \(\hat{p}\) is Hermitian.

b) **3/5 marks**
The answer correctly substitutes the operator into the integral and states that integration by parts is applied twice. However, the answer skips the intermediate steps of this process and does not clearly show how the boundary terms vanish, which is necessary to obtain the full 2 marks in the mark scheme.

**General Feedback:** This solution is very good, demonstrating a good understanding of the concepts. The explanation of the integration by parts in part b) could be more detailed to show the complete process.

\textbf{Response 3:}

**Solution 1:**

* a) (4 marks): Full marks. The solution correctly applies the definition of a Hermitian operator, uses integration by parts, and shows that both sides of the equation are equal, concluding that \( \hat{p} \) is Hermitian.

* b) (5 marks): Full marks. The solution again correctly applies the definition, uses integration by parts twice, and demonstrates that the operator \( \hat{Q} \) is also Hermitian.

* General feedback: Excellent solution. The steps are clear and well-explained, and the use of integration by parts is appropriate. The reasoning is sound and the conclusion is correct.

**Solution 2:**

* a) (4 marks): Full marks. The solution correctly applies the definition, uses integration by parts, and demonstrates that both sides of the equation are equal, thus proving that \( \hat{p} \) is Hermitian.

* b) (3 marks): The solution correctly applies the definition and mentions the use of integration by parts, but it lacks the detailed steps of the second integration by parts and the explicit demonstration of the equality of both sides.

* General feedback: The solution for part a) is well-explained and correct. However, part b) lacks some essential details and steps, preventing it from receiving full marks.

**Solution 3:**

* a) (4 marks): Full marks. The solution correctly applies the definition, uses integration by parts, and shows that both sides of the equation are equal, concluding that \( \hat{p} \) is Hermitian.

* b) (3 marks): The solution correctly applies the definition and mentions the use of integration by parts twice. However, it lacks the intermediate steps and the explicit demonstration of the equality of both sides after the second integration.

* General feedback: The solution for part a) is clear and correct. However, part b) needs more details and steps to justify the conclusion that \( \hat{Q} \) is Hermitian.

**Overall Summary:**

Solution 1 is the most complete and well-explained solution, earning full marks. Solutions 2 and 3 are also good but lack some essential details in part b), resulting in a deduction of 2 marks each.

\textbf{Response 4:}

**Solution 1:**

a) **4/4 marks** - Correctly uses integration by parts and simplification to show LHS=RHS. Clear conclusion that \(\hat{p}\) is Hermitian.

b) **5/5 marks** - Correctly uses integration by parts twice and simplification to show LHS=RHS. Clear conclusion that \(\hat{Q}\) is Hermitian. 

**Overall feedback:** This solution is very detailed and well explained. It is easy to follow and provides a clear understanding of the steps involved in the proof. Excellent work!

**Solution 2:**

a) **4/4 marks** - Correctly substitutes the operator into the definition and uses integration by parts. Clear explanation that the boundary terms vanish and a clear conclusion is stated. 

b) **5/5 marks** - The solution correctly outlines the steps involved in proving the Hermitian nature of the operator, using integration by parts twice and noting that the boundary terms vanish. The explanation is clear and concise.

**Overall feedback:** This solution is well-organized and easy to follow. The student demonstrates a good understanding of the concept of Hermitian operators and the use of integration by parts. Great work!

**Solution 3:** 

a) **4/4 marks** - Correctly uses integration by parts, explanation of why the boundary term is zero is clear. Clear conclusion that the operator is Hermitian is stated. 

b) **5/5 marks** - Correctly applies integration by parts twice, recognizing that the boundary terms vanish. The conclusion that the operator is Hermitian is clearly stated.

**Overall feedback:** This solution is well-structured and concise. The student demonstrates a good understanding of the problem and the necessary mathematical techniques. Well done!


\textbf{Response 5:}

**Solution 1:**
a) **4/4 marks:** The student correctly substitutes the operator and uses integration by parts to manipulate the expressions, clearly showing the equivalence of the LHS and RHS.
b) **5/5 marks:** The student applies integration by parts twice and correctly identifies that the boundary terms vanish due to the functions' behavior at infinity, demonstrating a clear understanding of the Hermitian operator definition.

**Overall feedback:** This is an excellent solution. The student demonstrates a thorough understanding of the concept of Hermitian operators and applies the definition rigorously to both parts of the problem.

**Solution 2:**
a) **3/4 marks:** The student correctly substitutes the operator and identifies the need for integration by parts but does not explicitly show the integration steps. The conclusion that \( \hat{p} \) is Hermitian is correct.
b) **4/5 marks:** The student correctly identifies the need for integration by parts twice and mentions that the boundary terms vanish. However, the integration steps are not shown, and the conclusion could be more explicit.

**Overall feedback:** This is a good solution, but it lacks some details in the integration steps. The student demonstrates a good understanding of the concept but could improve the clarity of the solution by showing the integration by parts explicitly.

**Solution 3:**
a) **4/4 marks:** The student correctly applies the definition of a Hermitian operator, uses integration by parts, and clearly shows the equivalence of the LHS and RHS.
b) **3/5 marks:** The student correctly identifies the need for integration by parts twice but does not explicitly show the integration steps. The conclusion that \( \hat{Q} \) is Hermitian is correct, but the reasoning could be more explicit by mentioning that the boundary terms vanish due to the functions' behavior at infinity.

**Overall feedback:** This is a good solution, but it lacks some details in the integration steps for part b). The student demonstrates a good understanding of Hermitian operators but could improve the clarity of the solution by showing the integration steps explicitly and providing a more detailed explanation for the vanishing boundary terms.


\subsection*{Question 4:}

\textbf{Response 1:}

Absolutely! Here's a grading and feedback breakdown for each student's solution:

**Solution 1: (20/20)**

*   **Part a:** (3/3) Correct expansion, clear reasoning, and correct conclusion.
*   **Part b:** (3/3) Same as part (a), well-executed.
*   **Part c:** (4/4) Correct expansion, factorization, and simplification using the commutation relation.
*   **Part d:** (4/4) Same as part (c), well-done.
*   **Part e:** (6/6) Correctly uses the commutator identity and results from previous parts to arrive at the correct conclusion.

**Feedback:** This is an excellent solution. The student demonstrates a thorough understanding of the commutation relations and their application. The steps are clear, and the reasoning is sound.

**Solution 2: (18/20)**

*   **Part a:** (3/3) Correct expansion and conclusion.
*   **Part b:** (3/3) Same as part (a), well-executed.
*   **Part c:** (4/4) Correct expansion, simplification, and use of the commutation relation.
*   **Part d:** (4/4) Same as part (c), well-done.
*   **Part e:** (4/6) Correctly uses the results from previous parts, but could be more explicit in explaining why each of the commutators with the squared momentum components is zero.

**Feedback:** This is a very good solution. The student demonstrates a strong understanding of the concepts. The only minor issue is the lack of explicit justification for the vanishing commutators in part (e).

**Solution 3: (19/20)**

*   **Part a:** (3/3) Correct expansion and conclusion.
*   **Part b:** (3/3) Same as part (a), well-executed.
*   **Part c:** (4/4) Correct expansion, simplification, and use of the commutation relation.
*   **Part d:** (4/4) Same as part (c), well-done.
*   **Part e:** (5/6) Correctly uses the results from previous parts. However, the explanation could be slightly more detailed to explicitly state that the commutators with the squared momentum components are zero due to the operators not affecting the x-component of angular momentum.

**Feedback:** This is another excellent solution. The student demonstrates a solid understanding of the concepts and applies them correctly. The explanation could be slightly more detailed in part (e) but is overall well-presented.

**Overall Feedback:** All three students have provided excellent solutions demonstrating a strong grasp of quantum mechanical commutation relations. Minor improvements in clarity and detail would elevate Solutions 2 and 3 to the level of Solution 1. 

\textbf{Response 2:}

**Solution 1:**
a) **3/3 marks**
The answer correctly expands the commutation relation and simplifies it to zero, demonstrating that \(\hat{L_x}\) and \(\hat{X}\) commute.

b) **3/3 marks**
The answer correctly expands the commutation relation and simplifies it to zero, demonstrating that \(\hat{L_x}\) and \(\hat{P_x}\) commute.

c) **4/4 marks**
The answer correctly expands the commutation relation, identifies the non-zero term, and applies the canonical commutation relation to obtain the correct result.

d) **4/4 marks**
The answer correctly expands the commutation relation, identifies the non-zero term, and applies the canonical commutation relation to obtain the correct result.

e) **6/6 marks**
The answer correctly rewrites the commutation relation, applies the identity for the commutator of an operator with a squared operator, and simplifies using the results from previous parts to obtain the correct result.

**General Feedback:** This solution is excellent, demonstrating a clear and thorough understanding of the concepts and calculations involved. The explanations are concise and easy to follow.

**Solution 2:**
a) **3/3 marks**
The answer correctly expands the commutation relation and simplifies it to zero, demonstrating that \(\hat{L_x}\) and \(\hat{X}\) commute.

b) **3/3 marks**
The answer correctly expands the commutation relation and simplifies it to zero, demonstrating that \(\hat{L_x}\) and \(\hat{P_x}\) commute.

c) **4/4 marks**
The answer correctly expands the commutation relation, identifies the non-zero term, and applies the canonical commutation relation to obtain the correct result.

d) **4/4 marks**
The answer correctly expands the commutation relation, identifies the non-zero term, and applies the canonical commutation relation to obtain the correct result.

e) **6/6 marks**
The answer correctly uses the linearity of the commutator and the results from previous parts to show that the commutator of \(\hat{L_x}\) with the total momentum squared operator is zero.

**General Feedback:** This solution is also excellent, demonstrating a clear and thorough understanding of the concepts and calculations involved. The explanations are concise and well-organized.

**Solution 3:**
a) **3/3 marks**
The answer correctly expands the commutation relation and simplifies it to zero, demonstrating that \(\hat{L_x}\) and \(\hat{X}\) commute.

b) **3/3 marks**
The answer correctly expands the commutation relation and simplifies it to zero, demonstrating that \(\hat{L_x}\) and \(\hat{P_x}\) commute.

c) **4/4 marks**
The answer correctly expands the commutation relation and applies the canonical commutation relation to obtain the correct result.

d) **4/4 marks**
The answer correctly expands the commutation relation and applies the canonical commutation relation to obtain the correct result.

e) **6/6 marks**
The answer correctly expands the commutation relation with the total momentum squared operator, uses the results from previous parts, and simplifies to obtain the correct result.

**General Feedback:** This solution is also excellent, demonstrating a clear and thorough understanding of the concepts and calculations involved. The explanations are detailed and accurate.

**Overall Assessment:** All three solutions are of high quality and demonstrate a strong understanding of the physics involved. There are no significant errors or omissions in any of the solutions. Congratulations to all students for their excellent work!

\textbf{Response 3:}

**Solution 1:**

* a) (3 marks): Full marks. The commutator is correctly expanded, the commutation relations are correctly stated, and the simplification leading to the zero result is clear.
* b) (3 marks): Full marks. The commutator is correctly expanded, the commutation relations are correctly stated, and the simplification to zero is clear.
* c) (4 marks): Full marks. The commutator is correctly expanded, the intermediate steps of rearranging terms and factoring are shown, and the correct commutation relation is used to arrive at the final answer.
* d) (4 marks): Full marks. The commutator is correctly expanded, the intermediate steps of rearranging terms and factoring are shown, and the correct commutation relation is used to arrive at the final answer.
* e) (6 marks): Full marks. The student correctly uses the commutator identity and the previously derived results to simplify the expression and obtain the final result. 

* General feedback: Excellent solution. All steps are clearly explained and the calculations are correct. The use of commutation relations is accurate and the presentation is well-organized.

**Solution 2:**

* a) (3 marks): Full marks. The commutator is correctly expanded, the commutation relations are correctly stated, and the simplification to zero is clear.
* b) (3 marks): Full marks. The commutator is correctly expanded, the commutation relations are correctly stated, and the simplification to zero is clear.
* c) (4 marks): Full marks. The commutator is correctly expanded, the intermediate steps of rearranging terms and factoring are shown, and the correct commutation relation is used to arrive at the final answer.
* d) (4 marks): Full marks. The commutator is correctly expanded, the intermediate steps of rearranging terms and factoring are shown, and the correct commutation relation is used to arrive at the final answer.
* e) (6 marks): Full marks. The student correctly uses the linearity of the commutator and the previously derived results to obtain the final result. 

* General feedback: Excellent solution. All steps are clearly explained and the calculations are correct. The use of commutation relations is accurate and the presentation is well-organized.

**Solution 3:**

* a) (3 marks): Full marks. The commutator is correctly expanded, the commutation relations are correctly stated, and the simplification to zero is clear.
* b) (3 marks): Full marks. The commutator is correctly expanded, the commutation relations are correctly stated, and the simplification to zero is clear.
* c) (4 marks): Full marks. The commutator is correctly expanded, the intermediate steps of rearranging terms and factoring are shown, and the correct commutation relation is used to arrive at the final answer.
* d) (4 marks): Full marks. The commutator is correctly expanded, the intermediate steps of rearranging terms and factoring are shown, and the correct commutation relation is used to arrive at the final answer.
* e) (6 marks): Full marks. The student correctly uses the commutator identity and the previously derived results to simplify the expression and obtain the final result. 

* General feedback: Excellent solution. All steps are clearly explained and the calculations are correct. The use of commutation relations is accurate and the presentation is well-organized.

**Overall Summary:**

All three solutions demonstrate a strong understanding of angular momentum operators and commutation relations. The calculations are accurate and the steps are well-explained. There are minor differences in the wording and presentation, but all three solutions are equally correct and deserve full marks.

\textbf{Response 4:}

**Solution 1:**

a) **3/3 marks** - The commutator is correctly expanded, and the commutation relations between \( \hat{X} \) and the operators in \( \hat{L_x} \) are correctly identified and used.

b) **3/3 marks** - Same as part a), but with \( \hat{P_x} \).

c) **4/4 marks** - The commutator is expanded correctly. The correct commutation relations are used to yield the answer.

d) **4/4 marks** - Same as part c), but with \( \hat{P_y} \).

e) **6/6 marks** - Correctly expands the commutation relation and applies the identity. Simplifies to the correct answer.

**Overall feedback:** This is an excellent solution, demonstrating a clear understanding of the commutation relations between angular momentum and position/momentum operators. The steps are well-explained, and the answer is concise and accurate.

**Solution 2:**

a) **3/3 marks** - Correct definition of the angular momentum operator and the commutator is given. The solution is correct.

b) **3/3 marks** - Same as part a) but with \( \hat{P_x} \).

c) **4/4 marks** - Correctly expands the commutator and uses the commutation relations to simplify.

d) **4/4 marks** - Same as part c), but with \( \hat{P_y} \).

e) **5/6 marks** - Correctly expands the commutation relation and simplifies using the results from parts a) and b). However, the statement that "the commutator of \( \hat{L_x} \) with the square of any component of the momentum operator also vanishes due to the properties of the angular momentum and momentum operators" is not entirely accurate. It should be clarified that this is because \( \hat{L_x} \) does not act on the momentum components directly, but rather on their relative positions.

**Overall feedback:** This solution is well-structured and demonstrates a good understanding of the topic. The minor inaccuracy in part e) does not detract significantly from the overall quality of the solution.

**Solution 3:**

a) **3/3 marks** - Correctly defines the angular momentum operator and expands the commutator. The solution is correct.

b) **3/3 marks** - Same as part a), but with \( \hat{P_x} \).

c) **4/4 marks** - Correctly expands the commutator, uses the commutation relations, and arrives at the correct answer.

d) **4/4 marks** - Same as part c), but with \( \hat{P_y} \).

e) **6/6 marks** - Correctly expands the commutator and uses the results from part b) to simplify and arrive at the correct answer.

**Overall feedback:** This solution is well-written and demonstrates a thorough understanding of the commutation relations involved. The steps are clear and easy to follow, and the solution is accurate. Excellent work!


\textbf{Response 5:}

**Solution 1:**
a) **3/3 marks:** The commutation relation is correctly expanded and simplified to zero.
b) **3/3 marks:** The commutation relation is correctly expanded and simplified to zero.
c) **4/4 marks:** The commutation relation is correctly expanded, simplified, and the commutation relation between \( \hat{Y} \) and \( \hat{P_{y}} \) is correctly used.
d) **4/4 marks:** The commutation relation is correctly expanded, simplified, and the commutation relation between \( \hat{Y} \) and \( \hat{P_{y}} \) is correctly used.
e) **6/6 marks:** The student correctly uses the expansion for \( \hat{P}^{2} \) and the commutator relation identity to simplify the expression to zero, showing a good understanding of the commutation relations between the angular momentum and linear momentum operators. 

**Overall feedback:** This is an excellent solution. The student demonstrates a thorough understanding of the commutation relations between the angular momentum and linear momentum operators. The derivations are clear, concise, and well-explained.

**Solution 2:**
a) **3/3 marks:** The commutation relation is correctly expanded and simplified to zero.
b) **3/3 marks:** The commutation relation is correctly expanded and simplified to zero.
c) **4/4 marks:** The commutation relation is correctly expanded, simplified, and the commutation relation between \( \hat{Y} \) and \( \hat{P_{y}} \) is correctly used.
d) **4/4 marks:** The commutation relation is correctly expanded, simplified, and the commutation relation between \( \hat{Y} \) and \( \hat{P_{y}} \) is correctly used.
e) **6/6 marks:** The student correctly uses the expansion for \( \hat{P}^{2} \) and the commutator relation identity to simplify the expression to zero, showing a good understanding of the commutation relations between the angular momentum and linear momentum operators.

**Overall feedback:** This is an excellent solution, similar to solution 1. The student demonstrates a thorough understanding of the commutation relations between the angular momentum and linear momentum operators. The derivations are clear, concise, and well-explained.

**Solution 3:**
a) **3/3 marks:** The commutation relation is correctly expanded and simplified to zero.
b) **3/3 marks:** The commutation relation is correctly expanded and simplified to zero.
c) **4/4 marks:** The commutation relation is correctly expanded, simplified, and the commutation relation between \( \hat{Y} \) and \( \hat{P_{y}} \) is correctly used.
d) **4/4 marks:** The commutation relation is correctly expanded, simplified, and the commutation relation between \( \hat{Y} \) and \( \hat{P_{y}} \) is correctly used.
e) **6/6 marks:** The student correctly uses the expansion for \( \hat{P}^{2} \) and the commutator relation identity to simplify the expression to zero, showing a good understanding of the commutation relations between the angular momentum and linear momentum operators.

**Overall feedback:** This is an excellent solution, similar to solution 1 and 2. The student demonstrates a thorough understanding of the commutation relations between the angular momentum and linear momentum operators. The derivations are clear, concise, and well-explained.



\subsection*{Question 5:}

\textbf{Response 1:}

Absolutely! Here's a grading and feedback breakdown for each student's solution:

**Solution 1: (7/7)**

*   **Part a:** (4/4) Correctly states the equation, substitutes the perturbation, simplifies the integral, and concludes that the correction is 0.
*   **Part b:** (3/3) Correctly applies perturbation theory, substitutes the terms, solves the integral, and states the final answer for both odd and even values of n.

**Feedback:** This is an excellent solution. The steps are clear, the calculations are accurate, and the conclusions are correct. The student demonstrates a thorough understanding of perturbation theory and its application to the infinite square well problem.

**Solution 2: (7/7)**

*   **Part a:** (4/4) Correctly applies perturbation theory, substitutes the perturbation, simplifies the integral, and concludes that the correction is 0.
*   **Part b:** (3/3) Correctly applies perturbation theory, substitutes the terms, uses the Dirac delta function property to simplify the integral, and states the final answer.

**Feedback:** This is another excellent solution. The student demonstrates a strong grasp of the material and presents the solution in a clear and concise manner. The calculations are accurate, and the conclusions are correct.

**Solution 3: (7/7)**

*   **Part a:** (4/4) Correctly states the equation, substitutes the perturbation, simplifies the integral using trigonometric identities, and concludes that the correction is 0.
*   **Part b:** (3/3) Correctly applies perturbation theory, substitutes the terms, uses the Dirac delta function property to simplify the integral, and states the final answer.

**Feedback:** This is also an excellent solution. The student shows a deep understanding of the concepts and the ability to apply them to the problem. The calculations are accurate, and the conclusions are well-justified.

**Overall Feedback:** All three students have provided outstanding solutions that demonstrate a thorough understanding of perturbation theory and its application to the infinite square well problem. Their solutions are clear, well-organized, and accurate. They have correctly applied the relevant formulas and techniques, resulting in the correct final answers.


\textbf{Response 2:}

**Solution 1:**
a) **4/4 marks**
The answer correctly states the equation for energy correction, substitutes the perturbation and wavefunction, and solves the integral to show that the correction is zero.

b) **3/3 marks**
The answer correctly applies perturbation theory, substitutes the correct terms into the integral, solves it, and states the results for odd and even values of \(n\).

**General Feedback:** This solution is excellent, demonstrating a clear and thorough understanding of the concepts and calculations involved. The explanations are concise and easy to follow.

**Solution 2:**
a) **4/4 marks**
The answer correctly states the equation for energy correction, substitutes the perturbation and wavefunction, and solves the integral to show that the correction is zero.

b) **3/3 marks**
The answer correctly applies perturbation theory, substitutes the correct terms into the integral, solves it, and provides the formula for the first-order correction.

**General Feedback:** This solution is also excellent, demonstrating a clear and thorough understanding of the concepts and calculations involved. The explanations are well-organized and easy to follow.

**Solution 3:**
a) **4/4 marks**
The answer correctly states the equation for energy correction, substitutes the perturbation and wavefunction, and solves the integral to show that the correction is zero.

b) **3/3 marks**
The answer correctly applies perturbation theory, substitutes the correct terms into the integral, solves it using the properties of the Dirac delta function, and simplifies the expression.

**General Feedback:** This solution is also excellent, demonstrating a clear and thorough understanding of the concepts and calculations involved. The explanations are detailed and accurate.

**Overall Assessment:** All three solutions are of high quality and demonstrate a strong understanding of perturbation theory and its application to the infinite square well problem. There are no significant errors or omissions in any of the solutions. Congratulations to all students for their excellent work!

\textbf{Response 3:}

**Solution 1:**

* a) (4 marks): Full marks. The student correctly states the equation for first-order energy correction and substitutes the relevant expressions. The trigonometric identities are used appropriately, and the integral is correctly evaluated to show that the correction is zero.
* b) (3 marks): Full marks. The student correctly applies perturbation theory, substitutes the correct terms, and evaluates the integral using the property of the Dirac delta function. The final expression for the energy correction is derived correctly.

* General feedback: Excellent solution. The steps are clear and well-explained, and the calculations are accurate. The use of trigonometric identities and the property of the Dirac delta function is appropriate.

**Solution 2:**

* a) (4 marks): Full marks. The student correctly states the equation for first-order energy correction and substitutes the relevant expressions. The integral is simplified using trigonometric identities, and the correct conclusion that the correction is zero is reached.
* b) (3 marks): Full marks. The student correctly applies perturbation theory, substitutes the correct terms, and evaluates the integral using the property of the Dirac delta function. The final expression for the energy correction is derived correctly.

* General feedback: Excellent solution. The steps are clear and well-explained, and the calculations are accurate. The use of trigonometric identities and the property of the Dirac delta function is appropriate.

**Solution 3:**

* a) (4 marks): Full marks. The student correctly states the equation for first-order energy correction and substitutes the relevant expressions. The trigonometric identities are used appropriately, and the integral is correctly evaluated to show that the correction is zero.
* b) (3 marks): Full marks. The student correctly applies perturbation theory, substitutes the correct terms, and evaluates the integral using the property of the Dirac delta function. The final expression for the energy correction is derived correctly.

* General feedback: Excellent solution. The steps are clear and well-explained, and the calculations are accurate. The use of trigonometric identities and the property of the Dirac delta function is appropriate.

**Overall Summary:**

All three solutions are excellent and demonstrate a strong understanding of perturbation theory. Each solution correctly addresses both parts of the question, providing clear and concise explanations. The use of mathematical tools and the application of perturbation theory are accurate. There is no significant difference in the quality of the solutions, and all three students deserve full marks.

\textbf{Response 4:}

**Solution 1:**

a) **4/4 marks** - The student correctly sets up the integral for the first-order energy correction and uses trigonometric identities to simplify it. The final answer of 0 is also correct.

b) **3/3 marks** - The student correctly applies the perturbation formula and uses the properties of the Dirac delta function to evaluate the integral. The final expression for the energy correction is also correct.

**Overall feedback:** This solution is well-presented and demonstrates a good understanding of the concepts involved. The calculations are clear and easy to follow, and the final answers are correct.

**Solution 2:**

a) **4/4 marks** - Correctly sets up and evaluates the integral to obtain the correct answer of 0.

b) **3/3 marks** - The student correctly identifies the formula for first-order correction and applies it to the problem. The integration is done correctly and yields the correct result. The final expression for the correction is given in a clear and concise manner.

**Overall feedback:** This solution is well-written and demonstrates a good understanding of perturbation theory. The calculations are accurate, and the explanations are clear. The student also shows a good understanding of the properties of the Dirac delta function.

**Solution 3:**

a) **4/4 marks** - The student correctly sets up the integral for the energy correction and simplifies it using trigonometric identities. The calculation is accurate and leads to the correct answer of 0.

b) **3/3 marks** - Correctly applies the formula for the first-order correction and evaluates the integral using the property of the Dirac delta function. The final expression for the energy correction is accurate.

**Overall feedback:** This is a well-organized and clearly presented solution. The student demonstrates a good understanding of perturbation theory and the relevant mathematical techniques. The calculations are accurate and easy to follow. 


\textbf{Response 5:}

**Solution 1:**
* a) **4/4 marks:** Correctly sets up the integral for the first-order correction, applies trigonometric identities, and accurately evaluates the integral to zero.
* b) **3/3 marks:** Correctly applies perturbation theory to obtain the integral, evaluates it using the properties of the Dirac delta function, and provides the final expression for the first-order correction.

**Overall feedback:** This solution is excellent, demonstrating a clear understanding of perturbation theory and its application to the infinite square well problem. The student's work is well-organized and easy to follow, with correct mathematical manipulations and clear explanations.

**Solution 2:**
* a) **4/4 marks:** Correctly sets up the integral for the first-order correction and evaluates it to zero. However, the use of trigonometric identities is not shown explicitly.
* b) **3/3 marks:** Correctly applies perturbation theory and uses the properties of the Dirac delta function to evaluate the integral. The final expression for the first-order correction is accurate.

**Overall feedback:** This solution is also very good, demonstrating a strong understanding of perturbation theory. The student's work is clear and concise, although the trigonometric simplification in part a) could be shown for completeness.

**Solution 3:**
* a) **3/4 marks:** Correctly sets up the integral for the first-order correction but does not explicitly show the simplification using trigonometric identities. While the final answer is correct, the missing steps could be clearer.
* b) **3/3 marks:** Correctly applies perturbation theory and uses the properties of the Dirac delta function to evaluate the integral. The final expression for the first-order correction is accurate.

**Overall feedback:** This solution is good, but it lacks some clarity in the simplification of the integral in part a). The student demonstrates a good understanding of perturbation theory but could improve the presentation by including the missing steps.

\textbf{Response 4:}

**Solution 1:**

a) **4/4 marks** - Correctly expands the expression for \( \hat{J^2} \) and uses the ladder operator identities to arrive at the desired result.

b) **3/4 marks** - Correctly identifies the action of \( \hat{L}^2 \), \( \hat{S}^2 \), and \( \hat{L_z}\hat{S_z} \) on the given state. However, the explanation for the vanishing of the \( \hat{L_+}\hat{S_-} \) and \( \hat{L_-}\hat{S_+} \) terms is incomplete. It should be mentioned that these operators raise/lower the \( m \) and \( m_s \) values, respectively, and acting on the state \( \left|l, -l; s, -s \right> \) would result in states outside the allowed range.

c) **3/9 marks** - Correctly identifies the basis kets but makes an error in calculating the matrix representation.

**Overall feedback:** This solution demonstrates a good understanding of the concepts but contains some errors in the calculation of the matrix representation in part c). The explanation in part b) could be more comprehensive.

**Solution 2:**

a) **4/4 marks** - Correctly derives the expression for \( \hat{J^2} \) using the given information.

b) **2/4 marks** - Attempts to explain the action of \( \hat{J^2} \) on the state \( \left|l, -l; s, -s \right> \) but fails to provide a rigorous argument for why the \( \hat{L_+}\hat{S_-} \) and \( \hat{L_-}\hat{S_+} \) terms do not contribute.

c) **0/9 marks** - Does not provide a correct matrix representation for the operator \( \hat{O} \).

**Overall feedback:** This solution shows some understanding of the problem but lacks rigor in part b) and contains errors in part c). The student should review the action of ladder operators on angular momentum states.

**Solution 3:**

a) **4/4 marks** - Correctly expands and simplifies the expression for \( \hat{J^2} \) using the ladder operator identities.

b) **4/4 marks** - Correctly identifies that \( \left|l, -l; s, -s \right> \) is an eigenvector of \( \hat{J^2} \) and provides a clear explanation based on the action of the operators.

c) **9/9 marks** - Correctly identifies the basis kets and calculates the matrix representation of \( \hat{O} \) accurately.

**Overall feedback:** This is an excellent solution. The student demonstrates a strong understanding of the concepts and calculations involved in angular momentum and spin operators. The explanations are clear and concise, and the matrix representation is accurate.

\textbf{Response 5:}

**Solution 1:**

a) **4/4 marks** - Correctly expands the expression for  \(\hat{J}^2\) and uses the ladder operator identities to arrive at the desired result.

b) **2/4 marks** - Correctly identifies the action of \(\hat{L}^2\), \(\hat{S}^2\), and \(\hat{L_z}\hat{S_z}\) on the given state. However, the explanation for the vanishing of the \(\hat{L_+}\hat{S_-}\) and \(\hat{L_-}\hat{S_+}\) terms is incomplete and needs more detail on the action of the raising and lowering operators on states with the lowest and highest allowed values of m and m_s.

c) **2/9 marks** - Correctly identifies the basis kets but makes an error in calculating the matrix representation. There should be only two non-zero off-diagonal elements: bℏ² at (1, 2) and -bℏ² at (5, 6).

**Overall feedback:** This solution demonstrates a good understanding of the concepts but contains calculation errors in part c). The explanation in part b) could be more detailed.

**Solution 2:**

a) **4/4 marks** - Correctly derives the expression for \(\hat{J}^2\) using the given information.

b) **2/4 marks** - Attempts to explain the action of \(\hat{J}^2\) on the state \(\left|l, -l; s, -s \right>\) but does not explicitly mention the effect of raising and lowering operators on the state with minimum m and m_s values, hence missing the opportunity to fully explain why the cross terms vanish.

c) **0/9 marks** - Does not provide a correct matrix representation for the operator \(\hat{O}\). The action of the operators on the basis kets is not calculated correctly.

**Overall feedback:** This solution shows some understanding of the problem but lacks a complete explanation in part b) and contains errors in part c). The student should focus on the action of ladder operators on angular momentum states.

**Solution 3:**

a) **4/4 marks** - Correctly expands and simplifies the expression for \(\hat{J}^2\) using the ladder operator identities.

b) **4/4 marks** - Correctly identifies that \(|l, -l; s, -s >\) is an eigenvector of \(\hat{J}^2\) and provides a clear explanation based on the action of the operators on the state with minimum m and m_s values, leading to the vanishing of cross terms.

c) **9/9 marks** - Correctly identifies the basis kets and calculates the matrix representation of \(\hat{O}\) accurately. The action of the operators on each basis ket is calculated correctly, leading to the correct matrix representation.

**Overall feedback:** This is an excellent solution. The student demonstrates a strong understanding of the concepts and calculations involved in angular momentum and spin operators. The explanations are clear and concise, and the matrix representation is accurate.


\subsection*{Question 6:}

\textbf{Response 1:}

Absolutely! Here's a grading and feedback breakdown for each student's solution:

**Solution 1: (16/17)**

*   **Part a:** (4/4) Correct expansion and simplification, leading to the desired result.
*   **Part b:** (4/4) Correctly identifies the action of ladder operators on the state, calculates eigenvalues, and concludes the state is an eigenvector.
*   **Part c:** (8/9) Correctly identifies the basis kets, chooses a vector representation, and calculates the action of  \( \hat{L^{2}} \) and \( \hat{S_{+}} \hat{L_z} \) to construct the matrix representation of \(\hat{O}\). One mark lost due to an incorrect value in the final matrix (the 1st row, 1st column element should be \(2a\hbar^2\) not \(4\hbar^2\) ).

**Feedback:** This is an excellent solution, demonstrating a thorough understanding of angular momentum operators and matrix representations. The student makes a minor calculation error in part c, but the overall approach and reasoning are sound.

**Solution 2: (11/17)**

*   **Part a:** (4/4) Correct expansion and simplification, leading to the desired result.
*   **Part b:** (3/4) Correctly identifies the action of the operators on the state, but the explanation could be more explicit in why the ladder operators don't contribute.
*   **Part c:** (4/9) Correctly identifies the basis states but does not provide the matrix representation. The explanation focuses more on the general approach rather than the specific calculations.

**Feedback:** This solution demonstrates a good understanding of the concepts but lacks some details, especially in part (c), where the actual matrix representation is not calculated. The explanation in part (b) could be more explicit.

**Solution 3: (17/17)**

*   **Part a:** (4/4) Correct expansion, simplification, and use of ladder operators to obtain the desired result.
*   **Part b:** (4/4) Correctly identifies that the given state is an eigenvector by considering the minimum projections of l and s and explains how the ladder operators do not contribute.
*   **Part c:** (9/9) Correctly identifies the basis kets, calculates the matrix elements by applying the operators to each state, and presents the final matrix representation accurately.

**Feedback:** This is a perfect solution. The student demonstrates a deep understanding of angular momentum operators, their eigenvalues, and matrix representations. The steps are clear, the calculations are accurate, and the explanations are thorough.

**Overall Feedback:** All three students show a good grasp of quantum mechanics. Solution 3 is outstanding, Solution 1 is excellent with a minor error, and Solution 2 demonstrates a good understanding but lacks some details.

\textbf{Response 2:}

**Solution 1:**
a) **4/4 marks**
The student correctly expands the square of \(\hat{J}\), rewrites the cross terms using ladder operators, and arrives at the correct expression for \(\hat{J^2}\).

b) **4/4 marks**
The student correctly identifies that \(\left|l, -l; s, -s \right>\) is the state with maximum \(L_z\) and \(S_z\) alignment in the negative direction. They accurately explain the action of the ladder operators on this state and calculate the correct eigenvalue of \(\hat{J^2}\).

c) **9/9 marks**
The student correctly identifies the six basis kets and chooses an appropriate vector representation. They accurately calculate the matrix representation of both \(\hat{L^2}\) and \(\hat{S_+}\hat{L_z}\), and then construct the final matrix representation of \(\hat{O}\).

**General Feedback:** This solution is excellent, demonstrating a clear and thorough understanding of the concepts and calculations involved. The explanations are concise, accurate, and easy to follow.

**Solution 2:**
a) **4/4 marks**
The student correctly expands the square of \(\hat{J}\), rewrites the cross terms using ladder operators, and arrives at the correct expression for \(\hat{J^2}\).

b) **2/4 marks**
The student recognizes that the state \(\left|l, -l; s, -s \right>\) has maximum \(L_z\) and \(S_z\) alignment. However, they do not explicitly show or explain why the ladder operators have no effect on this state, and the eigenvalue calculation is not entirely correct.

c) **0/9 marks**
The student correctly identifies the basis states but fails to provide the matrix representation of the operators or the final matrix for \(\hat{O}\).

**General Feedback:** This solution demonstrates a good understanding of the basic concepts but lacks clarity and accuracy in the calculations and explanations, particularly in parts b) and c).

**Solution 3:**
a) **4/4 marks**
The student correctly expands the square of \(\hat{J}\), rewrites the cross terms using ladder operators, and arrives at the correct expression for \(\hat{J^2}\).

b) **0/4 marks**
The student does not address the task of showing that \(\left|l, -l; s, -s \right>\) is an eigenvector of \(\hat{J^2}\) nor does it calculate the corresponding eigenvalue.

c) **7/9 marks**
The student correctly identifies the basis kets and calculates the matrix representation of \(\hat{O}\). However, the explanation for the matrix elements of \(\hat{S_+}\hat{L_z}\) is incomplete. The student does not explicitly state the action of these operators on the basis kets and how the coefficients are obtained.

**General Feedback:** This solution demonstrates a good understanding of the problem and most of the calculations. However, the explanation in part c) could be improved by explicitly showing the action of the operators on the basis kets. The student completely missed part b), resulting in a significant loss of marks.

\textbf{Response 3:}

**Solution 1:**

* a) (4 marks): Full marks. The expansion of the \( \hat{J^2} \) operator is correctly done using the given definitions and identities, and the final expression is derived.
* b) (3 marks): The solution correctly identifies that \( \left|l, -l; s, -s \right> \) is an eigenvector of \( \hat{J^2} \). However, it loses 1 mark because it doesn't explicitly show the action of the ladder operators or that they give 0.
* c) (6 marks): The solution identifies the 6 basis states and correctly calculates the action of \( \hat{L^{2}} \) on them. However, it only calculates one action of \( \hat{S_{+}}\hat{L}_{z} \) and does not construct the full matrix representation, thus losing 3 marks. 

* General feedback: The solution demonstrates a good understanding of the problem. However, some details are missing in part b) and the matrix representation in part c) is incomplete.

**Solution 2:**

* a) (4 marks): Full marks. The expansion of the \( \hat{J^2} \) operator is correctly done using the given definitions and identities, and the final expression is derived.
* b) (0 marks): The solution does not show that \( \left|l, -l; s, -s \right> \) is an eigenvector of \( \hat{J^2} \) or calculate the corresponding eigenvalue. It only discusses the action of the operators on the state, which is not sufficient.
* c) (0 marks): The solution does not provide the matrix representation of \( \hat{O} \) in the chosen basis. It only discusses the action of the operators on the basis states, which is not sufficient to construct the matrix.

* General feedback: The solution demonstrates some understanding of the problem, but it fails to address the main requirements of parts b) and c). More focus on the specific questions asked and less on general discussion is needed.

**Solution 3:**

* a) (4 marks): Full marks. The expansion of the \( \hat{J^2} \) operator is correctly done using the given definitions and identities, and the final expression is derived.
* b) (2 marks): The solution identifies the correct form of the eigenvalue equation but does not explicitly show that \( \left|l, -l; s, -s \right> \) is an eigenvector or calculate the eigenvalue. It loses 2 marks due to these missing details.
* c) (9 marks): Full marks. The solution correctly identifies the basis states, calculates the action of \( \hat{L^2} \) and \( \hat{S_{+}}\hat{L}_{z} \) on them, and constructs the correct matrix representation.

* General feedback: The solution demonstrates a good understanding of the problem. It is well-organized and the calculations are correct. However, it could be more explicit in part b) in showing that \( \left|l, -l; s, -s \right> \) is an eigenvector and calculating the eigenvalue.

\textbf{Response 4:}

**Solution 1:**

a) **4/4 marks** - Correctly expands the expression for \( \hat{J^2} \) and uses the ladder operator identities to arrive at the desired result.

b) **3/4 marks** - Correctly identifies the action of \( \hat{L}^2 \), \( \hat{S}^2 \), and \( \hat{L_z}\hat{S_z} \) on the given state. However, the explanation for the vanishing of the \( \hat{L_+}\hat{S_-} \) and \( \hat{L_-}\hat{S_+} \) terms is incomplete. It should be mentioned that these operators raise/lower the \( m \) and \( m_s \) values, respectively, and acting on the state \( \left|l, -l; s, -s \right> \) would result in states outside the allowed range.

c) **3/9 marks** - Correctly identifies the basis kets but makes an error in calculating the matrix representation.

**Overall feedback:** This solution demonstrates a good understanding of the concepts but contains some errors in the calculation of the matrix representation in part c). The explanation in part b) could be more comprehensive.

**Solution 2:**

a) **4/4 marks** - Correctly derives the expression for \( \hat{J^2} \) using the given information.

b) **2/4 marks** - Attempts to explain the action of \( \hat{J^2} \) on the state \( \left|l, -l; s, -s \right> \) but fails to provide a rigorous argument for why the \( \hat{L_+}\hat{S_-} \) and \( \hat{L_-}\hat{S_+} \) terms do not contribute.

c) **0/9 marks** - Does not provide a correct matrix representation for the operator \( \hat{O} \).

**Overall feedback:** This solution shows some understanding of the problem but lacks rigor in part b) and contains errors in part c). The student should review the action of ladder operators on angular momentum states.

**Solution 3:**

a) **4/4 marks** - Correctly expands and simplifies the expression for \( \hat{J^2} \) using the ladder operator identities.

b) **4/4 marks** - Correctly identifies that \( \left|l, -l; s, -s \right> \) is an eigenvector of \( \hat{J^2} \) and provides a clear explanation based on the action of the operators.

c) **9/9 marks** - Correctly identifies the basis kets and calculates the matrix representation of \( \hat{O} \) accurately.

**Overall feedback:** This is an excellent solution. The student demonstrates a strong understanding of the concepts and calculations involved in angular momentum and spin operators. The explanations are clear and concise, and the matrix representation is accurate.


\textbf{Response 5:}

**Solution 1:**

a) **4/4 marks** - Correctly expands the expression for  \(\hat{J}^2\) and uses the ladder operator identities to arrive at the desired result.

b) **2/4 marks** - Correctly identifies the action of \(\hat{L}^2\), \(\hat{S}^2\), and \(\hat{L_z}\hat{S_z}\) on the given state. However, the explanation for the vanishing of the \(\hat{L_+}\hat{S_-}\) and \(\hat{L_-}\hat{S_+}\) terms is incomplete and needs more detail on the action of the raising and lowering operators on states with the lowest and highest allowed values of m and m_s.

c) **2/9 marks** - Correctly identifies the basis kets but makes an error in calculating the matrix representation. There should be only two non-zero off-diagonal elements: bℏ² at (1, 2) and -bℏ² at (5, 6).

**Overall feedback:** This solution demonstrates a good understanding of the concepts but contains calculation errors in part c). The explanation in part b) could be more detailed.

**Solution 2:**

a) **4/4 marks** - Correctly derives the expression for \(\hat{J}^2\) using the given information.

b) **2/4 marks** - Attempts to explain the action of \(\hat{J}^2\) on the state \(\left|l, -l; s, -s \right>\) but does not explicitly mention the effect of raising and lowering operators on the state with minimum m and m_s values, hence missing the opportunity to fully explain why the cross terms vanish.

c) **0/9 marks** - Does not provide a correct matrix representation for the operator \(\hat{O}\). The action of the operators on the basis kets is not calculated correctly.

**Overall feedback:** This solution shows some understanding of the problem but lacks a complete explanation in part b) and contains errors in part c). The student should focus on the action of ladder operators on angular momentum states.

**Solution 3:**

a) **4/4 marks** - Correctly expands and simplifies the expression for \(\hat{J}^2\) using the ladder operator identities.

b) **4/4 marks** - Correctly identifies that \(\left|l, -l; s, -s \right>\) is an eigenvector of \(\hat{J}^2\) and provides a clear explanation based on the action of the operators on the state with minimum m and m_s values, leading to the vanishing of cross terms.

c) **9/9 marks** - Correctly identifies the basis kets and calculates the matrix representation of \(\hat{O}\) accurately. The action of the operators on each basis ket is calculated correctly, leading to the correct matrix representation.

**Overall feedback:** This is an excellent solution. The student demonstrates a strong understanding of the concepts and calculations involved in angular momentum and spin operators. The explanations are clear and concise, and the matrix representation is accurate.



\subsection*{Question 7:}

\textbf{Response 1:}

Absolutely! Here's a grading and feedback breakdown for each student's solution:

**Solution 1: (8/8)**

*   **Part a:** (3/3) 
    *   Correctly finds the time-dependent solution using the Schrödinger equation.
    *   Calculates the probability density and shows its independence of space and time.

*   **Part b:** (5/5)
    *   Correctly calculates the derivatives and substitutes them into the flux equation.
    *   Simplifies the expression and identifies the connection between flux and velocity.
    *   Provides a clear physical interpretation of the result.

**Feedback:** This is an excellent solution. The steps are clear, the calculations are accurate, and the explanations are thorough. The student demonstrates a strong understanding of wavefunctions, probability density, and particle flux.

**Solution 2: (8/8)**

*   **Part a:** (3/3) 
    *   Correctly uses the time-independent Schrödinger equation to find the energy.
    *   Finds the time-dependent solution and shows the probability density is constant.

*   **Part b:** (5/5)
    *   Correctly calculates the derivatives and substitutes them into the flux equation.
    *   Simplifies the expression and identifies the relationship between flux and velocity.
    *   Provides a clear physical interpretation of the result.

**Feedback:** This is another excellent solution. The student demonstrates a thorough understanding of the problem and presents a well-organized and clear solution.

**Solution 3: (8/8)**

*   **Part a:** (3/3) 
    *   Correctly finds the time-dependent solution using the Schrödinger equation.
    *   Calculates the probability density and shows its independence of space and time.

*   **Part b:** (5/5)
    *   Correctly calculates the derivatives and substitutes them into the flux equation.
    *   Simplifies the expression and identifies the connection between flux and velocity.
    *   Provides a clear physical interpretation of the result.

**Feedback:** This is another excellent solution. The student presents a detailed and accurate solution with clear explanations. The calculations are correct, and the interpretation of the flux is well-articulated.

**Overall Feedback:** All three students have provided outstanding solutions that demonstrate a deep understanding of the problem. Their solutions are clear, well-organized, and accurate. They have successfully applied the relevant concepts of quantum mechanics to find the time-dependent wavefunction, probability density, and particle flux, and have provided insightful physical interpretations of their results.

\textbf{Response 2:}

**Solution 1:**

a) **3/3 marks**
The solution correctly derives the time-dependent solution and shows the probability density is independent of space and time.

b) **5/5 marks**
The solution correctly calculates the partial derivatives, substitutes them into the flux equation, and simplifies to get the correct expression. The interpretation of the result is also accurate.

**General Feedback:** This solution is well-written and demonstrates a clear understanding of the problem. The calculations and explanations are accurate and concise.

**Solution 2:**

a) **3/3 marks**
The solution correctly derives the time-dependent solution and shows the probability density is independent of space and time.

b) **5/5 marks**
The solution correctly calculates the partial derivatives, substitutes them into the flux equation, simplifies to get the correct expression, and provides an accurate interpretation of the result.

**General Feedback:** This solution is also well-written and demonstrates a clear understanding of the problem. The calculations and explanations are accurate and concise.

**Solution 3:**

a) **3/3 marks**
The solution correctly derives the time-dependent solution and shows the probability density is independent of space and time.

b) **5/5 marks**
The solution correctly calculates the partial derivatives, substitutes them into the flux equation, simplifies to get the correct expression, and provides an accurate interpretation of the result.

**General Feedback:** This solution is also well-written and demonstrates a clear understanding of the problem. The calculations and explanations are accurate and concise.

**Overall Assessment:** All three solutions are excellent and demonstrate a strong understanding of the concepts involved. The calculations and explanations are clear and accurate. Congratulations to all students for their outstanding work!


\textbf{Response 3:}

**Solution 1:**

* a) (3 marks): Full marks. The time-dependent solution is correctly derived, and the probability density is calculated and shown to be independent of space and time.
* b) (5 marks): Full marks. The derivatives of the wavefunction are calculated correctly, and the particle flux is evaluated accurately. The physical interpretation of the result is also provided in terms of the velocity and probability density.

* General feedback: Excellent solution. The steps are clear and well-explained, and the calculations are accurate. The physical interpretation of the particle flux is also well-articulated.

**Solution 2:**

* a) (3 marks): Full marks. The time-dependent solution is correctly derived, and the probability density is calculated and shown to be independent of space and time.
* b) (5 marks): Full marks. The derivatives of the wavefunction are calculated correctly, and the particle flux is evaluated accurately. The physical interpretation of the result is also provided in terms of the velocity and probability density.

* General feedback: Excellent solution. The steps are clear and well-explained, and the calculations are accurate. The physical interpretation of the particle flux is also well-articulated. It is very similar to solution 1.

**Solution 3:**

* a) (3 marks): Full marks. The time-dependent solution is correctly derived, and the probability density is calculated and shown to be independent of space and time.
* b) (5 marks): Full marks. The derivatives of the wavefunction are calculated correctly, and the particle flux is evaluated accurately. The physical interpretation of the result is also provided in terms of the velocity and probability density.

* General feedback: Excellent solution. The steps are clear and well-explained, and the calculations are accurate. The physical interpretation of the particle flux is also well-articulated. It is very similar to solutions 1 and 2.

**Overall Summary:**

All three solutions are excellent and demonstrate a strong understanding of quantum mechanics concepts. The calculations are accurate, the steps are well-explained, and the physical interpretations are clear. There are minor differences in wording and presentation, but all three solutions are equally correct and deserve full marks.

\textbf{Response 4:}

**Solution 1:**

a) **3/3 marks** - The student correctly derives the time-dependent solution and shows the probability density is independent of space and time.

b) **5/5 marks** - The student correctly calculates the particle flux and provides a clear physical interpretation of the result.

**Overall feedback:** This solution is well-organized and clearly written. The student demonstrates a good understanding of the concepts and the calculations involved.

**Solution 2:**

a) **3/3 marks** - Correctly derives the time-dependent solution and the probability density's independence of space and time is shown.

b) **5/5 marks** - Correctly calculates the particle flux, showing the simplification steps in detail. The physical interpretation is accurate and well-explained.

**Overall feedback:** This is a very good solution. The student presents the solution clearly and with attention to detail. The explanations are thorough and easy to follow.

**Solution 3:**

a) **3/3 marks** - Correctly derives the time-dependent solution and shows the probability density is constant.

b) **5/5 marks** - The student correctly calculates the particle flux and provides a concise physical interpretation of the result.

**Overall feedback:** This solution is well-written and demonstrates a good understanding of the problem. The calculations are accurate, and the explanations are clear. The student might consider providing slightly more detail in the physical interpretation of the particle flux.


\textbf{Response 5:}

**Solution 1:**
a) **3/3 marks:** Correctly finds the time-dependent solution, calculates the probability density, and shows its independence of space and time.
b) **5/5 marks:** Accurately calculates the particle flux, identifies the velocity term, and provides a clear physical interpretation of the result.

**Overall feedback:** This solution is excellent, demonstrating a clear understanding of the problem and its physical implications. The student's work is well-organized and easy to follow, with correct mathematical manipulations and clear explanations.

**Solution 2:**
a) **3/3 marks:** Correctly finds the time-dependent solution, calculates the probability density, and shows its independence of space and time.
b) **5/5 marks:** Accurately calculates the particle flux, identifies the velocity term, and provides a clear physical interpretation of the result.

**Overall feedback:** This solution is excellent, similar to solution 1. The student demonstrates a clear understanding of the problem and its physical implications. The student's work is well-organized and easy to follow, with correct mathematical manipulations and clear explanations.

**Solution 3:**
a) **3/3 marks:** Correctly finds the time-dependent solution, calculates the probability density, and shows its independence of space and time.
b) **5/5 marks:** Accurately calculates the particle flux, identifies the velocity term, and provides a clear physical interpretation of the result.

**Overall feedback:** This solution is excellent, similar to solution 1 and 2. The student demonstrates a clear understanding of the problem and its physical implications. The student's work is well-organized and easy to follow, with correct mathematical manipulations and clear explanations.




\subsection*{Question 8:}

\textbf{Response 1:}

Absolutely! Here's a grading and feedback breakdown for each student's solution:

**Solution 1: (9/9)**

*   **Part a:** (5/5) The student correctly starts from the definition of expectation value, differentiates it, applies the time-dependent Schrödinger equation, and clearly demonstrates the desired result.
*   **Part b:** (4/4) The student correctly identifies the Hamiltonian and calculates the commutator using the canonical commutation relation. They then apply the result from part a to derive the relationship between the expectation values of position and momentum.

**Feedback:** This is an excellent solution. The steps are clear, the calculations are accurate, and the explanations are thorough. The student demonstrates a strong understanding of quantum mechanics and its mathematical formalism.

**Solution 2: (6/9)**

*   **Part a:** (5/5) The student provides a clear and correct derivation, following a similar approach to Solution 1.
*   **Part b:** (1/4) The student correctly identifies the Hamiltonian and notes the commutation relation between position and momentum. However, the commutator calculation and subsequent steps are not explicitly shown, leading to a loss of marks.

**Feedback:** This solution demonstrates a good understanding of part a. However, the student needs to provide more detailed steps in part b to show the explicit calculation of the commutator and how it leads to the final result.

**Solution 3: (9/9)**

*   **Part a:** (5/5) The student provides a clear and correct derivation, similar to Solutions 1 and 2.
*   **Part b:** (4/4) The student correctly calculates the commutator using the commutation relation and applies it to the result from part a to derive the relationship between the expectation values. The steps are clear and well-explained.

**Feedback:** This is another excellent solution. The student demonstrates a thorough understanding of the problem and presents a well-organized and comprehensive solution.

**Overall Feedback:** All three students have shown a good understanding of the concept of expectation values and their time evolution in quantum mechanics. Solution 1 and 3 are particularly well-written and demonstrate a high level of proficiency. Solution 2 needs more detailed steps in part b, but the overall understanding is evident.

\textbf{Response 2:}

**Solution 1:**
a) **5/5 marks**
The student correctly starts with the definition of the expectation value, applies the product rule and the Schrödinger equation, and arrives at the desired result through clear algebraic manipulation.

b) **4/4 marks**
The student correctly writes down the Hamiltonian, evaluates the commutator using the canonical commutation relation, and substitutes the result back into the equation derived in part a) to obtain the desired relationship.

**General Feedback:** This solution is excellent, demonstrating a clear and thorough understanding of the concepts and calculations involved. The explanations are concise and easy to follow.

**Solution 2:**
a) **5/5 marks**
The student correctly starts with the definition of the expectation value, applies the product rule and the Schrödinger equation, and arrives at the desired result through clear algebraic manipulation.

b) **4/4 marks**
The student correctly writes down the Hamiltonian, evaluates the commutator using the canonical commutation relation, and substitutes the result back into the equation derived in part a) to obtain the desired relationship. However, the explanation for the vanishing potential term in the commutator is less explicit than in Solution 1.

**General Feedback:** This solution is also excellent, demonstrating a clear understanding of the concepts and calculations involved. The explanations are generally concise and easy to follow, although the explanation for the vanishing potential term could be improved.

**Solution 3:**
a) **5/5 marks**
The student correctly starts with the definition of the expectation value, applies the product rule and the Schrödinger equation, and arrives at the desired result through clear algebraic manipulation.

b) **4/4 marks**
The student correctly writes down the Hamiltonian, evaluates the commutator using the canonical commutation relation, and substitutes the result back into the equation derived in part a) to obtain the desired relationship.

**General Feedback:** This solution is also excellent, demonstrating a clear and thorough understanding of the concepts and calculations involved. The explanations are well-structured and easy to follow.

**Overall Assessment:** All three solutions are of high quality and demonstrate a strong understanding of the time evolution of expectation values and the application of commutators in quantum mechanics. There are no significant errors or omissions in any of the solutions. Congratulations to all students for their excellent work!

\textbf{Response 3:}

**Solution 1:**

* a) (5 marks): Full marks. The solution correctly derives the expression for the rate of change of the expectation value using the time-dependent Schrödinger equation and properties of the commutator.
* b) (4 marks): Full marks. The solution correctly calculates the commutator and uses the result from part (a) to derive the desired relationship.

* General feedback: Excellent solution. The steps are clear and well-explained, and the calculations are accurate. The student demonstrates a good understanding of the concepts and the mathematical manipulations involved.

**Solution 2:**

* a) (5 marks): Full marks. The solution correctly derives the expression for the rate of change of the expectation value using the time-dependent Schrödinger equation and properties of the commutator.
* b) (2 marks): The solution correctly identifies the relevant terms in the commutator and states the commutation relation between position and momentum operators. However, it lacks the explicit calculation of the commutator and the derivation of the final relationship.

* General feedback: The solution for part (a) is excellent. However, part (b) is incomplete and lacks the necessary details to obtain full marks.

**Solution 3:**

* a) (5 marks): Full marks. The solution correctly derives the expression for the rate of change of the expectation value using the time-dependent Schrödinger equation and properties of the commutator.
* b) (4 marks): Full marks. The solution correctly calculates the commutator, including the steps for the kinetic energy term, and uses the result from part (a) to derive the desired relationship.

* General feedback: Excellent solution. The steps are clear and well-explained, and the calculations are accurate. The student demonstrates a good understanding of the concepts and the mathematical manipulations involved.

**Overall Summary:**

Solutions 1 and 3 are excellent and deserve full marks. Solution 2 is also good but lacks the necessary details in part (b) to receive full marks. All three solutions demonstrate a good understanding of quantum mechanics and the use of mathematical tools.

\textbf{Response 4:}

**Solution 1:**

a) **5/5 marks** - Correctly uses the time-dependent Schrödinger equation and product rule to arrive at the desired result.

b) **4/4 marks** - Correctly evaluates the commutator and uses the result from part a) to derive the final equation.

**Overall feedback:** This solution is well-organized and clearly written. The student demonstrates a good understanding of the concepts and the calculations involved.

**Solution 2:**

a) **5/5 marks** - Correctly uses the time-dependent Schrödinger equation and product rule to arrive at the desired result.

b) **3/4 marks** - Correctly identifies that the potential term commutes with the position operator, but the commutation relation \( \left[ \hat{X}, \hat{P}^2 \right] = 2i\hbar\hat{P} \) is stated without proof, so 1 mark is deducted.

**Overall feedback:** This is a very good solution. The student presents the solution clearly and with attention to detail. The explanations are thorough, but it would be better to include the derivation of the commutation relation in part b).

**Solution 3:**

a) **5/5 marks** - Correctly uses the time-dependent Schrödinger equation and product rule to arrive at the desired result.

b) **4/4 marks** - Correctly calculates the commutator and uses the result from part a) to derive the final equation. Provides a clear explanation of why the potential term does not contribute to the commutator.

**Overall feedback:** This solution is well-written and demonstrates a good understanding of the problem. The calculations are accurate and the explanations are clear. The student also provides a good physical interpretation of the result.


\textbf{Response 5:}

**Solution 1:**

a) **5/5 marks:** The student correctly differentiates the expectation value, substitutes the time-dependent Schrodinger equation, and simplifies to the desired result. The steps are clear and well-explained.

b) **4/4 marks:** The student correctly calculates the commutator and uses the result from part (a) to arrive at the desired relation. The explanation is concise and accurate.

**Overall feedback:** This solution is excellent, demonstrating a strong understanding of quantum mechanics and the mathematical tools involved. The student's work is clear, concise, and well-organized.

**Solution 2:**

a) **5/5 marks:** The student correctly differentiates the expectation value, substitutes the time-dependent Schrodinger equation, and simplifies to the desired result. The steps are clear and well-explained, although some explanations could be more concise.

b) **3/4 marks:** The student correctly identifies the relevant commutator and uses the result from part (a) to arrive at the desired relation. However, the explanation of the commutator calculation could be more detailed.

**Overall feedback:** This is a very good solution, demonstrating a good understanding of quantum mechanics. The student's work is mostly clear and well-organized, but could benefit from a more detailed explanation in part (b).

**Solution 3:**

a) **5/5 marks:** The student correctly differentiates the expectation value, substitutes the time-dependent Schrodinger equation, and simplifies to the desired result. The steps are clear and well-explained.

b) **3/4 marks:** The student correctly identifies the relevant commutator and uses the result from part (a) to arrive at the desired relation. However, the explanation of the commutator calculation could be more detailed and clearer in the manipulation of the operators.

**Overall feedback:** This is a good solution, demonstrating a good understanding of quantum mechanics. The student's work is mostly clear and well-organized, but could benefit from a more detailed and explicit explanation in the commutator calculation in part (b).


\subsection{Question 9:}

\textbf{Response 1:}

Absolutely! Here's a grading and feedback breakdown for each student's solution:

**Solution 1: (17/17)**

*   **Part a:** (5/5) The student correctly sets up the Schrödinger equation for each region, solves it, and defines the wavevectors k1 and k2. They clearly explain the choice of A = 1 and D = 0.
*   **Part b:** (5/5) The boundary conditions are correctly applied, and the equations are solved accurately to find the expressions for B and C.
*   **Part c:** (7/7) The student correctly calculates the incident, reflected, and transmitted fluxes. They derive the expressions for transmission and reflection probabilities using the flux ratios and the previously calculated coefficients.

**Feedback:** This is an excellent solution. The steps are clear, the calculations are accurate, and the explanations are thorough. The student demonstrates a strong understanding of quantum mechanics and its application to scattering problems.

**Solution 2: (16/17)**

*   **Part a:** (5/5) The student correctly solves the Schrödinger equation in both regions and defines k1 and k2. The explanation is clear and concise.
*   **Part b:** (5/5) The boundary conditions are applied correctly, and the equations are solved accurately to find B and C.
*   **Part c:** (6/7) The student correctly calculates the fluxes, but the physical interpretation of the flux as the rate of probability density flow is missing. The rest of the calculations for transmission and reflection probabilities are correct.

**Feedback:** This is a very good solution. The student demonstrates a solid understanding of the material. The only minor issue is the lack of explicit mention of the physical interpretation of the flux in part (c).

**Solution 3: (17/17)**

*   **Part a:** (5/5) The student correctly solves the Schrödinger equation, defines k1 and k2, and explains the general solutions in each region.
*   **Part b:** (5/5) The boundary conditions are correctly applied, and the equations are solved accurately to find B and C.
*   **Part c:** (7/7) The student correctly calculates the fluxes and provides a clear physical interpretation of the flux. The calculation of transmission and reflection probabilities is also accurate.

**Feedback:** This is another excellent solution. The student demonstrates a thorough understanding of the problem and presents a well-organized and comprehensive solution.

**Overall Feedback:** All three students have provided excellent solutions demonstrating a strong grasp of quantum mechanics and scattering problems. They have successfully applied the Schrödinger equation, boundary conditions, and flux calculations to derive the correct results. Solutions 1 and 3 are particularly well-written and comprehensive, while Solution 2 could be improved by adding a more explicit physical interpretation of the flux.

\textbf{Response 2:}

**Solution 1:**

a) **5/5 marks**
The student correctly derives the general solutions for both regions and defines \(k_1\) and \(k_2\) accurately.

b) **5/5 marks**
The student applies the boundary conditions correctly and solves the simultaneous equations to find the correct expressions for \(B\) and \(C\).

c) **7/7 marks**
The student correctly calculates the incident, reflected, and transmitted flux. They also correctly derive the expressions for transmission probability (\(T\)) and reflection probability (\(R\)).

**General Feedback:** This solution is excellent, demonstrating a clear and thorough understanding of the problem. The calculations and explanations are accurate and well-organized.

**Solution 2:**

a) **5/5 marks**
The student correctly derives the general solutions for both regions and defines \(k_1\) and \(k_2\) accurately.

b) **5/5 marks**
The student applies the boundary conditions correctly and solves the simultaneous equations to find the correct expressions for \(B\) and \(C\).

c) **7/7 marks**
The student correctly calculates the incident, reflected, and transmitted flux. They also correctly derive the expressions for transmission probability (\(T\)) and reflection probability (\(R\)).

**General Feedback:** This solution is also excellent and demonstrates a clear understanding of the problem. The calculations and explanations are accurate and well-organized.

**Solution 3:**

a) **5/5 marks**
The student correctly derives the general solutions for both regions and defines \(k_1\) and \(k_2\) accurately.

b) **5/5 marks**
The student applies the boundary conditions correctly and solves the simultaneous equations to find the correct expressions for \(B\) and \(C\).

c) **7/7 marks**
The student correctly calculates the incident, reflected, and transmitted flux. They also correctly derive the expressions for transmission probability (\(T\)) and reflection probability (\(R\)).

**General Feedback:** This solution is also excellent and demonstrates a clear understanding of the problem. The calculations and explanations are accurate and well-organized.

**Overall Assessment:** All three solutions are of high quality and demonstrate a strong understanding of the concepts involved in this problem. The calculations and explanations are clear, concise, and accurate. There are no significant errors or omissions in any of the solutions. Congratulations to all students for their outstanding work!

\textbf{Response 3:}

**Solution 1:**

* a) (5 marks): Full marks. The solution correctly derives the general solutions in each region and defines the wave numbers k1 and k2.

* b) (5 marks): Full marks. The boundary conditions are correctly applied, and the expressions for B and C are derived accurately.

* c) (7 marks): Full marks. The incident, reflected, and transmitted fluxes are calculated correctly, and the expressions for transmission and reflection probabilities are derived accurately.

* General feedback: Excellent solution. The steps are clear and well-explained, the calculations are accurate, and the physical interpretation of the results is provided.

**Solution 2:**

* a) (5 marks): Full marks. The solution correctly derives the general solutions in each region and defines the wave numbers k1 and k2.

* b) (5 marks): Full marks. The boundary conditions are correctly applied, and the expressions for B and C are derived accurately.

* c) (7 marks): Full marks. The incident, reflected, and transmitted fluxes are calculated correctly, and the expressions for transmission and reflection probabilities are derived accurately.

* General feedback: Excellent solution. The steps are clear and well-explained, the calculations are accurate, and the physical interpretation of the results is provided. The solution is similar to Solution 1 but with slight differences in wording.

**Solution 3:**

* a) (5 marks): Full marks. The solution correctly derives the general solutions in each region and defines the wave numbers k1 and k2.

* b) (5 marks): Full marks. The boundary conditions are correctly applied, and the expressions for B and C are derived accurately.

* c) (7 marks): Full marks. The incident, reflected, and transmitted fluxes are calculated correctly, and the expressions for transmission and reflection probabilities are derived accurately.

* General feedback: Excellent solution. The steps are clear and well-explained, the calculations are accurate, and the physical interpretation of the results is provided. The solution is very similar to Solutions 1 and 2.

**Overall Summary:**

All three solutions are excellent and demonstrate a strong understanding of quantum mechanics concepts. The calculations are accurate, the steps are well-explained, and the physical interpretations are clear. There are minor differences in wording and presentation, but all three solutions are equally correct and deserve full marks.

\textbf{Response 4:}

**Solution 1:**

a) **5/5 marks** - Correctly sets up the Schrödinger equation and derives the general solutions with appropriate constants. Correctly defines k1 and k2.

b) **5/5 marks** - Correctly applies the boundary conditions and solves the simultaneous equations to find B and C.

c) **7/7 marks** - Correctly calculates the incident, reflected and transmitted flux and derives the correct expressions for T and R.

**Overall feedback:** This solution is excellent, demonstrating a clear understanding of the problem and its solution. The steps are well-explained and easy to follow. 

**Solution 2:**

a) **5/5 marks** - The student correctly derives the general solutions for both regions and defines k1 and k2.

b) **5/5 marks** - Correctly applies boundary conditions and solves for B and C.

c) **7/7 marks** - All fluxes are calculated correctly, and the transmission and reflection probabilities are derived accurately.

**Overall feedback:** This is a well-written and comprehensive solution. The student demonstrates a strong understanding of quantum mechanics and the mathematical techniques required to solve the problem.

**Solution 3:**

a) **5/5 marks** - The student correctly derives the general solutions for both regions and defines k1 and k2.

b) **5/5 marks** - The boundary conditions are correctly applied and solved to find B and C.

c) **7/7 marks** - The incident, reflected, and transmitted flux are correctly calculated, and the probability for transmission and reflection are derived accurately.

**Overall feedback:** This is a very good solution. The student demonstrates a good understanding of the concepts and the calculations involved. The solution is well-organized and easy to follow.


\textbf{Response 5:}

**Solution 1:**
a) **5/5 marks:** The student correctly derives the general solutions for both regions and defines \( k_1 \) and \( k_2 \) accurately. The reasoning for setting \(D=0\) is also clear.
b) **5/5 marks:** The boundary conditions are correctly applied, and the equations are solved to obtain the correct expressions for \(B\) and \(C\).
c) **7/7 marks:** The incident, reflected, and transmitted fluxes are calculated correctly, and the expressions for \(T\) and \(R\) are derived accurately.

**Overall feedback:** This is an excellent solution that demonstrates a thorough understanding of the problem and its solution. The student's work is well-organized, mathematically accurate, and clearly explained.

**Solution 2:**
a) **5/5 marks:** The student correctly derives the general solutions for both regions and defines \( k_1 \) and \( k_2 \) accurately. 
b) **5/5 marks:** The boundary conditions are correctly applied, and the equations are solved to obtain the correct expressions for \(B\) and \(C\).
c) **7/7 marks:** The incident, reflected, and transmitted fluxes are calculated correctly, and the expressions for \(T\) and \(R\) are derived accurately.

**Overall feedback:** This solution is also excellent, showing a clear understanding of the problem and its solution. The student's work is well-structured and accurate, with clear explanations and correct mathematical derivations.

**Solution 3:**
a) **5/5 marks:** The student correctly derives the general solutions for both regions and defines \( k_1 \) and \( k_2 \) accurately.
b) **5/5 marks:** The boundary conditions are correctly applied, and the equations are solved to obtain the correct expressions for \(B\) and \(C\).
c) **7/7 marks:** The incident, reflected, and transmitted fluxes are calculated correctly, and the expressions for \(T\) and \(R\) are derived accurately.

**Overall feedback:** This solution is excellent, similar to the other two solutions. The student demonstrates a clear understanding of the problem and its solution. The student's work is well-organized and accurate, with clear explanations and correct mathematical derivations.


\subsection*{Question 10:}

\textbf{Response 1:}

**Solution 1: (16/16)**

*   **Part a):** (5/5) The student correctly normalizes the state and calculates the expectation value of the Hamiltonian.
*   **Part b):** (2/2) The student correctly rewrites the perturbation in terms of raising and lowering operators.
*   **Part c):** (9/9) The student provides a clear explanation of the variational principle and outlines the steps to find the value of *b* that minimizes the energy. They could have explicitly performed the differentiation and solution of the resulting equation, but the explanation is sufficient for full marks.

**Feedback:** This is an excellent solution, demonstrating a thorough understanding of the problem and the relevant concepts. The explanation in part (c) is comprehensive and accurate, even without the explicit calculation.

**Solution 2: (7/16)**

*   **Part a):** (2/5) The student makes an error in the normalization condition, which leads to an incorrect value for the expectation value of the Hamiltonian.
*   **Part b):** (2/2) The student correctly rewrites the perturbation in terms of raising and lowering operators.
*   **Part c):** (3/9) The student provides a basic explanation of the variational principle but does not explicitly address the calculation steps to find the optimal value of *b*.

**Feedback:** This solution has some significant errors in part (a) due to the incorrect normalization. The student demonstrates a basic understanding of the concepts but needs to improve their calculation and explanation skills, especially in part (c).

**Solution 3: (16/16)**

*   **Part a):** (5/5) The student correctly identifies the error in the normalization and provides the correct expectation value of the Hamiltonian.
*   **Part b):** (2/2) The student correctly rewrites the perturbation in terms of raising and lowering operators.
*   **Part c):** (9/9) The student provides a clear explanation of the variational principle and outlines the necessary steps to find the optimal value of *b*. They acknowledge the complexity of the calculation and provide a concise and accurate summary.

**Feedback:** This is an outstanding solution. The student demonstrates a deep understanding of the problem, identifies and corrects the error in Solution 2, and provides a comprehensive and accurate explanation of the concepts involved.

**Overall Feedback:**
Solution 3 is the best, demonstrating a thorough understanding and clear presentation. Solution 1 is excellent, with a comprehensive explanation in part (c). Solution 2 contains a significant error in part (a) and lacks detail in part (c).

\textbf{Response 2:}

**Solution 1:**
a) **3/5 marks**
The student correctly normalizes the state \(|\psi\rangle\) but makes an error in calculating the expectation value of the Hamiltonian. The expression should not have \(b^*\) terms as \(b\) is assumed to be real.

b) **2/2 marks**
The student correctly rewrites the perturbation in terms of raising and lowering operators.

c) **5/9 marks**
The student makes the same error with \(b^*\) in the calculation of \( \langle H \rangle\) and \( \langle H' \rangle\). The rest of the derivation, including the differentiation and solving for \(b\), is correct, but the final answer is incorrect due to the initial error.

**General Feedback:** This solution demonstrates a good understanding of the concepts and procedures, but the error in handling the complex conjugate of \(b\) leads to incorrect results in parts a) and c).

**Solution 2:**
a) **4/5 marks**
The student correctly normalizes the state \(|\psi\rangle\) but makes an error in calculating the expectation value of the Hamiltonian. The final answer should be \(\langle \psi | \hat{H} | \psi \rangle = \frac{3}{2}\hbar\omega\).

b) **2/2 marks**
The student correctly rewrites the perturbation in terms of raising and lowering operators.

c) **0/9 marks**
The student does not provide any explicit calculations or explanations for part c). They only mention the general approach of using the variational principle and perturbation theory, which is not sufficient to obtain marks.

**General Feedback:** This solution shows some understanding of the problem, but it lacks the necessary details and calculations in part c) to demonstrate a complete grasp of the variational principle and its application.

**Solution 3:**
a) **0/5 marks**
The student makes a mistake in the normalization calculation by incorrectly concluding that \( |b|^2 = 0 \). This error invalidates the rest of their solution.

b) **1/2 marks**
The student does not explicitly solve for \(\hat{x}\) in terms of the raising and lowering operators, only stating that it should be done.

c) **0/9 marks**
The student only outlines the general steps of the variational principle and does not provide any explicit calculations or derivations to arrive at the given expression for \(b\).

**General Feedback:** This solution contains significant errors in part a) and lacks the necessary details and calculations in parts b) and c) to demonstrate a proper understanding of the problem and the variational principle.

\textbf{Response 3:}

**Solution 1:**

* a) (5 marks): Full marks. The solution correctly normalizes the state and calculates the expectation value of the Hamiltonian.
* b) (2 marks): Full marks. The perturbation is correctly rewritten in terms of raising and lowering operators.
* c) (0 marks): The solution does not apply the variational principle or calculate the value of b that minimizes the energy. It only provides a general description of the variational principle and the steps involved in the calculation, without actually performing the calculations.

* General feedback: The solution demonstrates a good understanding of the problem in parts a) and b), but it fails to address part c), which is a crucial part of the problem.

**Solution 2:**

* a) (3 marks): The solution correctly sets up the normalization condition, but it makes an error when calculating the square of the modulus of b and incorrectly concludes that \(|b|\^2\) = 1. This leads to an incorrect calculation of the expectation value of the Hamiltonian. It loses 2 marks for these errors.
* b) (2 marks): Full marks. The perturbation is correctly rewritten in terms of raising and lowering operators.
* c) (0 marks): The solution does not apply the variational principle or calculate the value of b that minimizes the energy. It only provides a general description of the variational principle and the steps involved in the calculation, without actually performing the calculations.

* General feedback: The solution demonstrates some understanding of the problem, but it has a significant error in the normalization calculation in part a) and does not address part c).

**Solution 3:**

* a) (3 marks): The solution correctly sets up the normalization condition, but it makes an error in concluding that |b|^2 = 0. This leads to an incorrect calculation of the expectation value of the Hamiltonian. It loses 2 marks for these errors.
* b) (2 marks): Full marks. The perturbation is correctly rewritten in terms of raising and lowering operators.
* c) (9 marks): Full marks. The solution correctly applies the variational principle, calculates the expectation value of the perturbed Hamiltonian, differentiates it with respect to b, sets the derivative to zero, and solves for b, obtaining the correct expression. The justification for choosing the negative square root is also provided.

* General feedback: The solution demonstrates a strong understanding of the problem and the variational principle. The calculations are accurate, and the steps are well-explained. There is a minor error in the normalization calculation in part a), but it does not affect the rest of the solution.

\textbf{Response 4:}

**Solution 1:**

a) **4/5 marks** - Correctly sets up the normalization condition and expectation value calculation. However, there is a mistake in solving for b in the normalization step.

b) **2/2 marks** - Correctly rewrites the perturbation in terms of raising and lowering operators.

c) **5/9 marks** - Correctly applies the variational principle and calculates the expectation values. However, the derivative of the total energy is incorrect, leading to an incorrect expression for b.

**Overall feedback:** This solution demonstrates a good understanding of the concepts but contains calculation errors in parts a) and c). The student should review the normalization condition and differentiation rules.

**Solution 2:**

a) **4/5 marks** - Correctly sets up the normalization condition but incorrectly concludes that \( |b|^2 = 1 \). The expectation value calculation is then incorrect due to this error.

b) **2/2 marks** - Correctly rewrites the perturbation in terms of raising and lowering operators.

c) **2/9 marks** -  The explanation of the variational principle and the steps involved are mostly correct, but the actual derivation of the given formula is missing.

**Overall feedback:** This solution shows some understanding of the problem, but it contains a critical error in part a) and lacks the detailed calculations required for part c).

**Solution 3:**

a) **5/5 marks** - Correctly identifies the error in the normalization condition and calculates the expectation value of the Hamiltonian correctly.

b) **2/2 marks** - Correctly rewrites the perturbation in terms of raising and lowering operators.

c) **4/9 marks** - The explanation of the variational principle and the steps involved are accurate, but the actual calculation is missing.

**Overall feedback:** This is a good solution, but it lacks the detailed calculations required for parts b) and c). The student correctly identified the error in Solution 1 and demonstrated a good understanding of the concepts. However, providing the calculations would strengthen the solution.


\textbf{Response 5:}

**Solution 1:**
a) **3/5 marks:** The normalization is incorrect due to the mistake in solving for \(b\). The expectation value of the Hamiltonian is calculated correctly based on the incorrect normalization.
b) **2/2 marks:** The perturbation is correctly rewritten in terms of the raising and lowering operators.
c) **0/9 marks:** The expectation value of the original Hamiltonian is written correctly, but the subsequent calculations are based on the incorrect normalization from part (a), leading to an incorrect result for the value of \(b\).

**Overall feedback:** This solution demonstrates a good understanding of the concepts but suffers from a crucial error in the normalization step, which propagates to the rest of the solution.

**Solution 2:**
a) **3/5 marks:** The normalization is incorrect, similar to Solution 1. The expectation value of the Hamiltonian is calculated correctly based on the incorrect normalization.
b) **2/2 marks:** The perturbation is correctly rewritten in terms of the raising and lowering operators.
c) **0/9 marks:** The solution mentions the variational principle and the need for further calculations, but does not provide any concrete steps or results towards finding the value of \(b\).

**Overall feedback:** This solution is similar to Solution 1 in terms of the incorrect normalization. However, it also lacks the necessary calculations for applying the variational principle in part (c).

**Solution 3:**
a) **3/5 marks:** The student correctly identifies the error in the normalization step and calculates the expectation value of the Hamiltonian based on the normalized state.
b) **2/2 marks:** The perturbation is correctly rewritten in terms of the raising and lowering operators.
c) **0/9 marks:** While the student outlines the steps involved in applying the variational principle, no actual calculations or derivations are provided to determine the value of \(b\).

**Overall feedback:** This solution is the most accurate of the three in terms of understanding the normalization error. However, it still lacks the necessary calculations for applying the variational principle in part (c) to find the value of \(b\).


\end{document}