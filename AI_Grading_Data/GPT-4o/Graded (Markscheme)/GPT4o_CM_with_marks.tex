\documentclass[a4paper,11pt]{article}
\setlength\parindent{0pt}
\usepackage{amsmath, amssymb}
\usepackage[margin=1in,letterpaper]
{geometry} % decreases margins

\usepackage{setspace}
\setstretch{1.2}
\usepackage{spverbatim}

\begin{document}
\title{GPT-4o Grading (With Mark Scheme)}
\maketitle

This document compiles the GPT-4o responses when prompted to grade the solutions which have been generated. The markscheme is given to GPT-4o in the prompt. This is the grading for the Classical Mechanics (CM) questions. 

\section{CM}

\subsection*{Question 1:}

\textbf{Response 1:}

Solution 1:

**Grade: 3/4**

**Feedback:**

- **Definition of Conservative Force (1/1):** The student correctly defines a conservative force as one where the work done is independent of the path taken and adds that the work done in a closed loop is zero.

- **Mathematical Definition of Potential Energy (1/2):** The student states that \( \underline{F} = -\nabla U \), which is correct but does not explicitly use the integral form given in the mark scheme. They also miss the point about the arbitrary choice of origin.

- **Non-Conservative Forces and Potential Energy (1/1):** The explanation of why potential energy cannot be defined for non-conservative forces is clear and accurate.

Solution 2:

**Grade: 2/4**

**Feedback:**

- **Definition of Conservative Force (1/1):** The student correctly defines a conservative force and provides additional context about mechanical energy.

- **Mathematical Definition of Potential Energy (1/2):** The student describes the potential energy change using the work done, but does not use the integral form given in the mark scheme and does not mention the arbitrary origin. The gradient relationship \( \underline{F} = -\nabla PE \) is stated, but the integral definition is essential for full marks.

- **Non-Conservative Forces and Potential Energy (0/1):** The explanation is mostly correct but lacks clarity on why potential energy cannot be uniquely defined (no mention of multiple values or non-uniqueness).

Solution 3:

**Grade: 4/4**

**Feedback:**

- **Definition of Conservative Force (1/1):** The student correctly defines a conservative force, including the path independence.

- **Mathematical Definition of Potential Energy (2/2):** The student provides the integral form and the necessary context for understanding potential energy, including the arbitrary nature of the reference point.

- **Non-Conservative Forces and Potential Energy (1/1):** The explanation for why potential energy cannot be defined for non-conservative forces is clear and mentions path dependence and the resulting non-uniqueness.

\bigskip
\textbf{Response 2:}

Solution 1 Evaluation

**Score: 4/4**

**Feedback:**

- **Conservative Force Definition (1/1):** The explanation correctly defines a conservative force as one where work done is independent of the path taken.

- **Potential Energy Definition (2/2):** While the solution uses the gradient form instead of the integral, it still conveys the same concept accurately and explains the relationship between force and potential energy well.

- **Non-Conservative Force Explanation (1/1):** Clearly explains why potential energy cannot be defined for non-conservative forces, mentioning the path dependency of work done.

Solution 2 Evaluation

**Score: 3/4**

**Feedback:**

- **Conservative Force Definition (1/1):** Correctly defines a conservative force as one where the work done is path-independent.

- **Potential Energy Definition (1/2):** The potential energy is discussed, but the integral form of the potential energy is not provided. The explanation primarily relies on the gradient form and does not explicitly state the arbitrary origin point.

- **Non-Conservative Force Explanation (1/1):** Provides a clear explanation of why potential energy cannot be defined for non-conservative forces, mentioning path dependency.

Solution 3 Evaluation

**Score: 4/4**

**Feedback:**

- **Conservative Force Definition (1/1):** Accurately defines a conservative force, similar to the other solutions.

- **Potential Energy Definition (2/2):** Provides a detailed and mathematically correct definition using the integral form and explaining the differential relationship. Mentions the curl condition for conservative forces.

- **Non-Conservative Force Explanation (1/1):** Clearly explains why potential energy cannot be defined for non-conservative forces, addressing the path dependence and energy dissipation.

\bigskip
\textbf{Response 3:}

Solution 1

**Marks: 4/4**

**Feedback:**

- This solution correctly defines a conservative force as one where the work done is path-independent and includes the important note that the work done in a closed loop is zero.

- The mathematical definition of potential energy is provided correctly using the gradient, which is an acceptable form.

- The explanation for why non-conservative forces cannot have a potential energy is clear and accurately relates to path dependence and energy dissipation.

Solution 2

**Marks: 3/4**

**Feedback:**

- The definition of a conservative force is accurate.

- The explanation of potential energy is mostly correct but lacks the explicit integral form and the clarification that the chosen reference point is arbitrary.

- The reason why non-conservative forces do not have potential energy is well-explained, covering path dependence and energy dissipation.

- Missing the explicit integral form of potential energy deduction.

Solution 3

**Marks: 3/4**

**Feedback:**

- The definition of a conservative force is clear and correct.

- The explanation of potential energy involves correct use of work and the gradient, but does not explicitly state the integral form, nor does it mention that the origin is arbitrary.

- The discussion on non-conservative forces is thorough and correctly highlights path dependence and the energy dissipation issue.

- The explanation is good, but the lack of explicit integral form and arbitrary origin costs one mark.

\bigskip
\textbf{Response 4:}

Solution 1

**Grade: 3/4**

**Feedback:**

- **Conservative Force (1/1)**: Clearly defined as a force whose work is path-independent.

- **Potential Energy (1/2)**: Correctly states the relationship between force and potential energy using the gradient, but does not explicitly provide the integral form of the potential energy definition or mention that the origin is arbitrary.

- **Non-Conservative Force (1/1)**: Correct explanation about path dependence leading to non-unique potential energy.

Solution 2

**Grade: 2/4**

**Feedback:**

- **Conservative Force (1/1)**: Clearly defined as path-independent work.

- **Potential Energy (0/2)**: Explanation is somewhat correct but lacks the integral form for the potential energy definition and does not mention that the origin is arbitrary. It also inaccurately presents the relationship between potential energy and force as the change in potential energy being the negative work done, without a clear definition of potential energy.

- **Non-Conservative Force (1/1)**: Correctly explains why potential energy cannot be defined for non-conservative forces due to path dependence.

Solution 3

**Grade: 4/4**

**Feedback:**

- **Conservative Force (1/1)**: Correctly defined as a force where work is path-independent.

- **Potential Energy (2/2)**: Properly defines the potential energy using the integral form and includes the necessary details about the differential relationship and the condition that the force must be conservative.

- **Non-Conservative Force (1/1)**: Clearly explains the path dependence issue and why a unique potential energy cannot be defined for non-conservative forces.

\bigskip
\textbf{Response 5:}

Solution 1:

**Marks: 3/4**

**Feedback:**

- **Conservative force definition (1 mark):** Full mark. The student accurately defined a conservative force and correctly noted that the work done in a closed loop is zero.

- **Potential energy definition (2 marks):** Partial credit. The student described the relationship \( \underline{F} = -\nabla U \) correctly but did not provide the integral definition or mention that the origin point is arbitrary.

- **Non-conservative force (1 mark):** Full mark. The student correctly explained why a potential energy cannot be defined for non-conservative forces due to path dependency.

Solution 2:

**Marks: 2/4**

**Feedback:**

- **Conservative force definition (1 mark):** Full mark. The student correctly defined a conservative force and provided good examples.

- **Potential energy definition (2 marks):** Partial credit. The student correctly noted \( \underline{F} = -\nabla PE \) but used \(\Delta PE = -W\) without specifying the integral definition or addressing the arbitrariness of the origin point.

- **Non-conservative force (1 mark):** Full mark. The explanation regarding the path dependency and inability to define potential energy for non-conservative forces was accurate.

Solution 3:

**Marks: 3/4**

**Feedback:**

- **Conservative force definition (1 mark):** Full mark. The student correctly defined a conservative force and provided good examples.

- **Potential energy definition (2 marks):** Partial credit. The student described the differential relationship and mentioned the integral, but they did not clearly state the integral definition from the origin point or explicitly address its arbitrariness.

- **Non-conservative force (1 mark):** Full mark. The explanation of path dependency and the impossibility of defining potential energy for non-conservative forces was thorough and accurate.

\subsection*{Question 2:}

\textbf{Response 1:}

Solution 1

**Marks: 6/6**

**Feedback:**

- The solution correctly identifies and describes the three cases of damping: underdamped, critically damped, and overdamped.

- The equations provided for each case are accurate, and the qualitative descriptions of the behavior in each case are clear and precise.

- The explanation of the physical meaning behind the terms and the conditions for each type of damping is thorough and accurate.

Solution 2

**Marks: 6/6**

**Feedback:**

- The solution correctly explains the three damping cases: underdamped, critically damped, and overdamped, using the appropriate conditions (\(\Delta\)).

- The descriptions of the motion in each case are accurate and well-articulated, capturing the essential characteristics of the oscillations and decay.

- The physical interpretation and contextual examples, such as automotive shock absorbers, enhance the explanation and demonstrate a strong understanding of the concept.

Solution 3

**Marks: 5/6**

**Feedback:**

- The solution correctly identifies and describes the three damping cases and the corresponding conditions.

- The description of the underdamped case could be more explicit in mentioning that the amplitude decays exponentially.

- The reference to considering \(m=1\) for simplicity is unnecessary and could confuse the context since the original problem does not specify any such simplification.

- Overall, the explanations are clear and mostly accurate, but there is a slight imprecision in the description of the underdamped case compared to the mark scheme.

\bigskip
\textbf{Response 2:}

Solution 1:

Marks: **6/6**

**Feedback:**

- **Content Accuracy:** The solution correctly identifies and describes the three damping cases.

- **Clarity and Completeness:** Each case is well-explained, with clear distinctions made between underdamped, critically damped, and overdamped systems. The description of the sinusoidal motion in the underdamped case and the fastest return to equilibrium in the critically damped case are accurate and concise.

- **Technical Precision:** Uses appropriate mathematical conditions (\(b^2 < 4mk\), \(b^2 = 4mk\), \(b^2 > 4mk\)) and provides a qualitative description that aligns well with the conditions given in the problem statement.

Solution 2:

Marks: **6/6**

**Feedback:**

- **Content Accuracy:** The solution correctly identifies and describes the three damping cases using the discriminant \(\Delta = b^2 - 4mk\).

- **Clarity and Completeness:** Each damping case is well-explained with a qualitative description of the motion. The use of terms like "threshold between oscillatory and non-oscillatory motion" for critical damping is precise.

- **Technical Precision:** Uses appropriate mathematical conditions and provides clear qualitative descriptions. The mention of practical applications (e.g., automotive shock absorbers) enhances the understanding of critical damping.

Solution 3:

Marks: **6/6**

**Feedback:**

- **Content Accuracy:** The solution correctly identifies and describes the three damping cases.

- **Clarity and Completeness:** Each case is described with a focus on the qualitative behavior of the system. The use of the term "combination of exponential decay and sinusoidal function" for the underdamped case is precise and helpful.

- **Technical Precision:** Uses appropriate mathematical conditions (\(b^2 < 4k\), \(b^2 = 4k\), \(b^2 > 4k\)) and gives clear qualitative descriptions. The explanation is concise and correctly emphasizes the different behaviors for each damping scenario.

All solutions received full marks as they accurately and comprehensively addressed the problem, using clear and correct descriptions of the three damping cases.

\bigskip
\textbf{Response 3:}

Solution 1

**Marks: 6/6**

**Feedback:**

- The explanation clearly identifies the three cases of damping: underdamping, critical damping, and overdamping.

- The conditions \(b^2 < 4mk\), \(b^2 = 4mk\), and \(b^2 > 4mk\) are correctly stated.

- The descriptions for each type of damping are accurate and thorough, detailing the nature of the oscillations and return to equilibrium.

- Appropriate terminology and precision are used throughout.

Solution 2

**Marks: 6/6**

**Feedback:**

- The solution effectively describes the three cases of damping using the discriminant \(\Delta = b^2 - 4mk\), aligning with the problem's requirements.

- The conditions \(\Delta < 0\), \(\Delta = 0\), and \(\Delta > 0\) are correctly linked to the physical behaviors of the oscillator.

- Descriptions of underdamping, critical damping, and overdamping are accurate and detailed.

- The solution integrates real-world applications (e.g., automotive shock absorbers) to enhance understanding.

Solution 3

**Marks: 5/6**

**Feedback:**

- The solution correctly identifies the three cases of damping: underdamping, critical damping, and overdamping.

- The conditions \(b^2 < 4k\), \(b^2 = 4k\), and \(b^2 > 4k\) are correct but simplified by assuming \(m=1\) without explicitly stating it.

- The descriptions for each damping case are mostly accurate but lack explicit mention of the mass (\(m\)) which is an essential parameter in the given equation.

- While the qualitative descriptions are correct, the explanation would be clearer if it consistently included the mass term \(m\) in the conditions for each case.

\bigskip
\textbf{Response 4:}

Solution 1:

**Marks: 6/6**

**Feedback:**

- The solution correctly identifies and describes all three cases of damping (underdamping, critical damping, and overdamping).

- The conditions \(b^2 < 4mk\), \(b^2 = 4mk\), and \(b^2 > 4mk\) are accurately stated and the qualitative behavior of the system in each case is well explained.

- Good use of terminology and clear, concise descriptions of each damping case.
  
Solution 2:

**Marks: 6/6**

**Feedback:**

- The solution accurately identifies and describes the three cases of damping with proper conditions (\(\Delta < 0\), \(\Delta = 0\), \(\Delta > 0\)).

- The qualitative descriptions of underdamping, critical damping, and overdamping are clear and correct.

- The explanation of the system's behavior in each damping scenario is well-articulated and includes relevant physical interpretations, such as applications in automotive shock absorbers.
  
Solution 3:

**Marks: 5/6**

**Feedback:**

- The solution correctly identifies the three cases of damping with conditions \(b^2 < 4k\), \(b^2 = 4k\), and \(b^2 > 4k\).

- The descriptions of the qualitative behavior for underdamped and critically damped cases are clear and accurate.

- The overdamped case is described correctly but lacks a bit of depth compared to the other two solutions.

- The simplification assuming \(m=1\) should have been explicitly clarified in terms of units and consistency. This minor lapse led to the deduction of one mark.

\bigskip
\textbf{Response 5:}

Solution 1

Marks: 6/6

**Feedback:**

- Correctly identifies all three cases of damping.

- Clearly explains the behavior in each case.

- Uses the appropriate mathematical condition for each damping case.

- The qualitative descriptions are accurate and sufficiently detailed.

Solution 2

Marks: 6/6

**Feedback:**

- Provides a clear explanation of the damped harmonic oscillator.

- Correctly states the conditions for underdamping, critical damping, and overdamping.

- Descriptions are thorough and include both qualitative behavior and the significance of each damping type.

- The explanation of the discriminant is accurate and well-integrated into the descriptions.

Solution 3

Marks: 5/6

**Feedback:**

- Correctly identifies the three cases of damping.

- Provides accurate and detailed descriptions of the behavior in each damping case.

- The simplification with \( m=1 \) is unnecessary and slightly confusing, detracting from the clarity.

- The rest of the explanation is clear and correct, but the unnecessary simplification leads to a minor deduction.

\subsection*{Question 3:}

\textbf{Response 1:}

Solution 1

**Marks: 6/6**

**Part a: 2/2**

- Correctly uses the definition of the center of mass.

- Properly manipulates the expression to find the value of \( k \).

- Clear and accurate derivation.

**Part b: 4/4**

- Correctly calculates \( \underline{R} - \underline{r}_1 \) and \( \underline{R} - \underline{r}_2 \).

- Accurately determines the distances from the center of mass to each mass.

- All steps are shown clearly and correctly.

**Feedback:**

Excellent work. You clearly used the definition of the center of mass and manipulated the expression correctly. Your calculations and explanations for both parts are precise and thorough.

Solution 2

**Marks: 6/6**

**Part a: 2/2**

- Correctly uses the definition of the center of mass.

- Properly manipulates the expression to find the value of \( k \).

- Clear and accurate derivation.

**Part b: 4/4**

- Correctly calculates \( \underline{R} - \underline{r}_1 \) and \( \underline{R} - \underline{r}_2 \).

- Accurately determines the distances from the center of mass to each mass.

- All steps are shown clearly and correctly.

**Feedback:**

Well done. Your derivation and calculations are accurate and clearly presented. You correctly showed the distances from the center of mass to each mass.

Solution 3

**Marks: 5/6**

**Part a: 2/2**

- Correctly uses the definition of the center of mass.

- Properly manipulates the expression to find the value of \( k \).

- Clear and accurate derivation.

**Part b: 3/4**

- Correctly calculates the distances from the center of mass to each mass.

- The explanation is clear, but the vector calculations for \( \underline{R} - \underline{r}_1 \) and \( \underline{R} - \underline{r}_2 \) are not explicitly shown.

**Feedback:**

Good job. You correctly derived the value of \( k \) and found the distances from the center of mass to each mass. However, your solution would be stronger if you explicitly showed the intermediate vector calculations for \( \underline{R} - \underline{r}_1 \) and \( \underline{R} - \underline{r}_2 \).

\bigskip
\textbf{Response 2:}

**Solution 1:**

*Part a)*

- Correctly used the definition of the center of mass: 1 mark.

- Correctly manipulated to find the value of \( k \): 1 mark.

*Part b)*

- Calculated \(\underline{R} - \underline{r}_1\) using the results of part a): 1 mark.

- Correctly found the distance from \(\underline{r}_1\) to \(\underline{R}\): 1 mark.

- Calculated \(\underline{R} - \underline{r}_2\) using the results of part a): 1 mark.

- Correctly found the distance from \(\underline{r}_2\) to \(\underline{R}\): 1 mark.

**Total: 6/6**

*Feedback:* Excellent work! The solution is clear, correct, and complete. You have demonstrated a thorough understanding of the problem and provided the required manipulations and derivations accurately.

---

**Solution 2:**

*Part a)*

- Correctly used the definition of the center of mass: 1 mark.

- Correctly manipulated to find the value of \( k \): 1 mark.

*Part b)*

- Calculated \(\underline{R} - \underline{r}_1\) using the results of part a): 1 mark.

- Correctly found the distance from \(\underline{r}_1\) to \(\underline{R}\): 1 mark.

- Calculated \(\underline{R} - \underline{r}_2\) using the results of part a): 1 mark.

- Correctly found the distance from \(\underline{r}_2\) to \(\underline{R}\): 1 mark.

**Total: 6/6**

*Feedback:* Well done! The solution is concise and accurate. You have effectively used the provided formulas and correctly derived the necessary distances.

---

**Solution 3:**

*Part a)*

- Correctly used the definition of the center of mass: 1 mark.

- Correctly manipulated to find the value of \( k \): 1 mark.

*Part b)*

- Calculated \(\underline{R} - \underline{r}_1\) using the results of part a): 1 mark.

- Correctly found the distance from \(\underline{r}_1\) to \(\underline{R}\): 1 mark.

- Calculated \(\underline{R} - \underline{r}_2\) using the results of part a): 1 mark.

- Correctly found the distance from \(\underline{r}_2\) to \(\underline{R}\): 1 mark.

**Total: 6/6**

*Feedback:* Excellent job! The solution is precise and complete. You have shown a solid understanding of the problem and derived all necessary components accurately.

\bigskip
\textbf{Response 3:}

Solution 1 Evaluation

**Part a:**

- Correctly uses the definition of the center of mass.

- Manipulates the expression to the required form and finds \( k \).

**Mark: 2/2**

**Part b:**

- Calculates the vector \( \underline{R} - \underline{r}_1 \) and correctly shows the distance from \( \underline{r}_1 \) to \( \underline{R} \).

- Does not explicitly calculate the vector \( \underline{R} - \underline{r}_2 \) but infers the distance correctly using symmetry.

**Mark: 3/4**

**General Feedback:**

Good job on part a. For part b, while you correctly derived the distances, explicitly calculating the vector \( \underline{R} - \underline{r}_2 \) would have made the solution more complete.

**Total: 5/6**

---

Solution 2 Evaluation

**Part a:**

- Correctly uses the definition of the center of mass.

- Manipulates the expression to the required form and finds \( k \).

**Mark: 2/2**

**Part b:**

- Correctly calculates the distances from the center of mass to both \( \underline{r}_1 \) and \( \underline{r}_2 \).

- Does not explicitly show the vectors \( \underline{R} - \underline{r}_1 \) and \( \underline{R} - \underline{r}_2 \).

**Mark: 3/4**

**General Feedback:**

You correctly derived the distances in part b, but it would be better to show the intermediate vector calculations for clarity.

**Total: 5/6**

---

Solution 3 Evaluation

**Part a:**

- Correctly uses the definition of the center of mass.

- Manipulates the expression to the required form and finds \( k \).

**Mark: 2/2**

**Part b:**

- Correctly calculates the distances from the center of mass to both \( \underline{r}_1 \) and \( \underline{r}_2 \).

- Does not explicitly show the vectors \( \underline{R} - \underline{r}_1 \) and \( \underline{R} - \underline{r}_2 \).

**Mark: 3/4**

**General Feedback:**

Good job on both parts. For part b, ensure to include the vector calculations for completeness.

**Total: 5/6**

\bigskip
\textbf{Response 4:}

**Solution 1**

- **Part a)**

1. Uses the definition of center of mass: 1 mark

2. Correct manipulation to the given form: 1 mark

Score: 2/2

- **Part b)**

1. Correct calculation of \(\underline{R} - \underline{r}_1\): 1 mark

2. Correct distance from \(\underline{r}_1\) to center of mass: 1 mark

3. Correct calculation of \(\underline{R} - \underline{r}_2\): 1 mark

4. Correct distance from \(\underline{r}_2\) to center of mass: 1 mark

Score: 4/4

Total Score: 6/6

Feedback: Excellent solution. All steps are clearly and correctly shown.

**Solution 2**

- **Part a)**

1. Uses the definition of center of mass: 1 mark

2. Correct manipulation to the given form: 1 mark

Score: 2/2

- **Part b)**

1. Correct calculation of \(\underline{R} - \underline{r}_1\): 1 mark

2. Correct distance from \(\underline{r}_1\) to center of mass: 1 mark

3. Correct calculation of \(\underline{R} - \underline{r}_2\): 1 mark

4. Correct distance from \(\underline{r}_2\) to center of mass: 1 mark

Score: 4/4

Total Score: 6/6

Feedback: Well-done. Your solution is thorough and accurate, addressing all parts of the problem correctly.

**Solution 3**

- **Part a)**

1. Uses the definition of center of mass: 1 mark

2. Correct manipulation to the given form: 1 mark

Score: 2/2

- **Part b)**

1. Correct calculation of \(\underline{R} - \underline{r}_1\): 1 mark

2. Correct distance from \(\underline{r}_1\) to center of mass: 1 mark

3. Correct calculation of \(\underline{R} - \underline{r}_2\): 1 mark

4. Correct distance from \(\underline{r}_2\) to center of mass: 1 mark

Score: 4/4

Total Score: 6/6

Feedback: Comprehensive and precise solution. Every step is clearly explained and correctly executed.

\bigskip
\textbf{Response 5:}

Solution 1

**Part a)**

- Correctly uses the definition of center of mass and manipulates the expression to find \( k \).

- Clearly shows \( k = \frac{m_2}{m_1 + m_2} \).

**Marks: 2/2**

**Part b)**

- Correctly calculates \( \underline{R} - \underline{r}_1 \) and shows the distance \( d_{r1} \) from \( \underline{r}_1 \).

- Correctly calculates \( \underline{R} - \underline{r}_2 \) and shows the distance \( d_{r2} \) from \( \underline{r}_2 \).

**Marks: 4/4**

**Total Marks: 6/6**

**Feedback:**

Excellent work. You clearly explained each step and derived the required results accurately. All calculations and reasoning are correct.

Solution 2

**Part a)**

- Uses the definition of center of mass correctly.

- Clearly derives \( k = \frac{m_2}{m_1 + m_2} \).

**Marks: 2/2**

**Part b)**

- Correctly explains that the center of mass lies on the line connecting the two masses.

- Correctly calculates the distances \( d_{r1} \) and \( d_{r2} \).

**Marks: 4/4**

**Total Marks: 6/6**

**Feedback:**

Well done. Your solution is clear and thorough. All necessary steps and calculations are present and correct.

Solution 3

**Part a)**

- Correctly uses the definition of center of mass.

- Shows \( k = \frac{m_2}{m_1 + m_2} \) clearly.

**Marks: 2/2**

**Part b)**

- Correctly explains the positioning of the center of mass.

- Correctly calculates the distances \( d_{r1} \) and \( d_{r2} \).

**Marks: 4/4**

**Total Marks: 6/6**

**Feedback:**

Excellent job. Your solution is detailed and correct, with clear explanations and accurate calculations.

\subsection*{Question 4:}

\textbf{Response 1:}

Solution 1:

Part (a):

- **Marks:** 6/6

- **Feedback:** The derivation for the angular frequency \(\dot{\theta}\) is well done. The steps are clear and correctly apply the conservation of angular momentum to find the final expression.

Part (b):

- **Marks:** 4/5

- **Feedback:** The solution correctly identifies the angular component of the acceleration and attempts to calculate \(\ddot{\theta}\). The explanation is clear but does not explicitly finalize the simplification to show that it equals 0, which is a minor oversight.

Part (c):

- **Marks:** 6/7

- **Feedback:** The kinetic energy expression is derived accurately. The differentiation is correct, but there's an additional step where the simplification process could be clearer to match the provided expression perfectly.

**Total:** 16/18

---

Solution 2:

Part (a):

- **Marks:** 6/6

- **Feedback:** The student correctly applies the conservation of angular momentum and integrates to find \(r\). The final expression for \(\dot{\theta}\) is correct.

Part (b):

- **Marks:** 5/5

- **Feedback:** The angular component of the acceleration is calculated accurately. The solution clearly shows the differentiation process and the substitution, leading to the correct conclusion that it simplifies to 0.

Part (c):

- **Marks:** 6/7

- **Feedback:** The kinetic energy derivation is well-executed, and the time derivative is correctly found. The explanation could be slightly more detailed in showing the final steps leading to the provided expression.

**Total:** 17/18

---

Solution 3:

Part (a):

- **Marks:** 5/6

- **Feedback:** The conservation of angular momentum is used correctly, but the explanation is somewhat verbose and slightly confusing in places. The final expression is correct.

Part (b):

- **Marks:** 4/5

- **Feedback:** The angular component of the acceleration is mostly correct, but there is an unnecessary complication in the explanation which detracts from clarity. The final result should be more explicitly stated.

Part (c):

- **Marks:** 5/7

- **Feedback:** The kinetic energy derivation is mostly correct, but the explanation contains some unnecessary detail and confusion. The final differentiation step could be clearer and more concise to better match the provided expression.

**Total:** 14/18

\bigskip
\textbf{Response 2:}

Here are the detailed evaluations for each solution provided by the students, along with general feedback based on the given mark scheme.

**Solution 1:**

- **Part (a):** (6/6)

  - Correctly identifies that angular momentum \(L\) is conserved.
  
  - Derives the expression \(r = r_0 - \gamma t\) correctly.
  
  - Uses conservation of angular momentum to derive \(\dot{\theta} = \frac{r_0^2 \delta}{(r_0 - \gamma t)^2}\) correctly.

- **Part (b):** (4/5)

  - Correctly identifies the angular component of acceleration \(a_{\theta}\).
  
  - Derives \(\ddot{\theta}\) correctly.
  
  - Substitutes \(\ddot{\theta}\), \(\dot{r}\), and \(\dot{\theta}\) to find \(a_{\theta}\), but does not explicitly simplify to show it equals zero.

- **Part (c):** (7/7)

  - Correctly identifies the kinetic energy expression.
  
  - Substitutes correctly to derive the kinetic energy.
  
  - Differentiates to obtain \(\frac{dK}{dt}\) correctly matching the given expression.

**Total: 17/18**

**General Feedback:** Excellent work overall. The derivations are thorough and well-explained. Ensure to complete simplifications in future solutions for full clarity.

**Solution 2:**

- **Part (a):** (6/6)

  - Correctly identifies conservation of angular momentum.
  
  - Correctly derives \(r = r_0 - \gamma t\).
  
  - Uses these to find \(\dot{\theta} = \frac{r_0^2 \delta}{(r_0 - \gamma t)^2}\) accurately.

- **Part (b):** (5/5)

  - Correctly finds the angular component of acceleration \(a_{\theta}\).
  
  - Derives \(\ddot{\theta}\) correctly and substitutes it to find \(a_{\theta}\) explicitly simplifying to zero.

- **Part (c):** (6/7)

  - Correctly identifies the kinetic energy expression.
  
  - Substitutes to derive the kinetic energy.
  
  - Differentiates correctly, but misses sign consistency check which would align with the problem statement.

**Total: 17/18**

**General Feedback:** Very well done. Accurate and thorough derivations. Pay attention to the physical interpretation of signs in differentiation results.

**Solution 3:**

- **Part (a):** (6/6)

  - Correctly identifies conservation of angular momentum.
  
  - Correctly derives \(r = r_0 - \gamma t\).
  
  - Uses these to find \(\dot{\theta} = \frac{r_0^2 \delta}{(r_0 - \gamma t)^2}\) accurately.

- **Part (b):** (3/5)

  - Correctly identifies the components needed for \(a_{\theta}\).
  
  - Misunderstands and incorrectly simplifies \(a_{\theta}\) to zero without proper final step simplification or correct substitution verification.

- **Part (c):** (7/7)

  - Correctly identifies the kinetic energy expression.
  
  - Substitutes correctly to derive the kinetic energy.
  
  - Differentiates to obtain \(\frac{dK}{dt}\) correctly matching the given expression.

**Total: 16/18**

**General Feedback:** Good work. Correct and clear derivations in parts (a) and (c). Ensure proper verification and simplification steps to avoid errors, especially in parts involving multiple substitutions.

In conclusion, all solutions were well done, demonstrating a strong understanding of the physics principles involved. Each student effectively used the conservation of angular momentum and correctly derived the necessary expressions. Some minor issues in simplification and sign consistency should be carefully checked in future work to ensure full marks.

\bigskip
\textbf{Response 3:}

Here is the grading and feedback for each solution based on the provided mark scheme.

Solution 1

**Part (a):**

Marks: 6/6

Feedback: Excellent derivation of angular frequency \(\dot{\theta}\). The steps to express and simplify angular momentum using initial conditions were clearly and correctly executed.

**Part (b):**

Marks: 3/5

Feedback: The calculation of the angular component of acceleration is mostly correct. However, the expression for \(\ddot{\theta}\) was not clearly derived and directly used the provided expression without full justification.

**Part (c):**

Marks: 7/7

Feedback: The kinetic energy expression and its derivative were correctly derived and matched the given expression. Great job in showing all necessary steps.

**Total Marks: 16/18**

Solution 2

**Part (a):**

Marks: 6/6

Feedback: Excellent work in deriving the expression for \(\dot{\theta}\). The use of angular momentum conservation and initial conditions was well-handled.

**Part (b):**

Marks: 5/5

Feedback: Correct derivation of the angular component of the acceleration. The necessary differentiation was correctly performed and simplified.

**Part (c):**

Marks: 7/7

Feedback: The kinetic energy expression and its time derivative were correctly calculated, and the provided expression was matched. Good work in detailing the steps.

**Total Marks: 18/18**

Solution 3

**Part (a):**

Marks: 6/6

Feedback: Clear and correct derivation of \(\dot{\theta}\) using angular momentum conservation and initial conditions.

**Part (b):**

Marks: 1/5

Feedback: The angular component of the acceleration was not correctly calculated. The calculation oversimplified and did not properly derive \(\ddot{\theta}\).

**Part (c):**

Marks: 5/7

Feedback: The kinetic energy expression was correctly derived, but the derivative had a sign discrepancy. Although the magnitude and dependencies were correct, the sign needs to be revisited to ensure proper interpretation of energy transfer.

**Total Marks: 12/18**

---

Summary

- **Solution 1**: 16/18

- **Solution 2**: 18/18

- **Solution 3**: 12/18

Solution 2 was the most accurate and thorough, achieving full marks. Solution 1 was very strong but missed a couple of marks in part (b). Solution 3 had issues with part (b) and a sign discrepancy in part (c), leading to a lower score.

\bigskip
\textbf{Response 4:}

Solution 1: 

Part (a)

- Correctly identified that angular momentum is conserved and set up the relationship \( m r_0^2 \delta = m r^2 \dot{\theta} \) using the initial and arbitrary conditions.

- Properly substituted \( r(t) = r_0 - \gamma t \) and derived the correct expression for \( \dot{\theta} \).

Marks: 6/6

Part (b)

- Correctly started with the formula for the angular component of the acceleration.

- Properly found \( \ddot{\theta} \) by differentiating \( \dot{\theta} \) and correctly substituted values.

- Simplified the expression to show that the angular component of the acceleration is zero.

Marks: 5/5

Part (c)

- Correctly derived the expression for kinetic energy \( K = \frac{1}{2}m (\gamma^2 + \frac{r_0^4 \delta^2}{(r_0 - \gamma t)^2}) \).

- Correctly differentiated \( K \) to obtain \( \frac{dK}{dt} = \frac{m r_0^4 \gamma \delta^2}{(r_0 - \gamma t)^3} \).

Marks: 7/7

Total: 18/18

Feedback: Excellent solution. All parts are derived correctly and the final expressions match the expected results. Well done!

Solution 2:

Part (a)

- Correctly identified that angular momentum is conserved and set up the relationship \( m r_0^2 \delta = m r^2 \dot{\theta} \).

- Properly substituted \( r(t) = r_0 - \gamma t \) and derived the correct expression for \( \dot{\theta} \).

Marks: 6/6

Part (b)

- Correctly identified the formula for the angular component of the acceleration.

- Properly found \( \ddot{\theta} \) by differentiating \( \dot{\theta} \) and correctly substituted values.

- Simplified the expression to show that the angular component of the acceleration is zero.

Marks: 5/5

Part (c)

- Correctly derived the expression for kinetic energy \( K = \frac{1}{2}m (\gamma^2 + \frac{r_0^4 \delta^2}{(r_0 - \gamma t)^2}) \).

- Correctly differentiated \( K \) to obtain \( \frac{dK}{dt} = \frac{m r_0^4 \gamma \delta^2}{(r_0 - \gamma t)^3} \).

Marks: 7/7

Total: 18/18

Feedback: Excellent solution. All parts are derived correctly and the final expressions match the expected results. Well done!

Solution 3:

Part (a)

- Correctly identified that angular momentum is conserved and set up the relationship \( m r_0^2 \delta = m r^2 \dot{\theta} \).

- Properly substituted \( r(t) = r_0 - \gamma t \) and derived the correct expression for \( \dot{\theta} \).

Marks: 6/6

Part (b)

- Correctly identified the formula for the angular component of the acceleration.

- Properly found \( \ddot{\theta} \) by differentiating \( \dot{\theta} \) and correctly substituted values.

- Simplified the expression to show that the angular component of the acceleration is zero.

Marks: 5/5

Part (c)

- Correctly derived the expression for kinetic energy \( K = \frac{1}{2}m (\gamma^2 + \frac{r_0^4 \delta^2}{(r_0 - \gamma t)^2}) \).

- Correctly differentiated \( K \) to obtain \( \frac{dK}{dt} = \frac{m r_0^4 \gamma \delta^2}{(r_0 - \gamma t)^3} \).

Marks: 7/7

Total: 18/18

Feedback: Excellent solution. All parts are derived correctly and the final expressions match the expected results. Well done!

---

All three solutions are correct and each student has demonstrated a thorough understanding of the concepts and methods required to solve the problem. They correctly applied the conservation of angular momentum, derived the necessary expressions for velocity and acceleration components, and correctly calculated the kinetic energy and its time derivative. Each solution deserves full marks.

\bigskip
\textbf{Response 5:}

Solution 1 Feedback

Part (a)

- Correctly derived the expression for \(\dot{\theta}\) using conservation of angular momentum.

- Good use of initial conditions and integration of \(\dot{r}\).

**Score: 6/6**

Part (b)

- Identified the angular component of the acceleration.

- Incorrect simplification of the angular acceleration component. The step involving \(\ddot{\theta}\) was not executed correctly.

- Misinterpreted the problem by assuming \(\ddot{\theta}\) isn't necessary, leading to incorrect final result.

**Score: 2/5**

Part (c)

- Correctly expressed kinetic energy in polar coordinates.

- Correctly substituted expressions and derived the expression for kinetic energy.

- Differentiated correctly to match the given \(\frac{dK}{dt}\).

**Score: 7/7**

**Total Score: 15/18**

Solution 2 Feedback

Part (a)

- Correct derivation using angular momentum conservation.

- Correctly used initial conditions to find the expression for \(\dot{\theta}\).

**Score: 6/6**

Part (b)

- Correct identification of the angular component of the acceleration.

- Correct differentiation to find \(\ddot{\theta}\).

- Correctly simplified the angular acceleration component to zero.

**Score: 5/5**

Part (c)

- Correctly formulated the kinetic energy in polar coordinates.

- Correctly substituted and derived the expression for kinetic energy.

- Correctly differentiated to match the given \(\frac{dK}{dt}\).

**Score: 7/7**

**Total Score: 18/18**

Solution 3 Feedback

Part (a)

- Correct derivation using conservation of angular momentum.

- Correct use of initial conditions.

**Score: 6/6**

Part (b)

- Correct identification of the angular component of the acceleration.

- Incorrect differentiation for \(\ddot{\theta}\), leading to an incorrect simplification.

- Misinterpreted angular acceleration component.

**Score: 2/5**

Part (c)

- Correctly expressed kinetic energy in polar coordinates.

- Correct substitution and derivation of the expression for kinetic energy.

- Correct differentiation to match the given \(\frac{dK}{dt}\).

**Score: 7/7**

**Total Score: 15/18**

\subsection*{Question 5:}

\textbf{Response 1:}

Let's evaluate the three provided solutions against the given mark scheme.

Solution 1:

Part a)

- Correct application of Newton's second law and Hooke's Law.

- Equations of motion are derived correctly.

- Correctly identifies and manipulates forces on both objects.

**Marks: 2/2**

Part b)

- Correctly defines new coordinates \( y_1 \) and \( y_2 \).

- Correctly uncouples the equations and shows simple harmonic motion.

- Identifies the angular frequencies correctly.

**Marks: 4/4**

Part c)

- Provides the general solution for \( y_1 \).

- Correctly uses initial conditions to find \( A \) and \( B \).

- Arrives at the correct solution for \( y_1 \).

**Marks: 3/3**

**Total Marks: 9/9**

**Feedback:**

Excellent work! The solution is well-structured and correctly applies the principles of mechanics to derive the required results.

Solution 2:

Part a)

- Correct application of Newton's second law and Hooke's Law.

- Equations of motion are derived correctly.

- Forces and motions are clearly explained.

**Marks: 2/2**

Part b)

- Correctly defines new coordinates \( y_1 \) and \( y_2 \).

- Correctly uncouples the equations and shows simple harmonic motion.

- Identifies the angular frequencies correctly.

**Marks: 4/4**

Part c)

- Provides the general solution for \( y_1 \).

- Correctly uses initial conditions to find \( A \) and \( B \).

- Arrives at the correct solution for \( y_1 \).

**Marks: 3/3**

**Total Marks: 9/9**

**Feedback:**

Great job! The solution is thorough and accurate, demonstrating a solid understanding of the problem.

Solution 3:

Part a)

- Correct application of Newton's second law and Hooke's Law.

- Equations of motion are derived correctly.

- Forces and motions are explained, but the final forms could be simplified earlier.

**Marks: 2/2**

Part b)

- Correctly defines new coordinates \( y_1 \) and \( y_2 \).

- Correctly uncouples the equations and shows simple harmonic motion.

- Identifies the angular frequencies correctly.

**Marks: 4/4**

Part c)

- Provides the general solution for \( y_1 \).

- Correctly uses initial conditions to find \( A \) and \( B \).

- Arrives at the correct solution for \( y_1 \).

**Marks: 3/3**

**Total Marks: 9/9**

**Feedback:**

Well done! The solution is complete and clearly demonstrates the necessary steps to solve the problem.

---

All three solutions accurately follow the provided steps and correctly derive the necessary results. Each solution demonstrates a clear understanding of the mechanics involved and correctly applies the principles to find the equations of motion, uncouple them, and solve for \( y_1 \). All three solutions deserve full marks based on the criteria provided.

\bigskip
\textbf{Response 2:}

I have reviewed the provided solutions. Below are the grades and feedback for each solution according to the given mark scheme.

Solution 1

**a) Equations of Motion**:

- Correct derivation and application of Newton’s 2nd Law and Hooke’s Law.

- Final equations are correct.

Marks: 2/2

**b) Uncoupling the Coordinates**:

- Correctly defines new coordinates \( y_1 \) and \( y_2 \).

- Shows the process of adding and subtracting the equations.

- Correctly identifies the resulting uncoupled equations and shows they represent SHM.

Marks: 4/4

**c) Solution for \( y_1 \)**:

- Correct general solution for SHM.

- Correct use of initial conditions.

- Arrives at the correct final expression for \( y_1 \).

Marks: 3/3

**Total Marks: 9/9**

**Feedback**: Excellent solution. All parts are correctly solved with clear and accurate steps.

Solution 2

**a) Equations of Motion**:

- Correct application of Newton’s 2nd Law and Hooke’s Law.

- Correct final equations.

Marks: 2/2

**b) Uncoupling the Coordinates**:

- Defines new coordinates \( y_1 \) and \( y_2 \).

- Correct process of adding and subtracting equations.

- Correctly identifies resulting uncoupled equations and shows they represent SHM.

Marks: 4/4

**c) Solution for \( y_1 \)**:

- Correct general solution for SHM.

- Correct use of initial conditions.

- Arrives at the correct final expression for \( y_1 \).

Marks: 3/3

**Total Marks: 9/9**

**Feedback**: Very well done. The steps are clearly presented and correctly solved.

Solution 3

**a) Equations of Motion**:

- Correct derivation and application of Newton’s 2nd Law and Hooke’s Law.

- Correct final equations.

Marks: 2/2

**b) Uncoupling the Coordinates**:

- Defines new coordinates \( y_1 \) and \( y_2 \).

- Correct process of adding and subtracting equations.

- Correctly identifies resulting uncoupled equations and shows they represent SHM.

Marks: 4/4

**c) Solution for \( y_1 \)**:

- Correct general solution for SHM.

- Correct use of initial conditions.

- Arrives at the correct final expression for \( y_1 \).

Marks: 3/3

**Total Marks: 9/9**

**Feedback**: Excellent solution. The derivation and explanation are clear and correct.

Summary

All three solutions provided by the students are perfect, scoring full marks in each part. Each solution demonstrates a clear understanding of the problem, correct application of physical laws, and accurate mathematical derivation. Great job by all!

\bigskip
\textbf{Response 3:}

Let's evaluate each solution based on the provided mark scheme.

**Solution 1:**

**Part a:**

- Correctly applies Newton's second law and Hooke's law.

- Derives the equations of motion accurately.
  
Score: 2/2

**Part b:**

- Correctly adds and subtracts the equations to uncouple them.

- Correctly identifies \(y_1\) and \(y_2\) and provides their uncoupled equations.

- Mentions that both equations represent simple harmonic motion.
  
Score: 4/4

**Part c:**

- Provides the general solution for \(y_1\).

- Correctly uses the initial conditions to find the specific solution.
  
Score: 3/3

**Total: 9/9**

Feedback: Excellent solution. The derivations and explanations are clear, and all parts are answered correctly.

**Solution 2:**

**Part a:**

- Correctly applies Newton's second law and Hooke's law.

- Derives the equations of motion accurately.
  
Score: 2/2

**Part b:**

- Correctly adds and subtracts the equations to uncouple them.

- Correctly identifies \(y_1\) and \(y_2\) and provides their uncoupled equations.

- Mentions that both equations represent simple harmonic motion.
  
Score: 4/4

**Part c:**

- Provides the general solution for \(y_1\).

- Correctly uses the initial conditions to find the specific solution.
  
Score: 3/3

**Total: 9/9**

Feedback: Excellent solution. The derivations and explanations are clear, and all parts are answered correctly.

**Solution 3:**

**Part a:**

- Correctly applies Newton's second law and Hooke's law.

- Derives the equations of motion accurately.
  
Score: 2/2

**Part b:**

- Correctly adds and subtracts the equations to uncouple them.

- Correctly identifies \(y_1\) and \(y_2\) and provides their uncoupled equations.

- Mentions that both equations represent simple harmonic motion.
  
Score: 4/4

**Part c:**

- Provides the general solution for \(y_1\).

- Correctly uses the initial conditions to find the specific solution.
  
Score: 3/3

**Total: 9/9**

Feedback: Excellent solution. The derivations and explanations are clear, and all parts are answered correctly.

In summary, all three solutions are excellent, scoring a full 9 out of 9 marks. Each solution correctly applies Newton's second law and Hooke's law, uncouples the equations accurately, and uses the initial conditions to find the specific solution for \(y_1\).

\bigskip
\textbf{Response 4:}

Let's grade each of the solutions according to the mark scheme provided.

Solution 1

Part a) Equations of Motion

1. Correctly uses Newton's second law and Hooke's law.

2. Derives the correct equations of motion for both X and Y.

Marks: 2/2

Part b) Uncoupling the Coordinates

1. Correctly defines new coordinates \(y_1\) and \(y_2\).

2. Correctly derives the equations for \(y_1\) and \(y_2\).

3. Correctly shows these represent simple harmonic motion.

Marks: 4/4

Part c) Solution for \(y_1\)

1. Provides the general solution.

2. Correctly applies initial conditions to find specific solution.

Marks: 3/3

Total: 9/9

**Feedback:** Excellent work. The solution is detailed and correct, demonstrating a thorough understanding of the problem and the underlying physics.

Solution 2

Part a) Equations of Motion

1. Uses Newton's second law and Hooke's law correctly.

2. Derives the correct equations of motion for both X and Y.

Marks: 2/2

Part b) Uncoupling the Coordinates

1. Defines new coordinates \(y_1\) and \(y_2\) but makes a mistake in the derivation.

2. Incorrect equations for \(y_1\) and \(y_2\).

Marks: 2/4

Part c) Solution for \(y_1\)

1. Provides the general solution.

2. Correctly applies initial conditions to find specific solution.

Marks: 3/3

Total: 7/9

**Feedback:** Good attempt. There is an error in part b) regarding the uncoupling process, but parts a) and c) are handled well.

Solution 3

Part a) Equations of Motion

1. Uses Newton's second law and Hooke's law correctly.

2. Derives the correct equations of motion for both X and Y.

Marks: 2/2

Part b) Uncoupling the Coordinates

1. Correctly defines new coordinates \(y_1\) and \(y_2\).

2. Derives the correct equations for \(y_1\) and \(y_2\).

3. Correctly shows these represent simple harmonic motion.

Marks: 4/4

Part c) Solution for \(y_1\)

1. Provides the general solution.

2. Correctly applies initial conditions to find specific solution.

Marks: 3/3

Total: 9/9

**Feedback:** Excellent solution. Clearly demonstrates understanding and correctly solves each part of the problem.

Overall, Solutions 1 and 3 are perfect and demonstrate a clear understanding of the physics involved. Solution 2 is good but has a mistake in part b), which affects the overall score.

\bigskip
\textbf{Response 5:}

**Solution 1:**

a) Equations of Motion:

- Correct use of Newton's second law and Hooke's law.

- Derived equations are slightly incorrect but almost correct.

- Mark: 1/2

b) Uncoupling the Coordinates:

- Correct new coordinates and partially correct derivation of new equations.

- Equations mostly correct but lacks mention of SHM explicitly.

- Mark: 2/4

c) Solution for \( y_1 \):

- Correct general solution and application of initial conditions.

- Correct final solution.

- Mark: 3/3

**Total: 6/9**

**Feedback:**

- Good understanding of the concepts, but some minor mistakes in part a.

- Part b could be clearer in demonstrating SHM.

- Part c is well done.

---

**Solution 2:**

a) Equations of Motion:

- Correct equations and use of Newton's second law and Hooke's law.

- Mark: 2/2

b) Uncoupling the Coordinates:

- Correct new coordinates and correct derivation of new equations.

- Correctly identifies SHM.

- Mark: 4/4

c) Solution for \( y_1 \):

- Correct general solution and application of initial conditions.

- Correct final solution.

- Mark: 3/3

**Total: 9/9**

**Feedback:**

- Excellent solution. All parts are correctly solved and well-presented.

- Clear understanding and correct application of initial conditions.

---

**Solution 3:**

a) Equations of Motion:

- Correct use of Newton's second law and Hooke's law.

- Derived equations are slightly incorrect but almost correct.

- Mark: 1/2

b) Uncoupling the Coordinates:

- Correct new coordinates and correct derivation of new equations.

- Correctly identifies SHM.

- Mark: 4/4

c) Solution for \( y_1 \):

- Correct general solution and application of initial conditions.

- Correct final solution.

- Mark: 3/3

**Total: 8/9**

**Feedback:**

- Good solution with a few minor mistakes in part a.

- Part b and c are well done and show a clear understanding of SHM and initial conditions.

---

In summary, Solution 2 is the best with a perfect score, followed by Solution 3, and then Solution 1.



\subsection*{Question 6:}

\textbf{Response 1:}

Solution 1

Part a)

- Correctly identifies the forces involved and derives the correct expression for the largest speed.

- Marks: 2/2

Part b)

- Correctly identifies the forces involved, but the derived expression \( v = \sqrt{gR\sin(\phi)} \) is incorrect; it should be \( v = \sqrt{Rg\tan(\phi)} \).

- Marks: 2/3

Part c)

- Starts correctly but miscalculates and does not derive the correct final expression for \( v_{max} \) and \( v_{min} \). The attempt at algebraic simplification is flawed.

- Marks: 4/11

**Total Marks: 8/16**

**Feedback:** Good start and correct approach in part a. In part b, there's a small error in deriving the correct expression for the largest speed. Part c shows a strong initial understanding but lacks proper execution in algebraic simplification and final expression derivation.

Solution 2

Part a)

- Correctly identifies the forces involved and derives the correct expression for the largest speed.

- Marks: 2/2

Part b)

- Correctly identifies the forces involved, but the derived expression \( v = \sqrt{Rg\cos(\phi)} \) is incorrect; it should be \( v = \sqrt{Rg\tan(\phi)} \).

- Marks: 2/3

Part c)

- Incorrect derivation for \( v_{max} \) and \( v_{min} \). Simplifications do not match the required final expression.

- Marks: 4/11

**Total Marks: 8/16**

**Feedback:** Part a is well done. Part b contains a similar mistake to Solution 1, where the wrong trigonometric function is used. In part c, the approach is not executed correctly, leading to incorrect final expressions.

Solution 3

Part a)

- Correctly identifies the forces involved and derives the correct expression for the largest speed.

- Marks: 2/2

Part b)

- Correctly identifies the forces involved but uses incorrect trigonometric function leading to \( v = \sqrt{g R \sin(\phi)} \) instead of \( v = \sqrt{Rg\tan(\phi)} \).

- Marks: 2/3

Part c)

- Incorrect derivation for \( v_{max} \) and \( v_{min} \). Simplifications do not match the required final expression.

- Marks: 4/11

**Total Marks: 8/16**

**Feedback:** The approach in part a is correct. In part b, there is a trigonometric error. Part c fails to correctly derive and simplify the final expressions, leading to incorrect results.

Summary

All three solutions display a solid understanding of the initial parts of the problem, but they struggle with accurately deriving and simplifying the final expressions in part c. There is a common error in part b where the wrong trigonometric function is used, leading to incorrect expressions. Each solution shows room for improvement in algebraic manipulation and a more careful approach to force decomposition and vector analysis.

\bigskip
\textbf{Response 2:}

General Feedback:

For each solution, I will provide marks based on the following criteria:

- Part (a): Identification of forces and correct derivation of the largest speed.

- Part (b): Proper handling of the inclined plane, balancing forces correctly, and obtaining the correct speed.

- Part (c): Correct derivation of minimum and maximum speeds, proper balancing of forces, and showing the final equality accurately.

Solution 1

Part (a): 

- Identifies frictional force and centripetal force: 1 mark

- Correct derivation of the maximum speed: 1 mark

- Total: 2/2 marks

Part (b): 

- Identifies normal force as providing centripetal force: 1 mark

- Correctly balances the forces: 1 mark

- Correct derivation of speed: 1 mark

- Total: 3/3 marks

Part (c): 

- Identifies forces and balances for maximum speed: 2 marks

- Correct normal force and vertical force balance: 2 marks

- Derives correct expression for \( v_{max} \): 3 marks

- Correct expression for \( v_{min} \): 2 marks

- Misinterpretation in the final simplification of \( v_{max}^2 - v_{min}^2 \): 1 mark

- Total: 10/11 marks

**Total: 15/16 marks**

**Feedback:** Excellent work on parts (a) and (b). In part (c), the initial derivation was good, but the final simplification needed careful reevaluation.

Solution 2

Part (a): 

- Identifies frictional force and centripetal force: 1 mark

- Correct derivation of the maximum speed: 1 mark

- Total: 2/2 marks

Part (b): 

- Identifies normal force and centripetal force: 1 mark

- Correctly balances forces but incorrectly identifies the centripetal component as \( \cos(\phi) \) instead of \( \sin(\phi) \): 0 marks

- Incorrect derivation of speed: 0 marks

- Total: 1/3 marks

Part (c): 

- Identifies forces and balances for maximum speed: 2 marks

- Correct normal force and vertical force balance: 2 marks

- Derives correct expression for \( v_{max} \): 3 marks

- Correct expression for \( v_{min} \): 2 marks

- Misinterpretation in the final simplification of \( v_{max}^2 - v_{min}^2 \): 1 mark

- Total: 10/11 marks

**Total: 13/16 marks**

**Feedback:** Good work in part (a). In part (b), the error in identifying the correct centripetal component led to an incorrect derivation. In part (c), despite some misinterpretation in the final steps, the overall approach was sound.

Solution 3

Part (a): 

- Identifies frictional force and centripetal force: 1 mark

- Correct derivation of the maximum speed: 1 mark

- Total: 2/2 marks

Part (b): 

- Identifies normal force and centripetal force: 1 mark

- Correctly balances forces but identifies incorrect centripetal component: 0 marks

- Incorrect derivation of speed: 0 marks

- Total: 1/3 marks

Part (c): 

- Identifies forces and balances for maximum speed: 2 marks

- Correct normal force and vertical force balance: 2 marks

- Derives correct expression for \( v_{max} \): 3 marks

- Correct expression for \( v_{min} \): 2 marks

- Misinterpretation in the final simplification of \( v_{max}^2 - v_{min}^2 \): 1 mark

- Total: 10/11 marks

**Total: 13/16 marks**

**Feedback:** Good derivation in part (a). In part (b), misidentifying the correct force component led to an incorrect result. In part (c), despite the good derivation, the final simplification needed to be corrected. 

Summary

- **Solution 1:** 15/16 marks

- **Solution 2:** 13/16 marks

- **Solution 3:** 13/16 marks

\bigskip
\textbf{Response 3:}

Solution 1

**Part a)**: The solution correctly identifies the frictional force and centripetal force, and derives the expression for the largest speed.

- Marks: 2/2

**Part b)**: The solution correctly sets up the normal force and its components, but there is an error in identifying the gravitational force component contributing to centripetal force. It incorrectly states \( v = \sqrt{gR\sin(\phi)} \) instead of the correct \( v = \sqrt{gR\tan(\phi)} \).

- Marks: 1/3

**Part c)**: The solution gives a thorough attempt, but it has an error in the final algebra to reach the required expression \( \frac{2Rg\mu_s}{\cos^2(\phi) - \mu_s^2 \sin^2(\phi)} \). It calculates maximum and minimum speeds correctly but fails to properly simplify the expression.

- Marks: 7/11

**Total**: 10/16

**Feedback**: Good attempt overall. Ensure correct identification of force components in inclined plane problems and careful algebraic manipulation to reach the final expressions. 

Solution 2

**Part a)**: The solution correctly identifies the frictional force and centripetal force, and derives the expression for the largest speed.

- Marks: 2/2

**Part b)**: Correct identification of forces but there is a misunderstanding in gravitational component affecting centripetal force leading to incorrect final speed \( \sqrt{Rg\cos(\phi)} \) instead of \( \sqrt{Rg\tan(\phi)} \).

- Marks: 1/3

**Part c)**: The solution presents a thorough explanation but does not correctly reach the required final expression \( \frac{2Rg\mu_s}{\cos^2(\phi) - \mu_s^2 \sin^2(\phi)} \). Errors are found in the simplification steps.

- Marks: 7/11

**Total**: 10/16

**Feedback**: Solid understanding shown but be careful with trigonometric components and algebraic simplifications in inclined plane problems.

Solution 3

**Part a)**: The solution correctly identifies the frictional force and centripetal force, and derives the expression for the largest speed.

- Marks: 2/2

**Part b)**: Correctly sets up and identifies the forces but incorrectly concludes with \( \sqrt{gR \sin(\phi)} \) instead of \( \sqrt{gR\tan(\phi)} \).

- Marks: 1/3

**Part c)**: Properly identifies and derives expressions for \( v_{max} \) and \( v_{min} \), but like other solutions, does not reach the correct final expression for \( v_{max}^2 - v_{min}^2 \).

- Marks: 7/11

**Total**: 10/16

**Feedback**: Good attempt and detailed steps. Ensure correct trigonometric considerations and algebraic accuracy in final derivations for complete solution alignment. 

General Comments

All solutions correctly tackle parts of the problem, particularly in part a), but have issues with the correct trigonometric handling and algebraic simplifications in parts b) and c). Ensure careful identification and use of force components, especially in inclined plane scenarios, and double-check algebraic steps to reach the final required expressions.

\bigskip
\textbf{Response 4:}

Solution 1

**Part a:**

- Identification of frictional force \(F_f\) and centripetal acceleration \(F\): 1 mark

- Correct derivation of \(v = \sqrt{\mu_s g R}\): 1 mark

Total for part a: 2/2

**Part b:**

- Correctly identifies that the normal force provides the centripetal force: 1 mark

- Correct equation \(N \cos(\phi) = mg\): 1 mark

- Incorrect final expression \(v = \sqrt{gR\sin(\phi)}\) instead of \(v = \sqrt{Rg\tan(\phi)}\): 0 marks

Total for part b: 2/3

**Part c:**

- Incorrect maximum speed derivation: 1 mark (for the attempt)

- Incorrect minimum speed derivation: 1 mark (for the attempt)

- No proper calculation and demonstration of \(v_{max}^{2} - v_{min}^{2}\): 0 marks

Total for part c: 2/11

**Overall Total: 6/16**

**Feedback:**

- Good initial identification of forces in part a, but missed key steps in parts b and c.

- The derivation for part b missed the correct component of gravitational force.

- Part c's approach was fundamentally incorrect and lacked a detailed force balance.

---

Solution 2

**Part a:**

- Identification of frictional force \(F_f\) and centripetal acceleration \(F\): 1 mark

- Correct derivation of \(v = \sqrt{\mu_s g R}\): 1 mark

Total for part a: 2/2

**Part b:**

- Correctly identifies that the normal force provides the centripetal force: 1 mark

- Correct equation \(N \cos(\phi) = mg\): 1 mark

- Incorrect final expression \(v = \sqrt{Rg\cos(\phi)}\) instead of \(v = \sqrt{Rg\tan(\phi)}\): 0 marks

Total for part b: 2/3

**Part c:**

- Correctly sets up the problem but incorrectly derives maximum and minimum speeds: 2 marks

- Incomplete and incorrect demonstration of \(v_{max}^{2} - v_{min}^{2}\): 1 mark

Total for part c: 3/11

**Overall Total: 7/16**

**Feedback:**

- Good initial work on identifying forces and setting up equations.

- Part b's approach incorrectly used the wrong gravitational component.

- Part c had good attempts but did not correctly derive the speeds or their difference.

---

Solution 3

**Part a:**

- Identification of frictional force \(F_f\) and centripetal acceleration \(F\): 1 mark

- Correct derivation of \(v = \sqrt{\mu_s g R}\): 1 mark

Total for part a: 2/2

**Part b:**

- Correctly identifies that the normal force provides the centripetal force: 1 mark

- Correct equation \(N \cos(\phi) = mg\): 1 mark

- Incorrect final expression \(v = \sqrt{gR\sin(\phi)}\) instead of \(v = \sqrt{Rg\tan(\phi)}\): 0 marks

Total for part b: 2/3

**Part c:**

- Incorrect maximum speed derivation: 1 mark (for the attempt)

- Incorrect minimum speed derivation: 1 mark (for the attempt)

- No proper calculation and demonstration of \(v_{max}^{2} - v_{min}^{2}\): 0 marks

Total for part c: 2/11

**Overall Total: 6/16**

**Feedback:**

- Good work on part a, but part b missed the correct trigonometric component.

- Part c was fundamentally flawed and did not reach the correct final result.

- Ensure accurate vector breakdowns for forces in inclined plane problems.

---

Summary

- **Solution 1:** 6/16

- **Solution 2:** 7/16

- **Solution 3:** 6/16

Overall, each solution struggled with parts b and c. Solution 2 showed slightly better understanding in part c, hence the marginally higher score. Ensure all force components and correct trigonometric identities are used in future solutions.

\bigskip
\textbf{Response 5:}

**Solution 1:**

**Part a:**

- Correctly identified frictional force and centripetal force. (1 mark)

- Correctly derived the largest speed expression. (1 mark)

**Part b:**

- Incorrect use of the gravitational component for the centripetal force.

- Did not correctly derive the speed expression. (0 marks)

**Part c:**

- Identified the forces but incorrectly derived \(v_{max}\) and \(v_{min}\).

- Incorrect algebraic manipulation, did not show correct relationship for \(v_{max}^2 - v_{min}^2\). (2 marks)

**Total: 4/16**

**Feedback:**
Good start with identifying forces in part a. However, part b had an incorrect approach, and part c needed a more careful derivation and algebraic manipulation to show the required equality.

---

**Solution 2:**

**Part a:**

- Correctly identified frictional force and centripetal force. (1 mark)

- Correctly derived the largest speed expression. (1 mark)

**Part b:**

- Correctly identified normal force providing centripetal force but misinterpreted component use.

- Derived incorrect speed expression. (0 marks)

**Part c:**

- Attempted derivations for \(v_{max}\) and \(v_{min}\) but incorrect force components used.

- Did not correctly simplify or show required equality for \(v_{max}^2 - v_{min}^2\). (3 marks)

**Total: 5/16**

**Feedback:**

Correct understanding of forces in part a. In part b, incorrect handling of forces on an incline. Part c showed effort but had incorrect algebraic steps.

---

**Solution 3:**

**Part a:**

- Correctly identified frictional force and centripetal force. (1 mark)

- Correctly derived the largest speed expression. (1 mark)

**Part b:**

- Correctly identified gravitational component for centripetal force.

- Derived incorrect speed expression by using incorrect component. (0 marks)

**Part c:**

- Attempted derivations for \(v_{max}\) and \(v_{min}\) but used incorrect components.

- Incorrect simplification, did not show correct equality for \(v_{max}^2 - v_{min}^2\). (3 marks)

**Total: 5/16**

**Feedback:**

Good start with identifying forces in part a. However, incorrect handling of components for forces in parts b and c. Ensure careful analysis of forces on inclines and correct algebraic manipulations.

---

**Summary:**

All solutions correctly identified forces in part a but struggled with parts b and c, particularly with handling forces on an incline and algebraic manipulations. More attention to detail and correct force components is needed for complete and accurate derivations.


\subsection*{Question 7:}

\textbf{Response 1:}

Solution 1:

**Part a): 4/4**

- Correctly states the form of the energy-momentum 4-vector and includes the definitions and relationships involving \(\gamma\), \(E\), and \(\underline{p}\).

- Accurately computes the scalar product and demonstrates its invariance under Lorentz transformation.

**Part b): 5/5**

- Clearly presents the 4-momenta of the decaying particles.

- Correctly uses Lorentz transformations to derive the expressions of the 4-momenta in the lab frame.

**Part c): 3/4**

- Correctly identifies the relationship between energy and wavelength.

- Provides appropriate calculations but does not simplify the final wavelength expressions accurately.

- Minor miscalculation in the exact numeric values of wavelengths.

**Total: 12/13**

**Feedback:** Excellent job. The scalar product computation and Lorentz transformation were well-executed. A slight mistake in the final wavelength values cost a point.

---

Solution 2:

**Part a): 4/4**

- States the form of the energy-momentum 4-vector and appropriately defines all necessary quantities.

- Thoroughly computes the scalar product and demonstrates its invariance.

**Part b): 4/5**

- Defines 4-momenta in the rest frame correctly.

- Shows Lorentz transformations, but the explanation is a bit unclear and needs better clarity in deriving the lab frame 4-momenta.

**Part c): 4/4**

- Accurately calculates the energies and wavelengths in both frames.

- Correct application of the formulas and correct numeric results.

**Total: 12/13**

**Feedback:** Very well done. Part b could use a clearer explanation of the transformations, but overall, your approach and calculations are correct.

---

Solution 3:

**Part a): 4/4**

- States the correct form of the energy-momentum 4-vector and provides necessary definitions and relationships.

- Correctly computes the scalar product and demonstrates invariance.

**Part b): 4/5**

- Defines the 4-momenta in the rest frame correctly.

- Uses Lorentz transformations but lacks detail and clarity in deriving the lab frame 4-momenta.

**Part c): 3/4**

- Correctly identifies the relationship between energy and wavelength but makes a significant numerical error in the energy transformation.

- Final wavelengths are incorrect due to this error.

**Total: 11/13**

**Feedback:** Good job overall. Part b needs more clarity in the transformations, and part c contains a significant numerical error. Ensure accuracy in energy transformation to avoid such mistakes.

---

\bigskip
\textbf{Response 2:}

**Solution 1:**

a) Energy-Momentum 4-Vector and Lorentz Invariance

- Correctly identifies the energy-momentum 4-vector and its components.

- Correctly derives the scalar product and demonstrates its invariance.

**Marks:** 4/4

b) Four-Momenta of Decay Particles in Lab Frame

- Clearly states the 4-momenta in the rest frame.

- Correctly applies Lorentz transformations to find the 4-momenta in the lab frame.

**Marks:** 5/5

c) Wavelengths in Rest and Lab Frames

- Correctly calculates the photon energies in the rest frame.

- Correctly applies the Lorentz transformation to find energies in the lab frame.

- Correctly calculates wavelengths in both frames, but the wavelengths given do not match the problem's expected results, leading to confusion.

**Marks:** 3/4

**Total:** 12/13

**Feedback:** Excellent work. Thorough understanding and correct application of the Lorentz transformation. Minor calculation discrepancies in the wavelength results.

---

**Solution 2:**

a) Energy-Momentum 4-Vector and Lorentz Invariance

- Correctly identifies the energy-momentum 4-vector and its components.

- Correctly derives the scalar product and demonstrates its invariance.

**Marks:** 4/4

b) Four-Momenta of Decay Particles in Lab Frame

- Correctly states the 4-momenta in the rest frame.

- Correctly applies Lorentz transformations to find the 4-momenta in the lab frame.

**Marks:** 5/5

c) Wavelengths in Rest and Lab Frames

- Correctly calculates the photon energies in the rest frame.

- Correctly applies the Lorentz transformation to find energies in the lab frame.

- Correctly calculates wavelengths in both frames, but some final wavelength values are not consistent with expected results.

**Marks:** 3/4

**Total:** 12/13

**Feedback:** Very good understanding of the problem. Correct application of Lorentz transformation. Ensure final calculations match expected results more closely.

---

**Solution 3:**

a) Energy-Momentum 4-Vector and Lorentz Invariance

- Correctly identifies the energy-momentum 4-vector and its components.

- Correctly derives the scalar product and demonstrates its invariance.

**Marks:** 4/4

b) Four-Momenta of Decay Particles in Lab Frame

- Correctly states the 4-momenta in the rest frame.

- Correctly outlines the application of Lorentz transformations but lacks explicit transformation steps.

**Marks:** 4/5

c) Wavelengths in Rest and Lab Frames

- Correctly calculates the photon energies in the rest frame.

- Applies Lorentz transformation but provides incorrect energy values.

- Provides incorrect final wavelength values.

**Marks:** 2/4

**Total:** 10/13

**Feedback:** Good understanding, but needs more precision in applying Lorentz transformations and calculating final values. Explicit transformation steps would improve clarity.

---

**General Feedback for All Solutions:**

- Good understanding of the energy-momentum 4-vector and Lorentz invariance.

- Ensure to explicitly show all steps when applying Lorentz transformations.

- Double-check final calculations for consistency with expected results.

- Correct use of units and significant figures is important for full marks.

\bigskip
\textbf{Response 3:}

Here is the detailed evaluation of each solution based on the provided mark scheme.

Solution 1

**Part a)**

- Correct form of the energy-momentum 4-vector and definitions: 2 marks.

- Calculation of the scalar product and showing Lorentz invariance: 2 marks.

- Total: 4/4

**Part b)**

- Identifies 4-momenta of decay particles and sums to original 4-momentum: 2 marks.

- Correctly states the momenta are opposite: 1 mark.

- Inverse Lorentz transformation to lab frame with correct approach: 2 marks.

- Total: 5/5

**Part c)**

- Use of given values and calculation of gamma: 1 mark.

- Correct wavelength expression using \( \lambda = \frac{h}{p} \): 1 mark.

- Correct calculation of the wavelengths in both frames: 2 marks.

- Total: 4/4

**Overall: 13/13**

**Feedback:** Excellent solution with clear and correct calculations and explanations in all parts.

Solution 2

**Part a)**

- Correct form of the energy-momentum 4-vector and definitions: 2 marks.

- Calculation of the scalar product and showing Lorentz invariance: 2 marks.

- Total: 4/4

**Part b)**

- Identifies 4-momenta of decay particles and sums to original 4-momentum: 2 marks.

- Correctly states the momenta are opposite: 1 mark.

- Inverse Lorentz transformation to lab frame with correct approach: 2 marks.

- Total: 5/5

**Part c)**

- Use of given values and calculation of gamma: 1 mark.

- Correct wavelength expression using \( \lambda = \frac{h}{p} \): 1 mark.

- Correct calculation of the wavelengths in both frames: 2 marks.

- Total: 4/4

**Overall: 13/13**

**Feedback:** Clear and detailed solution with all parts correctly addressed. 

Solution 3

**Part a)**

- Correct form of the energy-momentum 4-vector and definitions: 2 marks.

- Calculation of the scalar product and showing Lorentz invariance: 2 marks.

- Total: 4/4

**Part b)**

- Identifies 4-momenta of decay particles and sums to original 4-momentum: 2 marks.

- Correctly states the momenta are opposite: 1 mark.

- Inverse Lorentz transformation approach explained but values not fully derived: 1 mark.

- Total: 4/5

**Part c)**

- Use of given values and calculation of gamma: 1 mark.

- Correct wavelength expression using \( \lambda = \frac{h}{p} \): 1 mark.

- Incorrect wavelength calculations due to an error in energy values: 0 marks.

- Total: 2/4

**Overall: 10/13**

**Feedback:** Generally good understanding and correct methods, but errors in the final wavelength calculations led to a loss of marks in part c).

**Final Evaluation Summary:**

- Solution 1: 13/13 - Excellent, full marks.

- Solution 2: 13/13 - Excellent, full marks.

- Solution 3: 10/13 - Good effort, minor errors in part c).

\bigskip
\textbf{Response 4:}

General Feedback and Marks

Solution 1

- **Part a:**

  - Correct form of the energy-momentum 4-vector: 1/1.
  
  - Expression of the 4-vector: 1/1.
  
  - Correct scalar product and proof of invariance: 2/2.
  
  - **Total for Part a: 4/4**

- **Part b:**

  - Correct 4-momenta setup in the rest frame: 2/2.
  
  - Identification of equal and opposite momenta and energy: 1/1.
  
  - Correct Lorentz transformation back to lab frame: 2/2.
  
  - **Total for Part b: 5/5**

- **Part c:**

  - Correct use of given values and correct gamma calculation: 1/1.
  
  - Correct wavelength expression: 1/1.
  
  - Correct calculation for wavelengths in rest and lab frames: 2/2.
  
  - **Total for Part c: 4/4**

**Overall Total: 13/13**

- Excellent work. All parts are correct and clearly explained. The scalar product invariance was well shown, and transformations were applied accurately.

Solution 2

- **Part a:**

  - Correct form of the energy-momentum 4-vector: 1/1.
  
  - Expression of the 4-vector: 1/1.
  
  - Correct scalar product and proof of invariance: 2/2.
  
  - **Total for Part a: 4/4**

- **Part b:**

  - Correct 4-momenta setup in the rest frame: 2/2.
  
  - Identification of equal and opposite momenta and energy: 1/1.
  
  - Lorentz transformation back to lab frame is correct but could be clearer: 1/2.
  
  - **Total for Part b: 4/5**

- **Part c:**

  - Correct use of given values and correct gamma calculation: 1/1.
  
  - Correct wavelength expression: 1/1.
  
  - Slight errors in final calculations, though concept is correct: 1/2.
  
  - **Total for Part c: 3/4**

**Overall Total: 11/13**

- Good work overall. Minor issues in the clarity of Lorentz transformation application and final wavelength calculation.

Solution 3

- **Part a:**

  - Correct form of the energy-momentum 4-vector: 1/1.
  
  - Expression of the 4-vector: 1/1.
  
  - Correct scalar product but with minor algebraic mishandling: 1/2.
  
  - **Total for Part a: 3/4**

- **Part b:**

  - Correct 4-momenta setup in the rest frame: 2/2.
  
  - Identification of equal and opposite momenta and energy: 1/1.
  
  - Lorentz transformation attempted but with errors: 1/2.
  
  - **Total for Part b: 4/5**

- **Part c:**

  - Correct use of given values and gamma calculation: 1/1.
  
  - Correct wavelength expression: 1/1.
  
  - Calculation and application issues with wavelengths: 1/2.
  
  - **Total for Part c: 3/4**

**Overall Total: 10/13**

- Overall good work but needs more careful handling of algebra and transformations. Final calculations need review for accuracy.

\bigskip
\textbf{Response 5:}

Solution 1:

**a) Energy-Momentum 4-Vector and Lorentz Invariance**

Marks: 4/4

Feedback: Excellent work in stating the energy-momentum 4-vector and showing the scalar product's invariance under Lorentz transformation. The derivation was thorough and accurate.

**b) Four-Momenta of Decay Particles in Lab Frame**

Marks: 5/5

Feedback: Comprehensive derivation of the four-momenta of the massless particles in both rest and lab frames. Correct application of the Lorentz transformation and conservation principles.

**c) Wavelengths in Rest and Lab Frames**

Marks: 3/4

Feedback: Good approach to calculating wavelengths, but the final numerical results were not accurate. Ensure precision in unit conversions and constants.

Total: 12/13

Solution 2:

**a) Energy-Momentum 4-Vector and Lorentz Invariance**

Marks: 4/4

Feedback: Clearly stated the energy-momentum 4-vector and provided a solid proof for the scalar product's invariance under Lorentz transformation.

**b) Four-Momenta of Decay Particles in Lab Frame**

Marks: 5/5

Feedback: Correctly determined the four-momenta in the lab frame, utilizing the Lorentz transformation correctly. Detailed and precise.

**c) Wavelengths in Rest and Lab Frames**

Marks: 2/4

Feedback: Attempted a thorough explanation, but the numerical values were off, and there was a minor error in the interpretation of energy distribution among photons.

Total: 11/13

Solution 3:

**a) Energy-Momentum 4-Vector and Lorentz Invariance**

Marks: 4/4

Feedback: Excellent derivation of the energy-momentum 4-vector and the proof for invariance under Lorentz transformation. Clear and correct.

**b) Four-Momenta of Decay Particles in Lab Frame**

Marks: 4/5

Feedback: The approach was solid, but the final expressions for the four-momenta in the lab frame were not fully simplified or clearly stated.

**c) Wavelengths in Rest and Lab Frames**

Marks: 2/4

Feedback: The explanation was good, but numerical values were incorrect. Ensure accuracy in converting and applying constants.

Total: 10/13

\subsection*{Question 8:}

\textbf{Response 1:}

Solution 1

Marks:

**Part a)**

- Position of mass: \(1/1\)

- Kinetic and potential energy calculation: \(3/3\)

- Euler-Lagrange equation: \(2/3\)

- Simplification and substitution: \(2/2\)

**Part b)**

- Limit case: \(1/1\)

**Total: 9/10**

**Feedback:**

Good solution overall. The derivation and calculations for kinetic and potential energy were well done. The Euler-Lagrange equations were derived correctly but the simplification had minor errors. The small-angle limit was handled well.

Solution 2

Marks:

**Part a)**

- Position of mass: \(0/1\)

- Kinetic and potential energy calculation: \(2/3\)

- Euler-Lagrange equation: \(2/3\)

- Simplification and substitution: \(1/2\)

**Part b)**

- Limit case: \(1/1\)

**Total: 6/10**

**Feedback:**

The solution lacked precision in defining the position of the mass initially. The kinetic energy calculation was not fully correct but showed an understanding of the principles. The Euler-Lagrange equation and its simplification were partly correct, but the final equation did not fully match the expected result. Good job on the small-angle limit.

Solution 3

Marks:

**Part a)**

- Position of mass: \(1/1\)

- Kinetic and potential energy calculation: \(2/3\)

- Euler-Lagrange equation: \(2/3\)

- Simplification and substitution: \(2/2\)

**Part b)**

- Limit case: \(1/1\)

**Total: 8/10**

**Feedback:**

The position of the mass was correctly identified. The calculations for kinetic and potential energy had minor errors but were close to the correct form. The Euler-Lagrange equations were derived correctly, and the simplification was handled well. The small-angle approximation was correct.

Overall, Solution 1 was the most accurate and complete, followed by Solution 3 and Solution 2.

\bigskip
\textbf{Response 2:}

I'll review the three solutions provided by the students and grade them according to the mark scheme.

Solution 1:

**Marks for Part a (out of 9):**

- Position of mass given: The student correctly identifies the position of the mass. [1 mark]

- Calculation of kinetic and potential energy: 
  
  - The kinetic energy is partially correct, but the student introduces unnecessary complexity and errors by directly including Python computation. The potential energy expression is correct. [2 marks]
  
  - The student makes an effort to derive the Lagrangian but incorrectly simplifies the expressions and uses unnecessary steps. [1 mark]

- Obtain and simplify Euler-Lagrange equations: 
  
  - The student correctly writes the Euler-Lagrange equations and simplifies them to obtain the final form of the equation of motion. [3 marks]

- Compute \( \ddot{x} \) and substitute: The student correctly computes \( \ddot{x} \) and substitutes it into the equation of motion. [2 marks]

Total for part a: 7/9

**Marks for Part b (out of 1):**

- Correctly shows the reduction to the classic pendulum problem and equation of motion. [1 mark]

Total for part b: 1/1

**Total Marks: 8/10**

**Feedback:** Good attempt overall. There are some unnecessary steps and minor errors in simplification. Clarity and precision in the derivation process could be improved.

Solution 2:

**Marks for Part a (out of 9):**


- Position of mass given: The student correctly identifies the position of the mass. [1 mark]

- Calculation of kinetic and potential energy: 
  
  - The kinetic energy part is mostly correct, though the student misses the interaction term between \( \dot{x} \) and \( \dot{\phi} \). The potential energy is correct. [2 marks]
  
  - The Lagrangian is presented correctly but lacks the cross term. [1 mark]

- Obtain and simplify Euler-Lagrange equations:
 
  - The student correctly writes the Euler-Lagrange equations but simplifies them without considering the cross term, resulting in an incorrect equation of motion. [2 marks]

- Compute \( \ddot{x} \) and substitute: The student does not perform this step correctly. [0 marks]

Total for part a: 6/9

**Marks for Part b (out of 1):**

- Correctly shows the reduction to the classic pendulum problem and equation of motion. [1 mark]

Total for part b: 1/1

**Total Marks: 7/10**

**Feedback:** The solution is on the right track but misses key terms in the kinetic energy and Euler-Lagrange equations. Simplifications and assumptions should be more carefully stated.

Solution 3:

**Marks for Part a (out of 9):**

- Position of mass given: The student correctly identifies the position of the mass. [1 mark]

- Calculation of kinetic and potential energy: 
  
  - The kinetic energy part is mostly correct but misses the interaction term between \( \dot{x} \) and \( \dot{\phi} \). The potential energy is correct. [2 marks]
 
  - The Lagrangian is presented correctly but lacks the cross term. [1 mark]

- Obtain and simplify Euler-Lagrange equations:
  
  - The student writes the Euler-Lagrange equations correctly but simplifies them without considering the cross term, resulting in an incorrect equation of motion. [2 marks]
  
- Compute \( \ddot{x} \) and substitute: The student does not perform this step correctly. [0 marks]

Total for part a: 6/9

**Marks for Part b (out of 1):**

- Correctly shows the reduction to the classic pendulum problem and equation of motion. [1 mark]

Total for part b: 1/1

**Total Marks: 7/10**

**Feedback:** The solution is clear and follows logical steps, but there are errors in the simplification of the Euler-Lagrange equations. The importance of cross terms should be emphasized in such derivations.

---

**Summary:**

- **Solution 1:** 8/10

- **Solution 2:** 7/10

- **Solution 3:** 7/10

General feedback: All students demonstrate an understanding of the Lagrangian mechanics framework but need to pay closer attention to the details of the kinetic energy calculation and the proper inclusion of all terms in the Lagrangian and Euler-Lagrange equations. Simplifications and assumptions should be clearly stated to avoid missing critical components.

\bigskip
\textbf{Response 3:}

Solution 1 Grading and Feedback

**Part a:**

1. The solution correctly starts with defining the Lagrangian and the kinetic and potential energies.

2. Correctly identifies the position of the mass.

3. The kinetic energy calculation is correct and matches the expected form.

4. Potential energy is correctly calculated.

5. Correct form of the Lagrangian is provided.

6. Correct use of Euler-Lagrange equations and proper differentiation.

7. Final equation of motion is correctly derived.

**Part b:**

1. Correctly simplifies the equation in the small-angle approximation.

**Marks: 9/10**

**Feedback:**

Excellent derivation and clear steps. The use of Python for algebraic simplification is a nice touch, though not shown explicitly here. Some steps could be more concisely presented, but overall, a strong and thorough solution.

---

Solution 2 Grading and Feedback

**Part a:**

1. Begins with a clear definition of the Lagrangian.

2. Calculates kinetic and potential energies correctly, but does not include the horizontal component of kinetic energy properly.

3. Incorrectly calculates the radial kinetic energy, missing the combined form of the kinetic energy.

4. The potential energy is correctly stated.

5. Errors in simplifying and applying the Euler-Lagrange equation, resulting in an incorrect final equation.

**Part b:**

1. Correctly applies the small-angle approximation and identifies the classic pendulum equation.

**Marks: 6/10**

**Feedback:**

The solution starts well but misses key components of the kinetic energy, leading to errors in the final equation of motion. The small-angle approximation is handled correctly, but more attention to detail is needed in the initial derivation.

---

Solution 3 Grading and Feedback

**Part a:**

1. Correctly identifies the initial position and potential energy.

2. Kinetic energy calculation is correct and matches the expected form.

3. Provides a correct form of the Lagrangian.

4. Correctly applies Euler-Lagrange equations and differentiates properly.

5. The final equation of motion is correctly derived.

**Part b:**

1. Correctly simplifies the equation in the small-angle approximation.

**Marks: 9/10**

**Feedback:**

A thorough and well-structured solution. The step-by-step approach is clear and easy to follow. A strong understanding of the problem and the necessary derivations is evident.

---

Summary

- **Solution 1: 9/10** - Thorough and detailed, with minor improvements needed for conciseness.

- **Solution 2: 6/10** - Correct concept but lacks attention to detail, leading to errors in the kinetic energy calculation.

- **Solution 3: 9/10** - Clear and accurate, demonstrating a strong grasp of the problem.

\bigskip
\textbf{Response 4:}

I will now grade each solution out of 10 marks, with concise feedback, according to the mark scheme provided.

Solution 1

**Marks Breakdown:**

- Position of mass: 1 mark

- Kinetic and potential energy to give Lagrangian: 2 marks (Detailed, but missing the correct form for potential energy and partially incorrect kinetic energy calculations)

- Euler-Lagrange equations: 2 marks (Good effort, but some simplification and steps are not fully correct)

- Compute and substitute \(\ddot{x}\): 1 mark

- Small-angle approximation: 1 mark

**Total Marks:** 6/10

**Feedback:**

Good attempt at breaking down the problem. The kinetic and potential energy calculations are not fully correct, which leads to errors in the Lagrangian. The use of Python for algebraic manipulations is innovative but not fully correct in execution. Small-angle approximation and classical pendulum relation are correctly identified.

Solution 2

**Marks Breakdown:**

- Position of mass: 0 marks (Did not explicitly write the correct position vector)

- Kinetic and potential energy to give Lagrangian: 1 mark (Incomplete kinetic energy, potential energy incorrect)

- Euler-Lagrange equations: 1 mark (Euler-Lagrange equations are not derived correctly)

- Compute and substitute \(\ddot{x}\): 0 marks (Not attempted)

- Small-angle approximation: 1 mark

**Total Marks:** 3/10

**Feedback:**

This solution lacked clear steps and correctness in the kinetic and potential energy derivations, and the Euler-Lagrange equations were not correctly obtained. The small-angle approximation was recognized but the connection to the classic pendulum was not properly shown.

Solution 3

**Marks Breakdown:**

- Position of mass: 1 mark

- Kinetic and potential energy to give Lagrangian: 3 marks (Correct approach, clear derivation)

- Euler-Lagrange equations: 3 marks (Clear and correct derivation and simplification)

- Compute and substitute \(\ddot{x}\): 1 mark

- Small-angle approximation: 1 mark

**Total Marks:** 9/10

**Feedback:**

Excellent solution. It correctly identified the position of the mass and used it to derive the kinetic and potential energy terms accurately. The Euler-Lagrange equations were well derived and simplified correctly. The small-angle approximation and reduction to the classic pendulum problem were clearly shown.

In summary:

- **Solution 1:** 6/10

- **Solution 2:** 3/10

- **Solution 3:** 9/10

\bigskip
\textbf{Response 5:}

I'll begin by evaluating the provided solutions against the given problem and mark scheme.

Solution 1 Evaluation

1. **Expression for the position of mass (1 mark)**

   - Correct: \( \underline{r} = (x + L \sin(\phi), -L \cos(\phi)) \)
   
   - **Mark: 1/1**

2. **Calculation of kinetic and potential energy (3 marks)**

   - Kinetic energy derivation is mostly correct, but there's a mistake in potential energy derivation (should be \( U = -mgL \cos(\phi) \) not \( mgL(1 - \cos(\phi)) \)).
   
   - **Mark: 2/3**

3. **Derive and simplify Euler-Lagrange equations (3 marks)**

   - Correct process but seems overly complicated. Should simplify the steps more effectively.
   
   - **Mark: 2/3**

4. **Compute \( \ddot{x} \) and substitute into equation of motion (2 marks)**

   - Correctly computes \( \ddot{x} \) and substitutes into the equation.
   
   - **Mark: 2/2**

5. **Classic pendulum problem (1 mark)**

   - Correctly reduces to the classic pendulum problem.
   
   - **Mark: 1/1**

**Total Marks: 8/10**

**Feedback:** Good understanding of Lagrangian mechanics. Small error in potential energy calculation. Simplify Euler-Lagrange steps.

Solution 2 Evaluation

1. **Expression for the position of mass (1 mark)**

   - Incorrect: missed horizontal component due to \( \phi \).
   
   - **Mark: 0/1**

2. **Calculation of kinetic and potential energy (3 marks)**

   - Kinetic energy derivation is correct, potential energy is incorrect (should be \( U = -mgL \cos(\phi) \)).
   
   - **Mark: 2/3**

3. **Derive and simplify Euler-Lagrange equations (3 marks)**

   - Correct derivation but missed the dependence on \( \dot{x} \) in the final equations.
   
   - **Mark: 1/3**

4. **Compute \( \ddot{x} \) and substitute into equation of motion (2 marks)**

   - Did not compute \( \ddot{x} \) explicitly in the context of the given problem.
   
   - **Mark: 0/2**

5. **Classic pendulum problem (1 mark)**

   - Correctly reduces to the classic pendulum problem.
   
   - **Mark: 1/1**

**Total Marks: 4/10**

**Feedback:** Lacks clarity in integrating the movement of support into the derivation. Needs better handling of kinetic and potential energies and the Euler-Lagrange process.

Solution 3 Evaluation

1. **Expression for the position of mass (1 mark)**

   - Correct: \( \underline{r} = (x + L \sin(\phi), -L \cos(\phi)) \)
   
   - **Mark: 1/1**

2. **Calculation of kinetic and potential energy (3 marks)**

   - Correct kinetic and potential energy calculations.
   
   - **Mark: 3/3**

3. **Derive and simplify Euler-Lagrange equations (3 marks)**

   - Correct derivation and simplification.
   
   - **Mark: 3/3**

4. **Compute \( \ddot{x} \) and substitute into equation of motion (2 marks)**

   - Correctly computes \( \ddot{x} \) and substitutes into the equation.
   
   - **Mark: 2/2**

5. **Classic pendulum problem (1 mark)**

   - Correctly reduces to the classic pendulum problem.
   
   - **Mark: 1/1**

**Total Marks: 10/10**

**Feedback:** Excellent solution. Clear and correct derivation of kinetic and potential energies, Euler-Lagrange equations, and classic pendulum limit.

Summary

- **Solution 1:** 8/10. Good effort, minor errors.

- **Solution 2:** 4/10. Needs improvement in understanding and applying the Lagrangian framework.

- **Solution 3:** 10/10. Excellent, comprehensive solution.


\subsection*{Question 9:}

\textbf{Response 1:}

Let's grade the solutions using the provided mark scheme.

Solution 1

Part a)

1. Application of laws of motion:

    - Correctly applied: \( T - mg = -ma \) and \( RT = I \alpha = \frac{1}{2} M R^2 \alpha \)
    
    - Full 2 marks.

2. Constraint:

    - Correctly applied: \( a = R \alpha \)
    
    - Full 1 mark.

3. Solving simultaneous equations:

    - Correct expressions for \( T \), \( a \), and \( \alpha \)
    
    - Full 3 marks.

Part b)

- Correctly substituted values to obtain numerical results: \( a = 3.27 \, ms^{-2} \), \( T = 19.6 \, N \), and \( \alpha = 16.35 \, rad/s^{-2} \)

- Full 1 mark.

**Total: 6 + 1 = 7 marks**

**Feedback:** Excellent work. All the required steps were properly detailed, and correct values were calculated.

Solution 2

Part a)

1. Application of laws of motion:

    - Correctly applied: \( m \cdot g - T = m \cdot a \) and \( T \cdot R = \left( \frac{1}{2} M R^2 \right) \cdot \alpha \)
    
    - Full 2 marks.

2. Constraint:

    - Correctly applied: \( a = R \cdot \alpha \)
    
    - Full 1 mark.

3. Solving simultaneous equations:

    - Correct expressions for \( a \), \( \alpha \), and \( T \)
    
    - Full 3 marks.

Part b)

- Correctly substituted values to obtain numerical results: \( a = 3.27 \, ms^{-2} \), \( T = 19.6 \, N \), and \( \alpha = 16.33 \, rad/s^{-2} \)

- Full 1 mark.

**Total: 6 + 1 = 7 marks**

**Feedback:** Excellent explanation and correct numerical results.

Solution 3

Part a)

1. Application of laws of motion:

    - Correctly applied: \( m g - T = m a \) and \( T \cdot R = \frac{1}{2} M R^2 \cdot \alpha \)
    
    - Full 2 marks.

2. Constraint:
    - Correctly applied: \( a = \alpha R \)
    
    - Full 1 mark.

3. Solving simultaneous equations:
    - Correct expressions for \( T \), \( a \), and \( \alpha \)
    
    - Full 3 marks.

Part b)

- Correctly substituted values to obtain numerical results: \( a = 3.27 \, ms^{-2} \), \( T = 19.6 \, N \), and \( \alpha = 16.35 \, rad/s^{-2} \)

- Full 1 mark.

**Total: 6 + 1 = 7 marks**

**Feedback:** Well-structured solution and correct calculations.

---

All three solutions correctly derive the necessary equations and calculate the values accurately. Each deserves full marks for both parts. 

Great job by all the students!

\bigskip
\textbf{Response 2:}

Let's begin by evaluating each solution according to the provided mark scheme:

Solution 1 Evaluation

**Part a:**

1. Correctly applies torque and rotational motion laws to find \(\alpha\) - 2 marks.

2. Correctly identifies the relationship \(a = R\alpha\) - 1 mark.

3. Solves the equations for \(\alpha\), \(T\), and \(a\), but with some steps missing, such as the exact form of the final expressions. The final results are correct - 2 marks.

**Part b:**

1. Correct substitution of values, precise results, correct units - 1 mark.

**Total: 6 marks**  

**Feedback:** This solution is mostly correct and follows the required steps accurately, but it misses out on explicitly stating the final derived expressions before substitution.

Solution 2 Evaluation

**Part a:**

1. Correctly applies laws of motion to block and disk, finds correct torque equation - 2 marks.

2. Identifies the relationship \(a = R\alpha\) - 1 mark.

3. Derives the correct final expressions for \(a\), \(T\), and \(\alpha\) - 3 marks.

**Part b:**

1. Correct substitution of values, precise results, correct units - 1 mark.

**Total: 7 marks**  

**Feedback:** This solution is comprehensive, follows all the required steps accurately, and provides clear final expressions before substituting values.

Solution 3 Evaluation

**Part a:**

1. Correctly applies laws of motion to block and disk, finds correct torque equation - 2 marks.

2. Identifies the relationship \(a = R\alpha\) - 1 mark.

3. Derives the correct final expressions for \(a\), \(T\), and \(\alpha\) - 3 marks.

**Part b:**

1. Correct substitution of values, precise results, correct units - 1 mark.

**Total: 7 marks**  

**Feedback:** This solution is also comprehensive and follows all the required steps accurately. It provides clear final expressions and correctly substitutes values.

Summary

- **Solution 1:** 6/7

- **Solution 2:** 7/7

- **Solution 3:** 7/7

\bigskip
\textbf{Response 3:}

Solution 1 Feedback:

**Part a):**

- Correctly applies the torque equation and derives an expression for \(\alpha\). (2/2)

- Correctly applies Newton's second law to the block. (2/2)

- Solves for \(T\), \(a\), and \(\alpha\) correctly, but the final expressions could be more explicitly written out. (2/3)

**Part b):**

- Numerical values are correctly calculated. (1/1)

**Marks: 6/7**

Solution 2 Feedback:

**Part a):**

- Correctly applies Newton's second law to both the block and the disk. (2/2)

- Correctly identifies and uses the constraint that \(a = R\alpha\). (1/1)

- Solves the simultaneous equations correctly for \(T\), \(a\), and \(\alpha\). (3/3)

**Part b):**

- Numerical values are correctly calculated. (1/1)

**Marks: 7/7**

Solution 3 Feedback:

**Part a):**

- Correctly applies the torque equation and Newton's second law to derive \(\alpha\). (2/2)

- Correctly applies Newton's second law to the block. (2/2)

- Solves for \(T\), \(a\), and \(\alpha\) correctly, though the derivations could be more streamlined. (3/3)

**Part b):**

- Numerical values are correctly calculated. (1/1)

**Marks: 7/7**

Summary:

- **Solution 1:** 6/7. Good approach but could be more explicit in the final expressions.

- **Solution 2:** 7/7. Clear and correct application of laws with correctly calculated numerical values.

- **Solution 3:** 7/7. Accurate and detailed, though could be more concise in derivations.

\bigskip
\textbf{Response 4:}

Let's start grading each solution using the provided mark scheme. Here's the analysis and grading for each solution:

Solution 1:

**Part a)**:

- Correctly applies Newton's second law for both the block and the disk: 2 marks.

- Uses the constraint \( a = R\alpha \) correctly: 1 mark.

- Solves the equations to find expressions for \( T \), \( a \), and \( \alpha \), but there's a minor error in the tension equation simplification which leads to \( T = \frac{mgM}{M + 2m} \) instead of \( T = \frac{\frac{1}{2}Mmg}{m + \frac{1}{2}M} \): 2 marks.
  
**Part b)**:

- Correctly substitutes the values and performs the calculations to find \( a \), \( T \), and \( \alpha \) with appropriate precision: 1 mark.

**Total Marks for Solution 1: 6/7**.

**Feedback**: Good understanding of the physics principles and correct application of laws. Minor mistake in simplifying the tension expression.

Solution 2:

**Part a)**:

- Correctly applies Newton's second law for both the block and the disk: 2 marks.

- Uses the constraint \( a = R\alpha \) correctly: 1 mark.

- Solves the equations to find expressions for \( T \), \( a \), and \( \alpha \) accurately: 3 marks.

**Part b)**:

- Correctly substitutes the values and performs the calculations to find \( a \), \( T \), and \( \alpha \) with appropriate precision: 1 mark.

**Total Marks for Solution 2: 7/7**.

**Feedback**: Excellent work. Clear and correct derivations with accurate calculations for the given values.

Solution 3:

**Part a)**:

- Correctly applies Newton's second law for both the block and the disk: 2 marks.

- Uses the constraint \( a = R\alpha \) correctly: 1 mark.

- Solves the equations to find expressions for \( T \), \( a \), and \( \alpha \), but makes an algebraic error in simplifying \( T \): 2 marks.

**Part b)**:

- Correctly substitutes the values and performs the calculations to find \( a \), \( T \), and \( \alpha \) with appropriate precision: 1 mark.

**Total Marks for Solution 3: 6/7**.

**Feedback**: Good effort. Correctly applied laws and derived expressions, but an algebraic simplification error affected the tension formula.

---

Summary:

- **Solution 1**: 6/7

- **Solution 2**: 7/7

- **Solution 3**: 6/7

If you need further details or another task, feel free to ask!

\bigskip
\textbf{Response 5:}

Grading the Solutions

Solution 1

**Part a:**

- Correct application of laws of motion for the acceleration of the block and torque on the disk. (2/2)

- Correctly states the constraint that the acceleration of the block equals the acceleration of the disk. (1/1)

- Correct solution of simultaneous equations, though the final expressions are presented without much clarity. (2/3)

**Part b:**

- Correct substitution of values and calculation. (1/1)

**Marks:**

- Part a: 5/6

- Part b: 1/1

- **Total: 6/7**

**Feedback:** Good solution overall with clear steps, but the simultaneous equation part could be more detailed and explicit.

Solution 2

**Part a:**

- Correct application of laws of motion for the block and torque on the disk. (2/2)

- Correctly states the constraint that the block's acceleration equals the disk's acceleration. (1/1)

- Correct solution of simultaneous equations with clear, step-by-step explanation. (3/3)

**Part b:**

- Correct substitution of values and calculation. (1/1)

**Marks:**

- Part a: 6/6

- Part b: 1/1

- **Total: 7/7**

**Feedback:** Excellent solution with detailed and clear derivation. Well done!

Solution 3

**Part a:**

- Correct application of laws of motion for the block and torque on the disk. (2/2)

- Correctly states the constraint that the block's acceleration equals the disk's acceleration. (1/1)

- Correct solution of simultaneous equations with clear explanation. (3/3)

**Part b:**

- Correct substitution of values and calculation. (1/1)

**Marks:**

- Part a: 6/6

- Part b: 1/1

- **Total: 7/7**

**Feedback:** Very clear and detailed solution with all steps correctly derived and explained. Excellent work!

---

Summary

- **Solution 1:** 6/7

- **Solution 2:** 7/7

- **Solution 3:** 7/7


\subsection*{Question 10:}

\textbf{Response 1:}

Solution 1

**Part a)**

- Correct application of Newton's second law: 1/1

- Correct formulation of the differential equation and general solution: 3/3

- Correct use of boundary conditions to determine the specific solution: 2/2

**Part b)**

- Correct use of initial conditions: 1/1

- Correct determination of constant vector \( \underline{A} \): 1/1

**Total:** 8/8

**Feedback:** 

- Excellent job. You have correctly derived and solved the differential equation, appropriately used boundary conditions, and found the expression for \( \underline{A} \). Your work is clear and accurate.

Solution 2

**Part a)**

- Correct application of Newton's second law: 1/1

- Correct formulation of the differential equation and general solution: 3/3

- Correct use of boundary conditions to determine the specific solution: 2/2

**Part b)**

- Correct use of initial conditions: 1/1

- Correct determination of constant vector \( \underline{A} \): 1/1

**Total:** 8/8

**Feedback:** 

- Well done. You have provided a detailed and accurate solution to the problem, correctly handling both the differential equation and the initial conditions.

Solution 3

**Part a)**

- Correct application of Newton's second law: 1/1

- Correct formulation of the differential equation and general solution: 3/3

- Correct use of boundary conditions to determine the specific solution: 2/2

**Part b)**

- Correct use of initial conditions: 1/1

- Correct determination of constant vector \( \underline{A} \): 1/1

**Total:** 8/8

**Feedback:** 

- Great job. Your solution is thorough and correct, with clear explanations and accurate use of initial conditions to find the constant vector \( \underline{A} \).

All three solutions correctly applied Newton's second law, solved the differential equation accurately, and appropriately used initial conditions to determine \( \underline{A} \). They all earned full marks for their clarity, accuracy, and adherence to the mark scheme.

\bigskip
\textbf{Response 2:}

Solution 1:

**Marks: 6/8**

**Part a (5/6):** 

- Correctly identified forces and applied Newton's second law.

- Properly derived the differential equation and general solution.

- Correctly applied the boundary condition to find \(\underline{B}\).

**Feedback:**

- Excellent detailed derivation, but the integration of initial conditions could be clearer.

**Part b (1/2):**

- Correctly used initial conditions to find \(\underline{A}\), but explanation could be more concise.

---

Solution 2:

**Marks: 7/8**

**Part a (5/6):** 

- Correctly derived and solved the differential equation.

- Clear and thorough explanation of vertical and horizontal components.

- Correctly applied boundary conditions to find \(\underline{B}\).

**Feedback:**

- Excellent detailed explanation. Could be slightly more concise, especially in handling vertical and horizontal components.

**Part b (2/2):**

- Correctly and concisely used initial conditions to find \(\underline{A}\).

---

Solution 3:

**Marks: 6/8**

**Part a (4/6):**

- Correctly derived and solved the differential equation.

- Clear explanation and verification of the solution.

- Missing explicit boundary condition to determine \(\underline{B}\) clearly.

**Feedback:**

- Good derivation and solution, but needs clearer application of boundary conditions.

**Part b (2/2):**

- Correctly used initial conditions to find \(\underline{A}\).

---

Overall, all three solutions demonstrate a good understanding of the problem, but clarity and completeness of explanation, especially in applying boundary conditions, could be improved.

\bigskip
\textbf{Response 3:}

**Solution 1:**

**Part a):** 

- Correctly states Newton's second law and sets up the differential equation: 1 mark.

- Solves the differential equation and shows the general solution: 3 marks.

- Correctly uses the boundary condition to determine \( \underline{B} \): 2 marks.

**Part b):**

- Uses the initial condition correctly to find \( \underline{A} \): 2 marks.

**Total: 8/8**

**Feedback:**

Excellent solution. All steps are clearly explained, and the correct use of boundary conditions shows a strong understanding of the problem. The general solution and substitution were well executed.

---

**Solution 2:**

**Part a):** 

- Correctly states Newton's second law and sets up the differential equation: 1 mark.

- Solves the differential equation and shows the general solution: 3 marks.

- Correctly uses the boundary condition to determine \( \underline{B} \): 2 marks.

**Part b):**

- Uses the initial condition correctly to find \( \underline{A} \): 2 marks.

**Total: 8/8**

**Feedback:**

This solution is very detailed and thorough. It follows all the necessary steps methodically and reaches the correct conclusion. The approach to solving the differential equation and the application of boundary conditions are correct.

---

**Solution 3:**

**Part a):** 

- Correctly states Newton's second law and sets up the differential equation: 1 mark.

- Solves the differential equation and shows the general solution: 3 marks.

- Correctly uses the boundary condition to determine \( \underline{B} \): 2 marks.

**Part b):**

- Uses the initial condition correctly to find \( \underline{A} \): 2 marks.

**Total: 8/8**

**Feedback:**

Well-done. The explanation is clear, and the steps are logically followed. The solution to the differential equation and the use of boundary conditions are accurately presented.

\bigskip
\textbf{Response 4:}

Here is the grading and feedback for each student's solution according to the given mark scheme:

Solution 1:

**Part a)**: 

1. Correctly identifies and applies Newton's second law: \( m\frac{d\underline{v}}{dt} = - mg \hat{\underline{k}} - b\underline{v} \). [1 mark]

2. Proceeds to solve the differential equation with general solution \( \underline{v}(t) = \underline{C}e^{-bt/m} + \underline{D} \). [2 marks]

3. Uses boundary conditions correctly, but only partially explains how \(\underline{D}\) is determined. [1 mark]

**Part b)**:

1. Correctly uses initial conditions to find \( \underline{A} \). [1 mark]

2. Derives \( \underline{A} = \underline{u} + \frac{mg}{b} \hat{\underline{k}} \) correctly. [1 mark]

**Total**: 6/8

**Feedback**: Good understanding and clear solution process, but the explanation of determining \(\underline{D}\) could be more detailed.

Solution 2:

**Part a)**:

1. Correctly states and applies Newton's second law: \( m\frac{d\underline{v}}{dt} = - mg \hat{\underline{k}} - b\underline{v} \). [1 mark]

2. Attempts to solve the differential equation, but splits into components unnecessarily and does not fully explain the solution process. [1 mark]

3. Correctly uses terminal velocity to identify \(\underline{B}\) but lacks clarity. [1 mark]

**Part b)**:

1. Correctly uses initial conditions to find \( \underline{A} \). [1 mark]

2. Derives \( \underline{A} = \underline{u} + \frac{mg}{b} \hat{\underline{k}} \) correctly. [1 mark]

**Total**: 5/8

**Feedback**: Good start, but overcomplicates the solution by splitting into components and lacks detailed explanation in solving the differential equation.

Solution 3:

**Part a)**:

1. Correctly states and applies Newton's second law: \( m\frac{d\underline{v}}{dt} = - mg \hat{\underline{k}} - b\underline{v} \). [1 mark]

2. Proceeds to solve the differential equation and provides the general solution correctly. [2 marks]

3. Uses boundary conditions correctly to find \(\underline{B}\). [2 marks]

**Part b)**:

1. Correctly uses initial conditions to find \( \underline{A} \). [1 mark]

2. Derives \( \underline{A} = \underline{u} + \frac{mg}{b} \hat{\underline{k}} \) correctly. [1 mark]

**Total**: 7/8

**Feedback**: Clear and detailed solution with proper application of boundary conditions and correct use of the general solution form.

---

Summary:

- Solution 1: 6/8

- Solution 2: 5/8

- Solution 3: 7/8

\bigskip
\textbf{Response 5:}

Solution 1

**Part a:**
- Correctly applies Newton's second law: 1 mark.

- Correctly identifies the forces acting on the ball and forms the differential equation: 1 mark.

- General solution presented but not clearly deriving the form: 2 marks.

- Correctly applies boundary conditions for terminal velocity but lacks clarity: 1 mark.

- Derives the form correctly in the end: 1 mark.
**Total for Part a: 5/6**

**Part b:**

- Correctly uses initial conditions: 1 mark.

- Correctly finds \( \underline{A} \): 1 mark.

**Total for Part b: 2/2**

**Overall Total: 7/8**

**Feedback:**

- Good application of Newton's second law.

- Solution derivation is mostly clear but can be improved for better clarity.

- Correctly handled the initial condition for part b.

Solution 2

**Part a:**

- Correctly applies Newton's second law: 1 mark.

- Correctly identifies the forces acting on the ball and forms the differential equation: 1 mark.

- General solution presented with reasonable explanation: 3 marks.

- Correctly applies boundary conditions for terminal velocity but slightly unclear: 1 mark.

**Total for Part a: 6/6**

**Part b:**

- Correctly uses initial conditions: 1 mark.

- Correctly finds \( \underline{A} \): 1 mark.

**Total for Part b: 2/2**

**Overall Total: 8/8**

**Feedback:**

- Clear and accurate derivation of the solution.

- Good application of boundary conditions and initial conditions.

- Well-structured and easy to follow.

Solution 3

**Part a:**

- Correctly applies Newton's second law: 1 mark.

- Correctly identifies the forces acting on the ball and forms the differential equation: 1 mark.

- General solution presented but slightly unclear: 2 marks.

- Correctly applies boundary conditions for terminal velocity but not clearly explained: 1 mark.

- Final form correctly derived: 1 mark.

**Total for Part a: 6/6**

**Part b:**

- Correctly uses initial conditions: 1 mark.

- Correctly finds \( \underline{A} \): 1 mark.

**Total for Part b: 2/2**

**Overall Total: 8/8**

**Feedback:**

- Good application of Newton's second law.

- Clear step-by-step solution but could be more concise.

- Correctly handled the initial condition for part b.

Summary

- **Solution 1:** 7/8. Clear but could improve in derivation clarity.

- **Solution 2:** 8/8. Excellent, clear, and accurate solution.

- **Solution 3:** 8/8. Detailed and correct, but a bit verbose.


\end{document}