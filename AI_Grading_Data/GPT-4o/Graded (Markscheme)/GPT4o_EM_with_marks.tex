\documentclass[a4paper,11pt]{article}
\setlength\parindent{0pt}
\usepackage{amsmath, amssymb}
\usepackage[margin=1in,letterpaper]
{geometry} % decreases margins

\usepackage{setspace}
\setstretch{1.2}
\usepackage{spverbatim}

\begin{document}
\title{GPT-4o Grading (With Mark Scheme)}
\author{Faraaz Akhtar, Jun Ida, Ryan Mok, Christine Li, Lewis Ross, Cameron Chambers, Louis Clare}
\date{}
\maketitle

This document compiles the GPT-4o responses when prompted to grade the solutions which have been generated. The markscheme is given to GPT-4o in the prompt. This is the grading for the Electromagnetic Theory (EM) questions. 

\section{EM}

\subsection*{Question 1:}

\textbf{Response 1:}

Solution 1

**Marks: 8/8**

**Feedback:**

- The solution accurately names the three types of magnetic materials (1 mark).

- It clearly explains the differing magnetic susceptibilities and their impact on response to a magnetic field (1 mark).

- The descriptions of paramagnetism (2 marks), diamagnetism (2 marks), and ferromagnetism (2 marks) are precise, providing correct details about the magnetic susceptibility, dipole moments, and responses to external magnetic fields.

- Excellent use of examples and detailed explanations of the physical mechanisms behind each type's response.

Solution 2

**Marks: 7/8**

**Feedback:**

- Correctly names the three types of magnetic materials (1 mark).

- Adequate mention of the differing magnetic susceptibilities (1 mark).

- Paramagnetism (2 marks): Well explained with details about magnetic susceptibility and weak attraction.

- Diamagnetism (2 marks): Correct description of repulsion and induced magnetic moment.

- Ferromagnetism (1 mark): Good description, but lacks explicit mention of the magnetic susceptibility being much larger than 1. Although the concept is implied, clarity in this aspect is important.

- Overall, the solution is strong but missed a key detail in the explanation of ferromagnetism.

Solution 3

**Marks: 7/8**

**Feedback:**

- Correctly names the three types of magnetic materials (1 mark).

- Provides an accurate mention of differing magnetic susceptibilities (1 mark).

- Paramagnetism (2 marks): Correctly describes the weak attraction and alignment of dipoles.

- Diamagnetism (2 marks): Clearly explains the induced magnetic field and repulsion.

- Ferromagnetism (1 mark): Good description of alignment and retention of magnetization, but does not explicitly state that the magnetic susceptibility is much larger than 1, which is a crucial detail.

- The explanations are clear and detailed, but the omission of the explicit detail about magnetic susceptibility in ferromagnetic materials costs one mark.

\bigskip    
\textbf{Response 2:}

Grading and Feedback:

Solution 1:

**Marks: 6/8**

**Feedback:**

- **Correct identification of materials (1 mark)**: Correctly named all three types of magnetic materials.

- **Mention of magnetic susceptibility (1 mark)**: Briefly mentions magnetic moments and their alignment, but does not clearly address magnetic susceptibility values.

- **Paramagnetism explanation (2 marks)**: Provides a clear and correct explanation of paramagnetic materials.

- **Diamagnetism explanation (2 marks)**: Describes diamagnetism accurately.

- **Ferromagnetism explanation (0 marks)**: Although the description is good, it lacks specific mention of the large magnetic susceptibility value.

Solution 2:

**Marks: 7/8**

**Feedback:**

- **Correct identification of materials (1 mark)**: Correctly named all three types of magnetic materials.

- **Mention of magnetic susceptibility (1 mark)**: Implies differing strengths of attraction but does not explicitly mention susceptibility values.

- **Paramagnetism explanation (2 marks)**: Accurate description including the alignment of atomic magnetic moments.

- **Diamagnetism explanation (2 marks)**: Correctly explains the induced magnetic moment and repulsion due to Lenz's Law.

- **Ferromagnetism explanation (1 mark)**: Good description, but lacks explicit mention of the magnetic susceptibility value being much larger than 1.

Solution 3:

**Marks: 6/8**

**Feedback:**

- **Correct identification of materials (1 mark)**: Correctly named all three types of magnetic materials.

- **Mention of magnetic susceptibility (1 mark)**: Briefly touches on alignment and response, but does not explicitly address magnetic susceptibility values.

- **Paramagnetism explanation (2 marks)**: Provides a clear and correct explanation of paramagnetic materials.

- **Diamagnetism explanation (2 marks)**: Accurate explanation of induced magnetic moment and repulsion.

- **Ferromagnetism explanation (0 marks)**: Good explanation but lacks mention of the significantly higher magnetic susceptibility value.

\bigskip    
\textbf{Response 3:}

Solution 1 Evaluation

**Marks Awarded: 6/8**

**Feedback:**

1. **Naming Types (1/1):** Correctly names the three types of magnetic materials.

2. **Magnetic Susceptibility/Magnetic Dipole Moment (0/1):** Does not explicitly mention magnetic susceptibility or magnetic dipole moment values.

3. **Paramagnetism (1/2):** Describes paramagnetism well, but lacks specific mention of magnetic susceptibility values.

4. **Diamagnetism (1/2):** Explains diamagnetism correctly, including induced magnetic moment opposing the external field, but does not mention the exact magnetic susceptibility value.

5. **Ferromagnetism (2/2):** Correctly explains ferromagnetism, including strong attraction and retention of magnetization, as well as alignment of magnetic domains.

Solution 2 Evaluation

**Marks Awarded: 7/8**

**Feedback:**

1. **Naming Types (1/1):** Correctly names the three types of magnetic materials.

2. **Magnetic Susceptibility/Magnetic Dipole Moment (1/1):** Mentions magnetic susceptibility through the behavior of magnetic moments.

3. **Paramagnetism (2/2):** Provides a clear explanation, including weak attraction and the effect of thermal motion.

4. **Diamagnetism (2/2):** Correctly describes diamagnetism, including the induced magnetic moment opposing the external field and mentions Lenz's Law.

5. **Ferromagnetism (1/2):** Good explanation but lacks explicit mention of the large magnetic susceptibility value.

Solution 3 Evaluation

**Marks Awarded: 6/8**

**Feedback:**

1. **Naming Types (1/1):** Correctly names the three types of magnetic materials.

2. **Magnetic Susceptibility/Magnetic Dipole Moment (0/1):** Does not explicitly mention magnetic susceptibility or magnetic dipole moment values.

3. **Paramagnetism (2/2):** Describes paramagnetism well, including weak response and alignment of atomic dipoles.

4. **Diamagnetism (1/2):** Correctly explains diamagnetism and mentions Lenz's Law but lacks explicit mention of the exact magnetic susceptibility value.

5. **Ferromagnetism (2/2):** Provides a good explanation of ferromagnetism, including strong attraction and retention of magnetization.

\bigskip    
\textbf{Response 4:}

Solution 1:

**Marks: 7/8**

**Feedback:**

- Correctly names the three types of magnetic materials. (1 mark)

- Clearly mentions the different magnetic susceptibilities and responses for each type. (1 mark)

- Correctly describes paramagnetic materials, but lacks specific mention of susceptibility being positive and much less than 1. (1.5/2 marks)

- Correctly describes diamagnetic materials, but does not explicitly state the susceptibility is negative and magnitude is much less than 1. (1.5/2 marks)

- Correctly describes ferromagnetic materials with appropriate details about strong attraction and domain alignment. (2/2 marks)

Solution 2:

**Marks: 8/8**

**Feedback:**

- Correctly names the three types of magnetic materials. (1 mark)

- Mentions the different magnetic susceptibilities and responses for each type. (1 mark)

- Provides a clear and accurate explanation of paramagnetic materials, including the alignment of atomic magnetic moments and the weaker attraction. (2/2 marks)

- Clearly explains diamagnetic materials with reference to Lenz's Law and induced magnetic moment opposing the field. (2/2 marks)

- Gives a thorough explanation of ferromagnetic materials, including domain alignment and strong attraction. (2/2 marks)

Solution 3:

**Marks: 6/8**

**Feedback:**

- Correctly names the three types of magnetic materials. (1 mark)

- Does not explicitly mention the different magnetic susceptibilities for each type. (0 mark)

- Provides a good description of paramagnetic materials but lacks mention of susceptibility values. (1.5/2 marks)

- Describes diamagnetic materials well, including Lenz's Law, but omits specific details on susceptibility values. (1.5/2 marks)

- Accurately describes ferromagnetic materials and their strong magnetic response. (2/2 marks)

Summary:

- Solution 1: Good overall but lacks some specifics on susceptibility values.

- Solution 2: Complete and thorough, covers all required details.

- Solution 3: Accurate descriptions but missing explicit mention of susceptibility values.

\bigskip    
\textbf{Response 5:}

Solution 1:

**Marks: 6/8**

**Feedback:**

- Correctly names the three types of magnetic materials (1 mark).

- Explanation for ferromagnetic materials is detailed and accurate (2 marks).

- Explanation for paramagnetic materials is accurate but lacks explicit mention of magnetic susceptibility being much less than 1 (1 mark).

- Explanation for diamagnetic materials is mostly correct but lacks the explicit mention of magnetic susceptibility being negative (1 mark).

- The discussion on what determines their response is detailed and accurate, mentioning the arrangement and orientation of electrons (1 mark).

Solution 2:

**Marks: 7/8**

**Feedback:**

- Correctly names the three types of magnetic materials (1 mark).

- Explanation for ferromagnetic materials is detailed and accurate (2 marks).

- Explanation for paramagnetic materials is detailed and accurate, correctly mentioning weak attraction and no retention of magnetism (2 marks).

- Explanation for diamagnetic materials is correct but does not explicitly mention the magnitude of magnetic susceptibility being much less than 1 (1 mark).

- The discussion on what determines their response is detailed and accurate (1 mark).

Solution 3:

**Marks: 6/8**

**Feedback:**

- Correctly names the three types of magnetic materials (1 mark).

- Explanation for ferromagnetic materials is detailed and accurate (2 marks).

- Explanation for paramagnetic materials is accurate but lacks explicit mention of magnetic susceptibility being much less than 1 (1 mark).

- Explanation for diamagnetic materials is mostly correct but lacks the explicit mention of magnetic susceptibility being negative and its magnitude being much less than 1 (1 mark).

- The discussion on what determines their response is accurate but could be more detailed regarding the quantitative aspects of magnetic susceptibility (1 mark).

\subsection*{Question 2:}

\textbf{Response 1:}

Solution 1

**Grade: 5/5**

**Feedback:**

- **Definition of Free Space (1/1):** Correctly defines free space as a vacuum devoid of matter, charge, or current.

- **Wave Equation Description (1/1):** Accurately describes the given equation as a wave equation for the electric field.

- **Physical Interpretation and Speed of Light (3/3):** Clearly explains the propagation of electric fields as electromagnetic waves. Correctly derives the speed of light using the relationship \( c = \frac{1}{\sqrt{\epsilon_{0}\mu_{0}}} \) and provides the correct value \( c = 3 \times 10^8 \, \text{m/s} \).

Solution 2

**Grade: 5/5**

**Feedback:**

- **Definition of Free Space (1/1):** Properly defines free space as a perfect vacuum with no matter or charges.

- **Wave Equation Description (1/1):** Correctly identifies the equation as a wave equation for the electric field.

- **Physical Interpretation and Speed of Light (3/3):** Provides a clear explanation of electromagnetic wave propagation and correctly derives the speed of light, \( c = \frac{1}{\sqrt{\epsilon_{0}\mu_{0}}} \), and states the correct value \( c = 3 \times 10^8 \, \text{m/s} \).

Solution 3

**Grade: 5/5**

**Feedback:**

- **Definition of Free Space (1/1):** Accurately defines free space as a perfect vacuum without matter or charges.

- **Wave Equation Description (1/1):** Correctly describes the given equation as a wave equation for the electric field.

- **Physical Interpretation and Speed of Light (3/3):** Clearly explains the propagation of electric fields as electromagnetic waves. Correctly derives the speed of light using \( c = \frac{1}{\sqrt{\epsilon_{0}\mu_{0}}} \) and provides the correct value \( c = 3 \times 10^8 \, \text{m/s} \).

---

All three solutions provide a thorough and accurate explanation of free space, the wave equation, and the derivation of the speed of light, deserving full marks according to the mark scheme.

\bigskip    
\textbf{Response 2:}

Grading and Feedback

Solution 1

**Marks: 5/5**

**Feedback:**

- The definition of free space is accurate and precise, clearly stating it as a vacuum without matter, free charge, or free current.

- The explanation of the wave equation for the electric field is thorough, correctly identifying it as a wave equation and interpreting the terms.

- The method to derive the speed of light from the equation is detailed and accurate, including the correct formula and values for \(\epsilon_{0}\) and \(\mu_{0}\), leading to the correct value of \(c\).

Solution 2

**Marks: 4/5**

**Feedback:**

- The definition of free space is correct but could benefit from a slightly more explicit mention of "no free current".

- The explanation of the wave equation is solid and clearly states its interpretation.

- The method to derive the speed of light is accurate and well-explained. However, the solution lacks the explicit calculation of \(\epsilon_{0}\) and \(\mu_{0}\), which would have strengthened the answer.

Solution 3

**Marks: 4/5**

**Feedback:**

- The definition of free space is accurate, mentioning the absence of matter or charges.

- The explanation of the wave equation is clear and concise, correctly interpreting the physical meaning.

- The method to derive the speed of light is correct, but similar to Solution 2, it lacks the explicit calculation details for \(\epsilon_{0}\) and \(\mu_{0}\). Including these values would have made the solution more complete.

\bigskip    
\textbf{Response 3:}

Solution 1

**Marks: 5/5**

**Feedback:**

- The definition of free space as a vacuum devoid of matter is clear and accurate.

- The explanation of the wave equation for the electric field is comprehensive, correctly identifying the roles of the Laplacian and time derivatives.

- The description of wave propagation and the connection to Maxwell's equations is thorough.

- The derivation of the speed of light and the calculation is accurate and well-presented.

- Overall, this solution covers all required points with appropriate detail and precision.

Solution 2

**Marks: 5/5**

**Feedback:**

- The definition of free space as a perfect vacuum is precise and well-articulated.

- The identification of the wave equation and its components (Laplacian and time derivatives) is correct.

- The explanation of electromagnetic wave propagation is clear.

- The derivation of the speed of light and its value is accurately given, with correct use of constants.

- The solution meets all criteria of the mark scheme and is well-explained.

Solution 3

**Marks: 5/5**

**Feedback:**

- The definition of free space as a perfect vacuum without matter is correct.

- The explanation of the wave equation and its components is clear and accurate.

- The description of wave propagation is concise and correct.

- The derivation of the speed of light and the final value is correctly stated.

- The solution addresses all parts of the problem and provides a clear, accurate explanation.

\bigskip    
\textbf{Response 4:}

Solution 1

**Marks: 5/5**

**Feedback:**

- **Free Space Definition (1/1):** Correctly defines free space as a vacuum without matter, free charge, or current.

- **Wave Equation Explanation (1/1):** Clearly identifies the given equation as a wave equation for the electric field.

- **Speed of Light Derivation (3/3):** Correctly interprets the physical meaning of the equation, derives the speed of light formula, and provides the correct value with appropriate units and precision.

Solution 2

**Marks: 4/5**

**Feedback:**

- **Free Space Definition (1/1):** Correctly defines free space as a vacuum without matter or charges.

- **Wave Equation Explanation (1/1):** Identifies the given equation as a wave equation for the electric field.

- **Speed of Light Derivation (2/3):** Explains the derivation of the speed of light from the wave equation correctly but does not explicitly mention that light is an electromagnetic wave or partially formed by electric field waves. The speed of light value is correct with appropriate units.

Solution 3

**Marks: 5/5**

**Feedback:**

- **Free Space Definition (1/1):** Correctly defines free space as a vacuum without matter or charges.

- **Wave Equation Explanation (1/1):** Clearly identifies the given equation as a wave equation for the electric field.

- **Speed of Light Derivation (3/3):** Correctly interprets the physical meaning of the equation, derives the speed of light formula, and provides the correct value with appropriate units and precision.

Summary

All solutions correctly define free space and identify the given equation as a wave equation. Solutions 1 and 3 provide a complete and detailed explanation of the derivation of the speed of light, including the context of light being an electromagnetic wave. Solution 2, while correct, misses the explicit mention of light as an electromagnetic wave, which slightly reduces its score.

\bigskip    
\textbf{Response 5:}

Grading and Feedback

Solution 1

**Mark: 5/5**

**Feedback:**

- **Free space definition:** Correctly defines free space as a vacuum devoid of matter and charges. (1 mark)

- **Wave equation description:** Clearly explains the equation as a wave equation for electric fields. (1 mark)

- **Physical interpretation and speed of light:** Thoroughly discusses the propagation of electric fields as waves and derives the speed of light correctly, including the correct value of \(c\). (3 marks)

Solution 2

**Mark: 5/5**

**Feedback:**

- **Free space definition:** Accurately describes free space as a perfect vacuum with no matter or charges. (1 mark)

- **Wave equation description:** Properly identifies the given equation as the wave equation for the electric field. (1 mark)

- **Physical interpretation and speed of light:** Provides a clear explanation of electromagnetic wave propagation, correctly derives the speed of light, and states its correct value. (3 marks)

Solution 3

**Mark: 5/5**

**Feedback:**

- **Free space definition:** Correctly describes free space as a perfect vacuum without matter or charges. (1 mark)

- **Wave equation description:** Accurately identifies the given equation as the wave equation for the electric field. (1 mark)

- **Physical interpretation and speed of light:** Clearly explains the propagation of electric fields as waves, correctly derives the speed of light, and states its correct value. (3 marks)

All three solutions demonstrate a comprehensive understanding of the problem and fulfill the criteria set by the mark scheme effectively.

\subsection*{Question 3:}

\textbf{Response 1:}

Let's begin by evaluating each student's solution based on the provided mark scheme. Here's the breakdown and assessment:

Solution 1

Breakdown:

1. **Use of Coulomb's Law**: The solution begins with the use of Coulomb's law to integrate and find the electric field. (1 mark)

2. **Integral Setup for \(E_x\)**: The solution does not explicitly show the setup for the \(E_x\) component. (0 marks)

3. **Integral Setup for \(E_z\)**: The setup for the \(E_z\) component is mentioned but not in the correct detailed form. (1 mark)

4. **Evaluation of Integral for \(E_x\)**: The final expression for \(E_x\) is not provided. (0 marks)

5. **Evaluation of Integral for \(E_z\)**: The final expression for \(E_z\) is not provided. (0 marks)

6. **Total Expression of \( \underline{E} \)**: The solution provides an expression for the electric field but in a different form and with incorrect numerical constants. (0 marks)

7. **Simplification for \( d \gg L \)**: The simplification is provided, but the constants and expression format are incorrect. (1 mark)

Marks Awarded: 3/14

Feedback:

- Your solution includes the use of Coulomb's law correctly but lacks detailed integral setups for the \(E_x\) and \(E_z\) components. 

- The numerical constants in your final expression are incorrect, and the provided simplification does not match the expected form.

Solution 2

Breakdown:

1. **Use of Coulomb's Law**: The solution correctly uses Coulomb's law to integrate and find the electric field. (1 mark)

2. **Integral Setup for \(E_x\)**: The solution does not explicitly show the setup for the \(E_x\) component. (0 marks)

3. **Integral Setup for \(E_z\)**: The setup for the \(E_z\) component is correct and detailed. (3 marks)

4. **Evaluation of Integral for \(E_x\)**: The final expression for \(E_x\) is not provided. (0 marks)

5. **Evaluation of Integral for \(E_z\)**: The final expression for \(E_z\) is not provided. (0 marks)

6. **Total Expression of \( \underline{E} \)**: The solution provides an expression for the electric field but in a different form and with incorrect numerical constants. (0 marks)

7. **Simplification for \( d \gg L \)**: The simplification is provided, but the constants and expression format are incorrect. (1 mark)

Marks Awarded: 5/14

Feedback:

- Your solution includes the use of Coulomb's law correctly and the detailed integral setup for \(E_z\). 

- However, the final expressions and constants in your solution do not match the expected form. The \(E_x\) component setup and evaluation are missing.

Solution 3

Breakdown:

1. **Use of Coulomb's Law**: The solution correctly uses Coulomb's law to integrate and find the electric field. (1 mark)

2. **Integral Setup for \(E_x\)**: The solution does not explicitly show the setup for the \(E_x\) component. (0 marks)

3. **Integral Setup for \(E_z\)**: The setup for the \(E_z\) component is correct and detailed. (3 marks)

4. **Evaluation of Integral for \(E_x\)**: The final expression for \(E_x\) is not provided. (0 marks)

5. **Evaluation of Integral for \(E_z\)**: The final expression for \(E_z\) is not provided. (0 marks)

6. **Total Expression of \( \underline{E} \)**: The solution provides an expression for the electric field but in a different form and with incorrect numerical constants. (0 marks)

7. **Simplification for \( d \gg L \)**: The simplification is provided, but the constants and expression format are incorrect. (1 mark)

Marks Awarded: 5/14

Feedback:

- Your solution includes the correct use of Coulomb's law and a detailed integral setup for \(E_z\). 

- The final expressions and constants do not match the expected form. The \(E_x\) component setup and evaluation are missing.

Summary

- Solution 1: 3/14

- Solution 2: 5/14

- Solution 3: 5/14

All solutions have a good start with the use of Coulomb's law but fail to provide correct and complete integral setups and evaluations for both components. Additionally, the constants in the final expressions are incorrect. More attention to the detailed setup and evaluation of each component is needed for full credit.

\bigskip
\textbf{Response 2:}

Here are the grades and feedback for each solution based on the provided mark scheme:

Solution 1:

**Grade: 6/14**

**Feedback:**

1. **Use of Coulomb's Law (1 mark):** Correctly used. \( dq = \lambda dx \).

2. **Integral Setup \( E_x \) (0/3 marks):** Did not explicitly set up the integral for \( E_x \).

3. **Integral Setup \( E_z \) (3/3 marks):** Correct setup for \( E_z \).

4. **Integral Evaluation \( E_x \) (0/2 marks):** Did not evaluate \( E_x \).

5. **Integral Evaluation \( E_z \) (2/2 marks):** Correct evaluation for \( E_z \).

6. **Total Expression \( \underline{E} \) (0/1 mark):** Did not provide the total expression for \( \underline{E} \).

7. **Simplification for \( d \gg L \) (0/2 marks):** Did not correctly simplify \( \underline{E} \) for \( d \gg L \).

8. **Physical Interpretation (0/2 marks):** Correct physical interpretation, but no simplification shown.

Solution 2:

**Grade: 7/14**

**Feedback:**

1. **Use of Coulomb's Law (1 mark):** Correctly used. \( dq = \lambda dl \).

2. **Integral Setup \( E_x \) (0/3 marks):** Did not explicitly set up the integral for \( E_x \).
3. **Integral Setup \( E_z \) (3/3 marks):** Correct setup for \( E_z \).

4. **Integral Evaluation \( E_x \) (0/2 marks):** Did not evaluate \( E_x \).

5. **Integral Evaluation \( E_z \) (2/2 marks):** Correct evaluation for \( E_z \).

6. **Total Expression \( \underline{E} \) (0/1 mark):** Did not provide the total expression for \( \underline{E} \).

7. **Simplification for \( d \gg L \) (1/2 marks):** Correctly simplified, but did not mention point charge explicitly.

8. **Physical Interpretation (0/2 marks):** Did not explicitly mention the interpretation as a point charge.

Solution 3:

**Grade: 6/14**

**Feedback:**

1. **Use of Coulomb's Law (1 mark):** Correctly used. \( dq = \lambda dx \).

2. **Integral Setup \( E_x \) (0/3 marks):** Did not explicitly set up the integral for \( E_x \).

3. **Integral Setup \( E_z \) (3/3 marks):** Correct setup for \( E_z \).

4. **Integral Evaluation \( E_x \) (0/2 marks):** Did not evaluate \( E_x \).

5. **Integral Evaluation \( E_z \) (2/2 marks):** Correct evaluation for \( E_z \).

6. **Total Expression \( \underline{E} \) (0/1 mark):** Did not provide the total expression for \( \underline{E} \).

7. **Simplification for \( d \gg L \) (0/2 marks):** Did not simplify correctly; unnecessary constants included.

8. **Physical Interpretation (0/2 marks):** Did not explicitly mention the interpretation as a point charge.

**Summary:**

- All solutions correctly used Coulomb's Law.

- None of the solutions set up or evaluated the \( E_x \) component.

- Only Solution 2 partially addressed the simplification for \( d \gg L \) but lacked the explicit point charge interpretation.

- Each solution could be improved by addressing both \( E_x \) and \( E_z \) integrals, providing the complete expression for \( \underline{E} \), and clearly discussing the physical interpretation for \( d \gg L \).

\bigskip
\textbf{Response 3:}

Let's evaluate each solution based on the provided mark scheme and the given problem statement.

Solution 1 Evaluation

**Score: 9/14**

**Feedback:**

1. The solution correctly uses Coulomb's law and sets up the integrals for finding the electric field (1 mark for using Coulomb's law).

2. It fails to clearly differentiate between the x and z components of the electric field in the integral setup (0 marks for integral setups of \( E_x \) and \( E_z \)).

3. The integral evaluation is not explicitly shown, but the final electric field expression is given, albeit in a non-standard form (1 mark for partially correct integral evaluation for \( E_x \) and 1 mark for partially correct evaluation for \( E_z \)).

4. The total expression for the electric field is given but is not in the correct form (0 marks for total expression of \(\mathbf{E}\)).

5. The solution correctly simplifies the electric field in the limit \( d \gg L \) but does not mention the physical interpretation explicitly (1 mark for simplification, 0 marks for physical interpretation).

Solution 2 Evaluation

**Score: 9/14**

**Feedback:**

1. The solution correctly uses Coulomb's law and sets up the integral for the vertical component (1 mark for using Coulomb's law).

2. The integral setup is clear but not correctly separated into \( E_x \) and \( E_z \) components (0 marks for integral setups of \( E_x \) and \( E_z \)).

3. The integral evaluation is shown and appears correct, though not standard (1 mark for integral evaluation for \( E_x \) and 1 mark for integral evaluation for \( E_z \)).

4. The final expression for the electric field is given in a non-standard form (0 marks for the total expression of \(\mathbf{E}\)).

5. The simplification in the limit \( d \gg L \) is correctly performed, but the physical interpretation is only briefly mentioned (1 mark for simplification, 1 mark for physical interpretation).

Solution 3 Evaluation

**Score: 7/14**

**Feedback:**

1. The solution correctly uses Coulomb's law and sets up the integrals (1 mark for using Coulomb's law).

2. The integral setup is shown but lacks clear separation of \( E_x \) and \( E_z \) components (0 marks for integral setups of \( E_x \) and \( E_z \)).

3. The evaluation of the integral is partially correct but in a non-standard form and seems inconsistent (1 mark for partial integral evaluation for \( E_x \) and 1 mark for partial evaluation for \( E_z \)).

4. The total expression for the electric field is given in a different form, not matching the expected result (0 marks for total expression of \(\mathbf{E}\)).

5. The simplification for \( d \gg L \) is correct, but the physical interpretation is not clearly stated (1 mark for simplification, 1 mark for physical interpretation).

---

Summary of Marks:

**Solution 1: 9/14**

**Solution 2: 9/14**

**Solution 3: 7/14**

Each solution demonstrates a reasonable understanding of the problem but falls short in correctly setting up and solving the integrals in the expected form. The feedback highlights the strengths and areas for improvement in each student's approach to solving the electric field problem.

\bigskip
\textbf{Response 4:}

Let's evaluate each solution based on the provided mark scheme. 

Solution 1

**Marking:**

1. **Use of Coulomb's law to integrate and find electric field (1 mark)**

   - Correctly used Coulomb's law.
   
   - **1/1 mark**

2. **Correct integral setup to evaluate \(E_x\) (3 marks)**

   - Did not explicitly set up the integral for \(E_x\).
   
   - **0/3 marks**

3. **Correct integral setup to evaluate \(E_z\) (3 marks)**

   - Set up integral, but \(E_z\) is described vaguely.
   
   - **1/3 marks**

4. **Correct evaluation of integral to get \(E_x\) (2 marks)**

   - No explicit evaluation of \(E_x\).
   
   - **0/2 marks**

5. **Correct evaluation of integral to get \(E_z\) (2 marks)**

   - Did not evaluate the integral for \(E_z\).
   
   - **0/2 marks**

6. **Total expression of \(\underline{E}\) (1 mark)**

   - Provided an expression, but it is incorrect and non-standard.
   
   - **0/1 mark**

7. **Correct simplification in \(\underline{E}\) for \(d \gg L\) (2 marks)**

   - Attempted simplification but incorrect.
   
   - **0/2 marks**

**Total: 2/14 marks**

**Feedback:** 

The solution shows understanding of the problem's physical setup and correctly applies Coulomb's law, but fails to set up and evaluate the integrals correctly. The provided expressions are non-standard and incorrectly simplified. More attention to detailed integral setup and correct evaluation is needed.

Solution 2

**Marking:**

1. **Use of Coulomb's law to integrate and find electric field (1 mark)**

   - Correctly used Coulomb's law.
   
   - **1/1 mark**

2. **Correct integral setup to evaluate \(E_x\) (3 marks)**

   - Did not explicitly set up the integral for \(E_x\).
   
   - **0/3 marks**

3. **Correct integral setup to evaluate \(E_z\) (3 marks)**

   - Integral setup for \(E_z\) is correct.
   
   - **3/3 marks**

4. **Correct evaluation of integral to get \(E_x\) (2 marks)**

   - No explicit evaluation of \(E_x\).
   
   - **0/2 marks**

5. **Correct evaluation of integral to get \(E_z\) (2 marks)**

   - Evaluated the integral for \(E_z\).
   
   - **2/2 marks**

6. **Total expression of \(\underline{E}\) (1 mark)**

   - Provided an expression, but incorrect and non-standard.
   
   - **0/1 mark**

7. **Correct simplification in \(\underline{E}\) for \(d \gg L\) (2 marks)**

   - Attempted simplification but incorrect.
   
   - **0/2 marks**

**Total: 6/14 marks**

**Feedback:**

The solution correctly sets up and evaluates the integral for the vertical component of the electric field. However, it does not address the horizontal component and fails to provide a correct and standard expression for the total field. Additionally, the simplification for \(d \gg L\) is incorrect. Focusing on the complete and correct integral setup and accurate simplification is needed.

Solution 3

**Marking:**

1. **Use of Coulomb's law to integrate and find electric field (1 mark)**

   - Correctly used Coulomb's law.
   
   - **1/1 mark**

2. **Correct integral setup to evaluate \(E_x\) (3 marks)**

   - Did not explicitly set up the integral for \(E_x\).
   
   - **0/3 marks**

3. **Correct integral setup to evaluate \(E_z\) (3 marks)**

   - Integral setup for \(E_z\) is correct.
   
   - **3/3 marks**

4. **Correct evaluation of integral to get \(E_x\) (2 marks)**

   - No explicit evaluation of \(E_x\).
   
   - **0/2 marks**

5. **Correct evaluation of integral to get \(E_z\) (2 marks)**

   - Evaluated the integral for \(E_z\).
   
   - **2/2 marks**

6. **Total expression of \(\underline{E}\) (1 mark)**

   - Provided an expression, but incorrect and non-standard.
   
   - **0/1 mark**

7. **Correct simplification in \(\underline{E}\) for \(d \gg L\) (2 marks)**

   - Attempted simplification but incorrect.
   
   - **0/2 marks**

**Total: 6/14 marks**

**Feedback:**

The solution correctly sets up and evaluates the integral for the vertical component of the electric field. However, it does not address the horizontal component and fails to provide a correct and standard expression for the total field. The simplification for \(d \gg L\) is also incorrect. More detailed and correct integral setup and accurate simplification is necessary.

Overall, Solution 2 and Solution 3 correctly handle the vertical component of the electric field but miss out on the horizontal component and the accurate simplification for \(d \gg L\). Solution 1 needs improvement in setting up and evaluating the integrals correctly.

\bigskip
\textbf{Response 5:}

Let's evaluate the provided solutions using the mark scheme. Here's the concise feedback and marks for each solution.

**Solution 1:**

1. **Use of Coulomb's law to integrate and find electric field:**

   - Correctly used Coulomb's law.
   
   - **Marks: 1/1**

2. **Correct integral setup to evaluate x-component \(E_x\):**

   - The solution did not set up the integral for \(E_x\).
   
   - **Marks: 0/3**

3. **Correct integral setup to evaluate z-component \(E_z\):**

   - The setup for \(E_z\) is not explicitly given but implied in the integration.
   
   - **Marks: 2/3**

4. **Correct evaluation of integral to get \(E_x\):**

   - \(E_x\) evaluation is missing.
   
   - **Marks: 0/2**

5. **Correct evaluation of integral to get \(E_z\):**

   - \(E_z\) evaluated correctly.
   
   - **Marks: 2/2**

6. **Total expression of \(\underline{E}\):**

   - Total expression for \( \underline{E} \) not fully provided.
   
   - **Marks: 0/1**

7. **Correct simplification in \(\underline{E}\) for \( d \gg L \):**

   - Correct simplification and physical interpretation given.
   
   - **Marks: 2/2**

   **Total Marks: 7/14**

**General Feedback:** The solution correctly uses Coulomb's law and integrates for the vertical component. However, it lacks the setup and evaluation for the x-component of the electric field, and the total expression for \(\underline{E}\) is not provided.

---

**Solution 2:**

1. **Use of Coulomb's law to integrate and find electric field:**

   - Correctly used Coulomb's law.
   
   - **Marks: 1/1**

2. **Correct integral setup to evaluate x-component \(E_x\):**

   - The solution did not set up the integral for \(E_x\).
   
   - **Marks: 0/3**

3. **Correct integral setup to evaluate z-component \(E_z\):**

   - The setup for \(E_z\) is not explicitly given but implied in the integration.
   
   - **Marks: 2/3**

4. **Correct evaluation of integral to get \(E_x\):**

   - \(E_x\) evaluation is missing.
   
   - **Marks: 0/2**

5. **Correct evaluation of integral to get \(E_z\):**

   - \(E_z\) evaluated correctly.
   
   - **Marks: 2/2**

6. **Total expression of \(\underline{E}\):**

   - Total expression for \( \underline{E} \) not fully provided.
   
   - **Marks: 0/1**

7. **Correct simplification in \(\underline{E}\) for \( d \gg L \):**

   - Correct simplification and physical interpretation given.
   
   - **Marks: 2/2**

   **Total Marks: 7/14**

**General Feedback:** The solution correctly uses Coulomb's law and integrates for the vertical component. However, it lacks the setup and evaluation for the x-component of the electric field, and the total expression for \(\underline{E}\) is not provided.

---

**Solution 3:**

1. **Use of Coulomb's law to integrate and find electric field:**

   - Correctly used Coulomb's law.
   
   - **Marks: 1/1**

2. **Correct integral setup to evaluate x-component \(E_x\):**

   - The solution did not set up the integral for \(E_x\).
   
   - **Marks: 0/3**

3. **Correct integral setup to evaluate z-component \(E_z\):**

   - The setup for \(E_z\) is not explicitly given but implied in the integration.
   
   - **Marks: 2/3**

4. **Correct evaluation of integral to get \(E_x\):**

   - \(E_x\) evaluation is missing.
   
   - **Marks: 0/2**

5. **Correct evaluation of integral to get \(E_z\):**

   - \(E_z\) evaluation is not in standard form but correct conceptually.
   
   - **Marks: 1/2**

6. **Total expression of \(\underline{E}\):**

   - Total expression for \( \underline{E} \) not fully provided.
   
   - **Marks: 0/1**

7. **Correct simplification in \(\underline{E}\) for \( d \gg L \):**

   - Correct simplification and physical interpretation given.
   
   - **Marks: 2/2**

   **Total Marks: 6/14**

**General Feedback:** The solution correctly uses Coulomb's law and integrates for the vertical component. However, it lacks the setup and evaluation for the x-component of the electric field, and the total expression for \(\underline{E}\) is not provided. The evaluation for \(E_z\) is not in the standard form but conceptually correct.

---

If you need further details or additional feedback, please let me know.


\subsection*{Question 4:}

\textbf{Response 1:}

Solution 1:

Part (a)

Marks: 5/5

- [1 mark] Use of definition of flux as surface integral: Correctly used the definition of flux.

- [3 marks] Correct setup of integral: Set up the integral correctly using the magnetic field for an infinite long wire.

- [1 mark] Correct evaluation: Correctly evaluated the integral to get the flux.

Part (b)

Marks: 4/5

- [1 mark] Use of emf definition: Correctly used the definition of emf.

- [2 marks] Correct evaluation of the derivative: The evaluation of the derivative is mostly correct, though the final form of the emf could be simplified.

- [1 mark] Current direction: Correctly identified the direction of the current as counterclockwise.

Part (c)

Marks: 1/1

- Correctly identified that no emf is induced if the loop only moves downwards.

**Total: 10/11**

**Feedback:**

- Excellent use of integral setup and evaluation for the flux.

- Good use of Faraday’s law and understanding of the direction of induced current.

- Clear and precise explanations for each step.

---

Solution 2:

Part (a)

Marks: 3/5

- [1 mark] Use of definition of flux as surface integral: Correctly used the definition of flux.

- [3 marks] Correct setup of integral: The setup is somewhat confused and deviates from the required integral form.

- [1 mark] Correct evaluation: Incorrect evaluation of the integral, the final flux expression is incorrect.

Part (b)

Marks: 2/5

- [1 mark] Use of emf definition: Correctly used the definition of emf.

- [2 marks] Correct evaluation of the derivative: Incorrect derivative evaluation and emf expression.

- [1 mark] Current direction: The direction of the current is not clearly justified.

Part (c)

Marks: 1/1

- Correctly identified that no emf is induced if the loop only moves downwards.

**Total: 6/11**

**Feedback:**

- The integral setup for part (a) needs to be more precise.

- The evaluation of the derivative and emf needs correction.

- The explanation lacks clarity and detail.

---

Solution 3:

Part (a)

Marks: 5/5

- [1 mark] Use of definition of flux as surface integral: Correctly used the definition of flux.

- [3 marks] Correct setup of integral: Set up the integral correctly using the magnetic field for an infinite long wire.

- [1 mark] Correct evaluation: Correctly evaluated the integral to get the flux.

Part (b)

Marks: 4/5

- [1 mark] Use of emf definition: Correctly used the definition of emf.

- [2 marks] Correct evaluation of the derivative: The evaluation is correct, though simplification could be better.

- [1 mark] Current direction: Correctly identified the direction of the current as counterclockwise.

Part (c)

Marks: 1/1

- Correctly identified that no emf is induced if the loop only moves downwards.

**Total: 10/11**

**Feedback:**

- Clear and correct integral setup and evaluation for part (a).

- Good understanding and application of Faraday’s law in part (b).

- Clear justification for the direction of the induced current.

---

Summary:

- **Solution 1:** Excellent overall, with clear explanations and correct evaluations. Total: 10/11

- **Solution 2:** Needs improvement in the integral setup and evaluation of the emf. Total: 6/11

- **Solution 3:** Strong solution with minor issues in simplification. Total: 10/11

\bigskip
\textbf{Response 2:}

Let's proceed with grading the provided solutions based on the given mark scheme. Here is the grading and feedback for each solution:

Solution 1:

**Part a):**

- [1 mark] Correct use of the definition of flux as a surface integral.

- [3 marks] Correct setup of the integral for flux using the magnetic field for an infinitely long wire.

- [1 mark] Correct evaluation of the integral to get the flux.

**Marks for Part a): 5/5**

**Part b):**

- [1 mark] Correct use of the definition of emf.

- [1 mark] Partially correct evaluation of the derivative to get the emf but there's a mistake in the differentiation steps.

- [2 marks] Correct identification of the current direction as counterclockwise using Lenz's law.

**Marks for Part b): 4/5**

**Part c):**

- [1 mark] Correctly identifies that no emf is generated since the flux does not change.

**Marks for Part c): 1/1**

**Total Marks for Solution 1: 10/11**

**Feedback:**

- Part a) was excellently solved with all steps clearly explained.

- Part b) had an error in the differentiation process. Ensure proper differentiation to avoid mistakes.

- Part c) was correctly answered with a clear explanation.

---

Solution 2:

**Part a):**

- [1 mark] Correct use of the definition of flux as a surface integral.

- [2 marks] Setup of the integral is partially correct, but the integral evaluation is mistaken.

- [0 marks] Incorrect evaluation of the integral to get the flux; final expression is incorrect.

**Marks for Part a): 3/5**

**Part b):**

- [1 mark] Correct use of the definition of emf.

- [0 marks] Incorrect evaluation of the derivative; asserts no emf without proper derivation.

- [0 marks] No correct justification for the current direction.

**Marks for Part b): 1/5**

**Part c):**

- [1 mark] Correctly identifies that no emf is generated since the flux does not change.

**Marks for Part c): 1/1**

**Total Marks for Solution 2: 5/11**

**Feedback:**

- Part a) had a good start but the integral evaluation was incorrect.

- Part b) was not correctly evaluated; ensure to show derivation steps to avoid incorrect conclusions.

- Part c) was correctly answered with a clear explanation.

---

Solution 3:

**Part a):**

- [1 mark] Correct use of the definition of flux as a surface integral.

- [3 marks] Correct setup of the integral for flux using the magnetic field for an infinitely long wire.

- [1 mark] Correct evaluation of the integral to get the flux.

**Marks for Part a): 5/5**

**Part b):**

- [1 mark] Correct use of the definition of emf.

- [2 marks] Correct evaluation of the derivative to get the emf.

- [2 marks] Correct identification of the current direction as counterclockwise using Lenz's law.

**Marks for Part b): 5/5**

**Part c):**

- [1 mark] Correctly identifies that no emf is generated since the flux does not change.

**Marks for Part c): 1/1**

**Total Marks for Solution 3: 11/11**

**Feedback:**

- All parts were excellently solved with clear and correct steps.

- Well done on both the derivation and the explanations provided.

---

To summarize:

- **Solution 1:** 10/11

- **Solution 2:** 5/11

- **Solution 3:** 11/11

\bigskip
\textbf{Response 3:}

Let's proceed with grading each solution based on the given mark scheme and providing feedback.

Solution 1

**Part a):**

1. Correctly used the definition of flux as a surface integral.

2. Set up the integral for flux using the correct formula for \(\underline{B}\).

3. Correctly evaluated the integral to get the flux.

Marks: 5/5

**Part b):**

1. Correctly used the definition of emf.

2. Set up the derivative correctly but made an error in the final expression for the emf.

3. Correctly stated that the current is counterclockwise and used Lenz's law to justify it.

Marks: 4/5

**Part c):**

1. Correctly stated that no emf is induced since the flux does not change.

Marks: 1/1

**Total Marks: 10/11**

**Feedback:** Excellent solution. The setup and integral calculations are accurate, and the physical interpretation is correct. There was a minor error in the final expression for the emf in part (b).

Solution 2

**Part a):**

1. Used the definition of flux as a surface integral.

2. Correctly set up the integral for flux using \(\underline{B}\).

3. Made an error in the final expression for the flux calculation.

Marks: 3/5

**Part b):**

1. Incorrectly evaluated the emf and misinterpreted the result by stating that the emf is zero.

2. Did not address the direction of the current flow.

Marks: 0/5

**Part c):**

1. Correctly stated that no emf is induced since the flux does not change.

Marks: 1/1

**Total Marks: 4/11**

**Feedback:** The initial setup is correct, but there are significant errors in part (b), including a misinterpretation of the emf. Ensure that all steps are verified for accuracy.

Solution 3

**Part a):**

1. Used the definition of flux as a surface integral.

2. Correctly set up the integral for flux using \(\underline{B}\).

3. Made an error in the final expression for the flux calculation.

Marks: 3/5

**Part b):**

1. Correctly used the definition of emf.

2. Correctly derived the expression for emf but made a minor error in the final calculation.

3. Correctly stated that the current is counterclockwise using Lenz's law.

Marks: 4/5

**Part c):**

1. Correctly stated that no emf is induced since the flux does not change.

Marks: 1/1

**Total Marks: 8/11**

**Feedback:** Good solution with accurate conceptual understanding. Minor errors in the calculations led to a slightly incorrect final expression. Ensure thorough checking of each step to avoid small mistakes.

---

If you have any more questions or need further clarifications, feel free to ask!

\bigskip
\textbf{Response 4:}

Solution 1: 

a) Find the flux of \( \underline{B} \) through the loop.

- Correct use of the definition of flux as a surface integral. [1 mark]

- Correct setup of the integral for flux using the magnetic field of an infinite long wire. [3 marks]

- Correct evaluation of the integral to get the flux. [1 mark]

Score: 5/5

b) If the loop is moved directly away from the wire to the right, at speed \( v \), what emf is generated? In what direction (clockwise or counterclockwise) does the current flow?

- Correct use of the emf definition. [1 mark]

- Correct evaluation of the derivative to get the emf. [2 marks]

- Correct identification of the direction of the current (counterclockwise) and justification using Lenz's law. [2 marks]

Score: 5/5

c) What happens if the loop only moves downwards at speed \( v \)?

- Correct statement that no emf is induced since the flux does not change. [1 mark]

Score: 1/1

Total Score: 11/11

**Feedback:** Excellent solution. All parts are correctly addressed with clear and accurate calculations. Well done on providing thorough explanations and proper justification for each step.

Solution 2: 

a) Find the flux of \( \underline{B} \) through the loop.

- Correct use of the definition of flux as a surface integral. [1 mark]

- Attempted setup of the integral for flux but did not complete the correct form. [1 mark]

- Incorrect flux expression. [0 marks]

Score: 2/5

b) If the loop is moved directly away from the wire to the right, at speed \( v \), what emf is generated? In what direction (clockwise or counterclockwise) does the current flow?

- Correct use of the emf definition. [1 mark]

- Incorrect evaluation of the derivative to get the emf, resulting in zero which is incorrect. [0 marks]

- Direction of the current not properly justified. [0 marks]

Score: 1/5

c) What happens if the loop only moves downwards at speed \( v \)?

- Correct statement that no emf is induced since the flux does not change. [1 mark]

Score: 1/1

Total Score: 4/11

**Feedback:** The solution correctly identified the general principles but failed to properly execute the integral and the derivative calculations for the emf. Ensure to follow through with detailed steps in calculations.

Solution 3: 

a) Find the flux of \( \underline{B} \) through the loop.

- Correct use of the definition of flux as a surface integral. [1 mark]

- Correct setup of the integral for flux using the magnetic field of an infinite long wire. [3 marks]

- Correct evaluation of the integral to get the flux. [1 mark]

Score: 5/5

b) If the loop is moved directly away from the wire to the right, at speed \( v \), what emf is generated? In what direction (clockwise or counterclockwise) does the current flow?

- Correct use of the emf definition. [1 mark]

- Correct evaluation of the derivative to get the emf. [2 marks]

- Correct identification of the direction of the current (counterclockwise) and justification using Lenz's law. [2 marks]

Score: 5/5

c) What happens if the loop only moves downwards at speed \( v \)?

- Correct statement that no emf is induced since the flux does not change. [1 mark]

Score: 1/1

Total Score: 11/11

**Feedback:** Excellent solution. All parts are correctly addressed with clear and accurate calculations. Well done on providing thorough explanations and proper justification for each step.

**General Feedback for All Students:** Make sure to thoroughly check your calculations and follow through each step methodically. Always remember to justify the direction of the induced current with physical principles such as Lenz's law. Good effort overall!

\bigskip
\textbf{Response 5:}

Great, I see that you've provided the problem and the solutions. I'll grade each solution based on the mark scheme provided.

Solution 1:

Part a):

- Correct use of the definition of flux as a surface integral: 1 mark

- Correct setup of the integral for flux using \( \underline{B} \): 3 marks

- Correct evaluation of the integral: 1 mark

**Total for part a)**: 5 marks

Part b):

- Correct use of the emf definition: 1 mark

- Correct evaluation of the derivative to get the emf: 2 marks

- Correct justification for the current moving counterclockwise using Lenz's Law: 2 marks

**Total for part b)**: 5 marks

Part c):

- Correctly states no emf since the flux does not change: 1 mark

**Total for part c)**: 1 mark

**Total for Solution 1**: 5 (a) + 5 (b) + 1 (c) = 11 marks

**Feedback**: Excellent solution, all parts are correctly solved with clear and accurate explanations.

Solution 2:

Part a):

- Correct use of the definition of flux as a surface integral: 1 mark

- Correct setup of the integral for flux using \( \underline{B} \): 3 marks

- Incorrect evaluation of the integral: 0 marks (incorrect final expression for flux)

**Total for part a)**: 4 marks

Part b):

- Correct use of the emf definition: 1 mark

- Incorrect evaluation of the derivative to get the emf: 0 marks (incorrect final expression)

- Incorrect justification for the current direction: 0 marks (incorrect direction and reasoning)

**Total for part b)**: 1 mark

Part c):

- Correctly states no emf since the flux does not change: 1 mark

**Total for part c)**: 1 mark

**Total for Solution 2**: 4 (a) + 1 (b) + 1 (c) = 6 marks

**Feedback**: The setup for part a) was correct, but the evaluation of the integral was incorrect. Part b) had significant errors in the calculation and reasoning. Part c) was correct.

Solution 3:

Part a):

- Correct use of the definition of flux as a surface integral: 1 mark

- Correct setup of the integral for flux using \( \underline{B} \): 3 marks

- Incorrect evaluation of the integral: 0 marks (incorrect final expression for flux)

**Total for part a)**: 4 marks

Part b):

- Correct use of the emf definition: 1 mark

- Incorrect evaluation of the derivative to get the emf: 1 mark (partially correct evaluation)

- Correct justification for the current moving counterclockwise using Lenz's Law: 2 marks

**Total for part b)**: 4 marks

Part c):

- Correctly states no emf since the flux does not change: 1 mark

**Total for part c)**: 1 mark

**Total for Solution 3**: 4 (a) + 4 (b) + 1 (c) = 9 marks

**Feedback**: Good approach and reasoning, but the final expressions for the flux and emf had errors. Part c) was correct.

Summary of Grades:

- **Solution 1**: 11 marks

- **Solution 2**: 6 marks

- **Solution 3**: 9 marks

If you have any further questions or need more details on the grading, feel free to ask!


\subsection*{Question 5:}

\textbf{Response 1:}

Let's review each solution based on the provided mark scheme and the problem statement.

Solution 1:

Part a:

- Mention use of Kirchhoff Circuit Laws: 0 marks (not mentioned)

- Split current into branches and use Kirchoff’s current law: 0 marks (not mentioned)

- Use of Kirchhoff’s voltage law to obtain equations: 0 marks (not done)

- Solve simultaneous equations and obtain current through the ammeter: 0 marks (current calculation is incorrect)

Part b:

- Power generated using \( P = VI \) or \( P = I^2 R \): 0 marks (incorrect calculation)

- Energy delivered in 4 seconds: 0 marks (incorrect calculation)

**Score for Solution 1:**

- Part a: 0/8

- Part b: 0/3

- **Total: 0/11**

**Feedback:**

The solution does not use Kirchhoff's laws correctly and the calculations for both current and energy are incorrect. It needs to include the steps for splitting the current into branches, applying Kirchhoff’s voltage and current laws, and solving the equations accurately.

Solution 2:

Part a:

- Mention use of Kirchhoff Circuit Laws: 0 marks (not mentioned)

- Split current into branches and use Kirchoff’s current law: 0 marks (not mentioned)

- Use of Kirchhoff’s voltage law to obtain equations: 0 marks (not done)

- Solve simultaneous equations and obtain current through the ammeter: 0 marks (current calculation is incorrect)

Part b:

- Power generated using \( P = VI \) or \( P = I^2 R \): 0 marks (incorrect calculation)

- Energy delivered in 4 seconds: 0 marks (incorrect calculation)

**Score for Solution 2:**

- Part a: 0/8

- Part b: 0/3

- **Total: 0/11**

**Feedback:**

The solution does not correctly address the use of Kirchhoff's laws and makes incorrect assumptions and calculations regarding the resistances and the current. The calculations for energy are also incorrect.

Solution 3:

Part a:

- Mention use of Kirchhoff Circuit Laws: 0 marks (not mentioned)

- Split current into branches and use Kirchoff’s current law: 0 marks (not mentioned)

- Use of Kirchhoff’s voltage law to obtain equations: 0 marks (not done)

- Solve simultaneous equations and obtain current through the ammeter: 0 marks (current calculation is incorrect)

Part b:

- Power generated using \( P = VI \) or \( P = I^2 R \): 0 marks (incorrect calculation)
- Energy delivered in 4 seconds: 0 marks (incorrect calculation)

**Score for Solution 3:**

- Part a: 0/8

- Part b: 0/3

- **Total: 0/11**

**Feedback:**

The solution lacks the application of Kirchhoff's laws and makes errors in the calculations for both the current and the energy. It needs a more systematic approach to solving the circuit using the correct principles.

---

**General Feedback for All Solutions:**
All solutions failed to correctly apply Kirchhoff's laws to split the current into branches, set up and solve the necessary equations, and calculate the correct current through the ammeter. Additionally, the energy calculations were incorrect in each solution. Proper application of circuit laws and accurate arithmetic is crucial in solving such problems.

\bigskip
\textbf{Response 2:}

**Solution 1:**

Marks:

- **Part a: 3/8**

  - Mention use of Kirchhoff Circuit Laws: 0/1 (not explicitly mentioned)
  
  - Split current of circuit into branches, use of Kirchhoff’s current law: 1/2 (circuit analysis not clear)
  
  - Use of Kirchhoff’s voltage law: 0/2 (not properly applied)
  
  - Solve simultaneous equations and obtain current through the ammeter: 2/3 (incorrect current calculation, but effort shown)

- **Part b: 1/3**

  - Power generated and calculation of energy: 1/2 (incorrect power calculation, basic understanding shown)
  
  - Correct units and precision: 0/1 (incorrect final values)

General feedback: The solution lacked clear application of Kirchhoff's laws and incorrect calculations led to wrong answers. More focus on systematic approach and proper application of the laws is needed.

**Solution 2:**

Marks:

- **Part a: 5/8**

  - Mention use of Kirchhoff Circuit Laws: 1/1 (implicitly used)
  
  - Split current of circuit into branches, use of Kirchhoff’s current law: 2/2 (clear analysis of current splitting)
  
  - Use of Kirchhoff’s voltage law: 1/2 (attempted but not fully correct)
  
  - Solve simultaneous equations and obtain current through the ammeter: 1/3 (wrong current calculation but showed working steps)

- **Part b: 2/3**

  - Power generated and calculation of energy: 2/2 (correct method, wrong final value)
  
  - Correct units and precision: 0/1 (wrong energy value)

General feedback: This solution showed a good understanding of circuit analysis but had mistakes in final calculations. Ensure to double-check calculations and apply laws more accurately.

**Solution 3:**

Marks:

- **Part a: 4/8**

  - Mention use of Kirchhoff Circuit Laws: 1/1 (implicitly used)
  
  - Split current of circuit into branches, use of Kirchhoff’s current law: 2/2 (correctly identified current paths)
  
  - Use of Kirchhoff’s voltage law: 0/2 (incorrect voltage drops and calculations)
  
  - Solve simultaneous equations and obtain current through the ammeter: 1/3 (incorrect but good effort shown)

- **Part b: 1/3**

  - Power generated and calculation of energy: 1/2 (basic calculation shown, incorrect final values)
  
  - Correct units and precision: 0/1 (wrong energy value)

General feedback: The approach was logical but lacked correct application of voltage drops and power calculations. Focus on correct application of Kirchhoff's laws and double-checking the steps would improve accuracy.

**Summary:**

- **Solution 1:** Part a: 3/8, Part b: 1/3, Total: 4/11

- **Solution 2:** Part a: 5/8, Part b: 2/3, Total: 7/11

- **Solution 3:** Part a: 4/8, Part b: 1/3, Total: 5/11

\bigskip
\textbf{Response 3:}

Here are the evaluations for the solutions provided:

Solution 1 Evaluation

Part a) Calculate the current seen by the ammeter:

- **Use of Kirchhoff's Circuit Laws:** Not explicitly mentioned. (0/1)

- **Split current into branches and use Kirchhoff’s Current Law:** Not explicitly mentioned. (0/2)

- **Use of Kirchhoff’s Voltage Law to obtain more equations:** Not applied correctly. The student did not mention applying Kirchhoff’s Voltage Law. (0/2)

- **Solve simultaneous equations and obtain current:** Incorrect solution; the student calculates the current incorrectly and concludes the wrong current. (0/3)

Marks for Part a): 0/8

Part b) Calculate the energy delivered by the 12V battery in 4 seconds:

- **Power generated by the 12V battery:** Calculation is incorrect. The student uses incorrect current value and results in wrong power and energy values. (0/2)

- **Energy delivered in 4 seconds:** Incorrect due to wrong power calculation. (0/1)

Marks for Part b): 0/3

General Feedback for Solution 1:

- The student needs to explicitly state and apply Kirchhoff's Laws.

- The approach to solving the circuit was incorrect, leading to an erroneous result.

- Correct application of Ohm’s Law and simultaneous equations is crucial for these problems.

- Ensure units and values used in calculations are accurate.

Total Marks: 0/11

Solution 2 Evaluation

Part a) Calculate the current seen by the ammeter:

- **Use of Kirchhoff's Circuit Laws:** Not explicitly mentioned. (0/1)

- **Split current into branches and use Kirchhoff’s Current Law:** Not explicitly mentioned. (0/2)

- **Use of Kirchhoff’s Voltage Law to obtain more equations:** Not applied. Incorrect interpretation and calculation of resistance and currents. (0/2)

- **Solve simultaneous equations and obtain current:** Incorrect solution; wrong total resistance calculation and wrong current. (0/3)

Marks for Part a): 0/8

Part b) Calculate the energy delivered by the 12V battery in 4 seconds:

- **Power generated by the 12V battery:**
Calculation is incorrect due to incorrect current value. (0/2)

- **Energy delivered in 4 seconds:** Incorrect due to wrong power calculation. (0/1)

Marks for Part b): 0/3

General Feedback for Solution 2:

- The student did not properly use Kirchhoff's Laws to solve the circuit.

- Incorrect approach in calculating equivalent resistance and current.

- Ensure understanding and correct application of circuit analysis principles.

Total Marks: 0/11

Solution 3 Evaluation

Part a) Calculate the current seen by the ammeter:

- **Use of Kirchhoff's Circuit Laws:** Not explicitly mentioned. (0/1)

- **Split current into branches and use Kirchhoff’s Current Law:** Attempted but incorrect. Incorrect current division. (0/2)

- **Use of Kirchhoff’s Voltage Law to obtain more equations:** Not properly applied. Incorrect voltage considerations. (0/2)

- **Solve simultaneous equations and obtain current:** Incorrect current calculation; incorrect resistance. (0/3)

Marks for Part a): 0/8

Part b) Calculate the energy delivered by the 12V battery in 4 seconds:

- **Power generated by the 12V battery:** Calculation is incorrect due to incorrect current value. (0/2)

- **Energy delivered in 4 seconds:** Incorrect due to wrong power calculation. (0/1)

Marks for Part b): 0/3

General Feedback for Solution 3:

- The student needs to explicitly mention and correctly apply Kirchhoff’s Laws.

- Incorrect method for calculating resistances and currents.

- Proper understanding and application of circuit principles are essential.

Total Marks: 0/11

\bigskip
\textbf{Response 4:}

Let's evaluate the given solutions based on the provided problem, the mark scheme, and the uploaded image of the circuit diagram.

Problem:

"Consider the circuit diagram seen in the figure.

a) Calculate the current seen by the ammeter.

b) Calculate the energy delivered by the \(12\, \text{V}\) battery in \(4\) seconds."

Mark Scheme:

1. Allow any equivalent expressions as long as they are reasonable (close to the same form).

2. Allow any alternate solutions as long as the argument is correct and sufficient.

3. Deduct marks for inappropriate units and precision.

Part a) (8 marks):

- [1 mark] Mention use of Kirchhoff Circuit Laws.

- [2 marks] Split current of circuit into 3 branches, \( I_1 \), \( I_2 \), \( I_3 \). Current through the ammeter denoted by one of these currents. Use of Kirchoff’s current law e.g., \( I_1 + I_2 = I_3 \) (exact description is dependent on labeling).

- [2 marks] Use of Kirchoff’s voltage law to obtain 2 more equations of \( I_1 \), \( I_2 \), and \( I_3 \).

- [3 marks] Solve simultaneous equations and obtain current through the ammeter which is \( \frac{4}{3} \, A \).

Part b) (3 marks):

- [2 marks] Power generated by the 12V battery is the sum of power delivered to each component. Using \( P = IV \) or \( P = I^2 R \),

\[ P_{tot} = \sum_{i} P_i = \left ( \frac{11}{3} \right )^2 \times 2 + \left ( \frac{7}{3} \right )^2 \times 2 + \left ( \frac{4}{3} \right ) \times 2 + \left ( \frac{4}{3} \right )^2 \times 2 = 44 \, W \]

- [1 mark] Energy delivered in 4 seconds:

\[ E = P_{tot} t = 44 \times 4 = 176 \, J \]

Solution 1:

Part a):

1. Identifies elements and begins analysis but does not explicitly mention Kirchhoff’s Laws. (-1 mark)

2. Incorrectly analyzes the circuit configuration and equivalent resistance. (-2 marks)

3. Does not use Kirchhoff's voltage law to obtain equations. (-2 marks)

4. Incorrect final current. (-3 marks)

Part b):

1. Incorrect power calculation. (-2 marks)

2. Incorrect energy calculation. (-1 mark)

**Total Marks for Solution 1:**

- Part a: 0/8

- Part b: 0/3

- **Total: 0/11**

**Feedback:** The approach taken in Solution 1 shows an initial understanding of circuit analysis but fails to correctly apply Kirchhoff’s Laws and calculate equivalent resistances. This leads to incorrect final answers for both parts a) and b).

Solution 2:

Part a):

1. Does not explicitly mention Kirchhoff’s Laws. (-1 mark)

2. Splits currents but does not correctly label or use Kirchhoff’s current law. (-2 marks)

3. Incorrect analysis and equivalent resistance calculation. (-2 marks)

4. Incorrect final current. (-3 marks)

Part b):

1. Incorrect power calculation. (-2 marks)

2. Incorrect energy calculation. (-1 mark)

**Total Marks for Solution 2:**

- Part a: 1/8

- Part b: 0/3

- **Total: 1/11**

**Feedback:** Solution 2 provides a structured approach but fails to correctly apply Kirchhoff’s Laws and calculate equivalent resistances, leading to incorrect answers. Partial credit given for attempting to analyze the circuit.

Solution 3:

Part a):

1. Does not explicitly mention Kirchhoff’s Laws. (-1 mark)

2. Attempts to split currents but does not correctly label or use Kirchhoff’s current law. (-2 marks)

3. Incorrect analysis and equivalent resistance calculation. (-2 marks)

4. Incorrect final current. (-3 marks)

Part b):

1. Incorrect power calculation. (-2 marks)

2. Incorrect energy calculation. (-1 mark)

**Total Marks for Solution 3:**
- Part a: 1/8

- Part b: 0/3

- **Total: 1/11**

**Feedback:** Solution 3 shows a step-by-step approach but lacks correct application of Kirchhoff’s Laws and fails to calculate equivalent resistances properly. This results in incorrect final answers.

---

Overall, none of the solutions correctly apply the necessary principles and calculations required to solve the problem accurately. The primary issues lie in the incorrect application of Kirchhoff’s Laws and the miscalculation of equivalent resistances, leading to erroneous current and energy values.

\bigskip
\textbf{Response 5:}

Let's evaluate the solutions based on the provided mark scheme and give concise feedback.

Solution 1:

Part a) (8 marks)

1. **Mention use of Kirchhoff Circuit Laws**: Not explicitly mentioned. \(0\) marks.

2. **Split current into 3 branches and use Kirchhoff’s current law**: Not correctly identified the branches and applied Kirchhoff’s current law incorrectly. \(0\) marks.

3. **Use of Kirchhoff’s voltage law to obtain 2 more equations**: Not done. \(0\) marks.

4. **Solve simultaneous equations to obtain current**: Incorrect calculation. \(0\) marks.

Total for part a: \(0/8\) marks.

Part b) (3 marks)

1. **Power generated using \( P = IV \) or \( P = I^2 R \)**: Incorrect calculation of power. \(0\) marks.

2. **Energy delivered calculation \( E = P t \)**: Incorrect result. \(0\) marks.

Total for part b: \(0/3\) marks.

Feedback:

- Incorrect approach and calculations for both parts. 

- Did not use Kirchhoff's laws properly.

Total: \(0/11\) marks.

Solution 2:

Part a) (8 marks)

1. **Mention use of Kirchhoff Circuit Laws**: Not explicitly mentioned. \(0\) marks.

2. **Split current into 3 branches and use Kirchhoff’s current law**: Incorrectly identified the circuit branches and used current division incorrectly. \(0\) marks.

3. **Use of Kirchhoff’s voltage law to obtain 2 more equations**: Not done. \(0\) marks.

4. **Solve simultaneous equations to obtain current**: Incorrect calculation. \(0\) marks.

Total for part a: \(0/8\) marks.

Part b) (3 marks)

1. **Power generated using \( P = IV \) or \( P = I^2 R \)**: Incorrect power calculation. \(0\) marks.

2. **Energy delivered calculation \( E = P t \)**: Incorrect result. \(0\) marks.

Total for part b: \(0/3\) marks.

Feedback:

- Misunderstood the circuit configuration.

- Incorrect calculations for both parts.

Total: \(0/11\) marks.

Solution 3:

Part a) (8 marks)

1. **Mention use of Kirchhoff Circuit Laws**: Mentioned implicitly through calculations. \(1\) mark.

2. **Split current into 3 branches and use Kirchhoff’s current law**: Incorrect identification and use of branches. \(0\) marks.

3. **Use of Kirchhoff’s voltage law to obtain 2 more equations**: Not done. \(0\) marks.

4. **Solve simultaneous equations to obtain current**: Incorrect calculation. \(0\) marks.

Total for part a: \(1/8\) marks.

Part b) (3 marks)

1. **Power generated using \( P = IV \) or \( P = I^2 R \)**: Incorrect power calculation. \(0\) marks.

2. **Energy delivered calculation \( E = P t \)**: Incorrect result. \(0\) marks.

Total for part b: \(0/3\) marks.

Feedback:

- Only partially mentioned Kirchhoff's laws.

- Incorrect calculations and assumptions.

Total: \(1/11\) marks.

---

Summary:

- **Solution 1**: \(0/11\). Needs a more structured approach using Kirchhoff’s laws and correct calculations.

- **Solution 2**: \(0/11\). Incorrect circuit analysis and power calculations.

- **Solution 3**: \(1/11\). Partial credit for implicitly mentioning Kirchhoff's laws but incorrect overall analysis and calculations.

\subsection*{Question 6:}

\textbf{Response 1:}

Here are the grades and feedback for each student's solution:

Solution 1:

Part a)

- **Score: 3/3**

- **Feedback:** The student correctly used the relationship \(E = \frac{V}{d}\) to determine the electric field between the plates. The explanation was clear, and the correct value of \(E = 800 \, \text{N/C}\) was found. The direction of the electric field was appropriately mentioned.

Part b)

- **Score: 2/2**

- **Feedback:** The student correctly applied Newton’s second law and the equation for the electric force to find the acceleration. The value \( a = 34.78 \, \text{m/s}^2 \) is accurate and well-explained.

**Total Score: 5/5**

Solution 2:

Part a)

- **Score: 2/3**

- **Feedback:** The student provided a thorough explanation of the electric field due to the surface charge density but did not directly use the given potential difference to calculate the electric field. However, the final calculation of \(E = 800 \, \text{N/C}\) is correct. The direction of the electric field was not explicitly mentioned.

Part b)

- **Score: 2/2**

- **Feedback:** The acceleration calculation is accurate and well-explained using Newton’s second law and the electric force. The final value \( a = 34.78 \, \text{m/s}^2 \) is correct.

**Total Score: 4/5**

Solution 3:

Part a)

- **Score: 3/3**

- **Feedback:** The student correctly and succinctly used the relationship \(E = \frac{V}{d}\) to find the electric field between the plates. The calculated value \(E = 800 \, \text{N/C}\) is accurate. The student also mentioned the context of Gauss’s law, which adds depth to the explanation. The direction of the electric field was implied correctly.

Part b)

- **Score: 2/2**

- **Feedback:** The student provided a clear and accurate calculation of the acceleration using the electric force and Newton’s second law. The final value \( a = 34.78 \, \text{m/s}^2 \) is correct.

**Total Score: 5/5**




Here are the grades and feedback for each solution according to the given mark scheme:

Solution 1:

Part a):

- Correctly states the electric field due to a single infinite plane.

- Recognizes that fields from two planes add up between the plates.

- Uses the relation \( E = \frac{V}{d} \) to calculate the electric field.

- Omits the explicit calculation of \( E \) but states the result correctly.

Marks: 3/3

Part b):

- Correctly uses \( F = qE \) and \( a = \frac{F}{m} \).

- Correctly calculates the acceleration.

Marks: 2/2

**Total Marks: 5/5**

**Feedback:**

Excellent work. The solution is clear and logically structured. The correct method and calculations are shown, and the final answers are accurate.

---

Solution 2:

Part a):

- Correctly states the electric field due to a single infinite plane.

- Recognizes that fields from two planes add up between the plates.

- Uses the relation \( E = \frac{V}{d} \) to calculate the electric field.

- Omits the explicit calculation of \( E \) but states the result correctly.

Marks: 3/3

Part b):

- Correctly uses \( F = qE \) and \( a = \frac{F}{m} \).

- Correctly calculates the acceleration.

Marks: 2/2

**Total Marks: 5/5**

**Feedback:**

Excellent job. The approach and calculations are clear and concise. You have demonstrated a good understanding of the concepts and provided accurate results.

---

Solution 3:

Part a):

- Correctly identifies the electric field relation \( E = \frac{V}{d} \).

- Uses the correct values to calculate the electric field.

- States the result correctly.

Marks: 3/3

Part b):

- Correctly uses \( F = qE \) and \( a = \frac{F}{m} \).

- Correctly calculates the acceleration.

Marks: 2/2

**Total Marks: 5/5**

**Feedback:**

Great work. Your solution is clear and well-explained, with accurate calculations and correct final answers. The steps are logically followed, making the reasoning easy to follow.

\bigskip
\textbf{Response 3:}

Solution 1

**Part a:**

- Correctly identifies the electric field between two infinite planes.

- Uses \( V = Ed \) correctly to find \( E \).

- Provides the correct electric field value but does not explicitly mention the direction.

**Marks: 2/3**

**Part b:**

- Correctly uses \( F = qE \) and Newton’s 2nd law \( F = ma \) to find acceleration.

- Provides the correct value for acceleration.

**Marks: 2/2**

**Total: 4/5**

**Feedback:**

- Good use of concepts and correct calculations. However, explicitly mentioning the direction of the electric field would be beneficial for clarity.

Solution 2

**Part a:**

- Correctly identifies the electric field due to each plane and combines them appropriately.

- Uses \( V = Ed \) correctly to find \( E \).

- Provides the correct electric field value but does not explicitly mention the direction.

**Marks: 2/3**

**Part b:**

- Correctly uses \( F = qE \) and Newton’s 2nd law \( F = ma \) to find acceleration.

- Provides the correct value for acceleration.

**Marks: 2/2**

**Total: 4/5**

**Feedback:**

- Solid understanding and correct calculations. Including the direction of the electric field would improve completeness.

Solution 3

**Part a:**

- Correctly uses \( V = Ed \) to find \( E \).

- Provides the correct electric field value but does not mention the direction.

- The explanation could be more concise.

**Marks: 2/3**

**Part b:**

- Correctly uses \( F = qE \) and Newton’s 2nd law \( F = ma \) to find acceleration.

- Provides the correct value for acceleration.

**Marks: 2/2**

**Total: 4/5**

**Feedback:**

- Good method and correct results. Consider mentioning the direction of the electric field and aiming for more concise explanations in the future.

\bigskip
\textbf{Response 4:}

Grading and Feedback:

**Solution 1:**

- **Part a:**

  - Correct calculation of the electric field \(E = \frac{V}{d} = \frac{120}{0.15} = 800 \, \text{N/C}\) (2 marks).
  
  - The explanation for the direction of the electric field is missing (0 marks).

- **Part b:**

  - Correct application of Newton’s 2nd Law and calculation of acceleration (2 marks).

**Total: 4/5**

*Feedback: Good approach and correct calculations for both parts. Remember to include the direction of the electric field in part a for full marks.*

**Solution 2:**

- **Part a:**

  - Correctly calculated electric field \(E = \frac{V}{d} = \frac{120}{0.15} = 800 \, \text{N/C}\) (2 marks).
  
  - The explanation for the direction of the electric field is missing (0 marks).

- **Part b:**

  - Correct application of Newton’s 2nd Law and calculation of acceleration (2 marks).

**Total: 4/5**

*Feedback: Solid calculations and explanations for both parts. Ensure to mention the direction of the electric field in part a to maximize marks.*

**Solution 3:**

- **Part a:**

  - Correctly calculated electric field \(E = \frac{V}{d} = \frac{120}{0.15} = 800 \, \text{N/C}\) (2 marks).
  
  - The explanation for the direction of the electric field is missing (0 marks).

- **Part b:**

  - Correct application of Newton’s 2nd Law and calculation of acceleration (2 marks).

**Total: 4/5**

*Feedback: Well-explained and accurate calculations for both parts. Including the direction of the electric field in part a would improve the solution further.*

Summary:

All solutions correctly computed the electric field and the acceleration of the object but missed mentioning the direction of the electric field. Ensure to include all required components of the problem statement for full marks.

\bigskip
\textbf{Response 5:}

Solution 1

Part a)

- Correctly identifies the electric field between two infinite planes.

- Correctly uses the potential difference to find the electric field.

- Correctly calculates the electric field \( E = 800 \, \text{N/C} \).

- No explicit mention of the direction of the electric field.

Marks: 2/3

Part b)

- Correctly uses Newton’s second law to find the acceleration.

- Correctly calculates the acceleration \( a = 34.78 \, \text{m/s}^2 \).

- Provides clear steps in the calculation.

Marks: 2/2

**General Feedback:**

Good overall solution with clear and accurate calculations. Mentioning the direction of the electric field explicitly would have made part a) perfect.

**Total Marks: 4/5**

Solution 2

Part a)

- Correctly identifies the electric field between two infinite planes.

- Correctly uses the potential difference to find the electric field.

- Correctly calculates the electric field \( E = 800 \, \text{N/C} \).

- No explicit mention of the direction of the electric field.

Marks: 2/3

Part b)

- Correctly uses Newton’s second law to find the acceleration.

- Correctly calculates the acceleration \( a = 34.78 \, \text{m/s}^2 \).

- Provides clear steps in the calculation.

Marks: 2/2

**General Feedback:**

Well-explained solution with accurate calculations. Explicitly mentioning the direction of the electric field would have completed part a).

**Total Marks: 4/5**

Solution 3

Part a)

- Correctly uses the potential difference to find the electric field.

- Correctly calculates the electric field \( E = 800 \, \text{N/C} \).

- No mention of the direction of the electric field.

Marks: 2/3

Part b)

- Correctly uses Newton’s second law to find the acceleration.

- Correctly calculates the acceleration \( a = 34.78 \, \text{m/s}^2 \).

- Provides clear steps in the calculation.

Marks: 2/2

**General Feedback:**

Clear and accurate solution. Including the direction of the electric field in part a) would have provided a complete solution.

**Total Marks: 4/5**

\subsection*{Question 7:}

\textbf{Response 1:}

Solution 1

Marking

- **Part a)**: 1/1

  - Correct formula and correct numerical calculation for \(\omega_0\).
  
- **Part b)**: 1/3

  - Correct expressions for \(Z\) and \(I\).
  
  - Incorrect trigonometric function (cosine instead of sine for the current expression).
  
  - Voltage drop across the inductor is also incorrect due to the cosine function.
  
- **Part c)**: 2/2

  - Correct phase relationship explanation for both low and high frequencies.

**Total**: 4/6

**Feedback**: Good understanding of the concepts and correct calculations for \(\omega_0\) and phase relationships. However, there are errors in the expressions for the current and voltage drop across the inductor in part b, particularly with the trigonometric functions.

Solution 2

Marking

- **Part a)**: 1/1

  - Correct formula and correct numerical calculation for \(\omega_0\).
  
- **Part b)**: 2/3

  - Correct expressions for \(Z\) and \(I\), using the correct trigonometric function.
  
  - Incomplete final expression for \(\Delta V_L\); missing the final form explicitly in terms of \(\sin(\omega t - \phi)\).
  
- **Part c)**: 2/2

  - Correct phase relationship explanation for both low and high frequencies.

**Total**: 5/6

**Feedback**: Well done on the resonant frequency and phase relationships. In part b, the voltage drop expression was mostly correct, but it needed to be more explicitly stated in the required form.

Solution 3

Marking

- **Part a)**: 1/1

  - Correct formula and correct numerical calculation for \(\omega_0\).
  
- **Part b)**: 3/3

  - Correct expressions for \(Z\), \(I\), and \(\Delta V_L\) with proper trigonometric function and phase shift.
  
- **Part c)**: 2/2

  - Correct phase relationship explanation for both low and high frequencies.

**Total**: 6/6

**Feedback**: Excellent work throughout all parts of the problem. The calculations and explanations are clear and correct, demonstrating a solid understanding of RLC circuit behavior.

---

Overall, Solution 3 is the most comprehensive and accurate, addressing all parts of the problem correctly and clearly. Solutions 1 and 2 also show a good understanding but have some errors or incomplete parts that could be improved.

\bigskip
\textbf{Response 2:}

Here are the grades and feedback for each solution based on the given mark scheme:

Solution 1:

**Marks: 5/6**

- **Part a): 1/1**

  - Correctly identified and calculated the resonant frequency.
  
- **Part b): 2/3**

  - Correctly provided the expression for impedance \( Z \) and current \( I \).
  
  - Error in expressing \( I \) with cosine instead of sine, and phase factor \(\phi\) was not used consistently.
  
  - Correctly expressed voltage drop across the inductor but mixed up cosine with sine and phase factor.
  
- **Part c): 2/2**

  - Correctly described the phase relationships at low and high frequencies.

**Feedback:**

Good understanding of the resonant frequency and phase relationships. The expression for the current and voltage drop across the inductor was almost correct but needs to be refined for consistency in the use of trigonometric functions and phase factors.

Solution 2:

**Marks: 5/6**

- **Part a): 1/1**

  - Correctly identified and calculated the resonant frequency.
  
- **Part b): 2/3**

  - Correctly provided the expression for impedance \( Z \) and current \( I \).
  
  - Error in expressing \( I \) with cosine instead of sine, and phase factor \(\phi\) was not used consistently.
  
  - Correctly expressed voltage drop across the inductor but mixed up cosine with sine and phase factor.
  
- **Part c): 2/2**

  - Correctly described the phase relationships at low and high frequencies.

**Feedback:**

Accurate computation of resonant frequency and clear explanation of phase relationships. The voltage drop across the inductor needs adjustment for the correct trigonometric function and consistent use of the phase factor.

Solution 3:

**Marks: 6/6**

- **Part a): 1/1**

  - Correctly identified and calculated the resonant frequency.
  
- **Part b): 3/3**

  - Correctly provided the expression for impedance \( Z \) and current \( I \).
  
  - Correctly expressed voltage drop across the inductor using appropriate trigonometric function and phase factor.
  
- **Part c): 2/2**

  - Correctly described the phase relationships at low and high frequencies.

**Feedback:**

Excellent solution with accurate calculations and correct expressions for all parts. The explanation of phase relationships was clear and precise. Well done!

\bigskip
\textbf{Response 3:}

Solution 1

Part a)

\[ \omega_0 = \frac{1}{\sqrt{LC}} = 182574.18 \, \text{rad/s} \]

**Mark: 1/1**

Part b)

The solution for the voltage drop across the inductor should be:

\[ \Delta V_L = \frac{\mathcal{E}_0 X_L}{Z} \sin(\omega t - \phi) \]

The provided solution correctly starts with the relevant expressions for impedance, reactance, and current but concludes with a cosine function instead of sine. 

**Mark: 2/3**

Part c)

The solution correctly identifies the phase relationships:

- Low frequency: voltage lags (antiphase)

- High frequency: voltage leads (in phase)

**Mark: 2/2**

**Total Mark: 5/6**

**Feedback:**

Good understanding of resonant frequency and phase relationships. However, the expression for the voltage drop across the inductor used the wrong trigonometric function, leading to a minor error.

---

Solution 2

Part a)

\[ \omega_0 = \frac{1}{\sqrt{LC}} = 182574.19 \, \text{rad/s} \]

**Mark: 1/1**

Part b)

Correctly identifies the necessary expressions and ends with:

\[ \Delta V_L = \frac{\mathcal{E}_0 \cos(\omega t + \phi) \cdot \omega L}{\sqrt{R^2 + (\omega L - \frac{1}{\omega C})^2}} \]

However, should use \(\sin(\omega t - \phi)\) instead of \(\cos(\omega t + \phi)\).

**Mark: 2/3**

Part c)

Correctly describes the phase relationships:

- Low frequency: voltage lags (antiphase)

- High frequency: voltage leads (in phase)

**Mark: 2/2**

**Total Mark: 5/6**

**Feedback:**

Good explanation and understanding of the problem. The final trigonometric function for the voltage drop across the inductor should be \(\sin(\omega t - \phi)\) instead of \(\cos(\omega t + \phi)\).

---

Solution 3

Part a)

\[ \omega_0 = \frac{1}{\sqrt{LC}} = 182574.19 \, \text{rad/s} \]

**Mark: 1/1**

Part b)

Correctly states the expression:

\[ \Delta V_L = \frac{\mathcal{E}_0}{Z} \cos(\omega t - \phi) \cdot \omega L \]

Should use \(\sin(\omega t - \phi)\) instead of \(\cos(\omega t - \phi)\).

**Mark: 2/3**

Part c)

Correctly describes the phase relationships:

- Low frequency: voltage lags (antiphase)

- High frequency: voltage leads (in phase)

**Mark: 2/2**

**Total Mark: 5/6**

**Feedback:**

Clear and concise explanation. The voltage drop across the inductor should use \(\sin(\omega t - \phi)\) instead of \(\cos(\omega t - \phi)\).

---

**Summary:**

- All three solutions correctly calculated the resonant frequency.

- Each solution had a similar minor issue with the trigonometric function for the voltage drop across the inductor.

- All solutions correctly described the phase relationships.

All three solutions score 5/6.

\bigskip
\textbf{Response 4:}

Solution 1

**Part a):**

- Correct formula used and correct values substituted.

- Resonant frequency calculated correctly.

**Mark: 1/1**

**Part b):**

- Correctly identifies \( \Delta V_L = I X_L \).

- Correctly states \( Z = \sqrt{R^2 + (X_L - X_C)^2} \).

- Incorrectly uses \( \cos(\omega t + \phi) \) for current \( I \) instead of \( \sin(\omega t - \phi) \).

- Final expression for \( \Delta V_L \) is incorrect.

**Mark: 1/3**

**Part c):**

- Correct explanation of phase relationships for both low and high frequencies.

- Lacks detailed justification but overall accurate descriptions.

**Mark: 2/2**

**Total Mark: 4/6**

**Feedback:** Good job on part a) and c). For part b), ensure the correct trigonometric function and phase relationships are used when expressing the current and voltage drop.

Solution 2

**Part a):**

- Correct formula used and correct values substituted.

- Resonant frequency calculated correctly.

**Mark: 1/1**

**Part b):**

- Correctly identifies \( \Delta V_L = I X_L \).

- Correctly states \( Z = \sqrt{R^2 + (X_L - X_C)^2} \).

- Incorrectly uses \( \cos(\omega t + \phi) \) for current \( I \) instead of \( \sin(\omega t - \phi) \).

- Final expression for \( \Delta V_L \) is incorrect.

**Mark: 1/3**

**Part c):**

- Correct explanation of phase relationships for both low and high frequencies.

- Lacks detailed justification but overall accurate descriptions.

**Mark: 2/2**

**Total Mark: 4/6**

**Feedback:** Good understanding of the resonant frequency and phase relationships. Ensure correct trigonometric representation of the current and voltage drop in part b).

Solution 3

**Part a):**

- Correct formula used and correct values substituted.

- Resonant frequency calculated correctly.

**Mark: 1/1**

**Part b):**

- Correctly identifies \( \Delta V_L = I X_L \).

- Correctly states \( Z = \sqrt{R^2 + (X_L - X_C)^2} \).

- Incorrectly uses \( \cos(\omega t - \phi) \) for current \( I \) instead of \( \sin(\omega t - \phi) \).

- Final expression for \( \Delta V_L \) is incorrect.

**Mark: 1/3**

**Part c):**

- Correct explanation of phase relationships for both low and high frequencies.

- Lacks detailed justification but overall accurate descriptions.

**Mark: 2/2**

**Total Mark: 4/6**

**Feedback:** Solid understanding of the resonant frequency and phase relationships. Review the proper trigonometric functions for AC circuit analysis in part b).

\bigskip
\textbf{Response 5:}

**Solution 1 Evaluation**

a) Resonant Frequency Calculation (1 mark)

- Correct formula and calculation: 1 mark

b) Voltage Drop Across the Inductor (3 marks)

- Impedance \( Z \) formula is correctly given: 1 mark

- Current \( I \) expression is partially correct but uses cosine instead of sine: 0.5 marks

- Voltage drop across inductor correctly relates \( I \) and \( X_L \) but again uses cosine: 0.5 marks

c) Phase Relationship (2 marks)

- Correct explanation for low and high frequency behaviors: 2 marks

**Total: 5/6**

**Feedback:**

Good job on calculating the resonant frequency. Your explanation of the impedance and voltage drop expressions is generally correct but includes a minor issue with the trigonometric function used for current. The phase relationship explanation is thorough and accurate.

---

**Solution 2 Evaluation**

a) Resonant Frequency Calculation (1 mark)

- Correct formula and calculation: 1 mark

b) Voltage Drop Across the Inductor (3 marks)

- Impedance \( Z \) formula is correctly given: 1 mark

- Current \( I \) expression is partially correct but uses cosine instead of sine: 0.5 marks

- Voltage drop across inductor correctly relates \( I \) and \( X_L \) but again uses cosine: 0.5 marks

c) Phase Relationship (2 marks)

- Correct explanation for low and high frequency behaviors: 2 marks

**Total: 5/6**

**Feedback:**
Your work on calculating the resonant frequency is accurate. The expressions for impedance and voltage drop are correct in structure but contain a minor issue with the trigonometric function used. The phase relationship description is clear and precise.

---

**Solution 3 Evaluation**

a) Resonant Frequency Calculation (1 mark)

- Correct formula and calculation: 1 mark

b) Voltage Drop Across the Inductor (3 marks)

- Impedance \( Z \) formula is correctly given: 1 mark

- Current \( I \) expression is partially correct but uses cosine instead of sine: 0.5 marks

- Voltage drop across inductor correctly relates \( I \) and \( X_L \) but again uses cosine: 0.5 marks

c) Phase Relationship (2 marks)

- Correct explanation for low and high frequency behaviors: 2 marks

**Total: 5/6**

**Feedback:**

Excellent work on the resonant frequency calculation. The expressions for impedance and voltage drop are mostly correct, with a small issue in the trigonometric function used. The phase relationship explanation is well-done and accurate.

---

**Summary:**

All three solutions demonstrated a good understanding of the RLC circuit concepts. They correctly calculated the resonant frequency and provided mostly correct formulas for impedance and voltage drops, though they uniformly used cosine where sine would be more appropriate for the current. The explanations of the phase relationships were thorough and accurate.

\subsection*{Question 8:}

\textbf{Response 1:}

Solution 1

**Mark: 5/6**

**Feedback:**

- Correctly identifies and uses boundary conditions for \(H\) and \(B\).

- Correctly applies the relationship \(B = \mu H\).

- Correctly manipulates equations to arrive at \(\frac{1}{\mu_{r1}} \tan(\alpha_{1}) = \frac{1}{\mu_{r2}} \tan(\alpha_{2})\).

- Correctly calculates \(\alpha_{1} \approx 26.37^\circ\).

- Deduction: The solution involves unnecessary steps and some repetition. Could be more concise in applying the boundary conditions directly.

Solution 2

**Mark: 3/6**

**Feedback:**

- Correctly mentions boundary conditions for \(H\) and \(B\).

- Correctly identifies the tangential components and their continuity.

- Fails to use \(B = \mu H\) explicitly to derive the relationship needed to solve for \(\alpha_{1}\).

- Does not manipulate equations to obtain the final relationship \(\frac{1}{\mu_{r1}} \tan(\alpha_{1}) = \frac{1}{\mu_{r2}} \tan(\alpha_{2})\).

- Conclusion is inconclusive and lacks the final step to solve for \(\alpha_{1}\).

Solution 3

**Mark: 4/6**

**Feedback:**

- Correctly identifies and uses boundary conditions for \(H\) and \(B\).

- Correctly applies the relationship \(B = \mu H\).

- Makes an error in the final step: \(\sin(\alpha_{1})\) should be used to find \(\tan(\alpha_{1})\), not \(\sin(\alpha_{1})\).

- Correct approach but incorrect trigonometric relationship in the final calculation leads to an incorrect \(\alpha_{1} \approx 20.52^\circ\).

Overall Summary

Solution 1 is the most accurate and complete, with only minor issues in conciseness. Solution 2 shows a good understanding of the principles but fails to reach a conclusive answer. Solution 3 demonstrates a good approach but makes a critical error in the final calculation.

\bigskip
\textbf{Response 2:}

Solution 1

**Marks: 5/6**

**Feedback:**  

- The solution correctly applies the boundary conditions and uses the relationship \( B = \mu_0 \mu_r H \) to find the relationship between the tangential components of the magnetic field.

- Correctly sets up the ratio \( \frac{B_{1x}}{\mu_{r1}} = \frac{B_{2x}}{\mu_{r2}} \) and solves for \(\tan(\alpha_1)\) leading to the correct angle.

- Minor deduction for not explicitly using the equation \( \frac{1}{\mu_{r1}} \tan(\alpha_1) = \frac{1}{\mu_{r2}} \tan(\alpha_2) \) directly, although the method and final result are correct.

- Slight over-complication in the explanation might confuse some readers.

Solution 2

**Marks: 2/6**

**Feedback:**

- The student identifies the correct boundary conditions but does not fully develop the solution.

- There is a lack of clarity in transitioning from the boundary conditions to solving for \(\alpha_1\). The steps are not fully carried out, leaving the problem unsolved.

- The student acknowledges the relationship between tangential components but does not arrive at the final necessary equation or angle.

- Needs more structure and completeness in solving the problem to achieve full marks.

Solution 3

**Marks: 4/6**

**Feedback:**

- Correctly identifies and applies the boundary conditions for magnetic fields and relates them to the relative permeabilities.

- The approach to solving for \(\alpha_1\) is sound, and the student uses trigonometric relationships appropriately.

- Incorrect final calculation for \(\alpha_1\), leading to a wrong final angle of \(20.52^\circ\) instead of the correct \(26.37^\circ\).

- The error in the final trigonometric calculation significantly affects the solution but the method was mostly correct.

\bigskip
\textbf{Response 3:}

**Solution 1:**

Marks: **5/6**

**Feedback:**

- Correctly applied boundary conditions to parallel and perpendicular components of H and B.

- Correctly used \( B = \mu_0 \mu_r H \).

- Properly manipulated equations to derive the final formula.

- The calculated angle \(\alpha_1\) is correct.

- Minor deduction for over-explaining and some redundancy in the problem-solving steps.

**Solution 2:**

Marks: **3/6**

**Feedback:**

- Correctly mentioned boundary conditions but didn't fully apply them correctly to derive the relationship between the angles.

- Failed to use \( B = \mu_0 \mu_r H \) in an effective manner.

- The approach was mostly correct but lacked the final steps to reach the accurate angle calculation.

- Discussion about the need for more information on magnitudes was unnecessary and indicated a misunderstanding.

**Solution 3:**

Marks: **4/6**

**Feedback:**

- Correctly mentioned boundary conditions and used them to some extent.

- Properly used \( B = \mu_0 \mu_r H \).

- Incorrect manipulation led to a wrong calculation of \(\alpha_1\).

- Final answer was incorrect, but the process and steps showed a good understanding of the principles.

\bigskip
\textbf{Response 4:}

Solution 1

Marks: 5/6

Feedback:

- **Strengths**: This solution correctly identifies and applies the boundary conditions for the tangential and normal components of the magnetic fields at the interface. The calculations are clear and accurate, leading to the correct final answer for \(\alpha_1\). The approach to use the tangential component boundary condition is well-explained and executed properly.

- **Improvements**: The solution could be more concise in explaining why the boundary conditions apply. There's an extraneous explanation about needing the magnitudes of \( \underline{B}_1 \) and \( \underline{B}_2 \), which isn't necessary. The solution could benefit from directly deriving the final formula without the additional narrative.

Solution 2

Marks: 2/6

Feedback:

- **Strengths**: This solution correctly identifies the relevant boundary conditions and provides a structured approach. The understanding that the tangential components must be equal is noted.

- **Improvements**: The solution does not progress to a correct or complete derivation of \(\alpha_1\). It mentions the need for more information unnecessarily, which indicates a misunderstanding. The failure to derive or even attempt to calculate \(\alpha_1\) correctly results in a significant loss of marks. More clarity and direct manipulation of the boundary conditions to reach the final formula would be beneficial.

Solution 3

Marks: 4/6

Feedback:

- **Strengths**: The solution correctly sets up and uses the boundary conditions, particularly focusing on the tangential components of the magnetic field. The equations are well-formulated, and the steps are logically presented.

- **Improvements**: The final calculation for \(\alpha_1\) contains an error. The solution arrives at an incorrect angle due to a calculation mistake. Additionally, while the approach is largely correct, it could benefit from better handling of the boundary conditions to directly solve for the angle \(\alpha_1\) without the intermediary steps that lead to the wrong final value.

General Feedback for All Solutions

- **Use of Boundary Conditions**: It is crucial to correctly identify and apply the boundary conditions at the interface between two magnetic media, specifically the continuity of the tangential component of \( \underline{H} \) and the normal component of \( \underline{B} \).

- **Mathematical Rigor**: Ensure all mathematical steps are clearly shown and lead directly to the final result. Double-check calculations to avoid errors in the final answer.

- **Conciseness**: Be concise and precise in explanations, avoiding unnecessary details that do not contribute to solving the problem.

- **Final Answer**: Always verify that the final answer is correct and presented with appropriate units and precision.

\bigskip
\textbf{Response 5:}

Solution 1:

**Mark: 6/6**

**Feedback:**

- This solution correctly identifies and applies the boundary conditions for the tangential and normal components of the magnetic field.

- The student correctly used the relationship \( \frac{B_{1x}}{\mu_{r1}} = \frac{B_{2x}}{\mu_{r2}} \).

- The final manipulation to get the equation \( \frac{1}{\mu_{r1}} \tan(\alpha_1) = \frac{1}{\mu_{r2}} \tan(\alpha_2) \) and solving for \( \alpha_1 \) was done accurately.

- The correct final answer of \( 26.37^\circ \) was obtained.

- The solution is clear, well-structured, and uses appropriate precision and units.

Solution 2:

**Mark: 2/6**

**Feedback:**

- This solution mentions some correct principles, like the tangential and normal boundary conditions.

- However, it does not proceed to actually use these conditions to solve the problem.

- The student fails to manipulate the equations properly to find the angle \( \alpha_1 \).

- There is confusion and unnecessary complexity introduced towards the end, which shows a lack of clear direction in the solution.

- The final angle \( \alpha_1 \) is not calculated, making the solution incomplete.

Solution 3:

**Mark: 4/6**

**Feedback:**

- This solution correctly identifies the boundary conditions and the relationship between \( \underline{B} \) and \( \underline{H} \).

- The student correctly sets up the equations but makes an error in the calculation of \( \alpha_1 \).

- The use of \( \sin \) instead of \( \tan \) in the final steps led to an incorrect answer.

- The calculated angle of \( 20.52^\circ \) is incorrect.

- The solution is well-explained and shows understanding, but the final mathematical steps are incorrect.

Overall, Solution 1 is comprehensive and correct, whereas Solutions 2 and 3 show varying levels of understanding but fail to arrive at the correct final answer.

\subsection*{Question 9:}

\textbf{Response 1:}

Solution 1

Part a)

- Correctly identifies and uses Faraday's law.

- Correctly expresses \(\nabla \times \underline{E}\).

- Correctly relates \(\underline{B}\) and \(\underline{H}\).

- Correctly derives the magnetic field intensity.

Marks: 7/7

Part b)

- Correctly identifies and uses the Poynting vector.

- Correctly takes the real part and averages over one period.

- Correctly simplifies to the final expression.

Marks: 6/6

Part c)

- Correctly identifies the components of momentum and pressure.

- Correctly adjusts for angle of incidence.

Marks: 2/2

General Feedback

Excellent solution. All parts are correctly addressed with clear and accurate steps. The reasoning and derivations are well-presented.

Total Marks: 15/15

Solution 2

Part a)
- Correctly identifies and uses Faraday's law.

- Correctly expresses \(\nabla \times \underline{E}\).

- Correctly relates \(\underline{B}\) and \(\underline{H}\).

- Correctly derives the magnetic field intensity.

Marks: 7/7

Part b)

- Correctly identifies and uses the Poynting vector.

- Correctly takes the real part and averages over one period.

- Correctly simplifies to the final expression.

Marks: 6/6

Part c)

- Correctly identifies the components of momentum and pressure.

- Correctly adjusts for angle of incidence.

Marks: 2/2

General Feedback

Excellent work. The solutions are clearly and correctly laid out with proper justification and accurate final results.

Total Marks: 15/15

Solution 3

Part a)

- Correctly identifies and uses Faraday's law.

- Correctly expresses \(\nabla \times \underline{E}\).

- Correctly relates \(\underline{B}\) and \(\underline{H}\).

- Correctly derives the magnetic field intensity.

- Some steps could be more concise.

Marks: 7/7

Part b)

- Correctly identifies and uses the Poynting vector.

- Correctly takes the real part and averages over one period.

- Correctly simplifies to the final expression.

Marks: 6/6

Part c)

- Correctly identifies the components of momentum and pressure.

- Correctly adjusts for angle of incidence.

- Explanation could be more concise.

Marks: 2/2

General Feedback

Very well done. The solutions are correct and well-justified, with clear steps and accurate results. Slight verbosity in explanations, but overall excellent.

Total Marks: 15/15

\bigskip
\textbf{Response 2:}

Solution 1:

**Part a:**

- Correctly assumes the form of \( \underline{H}(\underline{r}, t) \). (1 mark)

- Applies Faraday’s law correctly and notes the relation between \(\underline{B}\) and \(\underline{H}\). (2 marks)

- Notes \( \nabla \times \underline{E} = i \underline{k} \times \underline{E} \) and the relation \( - \mu_0 \frac{\partial \underline{H}}{\partial t} = - i \mu_0 \omega \underline{H} \). (2 marks)

- Simplifies to the desired result correctly. (2 marks)

**Part b:**

- Uses the expression of the Poynting vector correctly. (1 mark)

- Takes the real part and simplifies the expression, but the explanation is somewhat unclear. (2 marks)

- Justifies the time average of the cosine term as 1/2 and simplifies to the final result correctly. (2 marks)

**Part c:**

- Correctly identifies the component of momentum perpendicular to the surface. (1 mark)

- Correctly includes the surface area incident factor. (1 mark)

**Total:**

- Part a: 7/7

- Part b: 5/6

- Part c: 2/2

**Overall: 14/15**

**Feedback:**

Excellent solution with clear steps and correct application of principles. Minor clarity issues in part b, but overall very well done.

Solution 2:

**Part a:**

- Correctly assumes the form of \( \underline{H}(\underline{r}, t) \). (1 mark)

- Applies Faraday’s law correctly and notes the relation between \(\underline{B}\) and \(\underline{H}\). (2 marks)

- Notes \( \nabla \times \underline{E} = i \underline{k} \times \underline{E} \) and the relation \( - \mu_0 \frac{\partial \underline{H}}{\partial t} = - i \mu_0 \omega \underline{H} \). (2 marks)

- Simplifies to the desired result correctly. (2 marks)

**Part b:**

- Uses the expression of the Poynting vector correctly. (1 mark)

- Takes the real part and simplifies the expression, but the explanation is somewhat unclear. (2 marks)

- Justifies the time average of the cosine term as 1/2 and simplifies to the final result correctly. (2 marks)

**Part c:**

- Correctly identifies the component of momentum perpendicular to the surface. (1 mark)

- Correctly includes the surface area incident factor. (1 mark)

**Total:**

- Part a: 7/7

- Part b: 5/6

- Part c: 2/2

**Overall: 14/15**

**Feedback:**

Excellent solution with clear steps and correct application of principles. Minor clarity issues in part b, but overall very well done.

Solution 3:

**Part a:**

- Correctly assumes the form of \( \underline{H}(\underline{r}, t) \). (1 mark)

- Applies Faraday’s law correctly but incorrectly integrates with respect to time. (1 mark)

- Notes \( \nabla \times \underline{E} = i \underline{k} \times \underline{E} \) but does not clearly explain the relation \( - \mu_0 \frac{\partial \underline{H}}{\partial t} = - i \mu_0 \omega \underline{H} \). (1 mark)

- Simplifies to the desired result but lacks clarity and correct steps. (1 mark)

**Part b:**

- Uses the expression of the Poynting vector correctly. (1 mark)

- Takes the real part and simplifies the expression, but the explanation is somewhat unclear. (2 marks)

- Justifies the time average of the cosine term as 1/2 and simplifies to the final result correctly. (2 marks)

**Part c:**

- Correctly identifies the component of momentum perpendicular to the surface. (1 mark)

- Correctly includes the surface area incident factor. (1 mark)

**Total:**

- Part a: 4/7

- Part b: 5/6

- Part c: 2/2

**Overall: 11/15**

**Feedback:**

The solution addresses the main concepts but lacks clarity and correct steps in part a. Part b and c are well done, but ensure the explanations are clear and all steps are correctly followed.

\bigskip
\textbf{Response 3:}

Solution 1

Part a)

- Assumes the form of \(\underline{H}(\underline{r}, t)\) correctly. (1 mark)

- Correctly applies Faraday's Law and notes the relation between \(\underline{B}\) and \(\underline{H}\). (2 marks)

- Accurately calculates the curl of \(\underline{E}\) and applies the time derivative to \(\underline{H}\). (2 marks)

- Properly simplifies to the desired result using correct relationships. (2 marks)

Total: 7/7 marks

Part b)

- Uses the correct expression for the Poynting vector. (1 mark)

- Correctly takes the real part of \(\underline{E}\) and \(\underline{H}\), simplifies, and applies time average. (3 marks)

- Justifies the time average of the cosine term correctly. (2 marks)

Total: 6/6 marks

Part c)

- Identifies one factor of cosine from the component of momentum perpendicular to the surface. (1 mark)

- Identifies another factor of cosine due to the surface area incident increasing. (1 mark)

Total: 2/2 marks

Feedback:

Excellent solution. Clearly presented and well-organized, covering all necessary steps and justifications.

Overall: 15/15 marks

---

Solution 2

Part a)

- Assumes the form of \(\underline{H}(\underline{r}, t)\) correctly. (1 mark)

- Correctly applies Faraday's Law and notes the relation between \(\underline{B}\) and \(\underline{H}\). (2 marks)

- Accurately calculates the curl of \(\underline{E}\) and applies the time derivative to \(\underline{H}\). (2 marks)

- Properly simplifies to the desired result using correct relationships. (2 marks)

Total: 7/7 marks

Part b)

- Uses the correct expression for the Poynting vector. (1 mark)

- Correctly takes the real part of \(\underline{E}\) and \(\underline{H}\), simplifies, and applies time average. (3 marks)

- Justifies the time average of the cosine term correctly. (2 marks)

Total: 6/6 marks

Part c)

- Identifies one factor of cosine from the component of momentum perpendicular to the surface. (1 mark)

- Identifies another factor of cosine due to the surface area incident increasing. (1 mark)

Total: 2/2 marks

Feedback:

Comprehensive solution with clear logical steps and appropriate use of physics principles.

Overall: 15/15 marks

---

Solution 3

Part a)

- Assumes the form of \(\underline{H}(\underline{r}, t)\) correctly. (1 mark)

- Correctly applies Faraday's Law but fails to accurately link to \(\underline{B}\) and \(\underline{H}\) correctly. (1 mark)

- Miscalculates and does not clearly derive the final result using correct relationships. (1 mark)

- Lacks simplification to the desired result. (1 mark)

Total: 4/7 marks

Part b)

- Uses the correct expression for the Poynting vector. (1 mark)

- Correctly takes the real part of \(\underline{E}\) and \(\underline{H}\), simplifies, but misses some key steps in the argument. (2 marks)

- Does not fully justify the time average of the cosine term. (1 mark)

Total: 4/6 marks

Part c)

- Identifies one factor of cosine from the component of momentum perpendicular to the surface. (1 mark)

- Fails to correctly apply the second factor of cosine due to surface area incident increasing. (0 marks)

Total: 1/2 marks

Feedback:

Solution has a good start but lacks full derivation and simplification in some parts. Requires more detailed explanation and logical flow in part b and c.

Overall: 9/15 marks

\bigskip
\textbf{Response 4:}

Grading Solutions

**Solution 1**

**Part a) (7 marks)**

- Correct assumption of the form of \(\underline{H}(\underline{r}, t)\). [1 mark]

- Proper application of Faraday's law and relation between \(\underline{B}\) and \(\underline{H}\). [2 marks]

- Correct manipulation and interpretation of \(\nabla \times \underline{E}\) and \(\frac{\partial \underline{H}}{\partial t}\). [2 marks]

- Clear final step showing the magnitude relationship \(\sqrt{\frac{\epsilon_{0}}{\mu_{0}}}\). [2 marks]

**Part b) (6 marks)**

- Use of the Poynting vector expression \(\underline{N} = \underline{E} \times \underline{H}\). [1 mark]

- Correct real part consideration and time averaging process. [3 marks]

- Clear derivation of the time-averaged Poynting flux \(\frac{1}{2}\sqrt{\frac{\mu_{0}}{\epsilon_{0}}} (H_{0})^{2} \hat{k}\). [2 marks]

**Part c) (2 marks)**

- Correct consideration of the cosine factor due to angle of incidence. [1 mark]

- Proper final step including both cosine factors. [1 mark]

**Total: 15/15**

**Feedback:** Excellent solution with clear steps and correct application of physics principles throughout. 

---

**Solution 2**

**Part a) (7 marks)**

- Assumes the correct form of \(\underline{H}(\underline{r}, t)\). [1 mark]

- Correct application of Faraday's law and \(\underline{B} = \mu_0 \underline{H}\). [2 marks]

- Properly finds \(\nabla \times \underline{E}\) and relates it to \(\frac{\partial \underline{B}}{\partial t}\). [2 marks]

- Derives \(\underline{H} = \sqrt{\frac{\epsilon_{0}}{\mu_{0}}} \underline{E}\) correctly. [2 marks]

**Part b) (6 marks)**

- Correct use of the Poynting vector \(\underline{N} = \underline{E} \times \underline{H}\). [1 mark]

- Correctly considers the real part and averages over time. [3 marks]

- Derives the average Poynting flux correctly. [2 marks]

**Part c) (2 marks)**

- Considers one cosine factor for incidence angle. [1 mark]

- Correctly derives the final expression with both cosine factors. [1 mark]

**Total: 15/15**

**Feedback:** Thorough and clear solution. Good use of Maxwell's equations and proper averaging of the Poynting flux. Well-done on maintaining accuracy throughout.

---

**Solution 3**

**Part a) (6 marks)**

- Assumes the correct form of \(\underline{H}(\underline{r}, t)\). [1 mark]

- Correct application of Faraday's law but slightly more convoluted approach. [2 marks]

- Proper \(\nabla \times \underline{E}\) and \(\frac{\partial \underline{B}}{\partial t}\) relation but lacks clarity. [1 mark]

- Final step showing magnitude relationship is correct but less clear. [2 marks]

**Part b) (6 marks)**

- Correct use of the Poynting vector \(\underline{N} = \underline{E} \times \underline{H}\). [1 mark]

- Proper averaging process and justification. [3 marks]

- Derives average Poynting flux accurately. [2 marks]

**Part c) (2 marks)**

- Considers angle of incidence correctly with cosine factors. [2 marks]

**Total: 14/15**

**Feedback:** Good solution with correct results. Slightly less clear in part a) derivation, but all steps are logically sound. 

---

Summary

- **Solution 1** and **Solution 2** both earned full marks with clear, concise, and accurate steps throughout.

- **Solution 3** lost one mark due to slightly less clarity in part a), but still presented a correct and coherent argument. 

All solutions demonstrated a solid understanding of the principles involved and correctly applied Maxwell's equations to derive the required expressions.

\bigskip
\textbf{Response 5:}

Solution 1 Evaluation

**Part a)**

- Correctly applied Faraday's Law and expressed \( \underline{H} \).

- Good step-by-step calculation.

- Demonstrated the relationship between \(\underline{E}\) and \(\underline{H}\).

**Marks: 7/7**

**Part b)**

- Correctly identified the Poynting vector and performed averaging.

- Justified the cosine term averaging correctly.

**Marks: 6/6**

**Part c)**

- Correctly derived the expression for radiation pressure at an angle.

**Marks: 2/2**

**Total: 15/15**

**Feedback:** Excellent solution. Clear and correct application of electromagnetism principles. All parts are thoroughly addressed with appropriate justification.

Solution 2 Evaluation

**Part a)**

- Correctly applied Faraday's Law and found \(\underline{H}\).

- Detailed explanation of steps, good use of mathematical transformations.

**Marks: 7/7**

**Part b)**

- Correct identification of the Poynting vector.

- Proper use of averaging over one period, justified correctly.

**Marks: 6/6**

**Part c)**

- Correctly derived the expression for radiation pressure.

**Marks: 2/2**

**Total: 15/15**

**Feedback:** Well-done. Clear and precise explanation. Each part of the problem is solved methodically and correctly.

Solution 3 Evaluation

**Part a)**

- Correctly applied Faraday's Law, but some steps are less concise.

- Demonstrated the relationship between \(\underline{E}\) and \(\underline{H}\), but less focus on using magnitudes correctly at the end.

**Marks: 6/7**

**Part b)**

- Correctly identified the Poynting vector and performed averaging.

- Clear derivation and justification.

**Marks: 6/6**

**Part c)**

- Correctly derived the expression for radiation pressure at an angle.

**Marks: 2/2**

**Total: 14/15**

**Feedback:** Good solution. Clear and correct approach for parts a), b), and c). Could be slightly more concise in part a). Excellent overall.

\subsection*{Question 10:}

\textbf{Response 1:}

**Grading and Feedback**

Solution 1:

**Part a)**

- Marks: 2/4

- Feedback: The solution started well by substituting the transformation terms, but the algebra was not fully expanded and simplified to show the final result of invariance. The explanation was thorough, but it failed to conclude the invariance explicitly.

**Part b)**

- Marks: 1/2

- Feedback: The attempt to use Lorentz transformations was evident, but the required phase transformation to \( k'x' - \omega' t' \) wasn't clearly derived. There was some confusion in simplifying the expression.

**Part c)**

- Marks: 1/1

- Feedback: Correctly identified the phenomenon as the Doppler effect and provided a reasonable justification involving frequency shift as \( v/c \to 0 \).

**Total: 4/7**

**General Feedback:** The student showed a good understanding of the concepts but needs to improve on algebraic manipulation and clear derivations to achieve full marks.

---

Solution 2:

**Part a)**

- Marks: 3/4

- Feedback: The solution correctly substituted the transformation terms and made a substantial attempt to simplify the expression. However, the final algebraic simplification was not fully detailed to show explicit invariance.

**Part b)**

- Marks: 1/2

- Feedback: The solution showed the correct use of Lorentz transformations but failed to derive the transformed phase \( k'x' - \omega' t' \) explicitly. Some steps were skipped or not clearly presented.

**Part c)**

- Marks: 1/1

- Feedback: The explanation of the relationship between \( \omega \) and \( \omega' \) as \( v/c \to 0 \) was correct and well-explained, identifying it as the Doppler effect.

**Total: 5/7**

**General Feedback:** The student demonstrated a strong understanding and approach. Greater attention to detail in algebraic manipulation and derivations is needed to secure full marks.

---

Solution 3:

**Part a)**

- Marks: 4/4

- Feedback: This solution correctly substituted the transformation terms and provided a detailed algebraic simplification, explicitly showing that the scalar product is invariant. Excellent work.

**Part b)**

- Marks: 1/2

- Feedback: The initial steps of the phase transformation were correct, but the final simplified form \( k'x' - \omega' t' \) was not explicitly shown. More clarity in the final steps was needed.

**Part c)**

- Marks: 1/1

- Feedback: The relationship between \( \omega \) and \( \omega' \) as \( v/c \to 0 \) was well explained, identifying the phenomenon as the Doppler effect.

**Total: 6/7**

**General Feedback:** The student demonstrated excellent understanding and detailed algebraic work for part a. Slight improvements in the clarity of the final derivations for part b would ensure full marks.

\bigskip
\textbf{Response 2:}

Solution 1:

Part a:

The student correctly starts by identifying the transformations and the scalar products in frames \( S \) and \( S' \). They substitute the given transformations into the scalar product formula:
\[ \underline{E'} \cdot \underline{B'} = E'_{x}B'_{x} + E'_{y}B'_{y} + E'_{z}B'_{z} \]

They recognize that:
\[ E'_{x} = E_{x}, \quad B'_{x} = B_{x} \]
\[ E'_{y} = \gamma(E_{y} - vB_{z}), \quad B'_{y} = \gamma(B_{y} + \frac{v}{c^{2}}E_{z}) \]
\[ E'_{z} = \gamma(E_{z} + vB_{y}), \quad B'_{z} = \gamma(B_{z} - \frac{v}{c^{2}}E_{y}) \]

However, they fail to fully simplify the expression and show explicitly that:
\[ \underline{E'} \cdot \underline{B'} = \underline{E} \cdot \underline{B} \]

**Grade: 2/4**

Part b:

The student correctly identifies the need to use the Lorentz transformations:
\[ x' = \gamma(x - vt), \quad t' = \gamma \left(t - \frac{vx}{c^{2}} \right) \]
They correctly substitute these into the phase \( (kx - \omega t) \) and use the given expressions for \( k' \) and \( \omega' \):
\[ \omega' = \gamma(\omega - kv), \quad k' = \gamma(k - \frac{\omega v}{c^{2}}) \]

They do not explicitly simplify the expression to the desired form, but the approach and reasoning are correct.

**Grade: 2/2**

Part c:

The student correctly identifies the physical phenomenon as the Doppler effect and explains the limiting behavior as \( v/c \rightarrow 0 \).

**Grade: 1/1**

**Total Grade: 5/7**

**Feedback:** Good understanding of the problem and correct approach for parts b and c. Part a needs explicit simplification and final steps to show the invariance of the scalar product.

---

Solution 2:

Part a:

The student outlines the problem well, writing down the transformations correctly. They expand:
\[ \underline{E'} \cdot \underline{B'} = E_{x}B_{x} + \gamma(E_{y} - vB_{z}) \gamma(B_{y} + \frac{v}{c^{2}} E_{z}) + \gamma(E_{z} + vB_{y}) \gamma(B_{z} - \frac{v}{c^{2}} E_{y}) \]

However, they don't fully simplify the expression to show the final step explicitly.

**Grade: 2/4**

Part b:

The student correctly uses the Lorentz transformations and shows the derivation:
\[ x' = \gamma(x - vt), \quad t' = \gamma \left(t - \frac{vx}{c^{2}} \right) \]
They substitute these into the phase and derive the relations for \( k' \) and \( \omega' \).

**Grade: 2/2**

Part c:

The student correctly identifies the Doppler effect and provides an appropriate explanation for the limit \( v/c \rightarrow 0 \).

**Grade: 1/1**

**Total Grade: 5/7**

**Feedback:** Good approach and correct reasoning for parts b and c. Part a needs more detailed simplification to clearly demonstrate invariance.

---

Solution 3:

Part a:

The student writes down the transformations and the scalar product expression:
\[ \underline{E'} \cdot \underline{B'} = E_{x}B_{x} + \gamma(E_{y} - vB_{z})\gamma(B_{y} + \frac{v}{c^{2}}E_{z}) + \gamma(E_{z} + vB_{y})\gamma(B_{z} - \frac{v}{c^{2}}E_{y}) \]

However, they do not fully simplify to show:
\[ \underline{E'} \cdot \underline{B'} = \underline{E} \cdot \underline{B} \]

**Grade: 2/4**

Part b:

The student correctly identifies the need to transform the phase:
\[ \underline{E} = E_{0} \hat{\underline{y}} \exp(i(kx - \omega t)) \]
They use the Lorentz transformations and derive:
\[ \omega' = \gamma(\omega - kv), \quad k' = \gamma(k - \frac{\omega v}{c^{2}}) \]

They substitute these correctly but do not simplify the expression explicitly.

**Grade: 2/2**

Part c:

The student correctly identifies the Doppler effect and explains the behavior as \( v/c \rightarrow 0 \).

**Grade: 1/1**

**Total Grade: 5/7**

**Feedback:** Good understanding and correct steps for parts b and c. Part a needs explicit final steps to show the invariance clearly.

\bigskip
\textbf{Response 3:}

Solution 1

Part a)

To show the invariance of the scalar product \((\underline{E} \cdot \underline{B})\), we substitute the transformations given:

\[
\begin{aligned}
E'_{x} &= E_{x}, & E'_{y} &= \gamma(E_{y} - vB_{z}), & E'_{z} &= \gamma(E_{z} + vB_{y}) \\
B'_{x} &= B_{x}, & B'_{y} &= \gamma(B_{y} + \frac{v}{c^{2}} E_{z}), & B'_{z} &= \gamma(B_{z} - \frac{v}{c^{2}} E_{y})
\end{aligned}
\]

The scalar product in frame \(S'\) is:

\[
\underline{E'} \cdot \underline{B'} = E'_{x} B'_{x} + E'_{y} B'_{y} + E'_{z} B'_{z}
\]

Substituting the transformed fields:

\[
\begin{aligned}
\underline{E'} \cdot \underline{B'} &= E_{x} B_{x} + \gamma(E_{y} - vB_{z}) \gamma(B_{y} + \frac{v}{c^{2}} E_{z}) + \gamma(E_{z} + vB_{y}) \gamma(B_{z} - \frac{v}{c^{2}} E_{y}) \\
&= E_{x} B_{x} + \gamma^2 (E_{y} B_{y} + E_{y} \frac{v}{c^{2}} E_{z} - vB_{z} B_{y} - vB_{z} \frac{v}{c^{2}} E_{z} + E_{z} B_{z} + vE_{z} B_{y} - \frac{v}{c^{2}} E_{y} E_{z} - \frac{v^2}{c^{2}} E_{y} B_{z}) \\
&= E_{x} B_{x} + \gamma^2 \left(E_{y} B_{y} - \frac{v}{c^{2}} E_{y} E_{z} + E_{z} B_{z} - \frac{v^2}{c^{2}} E_{y} B_{z} + \frac{v}{c^{2}} E_{z} E_{y} + v E_{z} B_{y} - v B_{z} B_{y} - \frac{v^2}{c^{2}} B_{z} E_{z}\right)
\end{aligned}
\]

Noting the cancellations:

\[
\begin{aligned}
&= E_{x} B_{x} + \gamma^2 \left(E_{y} B_{y} + E_{z} B_{z}\right)
\end{aligned}
\]

Since \(\gamma^2 (E_{y} B_{y} + E_{z} B_{z}) = E_{y} B_{y} + E_{z} B_{z}\), we have:

\[
\underline{E'} \cdot \underline{B'} = E_{x} B_{x} + E_{y} B_{y} + E_{z} B_{z} = \underline{E} \cdot \underline{B}
\]

Part b)

The phase in \( S \) is \( kx - \omega t \). Using the Lorentz transformations:

\[
x' = \gamma(x - vt), \quad t' = \gamma(t - \frac{vx}{c^{2}})
\]

Substituting:

\[
kx - \omega t = k(\gamma(x' + vt')) - \omega(\gamma(t' + \frac{vx'}{c^2}))
\]

Simplifying:

\[
kx - \omega t = \gamma kx' + \gamma kvt' - \gamma \omega t' - \gamma \omega \frac{vx'}{c^2}
\]

Rearranging terms:

\[
(k - \frac{\omega v}{c^2})\gamma x' - \gamma(\omega - kv)t'
\]

This gives:

\[
k' = \gamma(k - \frac{\omega v}{c^2}), \quad \omega' = \gamma(\omega - kv)
\]

Part c)

As \( v/c \rightarrow 0 \):

\[
\omega' = \gamma (\omega - kv) \rightarrow \omega
\]

This represents the classical Doppler effect, where the frequency observed is approximately the same as the emitted frequency when relative velocity is negligible.

Feedback and Grades

**Solution 1:**

- **Part a):** 3/4 (Good substitution, but some simplification steps missing)

- **Part b):** 2/2 (Correct use of Lorentz transformation)

- **Part c):** 1/1 (Correctly identifies the Doppler effect)

- **Total:** 6/7

- **Feedback:** Clear understanding and correct approach. Simplification could be more explicit.

**Solution 2:**

- **Part a):** 3/4 (Correct substitution but missed some details in simplification)

- **Part b):** 2/2 (Accurate transformation)

- **Part c):** 1/1 (Correct identification of the Doppler effect)

- **Total:** 6/7

- **Feedback:** Good structure and approach, minor issues in the detailed algebra.

**Solution 3:**

- **Part a):** 3/4 (Correct setup, needs more detailed simplification)

- **Part b):** 2/2 (Correctly applies Lorentz transformation)

- **Part c):** 1/1 (Correctly identifies the physical phenomenon)

- **Total:** 6/7

- **Feedback:** Clear understanding, could benefit from more explicit algebraic steps.

\bigskip
\textbf{Response 4:}

Solution 1 Grading:

**Part a)**: 2/4 marks

- Substitution: 1/1 (Correctly substitutes the transformed fields into the scalar product).

- Algebraic Simplification: 1/3 (Attempts to show invariance but the algebra is incomplete and not fully simplified to demonstrate \( E_x B_x + E_y B_y + E_z B_z \)).

**Feedback**: Good start with the substitution. More detailed algebraic steps and simplification are needed to fully show the invariance of the scalar product.

**Part b)**: 1/2 marks

- Inverse Lorentz Transformation: 0/1 (Does not explicitly show the use of inverse Lorentz transformations for \( x \) and \( t \)).

- Simplification: 1/1 (Recognizes the goal but does not complete the steps to show the phase transformation).

**Feedback**: Clear understanding of the goal, but the solution lacks detailed steps to explicitly transform the phase using the Lorentz transformations.

**Part c)**: 1/1 mark

- Doppler Effect Explanation: 1/1 (Correctly identifies the Doppler effect and provides reasoning).

**Feedback**: Correct identification of the Doppler effect with an appropriate explanation.

**Total Marks**: 4/7

Solution 2 Grading:

**Part a)**: 3/4 marks

- Substitution: 1/1 (Correctly substitutes the transformed fields into the scalar product).

- Algebraic Simplification: 2/3 (Provides detailed algebraic steps, but the final simplification and explicit cancellation to demonstrate invariance is not complete).

**Feedback**: Detailed algebraic steps and a strong attempt to simplify, but the final explicit demonstration of invariance is incomplete.

**Part b)**: 2/2 marks

- Inverse Lorentz Transformation: 1/1 (Uses inverse Lorentz transformations correctly).

- Simplification: 1/1 (Rearranges and simplifies to derive the required phase relations).

**Feedback**: Well-executed use of Lorentz transformations and simplification to show the phase transformation.

**Part c)**: 1/1 mark

- Doppler Effect Explanation: 1/1 (Correctly identifies the Doppler effect and provides reasoning).

**Feedback**: Correct identification and explanation of the Doppler effect.

**Total Marks**: 6/7

Solution 3 Grading:

**Part a)**: 3/4 marks

- Substitution: 1/1 (Correctly substitutes the transformed fields into the scalar product).

- Algebraic Simplification: 2/3 (Provides algebraic steps but the explanation and simplification process are somewhat unclear and incomplete).

**Feedback**: Good substitution and algebraic steps but the simplification and final demonstration of invariance need to be clearer.

**Part b)**: 1/2 marks

- Inverse Lorentz Transformation: 1/1 (Uses Lorentz transformations correctly).

- Simplification: 0/1 (The explanation and final step to show the phase transformation are incomplete).

**Feedback**: Correct use of Lorentz transformations, but the solution does not fully simplify and show the phase transformation.

**Part c)**: 1/1 mark

- Doppler Effect Explanation: 1/1 (Correctly identifies the Doppler effect and provides reasoning).

**Feedback**: Correct identification and explanation of the Doppler effect.

**Total Marks**: 5/7

---

Summary:

- **Solution 1**: 4/7

  - Good start and correct identification of Doppler effect, but needs more detailed algebraic simplification in part a) and explicit steps in part b).

- **Solution 2**: 6/7

  - Detailed and well-executed steps, particularly strong in part b), but slightly incomplete in part a)'s final simplification.

- **Solution 3**: 5/7

  - Good substitution and initial algebra, correct Doppler effect identification, but needs clearer final steps in both parts a) and b).

\bigskip
\textbf{Response 5:}

Solution Grading and Feedback

**Solution 1:**

**Part a)**

- Marks: 2/4

- Feedback: The solution starts well by identifying the transformation equations but doesn't fully expand and simplify the algebra to explicitly show the invariance of the scalar product. The final simplified term should be presented.

**Part b)**

- Marks: 1/2

- Feedback: The approach is correct in using the Lorentz transformations, but the actual substitution and simplification to reach the final expressions for \(\omega'\) and \(k'\) are not completed. The steps should clearly show the transformed phase.

**Part c)**

- Marks: 1/1

- Feedback: Correctly identifies the Doppler effect and provides reasoning about the frequency shift when \(v/c \rightarrow 0\).

**Total Marks: 4/7**

**Solution 2:**

**Part a)**

- Marks: 3/4

- Feedback: The solution provides a thorough approach to substituting and expanding the scalar product. However, the simplification process to show the final invariance term \(E_x B_x + E_y B_y + E_z B_z\) is missing.

**Part b)**

- Marks: 2/2

- Feedback: The solution correctly applies Lorentz transformations and logically derives the phase transformation in the \(S'\) frame, reaching the required relations for \(\omega'\) and \(k'\).

**Part c)**

- Marks: 1/1

- Feedback: Accurately describes the physical phenomenon as the Doppler effect and correctly discusses the limit as \(v/c \rightarrow 0\).

**Total Marks: 6/7**

**Solution 3:**

**Part a)**

- Marks: 3/4

- Feedback: The solution correctly identifies the necessary transformations and starts the expansion process. It would benefit from more detailed algebraic steps to reach the final simplified invariance form.

**Part b)**

- Marks: 1/2

- Feedback: The conceptual explanation is good, but the actual derivation and clear presentation of how the phase transforms to \((k'x' - \omega' t')\) is not fully shown. The algebraic steps should be detailed.

**Part c)**

- Marks: 1/1

- Feedback: Correctly identifies the Doppler effect and the limit process, providing the necessary physical interpretation.

**Total Marks: 5/7**

Summary

- **Solution 1: 4/7** - Needs more detailed algebraic simplification and clear steps for phase transformation.

- **Solution 2: 6/7** - Almost perfect; just needs final simplification in part a.

- **Solution 3: 5/7** - Good approach but needs more detailed derivation for part b.

General feedback for all students:

1. Ensure all algebraic steps are clearly shown, especially when proving invariance.

2. When applying Lorentz transformations, carefully substitute and simplify to reach the final required forms.

3. Always conclude with the physical interpretation and context of the results obtained.

\end{document}