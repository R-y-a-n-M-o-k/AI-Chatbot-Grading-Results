\documentclass[a4paper,11pt]{article}
\setlength\parindent{0pt}
\usepackage{amsmath, amssymb}
\usepackage[margin=1in,letterpaper]
{geometry} % decreases margins

\usepackage{setspace}
\setstretch{1.2}
\usepackage{spverbatim}

\title{GPT-4o Grading (With Mark Scheme)}

\begin{document}

\maketitle
This document compiles the GPT-4o responses when prompted to grade the solutions which have been generated. No markscheme is given to GPT-4o in the prompt. This is the grading for the Quantum mechanics (QM) questions.

\section{QM}

\subsection*{Question 1:}

\subsubsection*{Response 1 :}

Solution 1 :

\textbf{Marks : 5/5}

\textbf{Feedback : }

\textbf{Part a : 3/3}

\begin{itemize}
    \item The commutator is correctly defined.
    \item The explanation of compatible operators is accurate and detailed.
    \item The implication about the commutator being zero is well-explained.
\end{itemize}

\textbf{Part b : 2/2}

\begin{itemize}
    \item Clearly states that the initial $p$ value is not necessarily the same as the final $p$ value.
    \item Explains how the wavefunction collapses and the impact on subsequent measurements accurately.
\end{itemize}

\textbf{General feedback :}

\begin{itemize}
    \item Excellent explanation of concepts with clear, detailed reasoning. This solution thoroughly addresses all parts of the problem.
\end{itemize}


Solution 2 :

\textbf{Marks : 5/5}

\textbf{Feedback : }

\textbf{Part a : 3/3}

\begin{itemize}
    \item Correctly defines the commutator.
    \item Explains compatible operators and their common eigenstates accurately.
    \item Clearly states that the commutator is zero for compatible operators.
\end{itemize}

\textbf{Part b : 2/2}

\begin{itemize}
    \item Accurately explains the implications of incompatibility on measurement results.
    \item Discusses the collapse of the wavefunction and how it affects subsequent measurements.
\end{itemize}

\textbf{General feedback :}

\begin{itemize}
    \item A solid, well-structured answer that covers all the required points with clarity and depth. The explanation is concise and complete.
\end{itemize}


Solution 3 :

\textbf{Marks : 5/5}

\textbf{Feedback : }

\textbf{Part a : 3/3}

\begin{itemize}
    \item Defines the commutator correctly.
    \item Provides an accurate definition of compatible operators.
    \item Explains the implication of the commutator being zero.
\end{itemize}

\textbf{Part b : 1/2}

\begin{itemize}
    \item Correctly states that the initial $p$ value is not necessarily the same as the final $p$ value.
    \item Accurately explains the impact of measuring incompatible operators on the wavefunction and subsequent measurements.
\end{itemize}

\textbf{General feedback :}

\begin{itemize}
    \item Well-written and complete answers. The solution addresses all aspects of the problem thoroughly and concisely.
\end{itemize}


\textbf{Summary :}

\begin{itemize}
    \item All three solutions provided excellent answers to the problem, with clear and accurate explanations for both parts. Each solution demonstrates a strong understanding of the concepts and effectively communicates the implications of commutators and operator compatibility in quantum mechanics. Therefore, each solution earns full marks: 5/5.
\end{itemize}


\subsubsection*{Response 2 :} 

Solution 1 :

\textbf{Marks : 5/5}

\textbf{Feedback : }

\textbf{Part a : 3/3}

\begin{itemize}
    \item Commutator Definition (1/1): The commutator is correctly defined as $[\hat{P},\hat{Q}] = \hat{P}\hat{Q} - \hat{Q}\hat{P}$
    \item Compatibility (1/1): The student explains compatibility by stating that compatible operators commute and their commutator is zero.
    \item Implication (1/1): The student correctly states that compatible operators share a common set of eigenstates and that their physical quantities can be determined simultaneously.
\end{itemize}

\textbf{Part b : 2/2}

\begin{itemize}
    \item Initial and Final p Value (1/1): The student correctly states that the initial $p$ value is not necessarily the same as the final $p$ value measured.
    \item Wavefunction Collapse (1/1): The explanation regarding the wavefunction collapse and the uncertainty in measurement outcomes is accurate and well-articulated.
\end{itemize}

\textbf{General feedback :}

\begin{itemize}
    \item This solution is thorough and accurate, providing a clear and comprehensive explanation of both parts of the problem.
\end{itemize}


Solution 2 :

\textbf{Marks : 5/5}

\textbf{Feedback :}

\textbf{Part a : 3/3}

\begin{itemize}
    \item Commutator Definition (1/1): The commutator is correctly defined as $[\hat{P},\hat{Q}] = \hat{P}\hat{Q} - \hat{Q}\hat{P}$
    \item Compatibility (1/1): The student explains compatibility by stating that compatible operators commute and their commutator is zero.
    \item Implication (1/1): The student correctly states that compatible operators share a common set of eigenstates and that their physical quantities can be determined simultaneously.
\end{itemize}

\textbf{Part b : 2/2}

\begin{itemize}
    \item Initial and Final p Value (1/1): The student correctly states that the initial $p$ value is not necessarily the same as the final $p$ value measured.
    \item Wavefunction Collapse (1/1): The explanation regarding the wavefunction collapse and the uncertainty in measurement outcomes is accurate and well-articulated.
\end{itemize}

\textbf{General feedback :}

\begin{itemize}
    \item This solution is precise and correctly explains the concepts, offering a solid understanding of the principles involved.
\end{itemize}


Solution 3 :

\textbf{Marks : 5/5}

\textbf{Feedback : }

\textbf{Part a : 3/3}

\begin{itemize}
    \item Commutator Definition (1/1): The commutator is correctly defined as $[\hat{P},\hat{Q}] = \hat{P}\hat{Q} - \hat{Q}\hat{P}$
    \item Compatibility (1/1): The student explains compatibility by stating that compatible operators commute and their commutator is zero.
    \item Implication (1/1): The student correctly states that compatible operators share a common set of eigenstates and that their physical quantities can be determined simultaneously.
\end{itemize}

\textbf{Part b : 2/2}

\begin{itemize}
    \item Initial and Final p Value (1/1): The student correctly states that the initial $p$ value is not necessarily the same as the final $p$ value measured.
    \item Wavefunction Collapse (1/1): The explanation regarding the wavefunction collapse and the uncertainty in measurement outcomes is accurate and well-articulated.
\end{itemize}

\textbf{General feedback :}

\begin{itemize}
    \item This solution is comprehensive and correctly addresses all aspects of the problem, demonstrating a clear understanding of the concepts.
\end{itemize}



\subsubsection*{Response 3 :} 

Solution 1 :

\textbf{Marks : 5/5}

\textbf{Feedback : }

\textbf{Part a : 3/3}

\begin{itemize}
    \item Definition of the commutator: $[\hat{P},\hat{Q}] = \hat{P}\hat{Q} - \hat{Q}\hat{P}$[1 mark]
    \item Definition of compatible operators: correctly identifies they commute and have a common set of eigenstates [1 mark]
    \item Implication on commutator: $[\hat{P},\hat{Q}] = 0$ [1 mark]
\end{itemize}

\textbf{Part b : 2/2}

\begin{itemize}
    \item Explanation that the initial $p$ value is not necessarily the same as the final $p$ value [1 mark]
    \item Correct explanation of wavefunction collapse and the uncertainty of the subsequent measurement [1 mark]
\end{itemize}

\textbf{General feedback :}

\begin{itemize}
    \item Excellent work. You provided clear definitions and correct implications for the commutator and compatibility. Your explanation of the impact of non-compatibility on measurement outcomes is thorough and accurate.
\end{itemize}


Solution 2 :

\textbf{Marks : 5/5}

\textbf{Feedback :}

\textbf{Part a : 3/3}

\begin{itemize}
    \item Definition of the commutator: $[\hat{P},\hat{Q}] = \hat{P}\hat{Q} - \hat{Q}\hat{P}$[1 mark]
    \item Definition of compatible operators: correctly identifies they commute and have a common set of eigenstates [1 mark]
    \item Implication on commutator: $[\hat{P},\hat{Q}] = 0$ [1 mark]
\end{itemize}

\textbf{Part b : 2/2}

\begin{itemize}
    \item Explanation that the initial $p$ value is not necessarily the same as the final $p$ value [1 mark]
    \item Correct explanation of wavefunction collapse and the uncertainty of the subsequent measurement [1 mark]
\end{itemize}


\textbf{General feedback :}

\begin{itemize}
    \item Well done. You accurately described the commutator, compatibility, and their implications. Your explanation for the effect of non-compatible measurements is detailed and precise.
\end{itemize}


Solution 3 :

\textbf{Marks : 5/5}

\textbf{Feedback : }

\textbf{Part a : 3/3}

\begin{itemize}
    \item Definition of the commutator: $[\hat{P},\hat{Q}] = \hat{P}\hat{Q} - \hat{Q}\hat{P}$[1 mark]
    \item Definition of compatible operators: correctly identifies they commute and have a common set of eigenstates [1 mark]
    \item Implication on commutator: $[\hat{P},\hat{Q}] = 0$ [1 mark]
\end{itemize}

\textbf{Part b : 2/2}

\begin{itemize}
    \item Explanation that the initial $p$ value is not necessarily the same as the final $p$ value [1 mark]
    \item Correct explanation of wavefunction collapse and the uncertainty of the subsequent measurement [1 mark]
\end{itemize}

\textbf{General feedback :}

\begin{itemize}
    \item Great job. Your definitions and explanations are clear and accurate, and you have effectively communicated the implications of operator compatibility and measurement sequence in quantum mechanics.
\end{itemize}


\subsubsection*{Response 4 :} 

Solution 1 :

\textbf{Marks : 5/5}

\textbf{Feedback : }

\textbf{Part a : 3/3}

\begin{itemize}
    \item Excellent explanation. The student correctly defines the commutator and explains compatibility and its implication for the commutator. Additionally, the description of what compatibility implies about the operators' eigenstates and measurements is clear.
\end{itemize}

\textbf{Part b : 2/2}

\begin{itemize}
    \item Clear and accurate explanation of the measurement process and the impact on the system state due to incompatible operators. The student correctly identifies that the final $p$ value is uncertain due to the measurement of $\hat{Q}$.

\end{itemize}

\textbf{General feedback :}

\begin{itemize}
    \item This solution demonstrates a comprehensive understanding of both the commutator and the implications of operator compatibility and non-compatibility.
\end{itemize}


Solution 2 :

\textbf{Marks : 5/5}

\textbf{Feedback :}

\textbf{Part a : 3/3}

\begin{itemize}
    \item The student provides a correct definition of the commutator and a clear explanation of what it means for operators to be compatible, including the implications for eigenstates and measurements.
\end{itemize}

\textbf{Part b : 2/2}

\begin{itemize}
    \item The explanation is accurate and thorough, covering the collapse of the wavefunction and the resulting uncertainty in subsequent measurements of $\hat{P}$. The student correctly notes the non-commutative nature and its consequences for measurement outcomes.
\end{itemize}


\textbf{General feedback :}

\begin{itemize}
    \item This solution is very detailed and demonstrates a strong grasp of the concepts involved, with a clear and concise explanation.
\end{itemize}


Solution 3 :

\textbf{Marks : 5/5}

\textbf{Feedback : }

\textbf{Part a : 3/3}

\begin{itemize}
    \item The commutator is correctly defined, and the explanation of compatibility and its implications is precise. The student covers the key points about eigenstates and the ability to measure observables simultaneously.
\end{itemize}

\textbf{Part b : 2/2}

\begin{itemize}
    \item The student provides a clear and correct explanation of the measurement process and the resulting impact on the system state when the operators are incompatible. The uncertainty in subsequent measurements is well-explained.
\end{itemize}

\textbf{General feedback :}

\begin{itemize}
    \item This solution is accurate and well-expressed, showing a solid understanding of the principles of quantum mechanics related to commutators and operator compatibility.
\end{itemize}


\subsubsection*{Response 5 :} 

Solution 1 :

\textbf{Marks : 5/5}

\textbf{Feedback : }

\textbf{Part a : 3/3}

\begin{itemize}
    \item Correct definition of the commutator. (1 mark)
    \item Clear explanation of compatibility and the condition for the commutator to be zero. (1 mark)
    \item Explanation of the implications of compatibility is accurate and well-expressed. (1 mark)
\end{itemize}

\textbf{Part b : 2/2}

\begin{itemize}
    \item Correctly states that the initial $p$ value may not be the same as the final $p$ value. (1 mark)
    \item Explanation of the wavefunction collapsing and the resulting uncertainty in the measurement outcomes is clear and correct. (1 mark)
\end{itemize}

\textbf{General feedback :}

\begin{itemize}
    \item Excellent work. The student provided clear and accurate definitions and explanations for both parts. The answers were well-articulated and demonstrated a good understanding of the concepts.
\end{itemize}


Solution 2 :

\textbf{Marks : 4/5}

\textbf{Feedback :}

\textbf{Part a : 2/3}

\begin{itemize}
    \item Correct definition of the commutator. (1 mark)
    \item Adequate explanation of compatibility but missed directly stating that compatible operators share a common, complete set of eigenfunctions. (0 mark)
    \item Explanation of what compatibility implies about the commutator is clear and correct. (1 mark)
\end{itemize}

\textbf{Part b : 2/2}

\begin{itemize}
    \item Correctly explains that the initial $p$ value may not be the same as the final $p$ value. (1 mark)
    \item Accurate and clear explanation about the wavefunction collapsing and the resulting uncertainty in subsequent measurements. (1 mark)
\end{itemize}


\textbf{General feedback :}

\begin{itemize}
    \item Good solution overall. The definition and explanation were mostly clear and correct, though it missed explicitly mentioning the complete set of eigenfunctions. The response to part b) was thorough and accurate.
\end{itemize}


Solution 3 :

\textbf{Marks : 5/5}

\textbf{Feedback : }

\textbf{Part a : 3/3}

\begin{itemize}
    \item Correct definition of the commutator. (1 mark)
    \item Clear explanation of compatibility and the condition for the commutator to be zero. (1 mark)
    \item Explanation of the implications of compatibility is accurate and well-expressed. (1 mark)
\end{itemize}

\textbf{Part b : 2/2}

\begin{itemize}
    \item Correctly states that the initial $p$ value may not be the same as the final $p$ value. (1 mark)
    \item Accurate and clear explanation of the wavefunction collapse and the resulting measurement uncertainty. (1 mark)
\end{itemize}

\textbf{General feedback :}

\begin{itemize}
    \item Excellent work. The student provided clear and accurate definitions and explanations for both parts. The answers were well-articulated and demonstrated a strong understanding of the concepts.
\end{itemize}


\subsection*{Question 2:}

\subsubsection*{Response 1:}

Solution 1 : 

\textbf{Marks : 4/4}

\textbf{Feedback : }

\textbf{Part a : 2/2}

\begin{itemize}
    \item [1 mark] Correctly states that the eigenvalues of a Hermitian operator are real.
    \item [1 mark] Clearly explains the physical interpretation that eigenvalues represent possible measurement outcomes.
\end{itemize}


\textbf{Part b : 2/2}

\begin{itemize}
    \item [1 mark] Correctly explains orthogonality with the appropriate mathematical expression.
    \item [1 mark] Clearly describes normalization with the appropriate mathematical expression.
\end{itemize}


\textbf{General feedback :}

\begin{itemize}
    \item Excellent work! The solution is comprehensive and accurately covers both the mathematical properties and physical interpretations. The explanation of orthonormality is clear and mathematically precise.
\end{itemize}


Solution 2 : 

\textbf{Marks : 4/4}

\textbf{Feedback : }

\textbf{Part a : 2/2}

\begin{itemize}
    \item [1 mark] Correctly states that the eigenvalues of a Hermitian operator are real.
    \item [1 mark] Provides a clear physical interpretation that eigenvalues represent observable physical quantities.
\end{itemize}


\textbf{Part b : 2/2}

\begin{itemize}
    \item [1 mark] Correctly explains orthogonality with the appropriate mathematical expression.
    \item [1 mark] Describes normalization correctly with the Kronecker delta notation.
\end{itemize}


\textbf{General feedback :}

\begin{itemize}
    \item Well done! The solution effectively communicates both the mathematical properties and the physical interpretations. The explanation of orthonormality is precise and correctly uses the Kronecker delta notation.
\end{itemize}


Solution 3 : 

\textbf{Marks : 4/4}

\textbf{Feedback : }

\textbf{Part a : 2/2}

\begin{itemize}
    \item [1 mark] Correctly states that the eigenvalues of a Hermitian operator are real.
    \item [1 mark] Clearly explains the physical significance that eigenvalues represent real, observable quantities.
\end{itemize}


\textbf{Part b : 2/2}

\begin{itemize}
    \item [1 mark] Correctly explains orthogonality with the appropriate mathematical expression.
    \item [1 mark] Describes normalization correctly with the Kronecker delta notation.
\end{itemize}


\textbf{General feedback :}

\begin{itemize}
    \item Great job! The solution accurately describes both the mathematical properties and the physical interpretations. The explanation of orthonormality is clear and precise, utilizing the Kronecker delta correctly.
\end{itemize}

\textbf{Summary :}
All three solutions correctly address the mathematical properties and physical interpretations of Hermitian operators and their eigenfunctions. They use appropriate mathematical expressions and provide clear, concise explanations.


\subsubsection*{Response 2 :}

Solution 1 : 

\textbf{Marks : 4/4}

\textbf{Feedback : }

\textbf{Part a : 2/2}

\begin{itemize}
    \item The eigenvalues of a Hermitian operator are real. (1 mark)
    \item The explanation about the physical interpretation of the eigenvalues being the possible outcomes of a measurement is correct. (1 mark)
\end{itemize}


\textbf{Part b : 2/2}

\begin{itemize}
    \item Correctly explains orthogonality : $<\phi_m|\phi_n> = 0$ for $m \neq n$. (1 mark)
    \item Correctly explains normalisation : $<\phi_n|\phi_n> = 1$. (1 mark)
\end{itemize}


\textbf{General feedback :}

\begin{itemize}
    \item Excellent solution. The explanation is detailed and covers all necessary points accurately. Well done.
\end{itemize}


Solution 2 : 

\textbf{Marks : 4/4}

\textbf{Feedback : }

\textbf{Part a : 2/2}

\begin{itemize}
    \item The eigenvalues of a Hermitian operator are real. (1 mark)
    \item Correctly mentions the physical interpretation that the eigenvalues represent observable physical quantities. (1 mark)
\end{itemize}


\textbf{Part b : 2/2}

\begin{itemize}
    \item Correctly explains orthogonality : $<\phi_m|\phi_n> = \delta_{nm}$. (1 mark)
    \item Correctly explains normalization. (1 mark)
\end{itemize}


\textbf{General feedback :}

\begin{itemize}
    \item Good solution with clear and correct explanations. It adequately covers the required concepts.
\end{itemize}


Solution 3 : 

\textbf{Marks : 4/4}

\textbf{Feedback : }

\textbf{Part a : 2/2}

\begin{itemize}
    \item The eigenvalues of a Hermitian operator are real. (1 mark)
    \item Correctly mentions the physical interpretation that the eigenvalues are real and represent observable quantities. (1 mark)
\end{itemize}


\textbf{Part b : 2/2}

\begin{itemize}
    \item Correctly explains orthogonality : $<\phi_i|\phi_j> = \delta_{ij}$. (1 mark)
    \item Correctly explains normalization. (1 mark)
\end{itemize}


\textbf{General feedback :}

\begin{itemize}
    \item Thorough and correct explanations. The solution is well-articulated and accurate.
\end{itemize}



\subsubsection*{Response 3 :}

Solution 1 : 

\textbf{Marks : 4/4}

\textbf{Feedback : }

\textbf{Part a : 2/2}

\begin{itemize}
    \item Real eigenvalues: Clearly stated and explained. [1 mark]
    \item Physical interpretation: Well-articulated explanation of the significance of eigenvalues in measurements. [1 mark]
\end{itemize}


\textbf{Part b : 2/2}

\begin{itemize}
    \item Orthogonality: Correctly explains the concept and provides the mathematical expression. [1 mark]
    \item Normalization: Clearly describes normalization and provides the correct mathematical expression. [1 mark]
\end{itemize}


\textbf{General feedback :}

\begin{itemize}
    \item This solution is well-written and complete, providing thorough explanations for both parts of the problem. Good use of mathematical expressions to support the explanations.
\end{itemize}


Solution 2 : 

\textbf{Marks : 4/4}

\textbf{Feedback : }

\textbf{Part a : 2/2}

\begin{itemize}
    \item Correctly identifies the property and explains it clearly. [1 mark]
    \item Explanation is good but lacks a bit of clarity compared to Solution 1. [1 mark]
\end{itemize}


\textbf{Part b : 2/2}

\begin{itemize}
    \item Orthogonality: Correctly describes the orthogonality condition. [1 mark]
    \item Normalization: Correct explanation and use of the Kronecker delta. [1 mark]
\end{itemize}


\textbf{General feedback :}

\begin{itemize}
    \item The solution is accurate and comprehensive. The explanations are clear, and the use of mathematical notation is appropriate and correct.
\end{itemize}


Solution 3 : 

\textbf{Marks : 4/4}

\textbf{Feedback : }

\textbf{Part a : 2/2}

\begin{itemize}
    \item Real eigenvalues: Correctly identifies the property and gives a thorough mathematical explanation. [1 mark]
    \item Physical interpretation: Explanation is good but a bit too verbose. [1 mark]
\end{itemize}


\textbf{Part b : 2/2}

\begin{itemize}
    \item Orthogonality: Correctly describes orthogonality and provides the necessary mathematical expression. [1 mark]
    \item Normalization: Explanation is correct and includes the Kronecker delta. [1 mark]
\end{itemize}


\textbf{General feedback :}

\begin{itemize}
    \item This solution provides a clear and accurate explanation, though it could be more concise. The mathematical details are well-presented and correct.
\end{itemize}

\textbf{Summary :}
All three solutions received full marks as they correctly answered both parts of the problem and provided accurate and detailed explanations.



\subsubsection*{Response 4 :}

Solution 1 : 

\textbf{Marks : 4/4}

\textbf{Feedback : }

\textbf{Part a : 2/2}

\begin{itemize}
    \item The eigenvalues are real: 1 mark
    \item Physical interpretation: 1 mark
\end{itemize}


\textbf{Part b : 2/2}

\begin{itemize}
    \item Orthogonality: 1 mark
    \item Normalization: 1 mark
\end{itemize}


\textbf{General feedback :}

\begin{itemize}
    \item The solution correctly identifies the real nature of the eigenvalues and provides a clear physical interpretation. The explanation for orthonormality is accurate, mentioning both orthogonality and normalization clearly.
\end{itemize}


Solution 2 : 

\textbf{Marks : 4/4}

\textbf{Feedback : }

\textbf{Part a : 2/2}

\begin{itemize}
    \item The eigenvalues are real: 1 mark
    \item Physical interpretation: 1 mark
\end{itemize}


\textbf{Part b : 2/2}

\begin{itemize}
    \item Orthogonality: 1 mark
    \item Normalization: 1 mark
\end{itemize}


\textbf{General feedback :}

\begin{itemize}
    \item This solution accurately explains both the mathematical and physical aspects of Hermitian operators and their eigenvalues. The explanation of orthonormality is clear and correct, covering both orthogonality and normalization.
\end{itemize}


Solution 3 : 

\textbf{Marks : 4/4}

\textbf{Feedback : }

\textbf{Part a : 2/2}

\begin{itemize}
    \item The eigenvalues are real: 1 mark
    \item Physical interpretation: 1 mark
\end{itemize}


\textbf{Part b : 2/2}

\begin{itemize}
    \item Orthogonality: 1 mark
    \item Normalization: 1 mark
\end{itemize}


\textbf{General feedback :}

\begin{itemize}
    \item The solution effectively explains the properties of Hermitian operators and their eigenvalues, as well as the concept of orthonormality. Both parts of the question are answered correctly and concisely.
\end{itemize}



\subsubsection*{Response 5 :}

Solution 1 : 

\textbf{Marks : 4/4}

\textbf{Feedback : }

\textbf{Part a : 2/2}

\begin{itemize}
    \item Correctly states that eigenvalues of Hermitian operators are real (1 mark).
    \item Correctly explains the physical interpretation of eigenvalues as possible measurement outcomes (1 mark).
\end{itemize}


\textbf{Part b : 2/2}

\begin{itemize}
    \item Accurately describes orthogonality and provides the integral form for orthogonal functions (1 mark).
    \item Correctly explains normalization and provides the integral form for normalized functions (1 mark).
\end{itemize}


\textbf{General feedback :}

\begin{itemize}
    \item Excellent explanation with clear mathematical and physical interpretations.
    \item Well-structured and concise.
\end{itemize}


Solution 2 : 

\textbf{Marks : 4/4}

\textbf{Feedback : }

\textbf{Part a : 2/2}

\begin{itemize}
    \item Correctly states that eigenvalues of Hermitian operators are real (1 mark).
    \item Correctly explains the physical interpretation of eigenvalues as observable quantities (1 mark).
\end{itemize}


\textbf{Part b : 2/2}

\begin{itemize}
    \item Accurately describes orthogonality using the Kronecker delta (1 mark).
    \item Correctly explains normalization and its significance (1 mark).
\end{itemize}


\textbf{General feedback :}

\begin{itemize}
    \item Clear and precise explanations with appropriate mathematical representations.
    \item Good use of the Kronecker delta to explain orthonormality.
\end{itemize}


Solution 3 : 

\textbf{Marks : 4/4}

\textbf{Feedback : }

\textbf{Part a : 2/2}

\begin{itemize}
    \item Correctly states that eigenvalues of Hermitian operators are real (1 mark).
    \item Correctly explains the physical interpretation of eigenvalues as observable quantities (1 mark).

\end{itemize}


\textbf{Part b : 2/2}

\begin{itemize}
    \item Accurately describes orthogonality using the Kronecker delta (1 mark).
    \item Correctly explains normalization and its significance (1 mark).
\end{itemize}


\textbf{General feedback :}

\begin{itemize}
    \item Well-structured and clear explanations.
    \item Good mathematical representation and interpretation of orthonormality.
\end{itemize}

\textbf{Summary :}
All three solutions provided accurate and complete answers to both parts of the problem, demonstrating a strong understanding of the properties and significance of Hermitian operators and their eigenfunctions in quantum mechanics.






\subsection*{Question 3:}

\subsubsection*{Response 1 :}

Solution 1 :

\textbf{Marks : 9/9}

\textbf{Feedback :}

\textbf{Part a : 4/4}

\begin{itemize}
    \item Substitution and Integration by Parts: 2/2
    \item Simplification and Final Statement: 2/2
\end{itemize}

\textbf{Feedback :}
The solution correctly substitutes the operator and performs integration by parts accurately. The final simplification is clear, and the conclusion that $\hat{p}$ is Hermitian is well-stated.


\textbf{Part b : 5/5}

\begin{itemize}
    \item Substitution and Integration by Parts (first pass): 2/2
    \item Integration by Parts (second pass) and Boundary Terms: 1/1
    \item Final Arrangement and Statement: 2/2
\end{itemize}

\textbf{Feedback :}
The explanation is thorough and detailed, demonstrating both integration by parts steps clearly and correctly. The conclusion that $\hat{Q}$ is Hermitian is well-supported.




Solution 2 :

\textbf{Marks : 9/9}

\textbf{Feedback :}

\textbf{Part a : 4/4}

\begin{itemize}
    \item Substitution and Integration by Parts: 2/2
    \item Simplification and Final Statement: 2/2
\end{itemize}

\textbf{Feedback :}
The solution correctly substitutes the operator and performs integration by parts accurately. The steps are clear, and the conclusion is correct.


\textbf{Part b : 5/5}

\begin{itemize}
    \item Substitution and Integration by Parts (first pass): 2/2
    \item Integration by Parts (second pass) and Boundary Terms: 1/1
    \item Final Arrangement and Statement: 2/2
\end{itemize}

\textbf{Feedback :}
The solution is detailed and follows the required steps. The integration by parts is performed correctly, and the conclusion is clearly supported.



Solution 3 :

\textbf{Marks : 7/9}

\textbf{Feedback :}

\textbf{Part a : 3/4}

\begin{itemize}
    \item Substitution and Integration by Parts: 2/2
    \item Simplification and Final Statement: 1/2
\end{itemize}

\textbf{Feedback :}
The solution correctly substitutes the operator and performs integration by parts. However, the final simplification step is not as clearly presented as in the previous solutions, and it lacks the final statement confirming $\hat{p}$ is Hermitian.


\textbf{Part b : 4/5}

\begin{itemize}
    \item Substitution and Integration by Parts (first pass): 2/2
    \item Integration by Parts (second pass) and Boundary Terms: 0/1
    \item Final Arrangement and Statement: 2/2
\end{itemize}

\textbf{Feedback :}
The solution is generally correct, but skipping the detailed steps of the second integration by parts and boundary terms leads to a loss of clarity. The conclusion is correct, but the missing steps make the argument less convincing.


\textbf{Summary :}

\begin{itemize}
    \item Solution 1: Excellent clarity and detail in both parts, covering all necessary steps. Total: 9/9
    \item Solution 2: Very good explanation and step-by-step process, resulting in a clear conclusion. Total: 9/9
    \item Solution 3: Good attempt but lacks some clarity and detail in the final simplification and second part's integration by parts. Total: 7/9
\end{itemize}



\subsubsection*{Response 2 :}

Solution 1 :

\textbf{Marks : 9/9}

\textbf{Feedback :}

\textbf{Part a : 4/4}

\begin{itemize}
    \item Substituted $\hat{p}$ into the LHS and RHS: 2 marks.
    \item Used integration by parts correctly and simplified: 2 marks.
    \item Correctly concluded that $\hat{p}$ is Hermitian.
\end{itemize}


\textbf{Part b : 5/5}

\begin{itemize}
    \item Substituted the operator $\hat{Q}$ into the integral: 2 marks.
    \item Applied integration by parts once and discussed boundary terms: 1 mark.
    \item Correctly applied integration by parts again, simplifying and concluding correctly: 2 marks.
    \item Correct conclusion that $\hat{Q}$ is Hermitian.
\end{itemize}

\textbf{Feedback :}
Excellent work! The solution is detailed and correctly applies the necessary steps to show that both operators are Hermitian. Well done on covering all parts of the problem comprehensively.




Solution 2 :

\textbf{Marks : 6/9}

\textbf{Feedback :}

\textbf{Part a : 3/4}

\begin{itemize}
    \item Substituted $\hat{p}$ into the LHS and RHS: 2 marks.
    \item Integration by parts was mentioned, but steps were less clearly laid out: 1 mark.
    \item Correctly concluded that $\hat{p}$ is Hermitian.
\end{itemize}


\textbf{Part b : 3/5}

\begin{itemize}
    \item Substituted the operator $\hat{Q}$ into the integral: 2 marks.
    \item Mentioned applying integration by parts twice but did not show clear steps: 1 mark.
    \item Correct conclusion that $\hat{Q}$ is Hermitian.
\end{itemize}

\textbf{Feedback :}

Good attempt. The solution shows understanding of the concepts, but it lacks detailed steps for integration by parts, especially in part (b). More clarity and detail in the intermediate steps would improve the explanation.



Solution 3 :

\textbf{Marks : 8/9}

\textbf{Feedback :}

\textbf{Part a : 4/4}

\begin{itemize}
    \item Substituted $\hat{p}$ into the LHS and RHS: 2 marks.
    \item Used integration by parts correctly and simplified: 2 marks.
    \item Correctly concluded that $\hat{p}$ is Hermitian.
\end{itemize}


\textbf{Part b : 4/5}

\begin{itemize}
    \item Substituted the operator $\hat{Q}$ into the integral: 2 marks.
    \item Applied integration by parts twice and discussed boundary terms: 1 mark.
    \item Mentioned the steps but skipped intermediate parts; nonetheless, conclusion correct: 1 mark.
\end{itemize}

\textbf{Feedback :}

Very good solution. It clearly explains the process in part (a). For part (b), while the explanation is correct, skipping intermediate steps affects clarity. Ensure to include all steps in the future for full marks.



\subsubsection*{Response 3 :}

Solution 1 :

\textbf{Marks : 9/9}

\textbf{Feedback :}

\textbf{Part a : 4/4}

\begin{itemize}
    \item Substituted $\hat{p}$ into the integral and correctly applied integration by parts (2 marks).
    \item Correctly simplified and showed LHS equals RHS, concluding $\hat{p}$ is Hermitian (2 marks).
\end{itemize}


\textbf{Part b : 5/5}

\begin{itemize}
    \item Substituted $\hat{Q}$ into the integral and applied integration by parts correctly twice (3 marks).
    \item Correctly simplified and showed LHS equals RHS, concluding $\hat{Q}$ is Hermitian (2 marks).
\end{itemize}

\textbf{Feedback :}

Excellent solution. You applied integration by parts correctly and showed all steps clearly, leading to the correct conclusion for both operators.



Solution 2 :

\textbf{Marks : 5/9}

\textbf{Feedback :}

\textbf{Part a : 2/4}

\begin{itemize}
    \item Substituted $\hat{p}$ into the integral but made an error in the use of integration by parts, resulting in incomplete matching of LHS and RHS (1 mark).
    \item Simplification and conclusion are present but not fully convincing due to the earlier error (1 mark).
\end{itemize}


\textbf{Part b : 3/5}

\begin{itemize}
    \item Substituted $\hat{Q}$ into the integral and mentioned integration by parts but did not show complete detailed steps (2 marks).
    \item Correctly concluded that $\hat{Q}$ is Hermitian but skipped key detailed steps (1 mark).
\end{itemize}

\textbf{Feedback :}

Good attempt. Your approach is correct but needs more detail, especially in the integration by parts steps to fully convince the conclusion.



Solution 3 :

\textbf{Marks : 7/9}

\textbf{Feedback :}

\textbf{Part a : 4/4}

\begin{itemize}
    \item Substituted $\hat{p}$ into the integral and correctly applied integration by parts (2 marks).
    \item Simplified and correctly showed LHS equals RHS, concluding $\hat{p}$ is Hermitian (2 marks).
\end{itemize}


\textbf{Part b : 3/5}

\begin{itemize}
    \item Substituted $\hat{Q}$ into the integral and mentioned integration by parts but skipped detailed intermediate steps (2 marks).
    \item Concluded that $\hat{Q}$ is Hermitian but with insufficient detail (1 mark).
\end{itemize}

\textbf{Feedback :}

Very good solution for part a). Part b) is correct in conclusion but lacks some intermediate details in integration by parts to be fully convincing.


\textbf{Summary :}
\begin{itemize}
    \item Solution 1 is excellent, covering all steps and reasoning clearly.
    \item Solution 2 has a good approach but lacks detail, especially in critical steps.
    \item Solution 3 is strong in part a) but needs more detail in part b) to fully justify the conclusions.
\end{itemize}


\subsubsection*{Response 4 :}

Solution 1 :

\textbf{Marks : 9/9}

\textbf{Feedback :}

\textbf{Part a : 4/4}

\begin{itemize}
    \item Correct substitution of $\hat{p}$ into the integral.
    \item Correct integration by parts with removal of vanishing terms.
    \item Correct simplification to show $\hat{p}$ is Hermitian. 
\end{itemize}


\textbf{Part b : 5/5}

\begin{itemize}
    \item Correct substitution of $\hat{Q}$ into the integral.
    \item Correct application of integration by parts twice.
    \item Correct simplification to show $\hat{Q}$ is Hermitian.
\end{itemize}

\textbf{Feedback :}

Excellent work. Clear and accurate use of integration by parts and logical progression through the problem. Both parts are well-explained and correctly solved.



Solution 2 :

\textbf{Marks : 7/9}

\textbf{Feedback :}

\textbf{Part a : 3/4}

\begin{itemize}
    \item Correct substitution of $\hat{p}$ into the integral.
    \item Attempted integration by parts but unclear in the presentation.
    \item The conclusion is correct, but the steps are a bit convoluted.
\end{itemize}


\textbf{Part b : 4/5}

\begin{itemize}
    \item Correct substitution of $\hat{Q}$ into the integral.
    \item Mentioned integration by parts but skipped detailed steps.
    \item Correct final conclusion but lacks detailed intermediate steps.
\end{itemize}

\textbf{Feedback :}

Good effort. Part a is mostly correct but could use a clearer explanation of integration by parts. Part b is correct in conclusion but would benefit from more detailed steps.



Solution 3 :

\textbf{Marks : 6/9}

\textbf{Feedback :}

\textbf{Part a : 3/4}

\begin{itemize}
    \item Correct substitution of $\hat{p}$ into the integral.
    \item Correct use of integration by parts but skipped some detailed explanation.
    \item The final conclusion is correct.
\end{itemize}


\textbf{Part b : 3/5}

\begin{itemize}
    \item Correct substitution of $\hat{Q}$ into the integral.
    \item Mentioned integration by parts but did not provide detailed steps.
    \item Conclusion is correct but the process lacks detail.
\end{itemize}

\textbf{Feedback :}

A solid attempt. Part a shows correct use of integration by parts but could be more detailed. Part b reaches the correct conclusion but skips important steps in the explanation. Adding more intermediate steps would improve clarity and completeness.




\subsubsection*{Response 5 :}

Solution 1 :

\textbf{Marks : 8/9}

\textbf{Feedback :}

\textbf{Part a : 3/4}

\begin{itemize}
    \item The solution correctly substitutes the operator $\hat{p} = -i\hbar \frac{d}{dx}$ into the definition of a Hermitian operator.
    \item The steps include using integration by parts and considering boundary terms, but there is a lack of explicitness in showing intermediate steps, especially in how integration by parts is applied.
    \item The final conclusion correctly identifies $\hat{p}$ as Hermitian.
\end{itemize}


\textbf{Part b : 5/5}

\begin{itemize}
    \item The solution effectively substitutes $\hat{Q} = \frac{d^2}{dx^2}$ and uses integration by parts twice, correctly considering boundary terms vanishing.
    \item The steps are clearly shown, and the conclusion that $\hat{Q}$ is Hermitian is correct.
\end{itemize}

\textbf{Feedback :}

Good solution overall, but more clarity on the integration by parts process in part (a) would make it excellent.



Solution 2 :

\textbf{Marks : 6/9}

\textbf{Feedback :}

\textbf{Part a : 2/4}

\begin{itemize}
    \item The solution correctly identifies the need to use integration by parts but incorrectly applies it in the context of proving Hermiticity.
    \item The integration by parts approach is not fully demonstrated, leading to a lack of clarity in the steps.
    \item he conclusion that $\hat{p}$ is Hermitian is correct but not sufficiently justified.
\end{itemize}


\textbf{Part b : 4/5}

\begin{itemize}
    \item The approach is correct, but the steps are not clearly detailed.
    \item It correctly uses integration by parts and concludes that $\hat{Q}$ is Hermitian.
\end{itemize}

\textbf{Feedback :}

The solution is on the right track but lacks clarity and detailed steps, especially in part (a). More explicit steps would strengthen the argument.



Solution 3 :

\textbf{Marks : 8/9}

\textbf{Feedback :}

\textbf{Part a : 4/4}

\begin{itemize}
    \item The solution correctly substitutes $\hat{p} = -i\hbar \frac{d}{dx}$ and uses integration by parts properly.
    \item Clear steps are shown, and the boundary terms are considered, leading to the correct conclusion.
\end{itemize}


\textbf{Part b : 4/5}

\begin{itemize}
    \item The substitution of $\hat{Q} = \frac{d^2}{dx^2}$ is correct, and integration by parts is used appropriately.
    \item However, the solution skips intermediate steps and jumps to the conclusion, which might confuse some readers.
\end{itemize}

\textbf{Feedback :}

Strong and correct approach overall, but more detailed intermediate steps in part (b) would enhance the solution.





\subsection*{Question 4:}


\subsubsection*{Response 1 :}

Solution 1 :

\textbf{Marks : 20/20}

\textbf{Feedback :}

\textbf{Part a : 3/3}

\begin{itemize}
    \item Expansion and substitution: 1/1
    \item Explanation of commutation: 1/1
    \item Conclusion: 1/1
\end{itemize}


\textbf{Part b : 3/3}

\begin{itemize}
    \item Expansion and substitution: 1/1
    \item Explanation of commutation: 1/1
    \item Conclusion: 1/1
\end{itemize}


\textbf{Part c : 4/4}

\begin{itemize}
    \item Expansion and substitution: 1/1
    \item Explanation of commutation: 1/1
    \item Factorizing and simplifying: 2/2
\end{itemize}


\textbf{Part d : 4/4}

\begin{itemize}
    \item Expansion and substitution: 1/1
    \item Explanation of commutation: 1/1
    \item Factorizing and simplifying: 2/2
\end{itemize}



\textbf{Part e : 6/6}

\begin{itemize}
    \item Rewriting the commutation: 1/1
    \item Applying the identity: 1/1
    \item Substituting commutator results: 3/3
    \item Simplifying: 1/1
\end{itemize}


\textbf{General feedback :}

\begin{itemize}
    \item Excellent work! Each part of the problem was correctly and clearly derived. All steps were shown in detail, and the solutions were well-explained and accurate.
\end{itemize}



Solution 2 :

\textbf{Marks : 20/20}

\textbf{Feedback :}

\textbf{Part a : 3/3}

\begin{itemize}
    \item Expansion and substitution: 1/1
    \item Explanation of commutation: 1/1
    \item Conclusion: 1/1
\end{itemize}


\textbf{Part b : 3/3}

\begin{itemize}
    \item Expansion and substitution: 1/1
    \item Explanation of commutation: 1/1
    \item Conclusion: 1/1
\end{itemize}


\textbf{Part c : 4/4}

\begin{itemize}
    \item Expansion and substitution: 1/1
    \item Explanation of commutation: 1/1
    \item Factorizing and simplifying: 2/2
\end{itemize}


\textbf{Part d : 4/4}

\begin{itemize}
    \item Expansion and substitution: 1/1
    \item Explanation of commutation: 1/1
    \item Factorizing and simplifying: 2/2
\end{itemize}



\textbf{Part e : 6/6}

\begin{itemize}
    \item Rewriting the commutation: 1/1
    \item Applying the identity: 1/1
    \item Substituting commutator results: 3/3
    \item Simplifying: 1/1
\end{itemize}


\textbf{General feedback :}

\begin{itemize}
    \item Excellent work! Your solutions are thorough and precise, and your explanations are clear and accurate. Keep up the good work.
\end{itemize}




Solution 3 :

\textbf{Marks : 20/20}

\textbf{Feedback :}

\textbf{Part a : 3/3}

\begin{itemize}
    \item Expansion and substitution: 1/1
    \item Explanation of commutation: 1/1
    \item Conclusion: 1/1
\end{itemize}


\textbf{Part b : 3/3}

\begin{itemize}
    \item Expansion and substitution: 1/1
    \item Explanation of commutation: 1/1
    \item Conclusion: 1/1
\end{itemize}


\textbf{Part c : 4/4}

\begin{itemize}
    \item Expansion and substitution: 1/1
    \item Explanation of commutation: 1/1
    \item Factorizing and simplifying: 2/2
\end{itemize}


\textbf{Part d : 4/4}

\begin{itemize}
    \item Expansion and substitution: 1/1
    \item Explanation of commutation: 1/1
    \item Factorizing and simplifying: 2/2
\end{itemize}



\textbf{Part e : 6/6}

\begin{itemize}
    \item Rewriting the commutation: 1/1
    \item Applying the identity: 1/1
    \item Substituting commutator results: 3/3
    \item Simplifying: 1/1
\end{itemize}


\textbf{General feedback :}

\begin{itemize}
    \item Well done! Your solutions are detailed and correct, and your explanations are comprehensive and clear. Excellent understanding of the problem and the commutation relations.
\end{itemize}



\subsubsection*{Response 2 :}

Solution 1 :

\textbf{Marks : 16/20}

\textbf{Feedback :}

\textbf{Part a : 3/3}

\begin{itemize}
    \item Expansion and substitution (1 mark): Correctly expanded and substituted.
    \item Commutation explanation (1 mark): Proper explanation provided.
    \item Simplification to zero (1 mark): Correctly simplified to zero.
\end{itemize}


\textbf{Part b : 3/3}

\begin{itemize}
    \item Expansion and substitution (1 mark): Correctly expanded and substituted.
    \item Commutation explanation (1 mark): Proper explanation provided.
    \item Simplification to zero (1 mark): Correctly simplified to zero.
\end{itemize}


\textbf{Part c : 3/4}

\begin{itemize}
    \item Expansion and substitution (1 mark): Correctly expanded and substituted.
    \item Order rearrangement (1 mark): Not explicitly detailed.
    \item Simplification (2 marks): Correctly simplified using commutator relations, but lack of intermediate steps.
\end{itemize}


\textbf{Part d : 3/4}

\begin{itemize}
    \item Expansion and substitution (1 mark): Correctly expanded and substituted.
    \item Order rearrangement (1 mark): Not explicitly detailed.
    \item Simplification (2 marks): Correctly simplified using commutator relations, but lack of intermediate steps.
\end{itemize}



\textbf{Part e : 4/6}

\begin{itemize}
    \item Expansion to components (1 mark): Correctly expanded to components.
    \item Commutator identity application (1 mark): Applied but not detailed.
    \item Intermediate steps (3 marks): Missing detailed steps.
    \item Final simplification (1 mark): Correctly simplified.
\end{itemize}


\textbf{General feedback :}

\begin{itemize}
    \item A thorough solution with correct results but lacking detailed intermediate steps for parts c) and d). More detailed application of commutation relations and identities in part e) would strengthen the argument.
\end{itemize}



Solution 2 :

\textbf{Marks : 16/20}

\textbf{Feedback :}

\textbf{Part a : 3/3}

\begin{itemize}
    \item Expansion and substitution (1 mark): Correctly expanded and substituted.
    \item Commutation explanation (1 mark): Proper explanation provided.
    \item Simplification to zero (1 mark): Correctly simplified to zero.
\end{itemize}


\textbf{Part b : 3/3}

\begin{itemize}
    \item Expansion and substitution (1 mark): Correctly expanded and substituted.
    \item Commutation explanation (1 mark): Proper explanation provided.
    \item Simplification to zero (1 mark): Correctly simplified to zero.
\end{itemize}


\textbf{Part c : 3/4}

\begin{itemize}
    \item Expansion and substitution (1 mark): Correctly expanded and substituted.
    \item Order rearrangement (1 mark): Not explicitly detailed.
    \item Simplification (2 marks): Correctly simplified using commutator relations, but lack of intermediate steps.
\end{itemize}


\textbf{Part d : 3/4}

\begin{itemize}
    \item Expansion and substitution (1 mark): Correctly expanded and substituted.
    \item Order rearrangement (1 mark): Not explicitly detailed.
    \item Simplification (2 marks): Correctly simplified using commutator relations, but lack of intermediate steps.
\end{itemize}



\textbf{Part e : 4/6}

\begin{itemize}
    \item Expansion to components (1 mark): Correctly expanded to components.
    \item Commutator identity application (1 mark): Applied but not detailed.
    \item Intermediate steps (3 marks): Missing detailed steps.
    \item Final simplification (1 mark): Correctly simplified.
\end{itemize}


\textbf{General feedback :}

\begin{itemize}
    \item Correct results, but a lack of intermediate steps and detailed commutation relation explanations in parts c), d), and e). More clarity in the step-by-step application would improve the solution.
\end{itemize}




Solution 3 :

\textbf{Marks : 16/20}

\textbf{Feedback :}

\textbf{Part a : 3/3}

\begin{itemize}
    \item Expansion and substitution (1 mark): Correctly expanded and substituted.
    \item Commutation explanation (1 mark): Proper explanation provided.
    \item Simplification to zero (1 mark): Correctly simplified to zero.
\end{itemize}


\textbf{Part b : 3/3}

\begin{itemize}
    \item Expansion and substitution (1 mark): Correctly expanded and substituted.
    \item Commutation explanation (1 mark): Proper explanation provided.
    \item Simplification to zero (1 mark): Correctly simplified to zero.
\end{itemize}


\textbf{Part c : 3/4}

\begin{itemize}
    \item Expansion and substitution (1 mark): Correctly expanded and substituted.
    \item Order rearrangement (1 mark): Not explicitly detailed.
    \item Simplification (2 marks): Correctly simplified using commutator relations, but lack of intermediate steps.
\end{itemize}


\textbf{Part d : 3/4}

\begin{itemize}
    \item Expansion and substitution (1 mark): Correctly expanded and substituted.
    \item Order rearrangement (1 mark): Not explicitly detailed.
    \item Simplification (2 marks): Correctly simplified using commutator relations, but lack of intermediate steps.
\end{itemize}



\textbf{Part e : 4/6}

\begin{itemize}
    \item Expansion to components (1 mark): Correctly expanded to components.
    \item Commutator identity application (1 mark): Applied but not detailed.
    \item Intermediate steps (3 marks): Missing detailed steps.
    \item Final simplification (1 mark): Correctly simplified.
\end{itemize}


\textbf{General feedback :}

\begin{itemize}
    \item Correct results overall, but lacking detailed intermediate steps and specific commutation relation explanations in parts c), d), and e). More clarity in the derivations would enhance the solution.
\end{itemize}



\subsubsection*{Response 3 :}

Solution 1 :

\textbf{Marks : 20/20}

\textbf{Feedback :}

\textbf{Part a : 3/3}

\begin{itemize}
    \item Correctly expanded and substituted $\hat{L_X}$: 1 mark.
    \item Explanation for why terms commute: 1 mark.
    \item Simplified correctly to 0: 1 mark.
\end{itemize}


\textbf{Part b : 3/3}

\begin{itemize}
    \item Correctly expanded and substituted $\hat{L_X}$: 1 mark.
    \item Explanation for why terms commute: 1 mark.
    \item Simplified correctly to 0: 1 mark.
\end{itemize}


\textbf{Part c : 4/4}

\begin{itemize}
    \item Correctly expanded and substituted $\hat{L_X}$: 1 mark.
    \item Changed order of terms and identified non-zero term: 1 mark.
    \item Correctly factorized and simplified: 2 marks.
\end{itemize}


\textbf{Part d : 4/4}

\begin{itemize}
    \item Correctly expanded and substituted $\hat{L_X}$: 1 mark.
    \item Changed order of terms and identified non-zero term: 1 mark.
    \item Correctly factorized and simplified: 2 marks.
\end{itemize}



\textbf{Part e : 6/6}

\begin{itemize}
    \item Rewriting commutation relation correctly: 1 mark.
    \item Applied identity correctly: 1 mark.
    \item Identified relevant commutators correctly: 3 marks.
    \item Simplified correctly to 0: 1 mark.
\end{itemize}


\textbf{General feedback :}

\begin{itemize}
    \item Very well-explained solutions, thorough and mostly correct. A minor point could be clarified in part c) regarding the factorization.
\end{itemize}



Solution 2 :

\textbf{Marks : 19/20}

\textbf{Feedback :}

\textbf{Part a : 3/3}

\begin{itemize}
    \item Correctly expanded and substituted $\hat{L_X}$: 1 mark.
    \item Explanation for why terms commute: 1 mark.
    \item Simplified correctly to 0: 1 mark.
\end{itemize}


\textbf{Part b : 3/3}

\begin{itemize}
    \item Correctly expanded and substituted $\hat{L_X}$: 1 mark.
    \item Explanation for why terms commute: 1 mark.
    \item Simplified correctly to 0: 1 mark.
\end{itemize}


\textbf{Part c : 4/4}

\begin{itemize}
    \item Correctly expanded and substituted $\hat{L_X}$: 1 mark.
    \item Changed order of terms and identified non-zero term: 1 mark.
    \item Factorization step missed a detailed explanation: 1 mark.
    \item Final simplification correct: 1 mark.
\end{itemize}


\textbf{Part d : 4/4}

\begin{itemize}
    \item Correctly expanded and substituted $\hat{L_X}$: 1 mark.
    \item Changed order of terms and identified non-zero term: 1 mark.
    \item Factorization step missed a detailed explanation: 1 mark.
    \item Final simplification correct: 1 mark.
\end{itemize}



\textbf{Part e : 5/6}

\begin{itemize}
    \item Rewriting commutation relation correctly: 1 mark.
    \item Applied identity correctly: 1 mark.
    \item Identified relevant commutators correctly: 2 marks.
    \item Simplified correctly to 0: 1 mark.
\end{itemize}


\textbf{General feedback :}

\begin{itemize}
    \item Good understanding and mostly correct solutions. Some steps, particularly in parts c) and d), lack detailed explanations on factorization and simplification.
\end{itemize}


Solution 3 :

\textbf{Marks : 20/20}

\textbf{Feedback :}

\textbf{Part a : 3/3}

\begin{itemize}
    \item Correctly expanded and substituted $\hat{L_X}$: 1 mark.
    \item Explanation for why terms commute: 1 mark.
    \item Simplified correctly to 0: 1 mark.
\end{itemize}


\textbf{Part b : 3/3}

\begin{itemize}
    \item Correctly expanded and substituted $\hat{L_X}$: 1 mark.
    \item Explanation for why terms commute: 1 mark.
    \item Simplified correctly to 0: 1 mark.
\end{itemize}


\textbf{Part c : 4/4}

\begin{itemize}
    \item Correctly expanded and substituted $\hat{L_X}$: 1 mark.
    \item Changed order of terms and identified non-zero term: 1 mark.
    \item Factorization step correct: 2 marks.
\end{itemize}


\textbf{Part d : 4/4}

\begin{itemize}
    \item Correctly expanded and substituted $\hat{L_X}$: 1 mark.
    \item Changed order of terms and identified non-zero term: 1 mark.
    \item Factorization step correct: 2 marks.
\end{itemize}



\textbf{Part e : 6/6}

\begin{itemize}
    \item Rewriting commutation relation correctly: 1 mark.
    \item Applied identity correctly: 1 mark.
    \item Identified relevant commutators correctly: 3 marks.
    \item Simplified correctly to 0: 1 mark.
\end{itemize}


\textbf{General feedback :}

\begin{itemize}
    \item Clear and well-explained solutions. Excellent understanding and thorough derivations. Some minor details could be more explicit in factorization steps.
\end{itemize}



\subsubsection*{Response 4 :}

Solution 1 :

\textbf{Marks : 19/20}

\textbf{Feedback :}

\textbf{Part a : 3/3}

\begin{itemize}
    \item Full marks for correctly expanding and simplifying the commutator.
\end{itemize}


\textbf{Part b : 3/3}

\begin{itemize}
    \item Full marks for correct expansion and simplification.
\end{itemize}


\textbf{Part c : 4/4}

\begin{itemize}
    \item Full marks for correct expansion and factorization.
\end{itemize}


\textbf{Part d : 4/4}

\begin{itemize}
    \item Full marks for correct expansion and factorization.
\end{itemize}



\textbf{Part e : 5/6}

\begin{itemize}
    \item One mark deducted for lack of clarity in combining and simplifying terms.
\end{itemize}


\textbf{General feedback :}

\begin{itemize}
    \item Excellent work. Each part was methodically solved, with clear expansions and correct final results. However, the last part could be clearer on how terms combine and simplify.
\end{itemize}



Solution 2 :

\textbf{Marks : 16/20}

\textbf{Feedback :}

\textbf{Part a : 2/3}

\begin{itemize}
    \item Incorrect simplification, missing explicit final step.
\end{itemize}


\textbf{Part b : 2/3}

\begin{itemize}
    \item Missing detailed explanation for zero result.
\end{itemize}


\textbf{Part c : 4/4}

\begin{itemize}
    \item Correct and clear derivation.
\end{itemize}


\textbf{Part d : 3/4}

\begin{itemize}
    \item Missed a detailed step-by-step simplification.
\end{itemize}



\textbf{Part e : 5/6}

\begin{itemize}
    \item The final part is correct, but some intermediate steps and explanations are omitted.
\end{itemize}


\textbf{General feedback :}

\begin{itemize}
    \item Good understanding and application of the commutator relations. The steps could be more detailed in a) and b), and clarity is needed in d). Overall, a strong solution.
\end{itemize}




Solution 3 :

\textbf{Marks : 15/20}

\textbf{Feedback :}

\textbf{Part a : 2/3}

\begin{itemize}
    \item Missing final explicit step.
\end{itemize}


\textbf{Part b : 2/3}

\begin{itemize}
    \item Lacks explanation for simplification.
\end{itemize}


\textbf{Part c : 3/4}

\begin{itemize}
    \item Correct approach, but explanation is somewhat terse.
\end{itemize}


\textbf{Part d : 3/4}

\begin{itemize}
    \item The simplification process is not fully detailed.
\end{itemize}



\textbf{Part e : 5/6}

\begin{itemize}
    \item Correct overall but could use more detailed explanation.
\end{itemize}


\textbf{General feedback :}

\begin{itemize}
    \item A solid attempt with mostly correct results. The approach is sound, but the explanations and intermediate steps could be more detailed, especially in parts a), b), and d). This would help clarify the reasoning and ensure no steps are skipped.
\end{itemize}



\subsubsection*{Response 5 :}

Solution 1 :

\textbf{Marks : 20/20}

\textbf{Feedback :}

\textbf{Part a : 3/3}

\begin{itemize}
    \item Expands the commutator correctly and substitutes $\hat{L_X}$ appropriately. (1 mark)
    \item States commutation properties correctly. (1 mark)
    \item Simplifies the expression to show the commutator is zero. (1 mark)
\end{itemize}


\textbf{Part b : 3/3}

\begin{itemize}
    \item Expands the commutator correctly and substitutes $\hat{L_X}$ appropriately. (1 mark)
    \item States commutation properties correctly. (1 mark)
    \item Simplifies the expression to show the commutator is zero. (1 mark)
\end{itemize}


\textbf{Part c : 4/4}

\begin{itemize}
    \item Expands the commutator correctly and substitutes $\hat{L_X}$ appropriately. (1 mark)
    \item Changes the order of terms correctly. (1 mark)
    \item Correctly factorizes and simplifies using the commutation relation. (2 marks)
\end{itemize}


\textbf{Part d : 4/4}

\begin{itemize}
    \item Expands the commutator correctly and substitutes $\hat{L_X}$ appropriately. (1 mark)
    \item Changes the order of terms correctly. (1 mark)
    \item Correctly factorizes and simplifies using the commutation relation. (2 marks)
\end{itemize}



\textbf{Part e : 6/6}

\begin{itemize}
    \item Rewrites the commutation relation in terms of components. (1 mark)
    \item Applies the commutation identity correctly. (1 mark)
    \item Uses the correct commutation relations to simplify the terms. (3 marks)
    \item Correctly simplifies to show the commutator is zero. (1 mark)
\end{itemize}


\textbf{General feedback :}

\begin{itemize}
    \item Excellent work! All steps are clearly explained and the commutation relations are correctly derived. Your use of commutation properties is precise and thorough.
\end{itemize}



Solution 2 :

\textbf{Marks : 16/20}

\textbf{Feedback :}

\textbf{Part a : 2/3}

\begin{itemize}
    \item Expands the commutator correctly and substitutes $\hat{L_X}$ appropriately. (1 mark)
    \item States commutation properties correctly. (1 mark)
    \item Simplifies the expression but lacks detail. (0 marks)
\end{itemize}


\textbf{Part b : 2/3}

\begin{itemize}
    \item Expands the commutator correctly and substitutes $\hat{L_X}$ appropriately. (1 mark)
    \item States commutation properties correctly. (1 mark)
    \item Simplifies the expression but lacks detail. (0 marks)
\end{itemize}


\textbf{Part c : 3/4}

\begin{itemize}
    \item Expands the commutator correctly and substitutes $\hat{L_X}$ appropriately. (1 mark)
    \item Changes the order of terms correctly but without clear steps. (0 marks)
    \item Factorizes and simplifies but without clear steps. (2 marks)
\end{itemize}


\textbf{Part d : 3/4}

\begin{itemize}
    \item Expands the commutator correctly and substitutes $\hat{L_X}$ appropriately. (1 mark)
    \item Changes the order of terms correctly but without clear steps. (0 marks)
    \item Factorizes and simplifies but without clear steps. (2 marks)
\end{itemize}



\textbf{Part e : 6/6}

\begin{itemize}
    \item Rewrites the commutation relation in terms of components. (1 mark)
    \item Applies the commutation identity correctly. (1 mark)
    \item Uses the correct commutation relations to simplify the terms. (3 marks)
    \item Correctly simplifies to show the commutator is zero. (1 mark)
\end{itemize}


\textbf{General feedback :}

\begin{itemize}
    \item Good effort. You correctly identify and use the necessary commutation properties. More detail in the intermediate steps, especially in parts c) and d), would strengthen your solutions.
\end{itemize}



Solution 3 :

\textbf{Marks : 16/20}

\textbf{Feedback :}

\textbf{Part a : 2/3}

\begin{itemize}
    \item Expands the commutator correctly and substitutes $\hat{L_X}$ appropriately. (1 mark)
    \item States commutation properties correctly. (1 mark)
    \item Simplifies the expression but lacks explicit steps. (0 marks)
\end{itemize}


\textbf{Part b : 2/3}

\begin{itemize}
    \item Expands the commutator correctly and substitutes $\hat{L_X}$ appropriately. (1 mark)
    \item States commutation properties correctly. (1 mark)
    \item Simplifies the expression but lacks explicit steps. (0 marks)
\end{itemize}


\textbf{Part c : 3/4}

\begin{itemize}
    \item Expands the commutator correctly and substitutes $\hat{L_X}$ appropriately. (1 mark)
    \item Changes the order of terms but lacks clarity. (0 marks)
    \item Factorizes and simplifies but lacks clarity. (2 marks)
\end{itemize}


\textbf{Part d : 3/4}

\begin{itemize}
    \item Expands the commutator correctly and substitutes $\hat{L_X}$ appropriately. (1 mark)
    \item Changes the order of terms but lacks clarity. (0 marks)
    \item Factorizes and simplifies but lacks clarity. (2 marks)
\end{itemize}



\textbf{Part e : 6/6}

\begin{itemize}
    \item Rewrites the commutation relation in terms of components. (1 mark)
    \item Applies the commutation identity correctly. (1 mark)
    \item Uses the correct commutation relations to simplify the terms. (3 marks)
    \item Correctly simplifies to show the commutator is zero. (1 mark)
\end{itemize}


\textbf{General feedback :}

\begin{itemize}
    \item Your solutions are correct, but the clarity and detail in the intermediate steps are lacking, especially in parts c) and d). Ensure to show each step clearly to demonstrate your understanding fully.
\end{itemize}



\subsection*{Question 5:}

\subsubsection*{Response 1 :}

Solution 1 :

\textbf{Marks : 7/7}

\textbf{Feedback :}

\textbf{Part a : 4/4}

\begin{itemize}
    \item [1 mark] Correctly stated the equation for energy correction.
    \item [1 mark] Correctly substituted $\hat{H'} = V_{0}sin(\frac{2\pi x}{L})$ into the integral.
    \item [2 marks] Correctly solved the integral, showing the perturbation equals 0.
\end{itemize}

\textbf{General feedback:}


\textbf{Part b : 3/3}

\begin{itemize}
    \item [1 mark] Correctly applied perturbation theory and substituted the terms into the integral.
    \item [1 mark] Correctly solved the integral and simplified.
    \item [1 mark] Correctly stated the results for even and odd values of $n$.
\end{itemize}

\textbf{General feedback:}

\begin{itemize}
    \item Excellent work. The solution is well-structured, and the integrals are handled correctly with appropriate trigonometric identities. The results are accurately derived and presented.
\end{itemize}



Solution 2 :

\textbf{Marks : 6/7}

\textbf{Feedback :}

\textbf{Part a : 4/4}

\begin{itemize}
    \item [1 mark] Correctly stated the equation for energy correction.
    \item [1 mark] Correctly substituted $\hat{H'} = V_{0}sin(\frac{2\pi x}{L})$ into the integral.
    \item [2 marks] Correctly solved the integral, showing the perturbation equals 0.
\end{itemize}

\textbf{General feedback:}


\textbf{Part b : 3/3}

\begin{itemize}
    \item [1 mark] Correctly applied perturbation theory and substituted the terms into the integral.
    \item [1 mark] Correctly solved the integral and simplified.
    \item [1 mark] Correctly stated the results for even and odd values of $n$.
\end{itemize}

\textbf{General feedback:}

\begin{itemize}
    \item Well done. The calculations and explanations are clear and precise. The use of trigonometric identities is appropriate, and the final expressions are correct.
\end{itemize}



Solution 3 :

\textbf{Marks : 7/7}

\textbf{Feedback :}

\textbf{Part a : 4/4}

\begin{itemize}
    \item [1 mark] Correctly stated the equation for energy correction.
    \item [1 mark] Correctly substituted $\hat{H'} = V_{0}sin(\frac{2\pi x}{L})$ into the integral.
    \item [2 marks] Correctly solved the integral, showing the perturbation equals 0.
\end{itemize}

\textbf{General feedback:}


\textbf{Part b : 3/3}

\begin{itemize}
    \item [1 mark] Correctly applied perturbation theory and substituted the terms into the integral.
    \item [1 mark] Correctly solved the integral and simplified.
    \item [1 mark] Correctly stated the results for even and odd values of $n$.
\end{itemize}

\textbf{General feedback:}

\begin{itemize}
    \item Excellent job. The problem is approached methodically, and the integrals are computed correctly. The final results are clearly presented and accurate.
\end{itemize}

\textbf{Summary :}
Overall, all three solutions are comprehensive and correctly solve the problem using appropriate methods from perturbation theory. Each student has demonstrated a clear understanding of the concepts and calculations involved.



\subsubsection*{Response 2 :}

Solution 1 :

\textbf{Marks : 7/7}

\textbf{Feedback :}

\textbf{Part a : 4/4}

\begin{itemize}
    \item Correctly stated the equation for energy correction and relevant terms. (1 mark)
    \item Proper substitution and setup of the integral for the given perturbation. (1 mark)
    \item Trigonometric identity usage was correct, but the integral simplification had unnecessary complexity and mistakes in the final simplification. (1 mark)
    \item Correct conclusion that the perturbation results in a zero correction. (1 mark)
\end{itemize}


\textbf{Part b : 3/3}

\begin{itemize}
    \item Correct application of perturbation theory and initial integral setup. (1 mark)
    \item Correct simplification and solution of the integral. (1 mark)
    \item Correct final expression for energy corrections with correct dependence on n. (1 mark)
\end{itemize}

\textbf{General feedback:}

\begin{itemize}
    \item Excellent understanding and application of perturbation theory. The steps were clear, although some unnecessary complexity was introduced in part a).
\end{itemize}



Solution 2 :

\textbf{Marks : 7/7}

\textbf{Feedback :}

\textbf{Part a : 4/4}

\begin{itemize}
    \item Correctly stated the equation for energy correction and relevant terms. (1 mark)
    \item Proper substitution and setup of the integral for the given perturbation. (1 mark)
    \item Trigonometric identity usage was correct, but the simplification process could be more straightforward. (1 mark)
    \item Correct conclusion that the perturbation results in a zero correction. (1 mark)
\end{itemize}


\textbf{Part b : 3/3}

\begin{itemize}
    \item Correct application of perturbation theory and initial integral setup. (1 mark)
    \item Correct simplification and solution of the integral. (1 mark)
    \item Correct final expression for energy corrections with correct dependence on n. (1 mark)
\end{itemize}

\textbf{General feedback:}

\begin{itemize}
    \item Thorough and well-explained solutions, demonstrating clear understanding of the problem and the methods required.
\end{itemize}



Solution 3 :

\textbf{Marks : 7/7}

\textbf{Feedback :}

\textbf{Part a : 4/4}

\begin{itemize}
    \item Correctly stated the equation for energy correction and relevant terms. (1 mark)
    \item Proper substitution and setup of the integral for the given perturbation. (1 mark)
    \item Trigonometric identity usage was correct, but the integral simplification had some redundancy and was not as clear. (1 mark)
    \item Correct conclusion that the perturbation results in a zero correction. (1 mark)
\end{itemize}


\textbf{Part b : 3/3}

\begin{itemize}
    \item Correct application of perturbation theory and initial integral setup. (1 mark)
    \item Correct simplification and solution of the integral. (1 mark)
    \item Correct final expression for energy corrections with correct dependence on n. (1 mark)
\end{itemize}

\textbf{General feedback:}

\begin{itemize}
    \item Solid solution overall, with good application of theory and clear conclusions. Simplification steps could be made more direct to enhance clarity.
\end{itemize}

\textbf{Summary :}
All solutions have scored full marks, reflecting a strong understanding of perturbation theory and the mathematical steps involved. Each student has demonstrated clarity in their derivations and correctness in their final expressions.



\subsubsection*{Response 3 :}

Solution 1 :

\textbf{Marks : 7/7}

\textbf{Feedback :}

\textbf{Part a : 4/4}

\begin{itemize}
    \item Correctly states the formula for the first-order correction to the ground state energy: 1 mark.
    \item Substitutes the perturbation into the integral and simplifies correctly: 1 mark.
    \item Attempts to solve the integral but does not clearly show that the integral equals zero: 1 mark.
    \item The result is correctly identified as zero: 1 mark.
\end{itemize}


\textbf{Part b : 3/3}

\begin{itemize}
    \item Correctly sets up the integral for the first-order correction: 1 mark.
    \item Solves the integral and simplifies correctly: 1 mark.
    \item Correctly identifies the energy correction for odd and even values of $n$: 1 mark.
\end{itemize}

\textbf{General feedback:}

\begin{itemize}
    \item Excellent solution. Both parts are clearly and correctly solved with appropriate steps and justification.
\end{itemize}



Solution 2 :

\textbf{Marks : 7/7}

\textbf{Feedback :}

\textbf{Part a : 4/4}

\begin{itemize}
    \item Correctly states the formula for the first-order correction to the ground state energy: 1 mark.
    \item Substitutes the perturbation into the integral and simplifies correctly: 1 mark.
    \item Does not explicitly show the step where the integral is evaluated to zero but indicates the result is zero: 1 mark.
    \item Correctly identifies the result as zero: 1 mark.
\end{itemize}


\textbf{Part b : 3/3}

\begin{itemize}
    \item Correctly sets up the integral for the first-order correction: 1 mark.
    \item Solves the integral and simplifies correctly: 1 mark.
    \item Correctly identifies the energy correction for odd and even values of $n$: 1 mark.
\end{itemize}

\textbf{General feedback:}

\begin{itemize}
    \item Very well done. The solution is clear and concise, with accurate calculations and justifications for both parts.
\end{itemize}



Solution 3 :

\textbf{Marks : 7/7}

\textbf{Feedback :}

\textbf{Part a : 4/4}

\begin{itemize}
    \item Correctly states the formula for the first-order correction to the ground state energy: 1 mark.
    \item Substitutes the perturbation into the integral and simplifies correctly: 1 mark.
    \item Attempts to solve the integral but does not explicitly show it equals zero: 1 mark.
    \item The result is correctly identified as zero: 1 mark.
\end{itemize}


\textbf{Part b : 3/3}

\begin{itemize}
    \item Correctly sets up the integral for the first-order correction: 1 mark.
    \item Solves the integral and simplifies correctly: 1 mark.
    \item Correctly identifies the energy correction for odd and even values of $n$: 1 mark.
\end{itemize}

\textbf{General feedback:}

\begin{itemize}
    \item Good job. The solution is thorough and demonstrates a strong understanding of the concepts involved. Both parts are solved correctly with clear reasoning.
\end{itemize}

\textbf{Summary :}
All three solutions effectively apply perturbation theory to solve the problem. The steps are well-justified, and the results are accurate. In future solutions, ensure that all simplification steps are clearly shown, especially when the integral evaluates to zero.


\subsubsection*{Response 4 :}

Solution 1 :

\textbf{Marks : 7/7}

\textbf{Feedback :}

\textbf{Part a : 4/4}

\begin{itemize}
    \item Equation for energy and relevance to ground state: 1 mark
    \item Substituting $\hat{H'}$ and setting up the integral: 1 mark
    \item Simplifying and solving the integral: 2 marks
\end{itemize}


\textbf{Part b : 3/3}

\begin{itemize}
    \item Applying perturbation theory and setting up the integral: 1 mark
    \item Correctly solving the integral: 1 mark
    \item Identifying corrections for odd and even values of $n$: 1 mark
\end{itemize}

\textbf{General feedback:}

\begin{itemize}
    \item Excellent work! You correctly set up and solved the integrals for both parts, demonstrating a clear understanding of perturbation theory. Your use of trigonometric identities and properties of the Dirac delta function was accurate and well-executed.
\end{itemize}



Solution 2 :

\textbf{Marks : 7/7}

\textbf{Feedback :}

\textbf{Part a : 4/4}

\textbf{Part a : 4/4}

\begin{itemize}
    \item Equation for energy and relevance to ground state: 1 mark
    \item Substituting $\hat{H'}$ and setting up the integral: 1 mark
    \item Simplifying and solving the integral: 2 marks
\end{itemize}


\textbf{Part b : 3/3}

\begin{itemize}
    \item Applying perturbation theory and setting up the integral: 1 mark
    \item Correctly solving the integral: 1 mark
    \item Identifying corrections for odd and even values of $n$: 1 mark
\end{itemize}

\textbf{General feedback:}

\begin{itemize}
    \item Excellent job! You have accurately followed the steps for perturbation theory, with clear and precise calculations. Your explanations and use of the trigonometric identities were spot on.
\end{itemize}



Solution 3 :

\textbf{Marks : 7/7}

\textbf{Feedback :}

\textbf{Part a : 4/4}

\textbf{Part a : 4/4}

\begin{itemize}
    \item Equation for energy and relevance to ground state: 1 mark
    \item Substituting $\hat{H'}$ and setting up the integral: 1 mark
    \item Simplifying and solving the integral: 2 marks
\end{itemize}


\textbf{Part b : 3/3}

\begin{itemize}
    \item Applying perturbation theory and setting up the integral: 1 mark
    \item Correctly solving the integral: 1 mark
    \item Identifying corrections for odd and even values of $n$: 1 mark
\end{itemize}

\textbf{General feedback:}

\begin{itemize}
    \item Great work! You correctly applied perturbation theory to find the first-order corrections. Your solution was detailed and accurate, and your calculations were clear and well-presented.
\end{itemize}


\textbf{Summary :}
Overall, all three solutions demonstrate a strong understanding of the principles and application of perturbation theory in quantum mechanics, accurately solving for the first-order corrections to the energy eigenvalues.


\subsubsection*{Response 5 :}

Solution 1 :

\textbf{Marks : 7/7}

\textbf{Feedback :}

\textbf{Part a : 4/4}

\begin{itemize}
    \item Correctly states the equation for energy correction and makes it relevant to the ground state energy.
    \item Correctly substitutes the perturbation into the integral.
    \item Properly simplifies the integral.
    \item Correctly concludes that the perturbation equals 0.
\end{itemize}


\textbf{Part b : 3/3}

\begin{itemize}
    \item Applies perturbation theory correctly and substitutes the terms into the integral.
    \item Solves the integral correctly and simplifies it to the required expression.
    \item States the correct results for all odd and even values of n.
\end{itemize}

\textbf{General feedback:}

\begin{itemize}
    \item Excellent solution. You correctly applied perturbation theory and properly evaluated the integrals. Your answers and explanations were clear and accurate.
\end{itemize}



Solution 2 :

\textbf{Marks : 6/7}

\textbf{Feedback :}

\textbf{Part a : 3/4}

\begin{itemize}
    \item Correctly states the equation for energy correction and makes it relevant to the ground state energy.
    \item Substitutes the perturbation into the integral.
    \item Attempts to simplify the integral but does not clearly show the steps leading to the conclusion that the perturbation equals 0.
\end{itemize}


\textbf{Part b : 3/3}

\begin{itemize}
    \item Applies perturbation theory correctly and substitutes the terms into the integral.
    \item Solves the integral correctly and simplifies it to the required expression.
    \item States the correct results for all odd and even values of n.
\end{itemize}

\textbf{General feedback:}

\begin{itemize}
    \item Good attempt. Your solution to part a is almost complete but lacks clarity in the steps leading to the conclusion. Part b is done correctly and comprehensively.
\end{itemize}



Solution 3 :

\textbf{Marks : 6/7}

\textbf{Feedback :}

\textbf{Part a : 3/4}

\begin{itemize}
    \item Correctly states the equation for energy correction and makes it relevant to the ground state energy.
    \item Substitutes the perturbation into the integral.
    \item Simplifies the integral but does not explicitly show the evaluation steps leading to the conclusion that the perturbation equals 0.
\end{itemize}


\textbf{Part b : 3/3}

\begin{itemize}
    \item Applies perturbation theory correctly and substitutes the terms into the integral.
    \item Solves the integral correctly and simplifies it to the required expression.
    \item States the correct results for all odd and even values of n.
\end{itemize}

\textbf{General feedback:}

\begin{itemize}
    \item Well done. Your solution to part a is correct but would benefit from a more detailed explanation of the integral simplification. Part b is thoroughly and accurately solved.
\end{itemize}



\subsection*{Question 6:}

\subsubsection*{Response 1:}


Solution 1 :

\textbf{Marks : 13/17}

\textbf{Feedback : }

\textbf{Part a : 3/4}

\begin{itemize}
    \item Correct initial expansion of $\hat{J^2}$
    \item Correct use of ladder operator definitions.
    \item Did not fully expand and simplify the cross product terms.
\end{itemize}

\textbf{Feedback :}

Good start and correct use of operators. Make sure to fully simplify cross terms for full marks.



\textbf{Part b : 3/4}

\begin{itemize}
    \item Correctly identified the state as an eigenvector.
    \item Correct eigenvalue derivation.
    \item Missed some intermediate steps for completeness.
\end{itemize}

\textbf{Feedback :}
Correct eigenvalue derivation, but include all intermediate steps for clarity.


\textbf{Part c : 7/9}

\begin{itemize}
    \item Identified basis kets correctly.
    \item Correct representation of $\hat{L^2}$ in the matrix.
    \item Incomplete or incorrect representation of $\hat{S_+}\hat{L_z}$
\end{itemize}

\textbf{Feedback :}
Good approach to matrix representation, but ensure calculations of off-diagonal elements for $\hat{S_+}\hat{L_z}$ are complete and correct.


Solution 2 :

\textbf{Marks : 13/17}

\textbf{Feedback : }

\textbf{Part a : 4/4}

\begin{itemize}
    \item Correct initial expansion of $\hat{J^2}$
    \item Correctly used ladder operators and simplified terms.
    \item Provided detailed steps and final answer.
\end{itemize}

\textbf{Feedback :}
Excellent work. Clear and detailed derivation.


\textbf{Part b : 2/4}

\begin{itemize}
    \item Correctly identified the state but lacked detailed explanation.
    \item Partial credit for identifying eigenvalue, but missed intermediate steps.
\end{itemize}

\textbf{Feedback :}
Provide more detailed intermediate steps and explanations to justify the eigenvalue derivation.


\textbf{Part c : 7/9}

\begin{itemize}
    \item Correctly identified basis kets.
    \item Correct representation of $\hat{L^2}$ in the matrix.
    \item Incomplete or incorrect representation of $\hat{S_+}\hat{L_z}$
\end{itemize}

\textbf{Feedback :}
Good identification of basis kets and initial matrix representation. Ensure the off-diagonal elements are fully correct.



Solution 3 :

\textbf{Marks : 13/17}

\textbf{Feedback : }

\textbf{Part a : 4/4}

\begin{itemize}
    \item Correct initial expansion of $\hat{J^2}$
    \item Correct use of ladder operators and simplification of terms.
    \item Detailed steps and final answer provided.
\end{itemize}

\textbf{Feedback :}
Excellent and detailed solution. Clear and correct throughout.


\textbf{Part b : 2/4}

\begin{itemize}
    \item Identified state as an eigenvector.
    \item Incomplete explanation of intermediate steps for eigenvalue.
\end{itemize}

\textbf{Feedback :}
Provide a more detailed explanation and intermediate steps for full marks.


\textbf{Part c : 7/9}

\begin{itemize}
    \item Correctly identified basis kets.
    \item Correct representation of $\hat{L^2}$ in the matrix.
    \item Incomplete or incorrect representation of $\hat{S_+}\hat{L_z}$
\end{itemize}

\textbf{Feedback :}
Good work on identifying basis kets and initial matrix setup. Double-check off-diagonal calculations.



\subsubsection*{Response 2:}


Solution 1 :

\textbf{Marks : 17/17}

\textbf{Feedback : }

\textbf{Part a : 4/4}

\begin{itemize}
    \item Correctly identified the expansion of $\hat{J^2}$: 1 mark
    \item Correctly used the identities for $\hat{L_x}, \hat{L_y}, \hat{S_x},$ and $\hat{S_y}$ : 2 marks
    \item Correct final expression for $\hat{J^2}$: 1 mark
\end{itemize}


\textbf{Part b : 4/4}

\begin{itemize}
    \item Correctly identified the eigenvalue equation for $\hat{J^2}$: 2 marks
    \item Correctly computed the eigenvalue of $\hat{J^2}$ for the given state: 2 marks
\end{itemize}


\textbf{Part c : 9/9}

\begin{itemize}
    \item Correct identification of basis kets: 1 mark
    \item Correct choice of vector representation for each basis ket: 1 mark
    \item Correct calculation of $\hat{L^2}$ matrix representation: 2 marks
    \item Correct calculation of $\hat{S_+}\hat{L_z}$ matrix representation: 4 marks
    Correct construction of the final matrix representation for $\hat{O}$: 1 mark
\end{itemize}

\textbf{Feedback :}
Excellent work! The solution is comprehensive and accurate, clearly demonstrating an understanding of the required concepts and correct application of quantum mechanical operators and their matrix representations. Well done!



Solution 2 :

\textbf{Marks : 11/17}

\textbf{Feedback : }

\textbf{Part a : 3/4}

\begin{itemize}
    \item Correctly identified the expansion of $\hat{J^2}$: 1 mark
    \item Used ladder operator identities, but did not show intermediate steps clearly: 1 mark
    \item Correct final expression for $\hat{J^2}$: 1 mark
\end{itemize}


\textbf{Part b : 2/4}

\begin{itemize}
    \item Identified the eigenvalue equation but lacked detailed explanation: 1 mark
    \item Correct computation of the eigenvalue for the given state but needs clearer presentation: 1 mark
\end{itemize}


\textbf{Part c : 6/9}

\begin{itemize}
    \item Correct identification of basis kets: 1 mark
    \item Correct choice of vector representation for each basis ket: 1 mark
    \item Correct calculation of $\hat{L^2}$ matrix representation: 2 marks
    \item Attempted calculation of $\hat{S_+}\hat{L_z}$ but made some errors: 2 marks
    \item Attempted construction of the final matrix representation but had errors: 0 marks
\end{itemize}

\textbf{Feedback :}
A good attempt overall. You showed a solid understanding of the concepts but lacked some clarity and detail in parts a and b. Part c was handled well, but there were minor errors in calculating and assembling the final matrix. Pay attention to intermediate steps and ensure accuracy in matrix construction.



Solution 3 :

\textbf{Marks : 11/17}

\textbf{Feedback : }

\textbf{Part a : 3/4}

\begin{itemize}
    \item Correctly identified the expansion of $\hat{J^2}$ : 1 mark
    \item Used ladder operator identities but skipped some steps: 1 mark
    \item Correct final expression for $\hat{J^2}$: 1 mark
\end{itemize}


\textbf{Part b : 2/4}

\begin{itemize}
    \item Identified the eigenvalue equation but lacked detailed explanation: 1 mark
    \item Correct computation of the eigenvalue for the given state but with less detailed justification: 1 mark
\end{itemize}


\textbf{Part c : 6/9}

\begin{itemize}
    \item Correct identification of basis kets: 1 mark
    \item Correct choice of vector representation for each basis ket: 1 mark
    \item Correct calculation of $\hat{L^2}$ matrix representation: 2 marks
    \item Incorrect or unclear calculation of $\hat{S_+}\hat{L_z}$ matrix representation: 2 marks
    \item Incorrect or incomplete construction of the final matrix representation: 0 marks
\end{itemize}

\textbf{Feedback :}
Your solution demonstrates a good understanding of the main concepts but lacks some clarity and detail, particularly in parts a and b. Part c was mostly correct, but there were inaccuracies in the final matrix representation. Ensure to show all steps clearly and double-check your calculations for accuracy.



\subsubsection*{Response 3:}


Solution 1 :

\textbf{Marks : 12/17}

\textbf{Feedback : }

\textbf{Part a : 3/4}

\begin{itemize}
    \item Correctly expanded $\hat{J^2}$and used the ladder operators.
    \item The derivation was clear and detailed but missed explicitly stating the simplification steps from the cross terms to the final expression.
\end{itemize}


\textbf{Part b : 3/4}

\begin{itemize}
    \item Correctly identified the action of $\hat{J^2}$ on the state $|l, -l;s, -s>$
    \item Explained why the ladder operators do not contribute.
    \item Calculated the eigenvalue but did not explicitly show the calculation of $\hat{L^2}, \hat{S^2}$ and $2\hat{L_z}\hat{S_z}$ acting on the state.
\end{itemize}


\textbf{Part c : 6/9}

\begin{itemize}
    \item Provided the basis states and explained the action of the operators.
    \item Matrix elements were calculated but some of the explanations and transitions could be clearer.
    \item Final matrix representation missed a few elements and there was some confusion in the off-diagonal elements.
\end{itemize}

\textbf{Feedback :}
Good overall understanding and detailed steps in parts a) and b). Part c) showed a reasonable approach but could improve on clarity and accuracy of matrix elements.


Solution 2 :

\textbf{Marks : 14/17}

\textbf{Feedback : }

\textbf{Part a : 4/4}

\begin{itemize}
    \item Correctly expanded and simplified $\hat{J^2}$ using ladder operators.
    \item Final expression matched the required form.
\end{itemize}


\textbf{Part b : 3/4}

\begin{itemize}
    \item Correctly identified the eigenvector and eigenvalue.
    \item Provided clear reasoning but missed explicitly stating the calculation steps for eigenvalues of $\hat{L^2}, \hat{S^2}$ and $2\hat{L_z}\hat{S_z}$
\end{itemize}


\textbf{Part c : 7/9}

\begin{itemize}
    \item Provided the basis states and explained the action of the operators.
    \item Correctly constructed the matrix representation but some off-diagonal elements' reasoning could be clearer.
    \item Final matrix representation was mostly correct.
\end{itemize}

\textbf{Feedback :}
Clear and concise solution. Good handling of matrix representation. Improve on explaining transitions in part c).



Solution 3 :

\textbf{Marks : 10/17}

\textbf{Feedback : }

\textbf{Part a : 3/4}

\begin{itemize}
    \item Correctly expanded $\hat{J^2}$ and used the ladder operators.
    \item Final expression was correct but derivation missed some steps and explanations.
\end{itemize}


\textbf{Part b : 3/4}

\begin{itemize}
    \item Correctly identified the eigenvector and eigenvalue.
    \item Provided reasoning but lacked detail in the calculation of individual operator actions.
\end{itemize}


\textbf{Part c : 6/9}

\begin{itemize}
    \item Provided the basis states and matrix representation.
    \item Some matrix elements were missing or not fully justified.
    \item Transition explanations needed more clarity.
\end{itemize}

\textbf{Feedback :}
Good understanding of concepts. More detailed derivations and clarity in part c) would improve the solution.



\subsubsection*{Response 4:}


Solution 1 :

\textbf{Marks : 12/17}

\textbf{Feedback : }

\textbf{Part a : 3/4}

\begin{itemize}
    \item Correctly expanded $\hat{J^2}$.
    \item Provided an explanation involving ladder operators, but didn't fully simplify the cross terms involving $\hat{L_\pm}$ and $\hat{S_\pm}$.
    \item Final result was correct but lacked complete derivation of intermediate steps.
\end{itemize}


\textbf{Part b : 3/4}

\begin{itemize}
    \item Correctly stated that $|l, -l;s, -s>$ is an eigenvector of $\hat{J^2}$
    \item Calculated eigenvalue correctly but missed some detailed steps in the reasoning about ladder operators contributing zero.
\end{itemize}


\textbf{Part c : 6/9}

\begin{itemize}
    \item Identified basis states correctly.
    \item Correctly computed the matrix representation for $a\hat{L^2}$.
    \item Started well on $b\hat{S_+}\hat{L_z}$ but didn't provide full calculations or correctly fill the matrix.
    \item Final matrix lacked some entries and had some inaccuracies.
\end{itemize}

\textbf{Feedback :}
Good understanding overall but needs more detailed step-by-step derivation, particularly in complex operator expansions and matrix calculations.


Solution 2 :

\textbf{Marks : 11/17}

\textbf{Feedback : }

\textbf{Part a : 3/4}

\begin{itemize}
    \item Correctly expanded $\hat{J^2}$.
    \item Explanation involving ladder operators was correct but somewhat brief and missed a complete derivation of cross terms.
    \item Final result was correct.
\end{itemize}


\textbf{Part b : 3/4}

\begin{itemize}
    \item Correctly stated that $|l, -l;s, -s>$ is an eigenvector of $\hat{J^2}$
    \item Correctly calculated eigenvalue but missed some detailed steps.
\end{itemize}


\textbf{Part c : 5/9}

\begin{itemize}
    \item Identified basis states but didn't fully provide a correct representation.
    \item Calculated the matrix for $a\hat{L^2}$ correctly.
    \item Attempted $b\hat{S_+}\hat{L_z}$ but lacked complete derivations and accurate final matrix.
\end{itemize}

\textbf{Feedback :}
Good attempt, but needs a more rigorous approach in part (c) and more detailed intermediate steps in parts (a) and (b).


Solution 3 :

\textbf{Marks : 13/17}

\textbf{Feedback : }

\textbf{Part a : 4/4}

\begin{itemize}
    \item Correctly expanded $\hat{J^2}$.
    \item Thoroughly explained the use of ladder operators.
    \item Provided correct and complete final derivation.
\end{itemize}


\textbf{Part b : 2/4}

\begin{itemize}
    \item Correctly identified $|l, -l;s, -s>$ as an eigenvector.
    \item Missed detailed steps in proving ladder operators contribute zero.
    \item Incorrectly combined some concepts.
\end{itemize}


\textbf{Part c : 7/9}

\begin{itemize}
    \item Identified basis states correctly.
    \item Computed matrix representation for $a\hat{L^2}$ accurately.
    \item Partially completed $b\hat{S_+}\hat{L_z}$ but didn't fully justify all steps.
    \item Final matrix was mostly correct but missed detailed derivations.
\end{itemize}

\textbf{Feedback :}
Strong understanding overall, particularly in operator expansions. Needs to ensure all steps are detailed and logically connected, especially in matrix derivations.


\subsubsection*{Response 5:}


Solution 1 :

\textbf{Marks : 16/17}

\textbf{Feedback : }

\textbf{Part a : 4/4}

\begin{itemize}
    \item Correctly derived the equation, using the given vector relations and ladder operators.
    \item Shows step-by-step simplification leading to the final result.
\end{itemize}


\textbf{Part b : 4/4}

\begin{itemize}
    \item Clearly shows the process of proving the state $|l, -l;s, -s>$ is an eigenvector of $\hat{J^2}$.
    \item Correctly identifies and calculates eigenvalues using provided quantum numbers.
\end{itemize}


\textbf{Part c : 8/9}

\begin{itemize}
    \item Identified the basis states and correctly represented each basis ket in vector form.
    \item Correctly calculated the action of $\hat{L^2}$ on basis kets and provided the matrix representation.
    \item Calculated $\hat{S_+}\hat{L_z}$ acting on basis kets and combined results to construct the matrix representation of $\hat{O}$.
\end{itemize}

\textbf{Feedback :}
Excellent solution with clear derivation and logical flow. Only minor mistake in the matrix element representation for $\hat{S_+}\hat{L_z}$, leading to slight deduction in marks for part c).



Solution 2 :

\textbf{Marks : 13/17}

\textbf{Feedback : }

\textbf{Part a : 3/4}

\begin{itemize}
    \item Correctly derived the equation using vector and ladder operator relations.
    \item Missing some intermediate steps which would show a clear simplification process.
\end{itemize}


\textbf{Part b : 4/4}

\begin{itemize}
    \item Provided clear steps showing that the state $|l, -l;s, -s>$ is an eigenvector of $\hat{J^2}$.
    \item Correctly identified the eigenvalues and applied them to the given state.
\end{itemize}


\textbf{Part c : 6/9}

\begin{itemize}
    \item Correctly identified the basis states and vector representation.
    \item Provided calculation for $\hat{L_2}$ on basis kets.
    \item Incorrectly constructed the final matrix representation of $\hat{O}$, lacking some transitions due to $\hat{S_+}\hat{L_z}$.
\end{itemize}

\textbf{Feedback :}
Good solution with a clear approach. Some details missing in part a) and an incomplete final matrix representation in part c).



Solution 3 :

\textbf{Marks : 11/17}

\textbf{Feedback : }

\textbf{Part a : 3/4}

\begin{itemize}
    \item Correctly derived the equation using vector and ladder operator relations.
    \item Explanation was brief, missing detailed steps for some intermediate simplifications.
\end{itemize}


\textbf{Part b : 3/4}

\begin{itemize}
    \item Correctly identified and applied eigenvalue calculations for the state $|l, -l;s, -s>$.
    \item Lacks clarity in explaining how the ladder operators do not contribute.
\end{itemize}


\textbf{Part c : 5/9}

\begin{itemize}
    \item Identified the basis states correctly.
    \item Provided calculations for $\hat{L^2}$ on basis kets.
    \item Incorrect construction of the matrix representation of $\hat{O}$, lacking detail and incorrect placements of off-diagonal elements.
\end{itemize}

\textbf{Feedback :}
The solution shows a good understanding of the problem, but needs more detailed steps and correct matrix construction in part c). More clarity in explaining why certain terms do not contribute in part b) would improve the answer.



\subsection*{Question 7:}

\subsubsection*{Response 1:}

Solution 1:

\textbf{Marks : 8/8}

\textbf{Feedback : }

\textbf{Part a : 3/3}

\begin{itemize}
    \item Correct time-dependent solution: 1 mark
    \item Correct expression for energy $E$ : 1 mark
    \item Correctly demonstrated independence of probability density: 1 mark
\end{itemize}


\textbf{Part b : 5/5}

\begin{itemize}
    \item Correct partial derivatives of $\Psi$: 1 mark
    \item Correct calculation of the probability flux: 2 marks
    \item Correct physical interpretation: 2 marks
\end{itemize}


\textbf{Feedback :}
Excellent work! Your solution is well-explained and thorough. You clearly demonstrated the key steps and concepts required for both parts of the problem.


Solution 2:

\textbf{Marks : 8/8}

\textbf{Feedback : }

\textbf{Part a : 3/3}

\begin{itemize}
    \item Correct time-dependent solution: 1 mark
    \item Correct expression for energy E: 1 mark
    \item Correctly demonstrated independence of probability density: 1 mark
\end{itemize}


\textbf{Part b : 5/5}

\begin{itemize}
    \item Correct partial derivatives of $\Psi$: 1 mark
    \item Correct calculation of the probability flux: 2 marks
    \item Correct physical interpretation: 2 marks
\end{itemize}


\textbf{Feedback :}
Well done! Your solution covers all the necessary steps and provides a clear and accurate explanation for both parts of the problem.


Solution 3:

\textbf{Marks : 8/8}

\textbf{Feedback : }

\textbf{Part a : 3/3}

\begin{itemize}
    \item Correct time-dependent solution: 1 mark
    \item Correct expression for energy $E$: 1 mark
    \item Correctly demonstrated independence of probability density: 1 mark
\end{itemize}


\textbf{Part b : 5/5}

\begin{itemize}
    \item Correct partial derivatives of $\Psi$: 1 mark
    \item Correct calculation of the probability flux: 2 marks
    \item Correct physical interpretation: 2 marks
\end{itemize}


\textbf{Feedback :}
Great job! Your solution is accurate and comprehensive, with a clear and correct explanation for both the time-dependent solution and the particle flux.



\subsubsection*{Response 2:}

Solution 1:

\textbf{Marks : 8/8}

\textbf{Feedback : }

\textbf{Part a : 3/3}

\begin{itemize}
    \item Correct form of the time-dependent solution: $\Psi(x,t) = Ae^{ikx-i\frac{\hbar k^2}{2m}t}$ (1 mark).
    \item Correct expression for $E$ and its substitution: $E = \frac{\hbar^2 k^2}{2m}$ (1 mark).
    \item Showing that $|\Psi(x,t)|^2 = |A|^2$ and it's independent of time and space (1 mark).
\end{itemize}


\textbf{Part b : 5/5}

\begin{itemize}
    \item Correct derivatives : $\frac{\partial \Psi}{\partial x} = ikAe^{ikx-i\frac{\hbar k^2}{2m}t}$ and $\frac{\partial \Psi*}{\partial x} = -ikAe^{-ikx+i\frac{\hbar k^2}{2m}t}$ (1 mark).
    \item Correct calculation of the flux : $\Psi* \frac{\partial\Psi}{\partial x} - \Psi \frac{\partial\Psi*}{\partial x} = 2ik|A|^2$ (2 marks).
    \item Correct final expression and physical interpretation: $\Gamma = \frac{\hbar k }{m} |A|^2$, explanation that $\Gamma$ is the number of particles per unit time and space (2 marks).
\end{itemize}


\textbf{Feedback :}
Excellent solution with clear steps and correct expressions. Good understanding of both the mathematical and physical aspects of the problem.


Solution 2:

\textbf{Marks : 8/8}

\textbf{Feedback : }

\textbf{Part a : 3/3}

\begin{itemize}
    \item Correct form of the time-dependent solution: $\Psi(x,t) = Ae^{ikx-i\frac{\hbar k^2}{2m}t}$ (1 mark).
    \item Correct expression for $E$ and its substitution: $E = \frac{\hbar^2 k^2}{2m}$ (1 mark).
    \item Showing that $|\Psi(x,t)|^2 = |A|^2$ and it's independent of time and space (1 mark).
\end{itemize}


\textbf{Part b : 5/5}

\begin{itemize}
    \item Correct derivatives : $\frac{\partial \Psi}{\partial x} = ikAe^{ikx-i\frac{\hbar k^2}{2m}t}$ and $\frac{\partial \Psi*}{\partial x} = -ikAe^{-ikx+i\frac{\hbar k^2}{2m}t}$ (1 mark).
    \item Correct calculation of the flux : $\Psi* \frac{\partial\Psi}{\partial x} - \Psi \frac{\partial\Psi*}{\partial x} = 2ik|A|^2$ (2 marks).
    \item Correct final expression and physical interpretation: $\Gamma = \frac{\hbar k }{m} |A|^2$, explanation that $\Gamma$ is the number of particles per unit time and space (2 marks).
\end{itemize}


\textbf{Feedback :}
Well-structured solution with correct methodology and final results. Clear physical interpretation of the flux.


Solution 3:

\textbf{Marks : 8/8}

\textbf{Feedback : }

\textbf{Part a : 3/3}

\textbf{Part a : 3/3}

\begin{itemize}
    \item Correct form of the time-dependent solution: $\Psi(x,t) = Ae^{ikx-i\frac{\hbar k^2}{2m}t}$ (1 mark).
    \item Correct expression for $E$ and its substitution: $E = \frac{\hbar^2 k^2}{2m}$ (1 mark).
    \item Showing that $|\Psi(x,t)|^2 = |A|^2$ and it's independent of time and space (1 mark).
\end{itemize}


\textbf{Part b : 5/5}

\begin{itemize}
    \item Correct derivatives : $\frac{\partial \Psi}{\partial x} = ikAe^{ikx-i\frac{\hbar k^2}{2m}t}$ and $\frac{\partial \Psi*}{\partial x} = -ikAe^{-ikx+i\frac{\hbar k^2}{2m}t}$ (1 mark).
    \item Correct calculation of the flux : $\Psi* \frac{\partial\Psi}{\partial x} - \Psi \frac{\partial\Psi*}{\partial x} = 2ik|A|^2$ (2 marks).
    \item Correct final expression and physical interpretation: $\Gamma = \frac{\hbar k }{m} |A|^2$, explanation that $\Gamma$ is the number of particles per unit time and space (2 marks).
\end{itemize}


\textbf{Feedback :}
Clear and concise solution. Correct application of mathematical principles and clear physical interpretation of the result.



\subsubsection*{Response 3:}

Solution 1:

\textbf{Marks : 8/8}

\textbf{Feedback : }

\textbf{Part a : 3/3}

\begin{itemize}
    \item Correct time-dependent wavefunction: $\Psi(x,t) = Ae^{i(kx-\frac{\hbar k^2}{2m}t)}$.
    \item Correctly shows the probability density $|\Psi(x,t)|^2 = |A|^2$ is constant.
    \item Clear and thorough explanation.
\end{itemize}


\textbf{Part b : 5/5}

\begin{itemize}
    \item Correct partial derivatives and calculation of probability flux.
    \item Correct final expression for $\Gamma$.
    \item Clear physical interpretation linking flux to particle velocity.
\end{itemize}


\textbf{Feedback :}
Excellent solution with clear and correct steps. Thorough explanations for both parts, and all key points covered.



Solution 2:

\textbf{Marks : 8/8}

\textbf{Feedback : }

\textbf{Part a : 3/3}

\begin{itemize}
    \item Correct time-dependent wavefunction: $\Psi(x,t) = Ae^{i(kx-\frac{\hbar k^2}{2m}t)}$.
    \item Correctly shows the probability density $|\Psi(x,t)|^2 = |A|^2$ is constant.
    \item Well-explained steps.
\end{itemize}


\textbf{Part b : 5/5}

\begin{itemize}
    \item Correct partial derivatives and calculation of probability flux.
    \item Correct final expression for $\Gamma$.
    \item Good physical interpretation linking flux to particle velocity.
\end{itemize}


\textbf{Feedback :}
Very well done. Correct answers with clear explanations and thorough understanding demonstrated.


Solution 3:

\textbf{Marks : 8/8}

\textbf{Feedback : }

\textbf{Part a : 3/3}

\begin{itemize}
    \item Correct time-dependent wavefunction: $\Psi(x,t) = Ae^{i(kx-\frac{\hbar k^2}{2m}t)}$.
    \item Correctly shows the probability density $|\Psi(x,t)|^2 = |A|^2$ is constant.
    \item Clear explanation.
\end{itemize}


\textbf{Part b : 5/5}

\begin{itemize}
    \item Correct partial derivatives and calculation of probability flux.
    \item Correct final expression for $\Gamma$.
    \item Clear physical interpretation linking flux to particle velocity.
\end{itemize}


\textbf{Feedback :}
Excellent work. Clear and correct solutions with thorough explanations. All points addressed accurately.

\textbf{Summary :}
All three solutions were perfect in both parts (a) and (b). Each solution demonstrated a strong understanding of the problem, correctly derived the necessary expressions, and provided clear and accurate explanations. Each solution earned full marks: 8 out of 8.


\subsubsection*{Response 4:}

Solution 1:

\textbf{Marks : 8/8}

\textbf{Feedback : }

\textbf{Part a : 3/3}

\begin{itemize}
    \item Correct time-dependent solution: 1/1 mark
    \item Correct expression for energy $E$ : 1/1 mark
    \item Correct probability density calculation: 1/1 mark
\end{itemize}


\textbf{Part b : 5/5}

\begin{itemize}
    \item Correct expressions for partial derivatives: 1/1 mark
    \item Correct calculation of probability flux: 2/2 marks
    \item Correct interpretation of the particle flux: 2/2 marks
\end{itemize}


\textbf{Feedback :}

\begin{itemize}
    \item Excellent explanation and clear derivation of the time-dependent wavefunction.
    \item Detailed and accurate calculation of the probability flux and its physical interpretation.
    \item Ensure to consistently denote complex conjugates clearly in the flux calculation to avoid any potential confusion.
\end{itemize}


Solution 2:

\textbf{Marks : 8/8}

\textbf{Feedback : }

\textbf{Part a : 3/3}

\begin{itemize}
    \item Correct time-dependent solution: 1/1 mark
    \item Correct expression for energy $E$: 1/1 mark
    \item Correct probability density calculation: 1/1 mark
\end{itemize}


\textbf{Part b : 5/5}

\begin{itemize}
    \item Correct expressions for partial derivatives: 1/1 mark
    \item Correct calculation of probability flux: 2/2 marks
    \item Correct interpretation of the particle flux: 2/2 marks
\end{itemize}


\textbf{General feedback :}

\begin{itemize}
    \item Very well-structured and clear explanation throughout the solution.
    \item Accurate mathematical derivations and correct application of the Schrödinger equation.
    \item Excellent interpretation of physical significance in terms of velocity and particle flow.
\end{itemize}


Solution 3:

\textbf{Marks : 8/8}

\textbf{Feedback : }

\textbf{Part a : 3/3}

\begin{itemize}
    \item Correct time-dependent solution: 1/1 mark
    \item Correct expression for energy $E$: 1/1 mark
    \item Correct probability density calculation: 1/1 mark
\end{itemize}


\textbf{Part b : 5/5}

\begin{itemize}
    \item Correct expressions for partial derivatives: 1/1 mark
    \item Correct calculation of probability flux: 2/2 marks
    \item Correct interpretation of the particle flux: 2/2 marks
\end{itemize}


\textbf{Feedback :}

\begin{itemize}
    \item Clear step-by-step derivation of the time-dependent wavefunction.
    \item Correctly handled the partial derivatives and flux calculation.
    \item Thorough physical interpretation, making the connection between mathematical results and physical meaning.
    \item Ensure consistency in notations, especially with complex conjugates in flux derivations, to maintain clarity.
\end{itemize}



\subsubsection*{Response 5:}

Solution 1:

\textbf{Marks : 8/8}

\textbf{Feedback : }

\textbf{Part a : 3/3}

\begin{itemize}
    \item Correctly identified the time-dependent solution. Clearly showed that the probability density is independent of space and time.
\end{itemize}


\textbf{Part b : 5/5}

\begin{itemize}
    \item Correctly computed the particle flux and provided a clear physical interpretation.
\end{itemize}



Solution 2:

\textbf{Marks : 8/8}

\textbf{Feedback : }

\textbf{Part a : 3/3}

\begin{itemize}
    \item The solution is correct and demonstrates the independence of the probability density from space and time clearly.
\end{itemize}


\textbf{Part b : 5/5}

\begin{itemize}
    \item The particle flux calculation and physical interpretation are both correct and well-explained.
\end{itemize}


Solution 3:

\textbf{Marks : 8/8}

\textbf{Feedback : }

\textbf{Part a : 3/3}

\begin{itemize}
    \item Correct time-dependent solution and clear demonstration of the probability density being constant.
\end{itemize}


\textbf{Part b : 5/5}

\begin{itemize}
    \item The flux calculation is accurate, and the physical interpretation is well-articulated.
\end{itemize}




\subsection*{Question 8:}

\subsubsection*{Response 1 :}


Solution 1 :

\textbf{Feedback :}

\textbf{Marks : 8/9}

\textbf{Part a : 4/5}

\begin{itemize}
    \item Correctly starts with the definition of expectation value and applies the product rule (2 marks).
    \item Successfully substitutes the time-dependent Schrödinger equation (1 mark).
    \item Correctly simplifies to the final commutator form but with minor algebraic simplification errors (1 mark).
\end{itemize}


\textbf{Part b : 4/4}

\begin{itemize}
    \item Correctly identifies and computes the commutator $[\hat{X}, \hat{H}]$ (2 marks).
    \item Correctly substitutes the commutator result into the expectation value formula (1 mark).
    \item The solution is clearly presented and correctly derives the final result (1 mark).
\end{itemize}

\textbf{Feedback :}
Good understanding and application of quantum mechanics principles. Minor algebraic errors in part a, but overall well-executed.



Solution 2 :

\textbf{Feedback :}

\textbf{Marks : 6/9}

\textbf{Part a : 4/5}

\begin{itemize}
    \item Correctly differentiates the expectation value and applies the product rule (2 marks).
    \item Uses the time-dependent Schrödinger equation accurately (1 mark).
    \item Minor simplification errors in arriving at the final commutator form (1 mark).
\end{itemize}


\textbf{Part b : 2/4}

\begin{itemize}
    \item Begins correctly but lacks a detailed and accurate computation of the commutator $[\hat{X}, \hat{H}]$ (1 mark).
    \item Attempts to substitute into the expectation value formula but lacks clarity and precision (1 mark).
\end{itemize}

\textbf{Feedback :}
Clear structure and logical flow. However, part b is less detailed and accurate in its commutator computation, needing a more thorough approach.



Solution 3 :

\textbf{Feedback :}

\textbf{Marks : 7/9}

\textbf{Part a : 4/5}

\begin{itemize}
    \item Begins with the definition and applies the product rule correctly (2 marks).
    \item Proper use of the Schrödinger equation (1 mark).
    \item Simplification to the final form is mostly correct but has minor errors (1 mark).
\end{itemize}


\textbf{Part b : 3/4}

\begin{itemize}
    \item Correctly identifies the Hamiltonian and begins to compute the commutator (2 marks).
    \item Correctly applies commutation relations but minor algebraic simplification errors (1 mark).
\end{itemize}

\textbf{Feedback :}
Solid understanding demonstrated. Clear presentation with minor errors in algebraic simplification. Part b shows a good approach but needs a bit more precision in the final steps.



\subsubsection*{Response 2 :}

Solution 1 :

\textbf{Feedback :}

\textbf{Marks : 9/9}

\textbf{Part a : 5/5}

\begin{itemize}
    \item Step 1 (2 marks): Correctly starts with the definition and differentiates the expectation value.
    \item Step 2 (2 marks): Applies the product rule and substitutes Schrödinger equations.
    \item Step 3 (1 mark): Correctly simplifies to the final expression involving the commutator.
\end{itemize}


\textbf{Part b : 4/4}

\begin{itemize}
    \item Step 1 (1 mark): Correctly states the relationship.
    \item Step 2 (2 marks): Computes the commutator and finds $\frac{i\hbar \hat{P}}{m}$.
    \item Step 3 (1 mark): Correctly substitutes to show the final result.
\end{itemize}

\textbf{Feedback :}
Excellent job! The solution is clear, thorough, and correctly applies quantum mechanical principles. All necessary steps are logically and accurately presented.



Solution 2 :

\textbf{Feedback :}

\textbf{Marks : 7/9}

\textbf{Part a : 4/5}

\begin{itemize}
    \item Step 1 (2 marks): Differentiates the expectation value correctly.
    \item Step 2 (1 mark): Applies the product rule but omits substitution of Schrödinger equations.
    \item Step 3 (1 mark): Simplifies but lacks clarity in showing steps to the commutator form.
\end{itemize}


\textbf{Part b : 3/4}

\begin{itemize}
    \item Step 1 (1 mark): Correctly starts with the Hamiltonian operator.
    \item Step 2 (1 mark): Attempts the commutator but lacks detail in the canonical commutation relation.
    \item Step 3 (1 mark): Correctly identifies the final substitution but lacks intermediate steps.
\end{itemize}

\textbf{Feedback :}
Good effort, but the solution needs more clarity and detail in deriving key steps, especially in part b). Ensure to show all intermediate steps and clearly state any assumptions or commutation relations used.



Solution 3 :

\textbf{Feedback :}

\textbf{Marks : 8/9}

\textbf{Part a : 4/5}

\begin{itemize}
    \item Step 1 (2 marks): Correctly starts with the definition and differentiates the expectation value.
    \item Step 2 (1 mark): Applies the product rule and substitutes Schrödinger equations.
    \item Step 3 (1 mark): Simplifies to the commutator form but could be clearer.
\end{itemize}


\textbf{Part b : 4/4}

\begin{itemize}
    \item Step 1 (1 mark): Correctly starts with the Hamiltonian operator.
    \item Step 2 (2 marks): Properly evaluates the commutator with detailed steps.
    \item Step 3 (1 mark): Correctly substitutes to show the final result.
\end{itemize}

\textbf{Feedback :}
Very good solution with detailed steps, especially in part b). However, part a) could benefit from a bit more clarity in presenting the final simplification steps.



\subsubsection*{Response 3 :}

Solution 1 :

\textbf{Feedback :}

\textbf{Marks : 9/9}

\textbf{Part a : 5/5}

\begin{itemize}
    \item Correctly starts with the definition of expectation value and proceeds with differentiation: 2 marks.
    \item Correct substitution of Schrödinger equation: 1 mark.
    \item Correct substitution of Schrödinger equation: 1 mark.
\end{itemize}


\textbf{Part b : 4/4}

\begin{itemize}
    \item Starts with the correct expression: 1 mark.
    \item Correctly evaluates the commutator $[\hat{X}, \hat{H}]$ including step-by-step commutation relation: 2 marks.
    \item Correct final substitution into the derived formula: 1 mark.
\end{itemize}


\textbf{Feedback :}
Excellent work. All steps are clearly shown with correct application of the Schrödinger equation and commutation relations. Solution is thorough and accurate.



Solution 2 :

\textbf{Feedback :}

\textbf{Marks : 9/9}

\textbf{Part a : 5/5}

\begin{itemize}
    \item Correctly starts with the definition of expectation value and proceeds with differentiation: 2 marks.
    \item Correct substitution of Schrödinger equation: 1 mark.
    \item Accurate simplification to final commutator expression: 2 marks.
\end{itemize}


\textbf{Part b : 4/4}

\begin{itemize}
    \item Starts with the correct expression: 1 mark.
    \item Correctly evaluates the commutator $[\hat{X}, \hat{H}]$ but lacks clarity in intermediate steps: 2 marks.
    \item Correct final substitution into the derived formula: 1 mark.
\end{itemize}

\textbf{Feedback :}
Well done. The solution is mostly clear and correct. Ensure intermediate steps are explained more thoroughly, especially in part b).



Solution 3 :

\textbf{Feedback :}

\textbf{Marks : 9/9}

\textbf{Part a : 5/5}

\begin{itemize}
    \item Correctly starts with the definition of expectation value and proceeds with differentiation: 2 marks.
    \item Correct substitution of Schrödinger equation: 1 mark.
    \item Accurate simplification to final commutator expression: 2 marks.
\end{itemize}


\textbf{Part b : 4/4}

\begin{itemize}
    \item Starts with the correct expression: 1 mark.
    \item Correctly evaluates the commutator $[\hat{X}, \hat{H}]$ including step-by-step commutation relation: 2 marks.
    \item Correct final substitution into the derived formula: 1 mark.
\end{itemize}

\textbf{Feedback :}
Excellent job. Your explanation is clear and methodical. The detailed steps in evaluating the commutator are well presented.

\textbf{Summary :}
All solutions provided are accurate and thorough. Each student has demonstrated a clear understanding of the concepts and correctly applied the relevant equations and principles to solve the problem.




\subsubsection*{Response 4 :}


Solution 1 :

\textbf{Feedback :}

\textbf{Marks : 9/9}

\textbf{Part a : 5/5}

\begin{itemize}
    \item Correctly starts with the definition of the expectation value.
    \item Uses the product rule for differentiation appropriately.
    \item Correctly substitutes the time-dependent Schrödinger equation.
    \item Simplifies to the required commutator form.
\end{itemize}


\textbf{Part b : 4/4}

\begin{itemize}
    \item Defines the Hamiltonian correctly.
    \item Evaluates the commutator $[\hat{X}, \hat{H}]$ accurately.
    \item Uses the commutator to derive the final result correctly.
\end{itemize}

\textbf{Feedback :}
Excellent work. Both parts of the solution are correct, clearly explained, and mathematically rigorous.



Solution 2 :

\textbf{Feedback :}

\textbf{Marks : 9/9}

\textbf{Part a : 5/5}

\begin{itemize}
    \item Starts with the definition of the expectation value.
    \item Uses the product rule for differentiation appropriately.
    \item Correctly substitutes the time-dependent Schrödinger equation.
    \item Simplifies to the required commutator form.
\end{itemize}


\textbf{Part b : 4/4}

\begin{itemize}
    \item Defines the Hamiltonian correctly.
    \item Evaluates the commutator $[\hat{X}, \hat{H}]$ accurately.
    \item Uses the commutator to derive the final result correctly.
\end{itemize}

\textbf{Feedback :}
Excellent. The solution is comprehensive, clear, and correctly applies the relevant quantum mechanics principles.



Solution 3 :

\textbf{Feedback :}

\textbf{Marks : 9/9}

\textbf{Part a : 5/5}

\begin{itemize}
    \item Starts with the definition of the expectation value.
    \item Uses the product rule for differentiation appropriately.
    \item Correctly substitutes the time-dependent Schrödinger equation.
    \item Simplifies to the required commutator form.
\end{itemize}


\textbf{Part b : 4/4}

\begin{itemize}
    \item Defines the Hamiltonian correctly.
    \item Evaluates the commutator $[\hat{X}, \hat{H}]$ accurately.
    \item Uses the commutator to derive the final result correctly.
\end{itemize}

\textbf{Feedback :}
Excellent. Both parts are well-explained and demonstrate a solid understanding of the quantum mechanics involved.


\subsubsection*{Response 5 :}


Solution 1 :

\textbf{Feedback :}

\textbf{Marks : 9/9}

\textbf{Part a : 5/5}

\begin{itemize}
    \item Correct differentiation and application of the product rule: 2 marks
    \item Correct substitution of the time-dependent Schrödinger equation: 1 mark
    \item Correct algebraic simplification to the commutator form: 2 marks
\end{itemize}


\textbf{Part b : 4/4}

\begin{itemize}
    \item Correct starting point and setup using the Hamiltonian: 1 mark
    \item Correct evaluation of the commutator $[\hat{X}, \hat{H}]$: 2 marks
    \item Correct substitution and final result: 1 mark
\end{itemize}

\textbf{Feedback :}
Excellent work. The solution is thorough and well-organized. Each step is clearly justified, and the mathematical manipulations are correct.



Solution 2 :

\textbf{Feedback :}

\textbf{Marks : 8/9}

\textbf{Part a : 5/5}

\begin{itemize}
    \item Correct differentiation and application of the product rule: 2 marks
    \item Correct substitution of the time-dependent Schrödinger equation: 1 mark
    \item Correct algebraic simplification to the commutator form: 2 marks
\end{itemize}


\textbf{Part b : 3/4}

\begin{itemize}
    \item Correct starting point and setup using the Hamiltonian: 1 mark
    \item Correct evaluation of the commutator $[\hat{X}, \hat{H}]$: 1 mark
    \item Incorrectly simplifying the commutator evaluation, not clearly using $[\hat{X}, \hat{P^2}] = 2i \hbar \hat{P}$: 0 marks
    \item Correct substitution and final result based on an incomplete commutator evaluation: 1 mark
\end{itemize}

\textbf{Feedback :}
Good attempt. The first part is solid and well-explained. However, the commutator evaluation in part b) is incomplete. The final result is correct, but the derivation lacks rigor.



Solution 3 :

\textbf{Feedback :}

\textbf{Marks : 9/9}

\textbf{Part a : 5/5}

\begin{itemize}
    \item Correct differentiation and application of the product rule: 2 marks
    \item Correct substitution of the time-dependent Schrödinger equation: 1 mark
    \item Correct algebraic simplification to the commutator form: 2 marks
\end{itemize}


\textbf{Part b : 4/4}

\begin{itemize}
    \item Correct starting point and setup using the Hamiltonian: 1 mark
    \item Correct evaluation of the commutator $[\hat{X}, \hat{H}]$: 2 marks
    \item Correct substitution and final result: 1 mark
\end{itemize}

\textbf{Feedback :}
Excellent work. The solution is methodical and accurate. All steps are justified, and the mathematical manipulations are correct.


\subsection*{Question 9 :}

\subsubsection*{Response 1:}

Solution 1 :

\textbf{Marks : 15/17}

\textbf{Feedback :}

\textbf{Part a : 5/5}

\begin{itemize}
    \item Correctly states the general solutions for both regions. (2 marks)
    \item Defines $k_1$ correctly. (1 mark)
    \item Defines $k_2$ correctly. (1 mark)
    \item Explains why $D = 0$. (1 mark)
\end{itemize}

\textbf{Part b : 5/5}

\begin{itemize}
    \item Correctly calculates the derivatives. (1 mark)
    \item Correctly applies boundary conditions and derives equations. (2 marks)
    \item Correctly solves for $B$ and $C$. (2 marks)
\end{itemize}

\textbf{Part c : 5/7}

\begin{itemize}
    \item Correctly substitutes for the incident flux. (1 mark)
    \item Correctly substitutes for the reflected flux but omits some key details about sign changes. (1 mark)
    \item Correctly substitutes for the transmitted flux but omits some key details about sign changes. (1 mark)
    \item Derives $T$ and $R$ correctly. (2 marks)
\end{itemize}

\textbf{General Feedback :}

\begin{itemize}
    \item Well-structured and detailed solution.
    \item Missing some details in flux calculations.
    \item Clear understanding and correct application of boundary conditions.
\end{itemize}


Solution 2 :

\textbf{Marks : 15/17}

\textbf{Feedback :}

\textbf{Part a : 5/5}

\begin{itemize}
    \item Correctly states the general solutions for both regions. (2 marks)
    \item Defines $k_1$ correctly. (1 mark)
    \item Defines $k_2$ correctly. (1 mark)
    \item Explanation for $D=0$ is implicit but clear from context. (1 mark)
\end{itemize}

\textbf{Part b : 5/5}

\begin{itemize}
    \item Correctly calculates the derivatives. (1 mark)
    \item Correctly applies boundary conditions and derives equations. (2 marks)
    \item Correctly solves for $B$ and $C$. (2 marks)
\end{itemize}

\textbf{Part c : 5/7}

\begin{itemize}
    \item Correctly substitutes for the incident flux. (1 mark)
    \item Correctly substitutes for the reflected flux but uses incorrect signs in flux calculation. (1 mark)
    \item Correctly substitutes for the transmitted flux but uses incorrect signs in flux calculation. (1 mark)
    \item Derives $T$ and $R$ correctly. (2 marks)
\end{itemize}

\textbf{General feedback :}

\begin{itemize}
    \item Good structure and thorough explanations.
    \item Misunderstandings in flux calculations lead to incorrect signs.
    \item Accurate boundary condition applications.
\end{itemize}


Solution 3 :

\textbf{Marks : 15/17}

\textbf{Feedback :}

\textbf{Part a : 5/5}

\begin{itemize}
    \item Correctly states the general solutions for both regions. (2 marks)
    \item Defines $k_1$ correctly. (1 mark)
    \item Defines $k_2$ correctly. (1 mark)
    \item Explains why $D = 0$. (1 mark)
\end{itemize}

\textbf{Part b : 5/5}

\begin{itemize}
    \item Correctly calculates the derivatives. (1 mark)
    \item Correctly applies boundary conditions and derives equations. (2 marks)
    \item Correctly solves for $B$ and $C$. (2 marks)
\end{itemize}


\textbf{Part c : 5/7}

\begin{itemize}
    \item Correctly substitutes for the incident flux. (1 mark)
    \item Correctly substitutes for the reflected flux but uses incorrect signs in flux calculation. (1 mark)
    \item Correctly substitutes for the transmitted flux but uses incorrect signs in flux calculation. (1 mark)
    \item Derives $T$ and $R$ correctly. (2 marks)
\end{itemize}

\textbf{General feedback :}

\begin{itemize}
    \item Well-organized and clear explanation.
    \item Misunderstandings in flux calculations similar to other solutions.
    \item Effective boundary condition use and problem-solving strategy.
\end{itemize}


\subsubsection*{Response 2:}

Solution 1 :

\textbf{Marks : 17/17}

\textbf{Feedback :}

\textbf{Part a : 5/5}

\begin{itemize}
    \item Correct definition of $k_1$ and $k_2$: 2 marks
    \item Correct general solution in both regions: 2 marks
    \item Explanation and choosing $D = 0$ for $x>0$: 1 mark
\end{itemize}

\textbf{Part b : 5/5}

\begin{itemize}
    \item Correctly derived boundary conditions: 2 marks
    \item Correctly solved for $B$ and $C$: 3 marks
\end{itemize}

\textbf{Part c : 7/7}

\begin{itemize}
    \item Correct incident flux calculation: 1 mark
    \item Correct reflected flux calculation: 2 marks
    \item Correct transmitted flux calculation: 2 marks
    \item Correctly derived $T$ and $R$: 2 marks
\end{itemize}

\textbf{General feedback :}
Excellent solution with clear and correct derivations. Well-organized and accurate use of equations and boundary conditions.


Solution 2 :

\textbf{Marks : 17/17}

\textbf{Feedback :}

\textbf{Part a : 5/5}

\begin{itemize}
    \item Correct definition of $k_1$ and $k_2$: 2 marks
    \item Correct general solution in both regions: 2 marks
    \item Explanation and choosing $D = 0$ for $x>0$: 1 mark
\end{itemize}

\textbf{Part b : 5/5}

\begin{itemize}
    \item Correctly derived boundary conditions: 2 marks
    \item Correctly solved for $B$ and $C$: 3 marks
\end{itemize}

\textbf{Part c : 7/7}

\begin{itemize}
    \item Correct incident flux calculation: 1 mark
    \item Correct reflected flux calculation: 2 marks
    \item Correct transmitted flux calculation: 2 marks
    \item Correctly derived $T$ and $R$: 2 marks
\end{itemize}

\textbf{General feedback :}
Comprehensive and accurate solution with detailed steps and correct application of boundary conditions and flux calculations.


Solution 3 :

\textbf{Marks : 17/17}

\textbf{Feedback :}

\textbf{Part a : 5/5}

\begin{itemize}
    \item Correct definition of $k_1$ and $k_2$: 2 marks
    \item Correct general solution in both regions: 2 marks
    \item Explanation and choosing $D = 0$ for $x>0$: 1 mark
\end{itemize}

\textbf{Part b : 5/5}

\begin{itemize}
    \item Correctly derived boundary conditions: 2 marks
    \item Correctly solved for $B$ and $C$: 3 marks
\end{itemize}

\textbf{Part c : 7/7}

\begin{itemize}
    \item Correct incident flux calculation: 1 mark
    \item Correct reflected flux calculation: 2 marks
    \item Correct transmitted flux calculation: 2 marks
    \item Correctly derived $T$ and $R$: 2 marks
\end{itemize}

\textbf{General feedback :}
Well-structured and precise solution. Each step is clearly justified, leading to correct results.



\subsubsection*{Response 3:}

Solution 1 :

\textbf{Marks : 17/17}

\textbf{Feedback :}

\textbf{Part a : 5/5}

\begin{itemize}
    \item Correct general solution for both regions: 2/2
    \item Correct expressions for $k_1$ and $k_2$: 2/2
    \item Chose $D = 0$ for $x > 0$: 1/1
\end{itemize}

\textbf{Part b : 5/5}

\begin{itemize}
    \item Correct boundary conditions:  1/1
    \item Correct derivatives: 1/1
    \item Correct continuity equations: 2/2
    \item Solved for $B$ and $C$: 1/1
\end{itemize}

\textbf{Part c : 7/7}

\begin{itemize}
    \item Incident flux calculation: 1/1
    \item Reflected flux calculation: 2/2
    \item Transmitted flux calculation: 2/2
    \item Correct probabilities for transmission and reflection: 2/2
\end{itemize}

\textbf{General feedback :}
Excellent work! You provided a thorough and clear solution, with all steps correctly presented and calculated.


Solution 2 :

\textbf{Marks : 17/17}

\textbf{Feedback :}

\textbf{Part a : 5/5}

\begin{itemize}
    \item Correct general solution for both regions: 2/2
    \item Correct expressions for $k_1$ and $k_2$: 2/2
    \item Chose $D = 0$ for $x > 0$: 1/1
\end{itemize}

\textbf{Part b : 5/5}

\begin{itemize}
    \item Correct boundary conditions:  1/1
    \item Correct derivatives: 1/1
    \item Correct continuity equations: 2/2
    \item Solved for $B$ and $C$: 1/1
\end{itemize}

\textbf{Part c : 7/7}

\begin{itemize}
    \item Incident flux calculation: 1/1
    \item Reflected flux calculation: 2/2
    \item Transmitted flux calculation: 2/2
    \item Correct probabilities for transmission and reflection: 2/2
\end{itemize}

\textbf{General feedback :}
Excellent work! You have shown a deep understanding of the problem and provided a clear and comprehensive solution.


Solution 3 :

\textbf{Marks : 17/17}

\textbf{Feedback :}

\textbf{Part a : 5/5}

\begin{itemize}
    \item Correct general solution for both regions: 2/2
    \item Correct expressions for $k_1$ and $k_2$: 2/2
    \item Chose $D = 0$ for $x > 0$: 1/1
\end{itemize}

\textbf{Part b : 5/5}

\begin{itemize}
    \item Correct boundary conditions:  1/1
    \item Correct derivatives: 1/1
    \item Correct continuity equations: 2/2
    \item Solved for $B$ and $C$: 1/1
\end{itemize}

\textbf{Part c : 7/7}

\begin{itemize}
    \item Incident flux calculation: 1/1
    \item Reflected flux calculation: 2/2
    \item Transmitted flux calculation: 2/2
    \item Correct probabilities for transmission and reflection: 2/2
\end{itemize}

\textbf{General feedback :}
Excellent work! Your solution is well-structured and you have demonstrated a solid understanding of the problem and its solution.

\subsubsection*{Response 4:}

Solution 1 :

\textbf{Marks : 17/17}

\textbf{Feedback :}

\textbf{Part a : 5/5}

\begin{itemize}
    \item Correctly identified and solved the Schrödinger equation in both regions.
    \item Proper definitions of $k_1$ and $k_2$.
\end{itemize}

\textbf{Part b : 5/5}

\begin{itemize}
    \item Correct application of boundary conditions and calculations for $B$ and $C$.
\end{itemize}

\textbf{Part c : 7/7}

\begin{itemize}
    \item Correctly calculated incident, reflected, and transmitted flux.
    \item Correctly derived the expressions for $T$ and $R$.
\end{itemize}

\textbf{General feedback :}
Excellent job. Clear and concise presentation of each part with correct calculations and reasoning.


Solution 2 :

\textbf{Marks : 17/17}

\textbf{Feedback :}

\textbf{Part a : 5/5}

\begin{itemize}
    \item Correctly identified and solved the Schrödinger equation in both regions.
    \item Proper definitions of $k_1$ and $k_2$.
\end{itemize}

\textbf{Part b : 5/5}

\begin{itemize}
    \item Correct application of boundary conditions and calculations for $B$ and $C$.
\end{itemize}

\textbf{Part c : 7/7}

\begin{itemize}
    \item Correctly calculated incident, reflected, and transmitted flux.
    \item Correctly derived the expressions for $T$ and $R$.
\end{itemize}

\textbf{General feedback :}
Well-organized solution with correct steps and clear explanations.

Solution 3 :

\textbf{Marks : 17/17}

\textbf{Feedback :}

\textbf{Part a : 5/5}

\begin{itemize}
    \item Correctly identified and solved the Schrödinger equation in both regions.
    \item Proper definitions of $k_1$ and $k_2$.
\end{itemize}

\textbf{Part b : 5/5}

\begin{itemize}
    \item Correct application of boundary conditions and calculations for $B$ and $C$.
\end{itemize}

\textbf{Part c : 7/7}

\begin{itemize}
    \item Correctly calculated incident, reflected, and transmitted flux.
    \item Correctly derived the expressions for $T$ and $R$.
\end{itemize}

\textbf{General feedback :}
Clear and detailed solution with correct calculations and logical flow.



\subsubsection*{Response 5:}

Solution 1 :

\textbf{Marks : 17/17}

\textbf{Feedback :}

\textbf{Part a : 5/5}

\begin{itemize}
    \item Correctly identifies and defines $k_1$ and $k_2$.
    \item Provides the general solutions for $\psi_1 (x)$ and $\psi_2 (x)$.
\end{itemize}

\textbf{Part b : 5/5}

\begin{itemize}
    \item Accurately applies boundary conditions and derives the expressions for $B$ and $C$.
\end{itemize}

\textbf{Part c : 7/7}

\begin{itemize}
    \item Correctly calculates the flux for the incident, reflected, and transmitted beams.
    \item Derives the correct expressions for $T$ and $R$.
\end{itemize}

\textbf{General feedback :}
Excellent solution with clear, systematic derivations and correct final results. Well done.


Solution 2 :

\textbf{Marks : 17/17}

\textbf{Feedback :}

\textbf{Part a : 5/5}

\begin{itemize}
    \item Correctly identifies and defines $k_1$ and $k_2$.
    \item Provides the general solutions for $\psi_1 (x)$ and $\psi_2 (x)$.
\end{itemize}

\textbf{Part b : 5/5}

\begin{itemize}
    \item Accurately applies boundary conditions and derives the expressions for $B$ and $C$.
\end{itemize}

\textbf{Part c : 7/7}

\begin{itemize}
    \item Correctly calculates the flux for the incident, reflected, and transmitted beams.
    \item Derives the correct expressions for $T$ and $R$.
\end{itemize}

\textbf{General feedback :}
Clear and accurate solution with correct derivations and results.


Solution 3 :

\textbf{Marks : 17/17}

\textbf{Feedback :}

\textbf{Part a : 5/5}

\begin{itemize}
    \item Correctly identifies and defines $k_1$ and $k_2$.
    \item Provides the general solutions for $\psi_1 (x)$ and $\psi_2 (x)$.
\end{itemize}

\textbf{Part b : 5/5}

\begin{itemize}
    \item Accurately applies boundary conditions and derives the expressions for $B$ and $C$.
\end{itemize}

\textbf{Part c : 7/7}

\begin{itemize}
    \item Correctly calculates the flux for the incident, reflected, and transmitted beams.
    \item Derives the correct expressions for $T$ and $R$.
\end{itemize}

\textbf{General feedback :}
Comprehensive and correct solution. Good work.


\textbf{Summary :}
All three solutions are correct and follow a clear and systematic approach to solving the problem. Each solution correctly identifies the wave vectors, applies boundary conditions accurately, and derives the flux and probability expressions correctly. Therefore, each solution receives full marks.


\subsection*{Question 10 :}

\subsubsection*{Response 1 :}

Solution 1 :

\textbf{Marks : 12/16}

\textbf{Feedback :}

\textbf{Part a : 4/5}

\begin{itemize}
    \item The normalization process is correct but contains an algebraic error when simplifying. The expectation value of the Hamiltonian is correctly approached but lacks the final simplified form.
\end{itemize}

\textbf{Part b : 2/2}

\begin{itemize}
    \item The expression for the perturbation in terms of ladder operators is correct and clearly presented.
\end{itemize}

\textbf{Part c : 6/9}

\begin{itemize}
    \item The initial steps are correct, including the expectation values and the form of the total Hamiltonian. However, the differentiation and final algebra are not fully detailed, leading to an incomplete derivation of the value of $b$.
\end{itemize}


Solution 2 :

\textbf{Marks : 10/16}

\textbf{Feedback :}

\textbf{Part a : 3/5}

\begin{itemize}
    \item Correct steps are taken for normalization but some algebraic mistakes and an incorrect interpretation of normalization lead to a less clear solution. The expectation value of the Hamiltonian is correctly formulated but not simplified.
\end{itemize}

\textbf{Part b : 2/2}

\begin{itemize}
    \item The perturbation is correctly rewritten in terms of the raising and lowering operators.
\end{itemize}

\textbf{Part c : 5/9}

\begin{itemize}
    \item The solution starts correctly but lacks a complete and detailed calculation. The final result is not derived, and steps are skipped, leading to an incomplete answer.
\end{itemize}



Solution 3 :

\textbf{Marks : 8/16}

\textbf{Feedback :}

\textbf{Part a : 4/5}

\begin{itemize}
    \item Correct normalization and correct formulation of the expectation value, but missing final simplified form. Some redundancy in explaining normalization.
\end{itemize}

\textbf{Part b : 1/2}

\begin{itemize}
    \item Correct idea but contains a mistake in the coefficient and lacks the final simplified form in terms of $g$.
\end{itemize}

\textbf{Part c : 3/9}

\begin{itemize}
    \item The explanation is more qualitative and lacks the detailed mathematical derivation needed to solve for $b$. Steps are described but not sufficiently executed.
\end{itemize}


\subsubsection*{Response 2 :}

Solution 1 :

\textbf{Marks : 14/16}

\textbf{Feedback :}

\textbf{Part a : 4/5}

\begin{itemize}
    \item Normalization (2/2): Correct normalization process and equation, correct form for normalized state.
    \item Expectation value of Hamiltonian (2/3): Correct form but the final step calculation is not shown.
\end{itemize}

\textbf{Part b : 2/2}

\begin{itemize}
    \item Raising and lowering operators (2/2): Correctly expressed perturbation in terms of ladder operators.
\end{itemize}

\textbf{Part c : 8/9}

\begin{itemize}
    \item Expectation value of original Hamiltonian (1/1): Correct.
    \item Calculation of perturbation expectation (3/3): Correct.
    \item Total expectation value (1/1): Correct.
    \item Finding $b$ by differentiation (1/1): Correct.
    \item Solving for $b$ (2/2): Correct.
    \item Justification for negative square root (0/1): Justification for choosing the negative root is missing.
\end{itemize}

\textbf{General feedback :}
Good understanding of normalization and perturbation concepts. Missed detailed justification for the final form of $b$.

Solution 2 :

\textbf{Marks : 13/16}

\textbf{Feedback :}

\textbf{Part a : 3/5}

\begin{itemize}
    \item Normalization (2/2): Correct normalization steps.
    \item Expectation value of Hamiltonian (1/3): Correct method but final value is incorrect.
\end{itemize}

\textbf{Part b : 2/2}

\begin{itemize}
    \item Raising and lowering operators (2/2): Correct expression of the perturbation.
\end{itemize}

\textbf{Part c : 8/9}

\begin{itemize}
    \item Expectation value of original Hamiltonian (1/1): Correct.
    \item Calculation of perturbation expectation (3/3): Correct.
    \item Total expectation value (1/1): Correct.
    \item Finding $b$ by differentiation (1/1): Correct.
    \item Solving for $b$ (2/2): Correct.
    \item Justification for negative square root (0/1): Justification is missing.
\end{itemize}

\textbf{General feedback :}
Good step-by-step approach, but the error in the expectation value calculation reduced marks. Missing the justification for the negative root in part c.


Solution 3 :

\textbf{Marks : 13/16}

\textbf{Feedback :}

\textbf{Part a : 3/5}

\begin{itemize}
    \item Normalization (2/2): Correct normalization process.
    \item Expectation value of Hamiltonian (1/3): Correct method but errors in algebra and final value.
\end{itemize}

\textbf{Part b : 1/2}

\begin{itemize}
    \item Raising and lowering operators (1/2): Correct form but missing full simplification.
\end{itemize}

\textbf{Part c : 9/9}

\begin{itemize}
    \item Expectation value of original Hamiltonian (1/1): Correct.
    \item Calculation of perturbation expectation (3/3): Correct.
    \item Total expectation value (1/1): Correct.
    \item Finding $b$ by differentiation (1/1): Correct.
    \item Solving for $b$ (2/2): Correct.
    \item Justification for negative square root (1/1): Provided a clear justification.
\end{itemize}

\textbf{General feedback :}
Very thorough solution with clear steps. Small algebra errors in part a) and incomplete expression in part b). Excellent work on part c) with justification included.



\subsubsection*{Response 3 :}

Solution 1 :

\textbf{Marks : 16/16}

\textbf{Feedback :}

\textbf{Part a : 5/5}

\begin{itemize}
    \item Correctly identifies normalization condition and sets inner product to 1.
    \item Correctly expands inner product and finds the normalization factor $a$.
    \item Properly computes the expectation value of the Hamiltonian.
\end{itemize}

\textbf{Part b : 2/2}

\begin{itemize}
    \item Correctly rewrites $\hat{x}$ in terms of ladder operators.
    \item Correctly substitutes into $\hat{H}'$.
\end{itemize}

\textbf{Part c : 9/9}

\begin{itemize}
    \item Correctly writes the expectation value of the original Hamiltonian.
    \item Properly calculates the expectation value of the perturbation.
    \item Correctly finds the total expectation value.
    \item Differentiates correctly to find the minimizing condition.
    \item Solves the resulting equation for $b$.
    \item Justifies the choice of negative square root.
\end{itemize}

\textbf{General feedback :}
Excellent solution, well-explained and mathematically accurate in all parts.


Solution 2 :

\textbf{Marks : 8/16}

\textbf{Feedback :}

\textbf{Part a : 2/5}

\begin{itemize}
    \item Correctly identifies the normalization condition and sets inner product to 1.
    \item Expands inner product but introduces an error in finding the value of $b$. (Assumes $b$ should be such that $1 + |b|^2 = 1$ leading to incorrect conclusion.)
    \item Properly computes the expectation value of the Hamiltonian using incorrect $b$.
\end{itemize}

\textbf{Part b : 2/2}

\begin{itemize}
    \item Correctly rewrites $\hat{x}$ in terms of ladder operators.
    \item Correctly substitutes into $\hat{H}'$.
\end{itemize}

\textbf{Part c : 4/9}

\begin{itemize}
    \item Writes the expectation value of the original Hamiltonian but uses incorrect normalization.
    \item Calculates the expectation value of the perturbation.
    \item Finds total expectation value with incorrect normalization.
    \item Differentiates to find minimizing condition but does not solve correctly for $b$.
\end{itemize}

\textbf{General feedback :}
Correct approach but flawed in the normalization and subsequent calculations. Ensure proper normalization to avoid errors in following steps.



Solution 3 :

\textbf{Marks : 8/16}

\textbf{Feedback :}

\textbf{Part a : 2/5}

\begin{itemize}
    \item Correctly identifies normalization condition and sets inner product to 1.
    \item Expands inner product but introduces an error in finding the value of $b$ (similar to Solution 2).
    \item Properly computes the expectation value of the Hamiltonian using incorrect $b$.
\end{itemize}

\textbf{Part b : 2/2}

\begin{itemize}
    \item Correctly rewrites $\hat{x}$ in terms of ladder operators.
    \item Correctly substitutes into $\hat{H}'$.
\end{itemize}

\textbf{Part c : 4/9}

\begin{itemize}
    \item Writes the expectation value of the original Hamiltonian but uses incorrect normalization.
    \item Calculates the expectation value of the perturbation.
    \item Finds total expectation value with incorrect normalization.
    \item Differentiates to find minimizing condition but does not solve correctly for $b$
\end{itemize}

\textbf{General feedback :}
Correct methodology but marred by the initial error in normalization, leading to incorrect results in subsequent steps. Focus on accurate normalization.


\subsubsection*{Response 4 :}

Solution 1 :

\textbf{Marks : 10/16}

\textbf{Feedback :}

\textbf{Part a : 3/5}

\begin{itemize}
    \item The normalization process is correctly started with the inner product but mistakenly concludes with $1 + |b|^2 = 1$.
    \item Correctly recalculates the expectation value of the Hamiltonian, but the normalization is incorrect.
\end{itemize}

\textbf{Part b : 2/2}

\begin{itemize}
    \item Correctly expresses $\hat{x}$ in terms of ladder operators.
    \item Correctly rewrites the perturbation in terms of $\hat{a_+}$ and $\hat{a_-}$.
\end{itemize}

\textbf{Part c : 5/9}

\begin{itemize}
    \item Correctly writes the expectation value of the original Hamiltonian.
    \item The calculation of the perturbation expectation is not explicitly shown.
    \item The total expectation value is correctly obtained but the minimization process and solving for $b$ are not detailed.
    \item The final value for $b$ is mentioned but not derived.
\end{itemize}

\textbf{General feedback :}
The solution shows a good understanding of the problem but lacks detailed calculations for the normalization in part a) and the variational principle in part c). Ensure each step is fully explained and justified.


Solution 2 :

\textbf{Marks : 9/16}

\textbf{Feedback :}

\textbf{Part a : 3/5}

\begin{itemize}
    \item Correctly starts with the normalization but mistakenly concludes $1 + |b|^2 = 1$.
    \item Correctly recalculates the expectation value of the Hamiltonian but with the incorrect normalization step.
\end{itemize}

\textbf{Part b : 2/2}

\begin{itemize}
    \item Correctly expresses $\hat{x}$ in terms of ladder operators.
    \item Correctly rewrites the perturbation in terms of $\hat{a_+}$ and $\hat{a_-}$.
\end{itemize}

\textbf{Part c : 4/9}

\begin{itemize}
    \item Correctly mentions the variational principle but does not explicitly show the detailed calculations.
    \item The final expression for $b$ is mentioned but not derived.
\end{itemize}

\textbf{General feedback :}
Good attempt but lacks detail in the derivations. Ensure to properly show all steps and verify the normalization process. The explanation for part c) needs more detailed calculations and justification.



Solution 3 :

\textbf{Marks : 9/16}

\textbf{Feedback :}

\textbf{Part a : 3/5}

\begin{itemize}
    \item Correctly starts with the normalization but mistakenly concludes $1 + |b|^2 = 1$.
    \item Correctly recalculates the expectation value of the Hamiltonian but with the incorrect normalization step.
\end{itemize}

\textbf{Part b : 2/2}

\begin{itemize}
    \item Correctly expresses $\hat{x}$ in terms of ladder operators.
    \item Correctly rewrites the perturbation in terms of $\hat{a_+}$ and $\hat{a_-}$.
\end{itemize}

\textbf{Part c : 4/9}

\begin{itemize}
    \item Correctly mentions the variational principle but does not show the detailed steps.
    \item Mentions the final expression for $b$ without detailed derivation.
\end{itemize}

\textbf{General feedback :}
The solution shows understanding but lacks detailed calculations and justification for normalization and the variational principle. Ensure to work through each step thoroughly and present the derivations clearly.

\textbf{Summary :}
All solutions show a good understanding of the problem but have similar issues with incorrect normalization in part a) and lack detailed derivations for the variational principle in part c). Each solution correctly handles part b). To improve, ensure each step is fully explained, especially the normalization and variational principle steps, and verify calculations thoroughly.



\subsubsection*{Response 5 :}

Solution 1 :

\textbf{Marks : 14/16}

\textbf{Feedback :}

\textbf{Part a : 4/5}

\begin{itemize}
    \item Correct approach to normalization, with the correct expression for $a$.
    \item Correct final normalized state.
    \item The expectation value is derived correctly.
\end{itemize}

\textbf{Part b : 2/2}

\begin{itemize}
    \item Correct expression for $\hat{x}$ in terms of ladder operators.
    \item Correctly substituted into $\hat{H}'$.
\end{itemize}

\textbf{Part c : 8/9}

\begin{itemize}
    \item Correct expression for $<H>$.
    \item Correctly calculated $<H'>$.
    \item Correct expression for $<H_tot>$.
    \item Correct differentiation and solving for $b$.
    \item Justified the negative square root: Not addressed.
\end{itemize}

\textbf{General feedback :}
The solution is well-structured and mostly correct. However, the explanation for choosing the negative root in part c) is missing.


Solution 2 :

\textbf{Marks : 3/16}

\textbf{Feedback :}

\textbf{Part a : 1/5}

\begin{itemize}
    \item Correct approach, but the condition $|b|^2 = 1$ is incorrect.
    \item Incorrect normalization leads to an incorrect expectation value.
\end{itemize}

\textbf{Part b : 2/2}

\begin{itemize}
    \item Correct expression for $\hat{x}$ in terms of ladder operators.
    \item Correctly substituted into $\hat{H}'$.
\end{itemize}

\textbf{Part c : 0/9}

\begin{itemize}
    \item Expectation value of original Hamiltonian: Not explicitly shown.
    \item Expectation value of perturbation: Not explicitly shown.
    \item Total expectation value of the Hamiltonian: Not explicitly shown.
    \item Minimizing $<H_{tot}>$: Not explicitly shown.
    \item Choosing the negative root: Not addressed.
\end{itemize}

\textbf{General feedback :}
The solution contains several conceptual errors and lacks detail in part c). The normalization condition in part a) is incorrectly applied, leading to incorrect results.



Solution 3 :

\textbf{Marks : 3/16}

\textbf{Feedback :}

\textbf{Part a : 1/5}

\begin{itemize}
    \item Correct approach, but the normalization condition is incorrectly applied.
    \item Incorrect due to wrong normalization.
\end{itemize}

\textbf{Part b : 2/2}

\begin{itemize}
    \item Correct expression for $\hat{x}$ in terms of ladder operators.
    \item Correctly substituted into $\hat{H}'$.
\end{itemize}

\textbf{Part c : 0/9}

\begin{itemize}
    \item Expectation value of original Hamiltonian: Not explicitly shown.
    \item Expectation value of perturbation: Not explicitly shown.
    \item Total expectation value of the Hamiltonian: Not explicitly shown.
    \item Minimizing $<H_{tot}>$: Not explicitly shown.
    \item Choosing the negative root: Not addressed.
\end{itemize}

\textbf{General feedback :}
The solution contains several conceptual errors and lacks detail in part c). The normalization condition in part a) is incorrectly applied, leading to incorrect results. The steps in part c) are not shown explicitly.


\end{document}