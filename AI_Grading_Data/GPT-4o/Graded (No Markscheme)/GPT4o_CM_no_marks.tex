\documentclass[a4paper,11pt]{article}
\setlength\parindent{0pt}
\usepackage{amsmath, amssymb}
\usepackage[margin=1in,letterpaper]
{geometry} % decreases margins

\usepackage{setspace}
\setstretch{1.2}
\usepackage{spverbatim}

\title{GPT-4o Grading (No Mark Scheme)}

\begin{document}

\maketitle
This document compiles the GPT-4o responses when prompted to grade the solutions which have been generated. No markscheme is given to GPT-4o in the prompt. This is the grading for the Classical Mechanics (CM) questions. 

\section{CM}
\subsection*{Question 1:}

\textbf{Response 1:}

Solution 1:

**Grade: 4/4**

**Feedback:**

- The definition of a conservative force is accurate and well-explained, mentioning the path independence and work done in a closed loop being zero.

- The mathematical definition of potential energy is correctly stated, with a clear explanation that \( \underline{F} = -\nabla U \).

- The explanation of why potential energy cannot be defined for non-conservative forces is thorough and logically presented.

- Overall, this solution is clear, concise, and covers all required aspects of the problem.

Solution 2:

**Grade: 3/4**

**Feedback:**

- The definition of a conservative force is correct, but the explanation involving the conservation of mechanical energy could be misleading if not clarified further.

- The mathematical definition of potential energy is mostly correct, but it introduces potential confusion by using \(\Delta PE = -W\) without explicitly stating the work done as a line integral.

- The explanation of why potential energy cannot be defined for non-conservative forces is accurate but could benefit from a more explicit mention of energy dissipation mechanisms.

- Overall, this solution is good but could be improved with more precise mathematical expressions and clearer distinction between conservative and non-conservative forces.

Solution 3:

**Grade: 4/4**

**Feedback:**

- The definition of a conservative force is accurate and well-explained, with a mention of path independence and energy conservation.

- The mathematical definition of potential energy is correctly and thoroughly derived, including the differential form \( dU = -\underline{F} \cdot d\underline{r} \) and the condition \( \nabla \times \underline{F} = 0 \).

- The explanation of why potential energy cannot be defined for non-conservative forces is clear and logically sound, mentioning the path dependence and energy dissipation.

- Overall, this solution is comprehensive, mathematically rigorous, and effectively addresses all aspects of the problem.

\bigskip
\textbf{Response 2:}

Solution 1

**Grade: 4/4**

**Feedback:**

- Comprehensive definition of a conservative force.

- Correct mathematical representation of potential energy.

- Clear explanation of why potential energy cannot be defined for non-conservative forces.

- Detailed and accurate, covering all aspects of the problem.

Solution 2

**Grade: 3/4**

**Feedback:**

- Good definition of a conservative force and mention of examples.

- Correct mathematical definition of potential energy but lacks clarity and depth compared to Solution 1.

- Correct reasoning for non-conservative forces but less thorough explanation.

- Slightly less precise and detailed than Solution 1.

Solution 3

**Grade: 3/4**

**Feedback:**

- Accurate definition of a conservative force and inclusion of examples.

- Correct mathematical expression for potential energy but slightly confusing presentation.

- Good explanation of why non-conservative forces cannot have potential energy, mentioning the curl condition.

- Detailed but could be more concise and clear, particularly in mathematical explanations.

\bigskip
\textbf{Response 3:}

Solution 1:

**Grade: 4/4**

**Feedback:**

This solution is clear and comprehensive. It correctly defines a conservative force and explains the path independence of the work done. The mathematical definition of potential energy and the relation \( \underline{F} = -\nabla U \) is accurate. The explanation of why potential energy cannot be defined for non-conservative forces is thorough, mentioning the path dependency and the dissipation of energy. The examples given are relevant and well-explained.

Solution 2:

**Grade: 3/4**

**Feedback:**

This solution provides a correct definition of a conservative force and explains the concept of path independence. The mathematical explanation is mostly correct, but the definition of potential energy could be clearer. The solution states \( \Delta PE = -W \), which is correct, but it doesn't explicitly relate this to the gradient of potential energy until later. The explanation of why potential energy cannot be defined for non-conservative forces is correct but could be more detailed. Overall, it is a good solution but lacks some clarity in the mathematical explanation.

Solution 3:

**Grade: 4/4**

**Feedback:**

This solution is very detailed and well-explained. It correctly defines a conservative force and the path independence of work. The mathematical explanation is precise, and it correctly derives the relationship \( dU = -\underline{F} \cdot d\underline{r} \). The integration part is a nice touch, showing a deeper understanding. The explanation of why non-conservative forces cannot have a potential energy is accurate, including the mention of the curl condition \( \nabla \times \underline{F} = 0 \). This solution is thorough and covers all necessary points effectively.

\bigskip
\textbf{Response 4:}

Solution 1

**Grade: 4/4**

**Feedback:**

- Excellent definition of a conservative force, correctly noting the path independence and zero work done over a closed loop.

- The mathematical definition of potential energy using the gradient is accurately described.

- The explanation for why non-conservative forces cannot have a potential energy is clear and includes the concept of energy dissipation.

- Well-structured and comprehensive.

Solution 2

**Grade: 3/4**

**Feedback:**

- The definition of a conservative force is mostly correct but lacks the explicit mention of zero work done over a closed loop.

- The mathematical definition of potential energy is a bit vague. Introducing \( \Delta PE = -W \) without clarity on the reference point or detailed explanation can be confusing.

- The explanation for non-conservative forces is correct but could benefit from more detail on how non-conservative forces convert energy into other forms.

- Good overall, but lacks some clarity and detail in the mathematical explanation.

Solution 3

**Grade: 4/4**

**Feedback:**

- Accurate definition of a conservative force, including path independence and conservation of mechanical energy.

- The mathematical definition of potential energy is thorough and well-explained with clear notation.

- The explanation for non-conservative forces is detailed, mentioning the curl condition and path dependence, and the conversion of mechanical energy into other forms.

- Well-structured and comprehensive, with clear explanations and correct use of mathematical concepts.

\bigskip
\textbf{Response 5:}

Here are the grades and feedback for each solution:

**Solution 1:**

**Grade: 4/4**

**Feedback:**

- Clear and accurate definition of a conservative force, mentioning the path-independence of work and the closed-loop condition.

- Correct mathematical definition of potential energy for a conservative force.

- Thorough explanation of why potential energy cannot be defined for non-conservative forces, including a discussion of energy dissipation.

**Solution 2:**

**Grade: 3/4**

**Feedback:**

- Accurate definition of a conservative force and good examples provided.

- Correct mathematical expression for potential energy but lacks clarity in defining it explicitly.

- Good explanation of why non-conservative forces do not have potential energy, though the link between mechanical energy conservation and potential energy could be clearer.

**Solution 3:**

**Grade: 4/4**

**Feedback:**

- Precise definition of a conservative force, highlighting the path-independence of work.

- Detailed mathematical treatment of potential energy, including the differential form and condition \(\nabla \times \underline{F} = 0\).

- Comprehensive explanation of why potential energy cannot be defined for non-conservative forces, including the role of energy conversion.

\subsection*{Question 2:}

\textbf{Response 1:}

**Solution 1: 5/6**

- **Feedback**:

  - The solution correctly identifies and describes the three cases of damping.
  
  - The explanation of underdamping, critical damping, and overdamping is clear and precise.
  
  - The explanation that the system's behavior depends on the relationship between \(b\), \(m\), and \(k\) is accurate.
  
  - However, it incorrectly describes the conditions for damping cases in terms of \(b^2\) rather than \(\Delta = b^2 - 4mk\).
  
  - The solution does not explicitly mention the discriminant \(\Delta\).

**Solution 2: 6/6**

- **Feedback**:

  - The solution correctly identifies and describes the three cases of damping.
  
  - It uses the discriminant \(\Delta = b^2 - 4mk\) correctly to describe the conditions for each case.
  
  - The explanation of the physical behavior in each damping case is clear and accurate.
  - It mentions practical applications, such as in automotive shock absorbers, adding depth to the explanation.
  
  - The solution is well-organized and concise, covering all necessary points.

**Solution 3: 5/6**

- **Feedback**:

  - The solution correctly identifies and describes the three cases of damping.
  
  - The explanation of underdamping, critical damping, and overdamping is accurate and detailed.
  
  - It correctly mentions the discriminant \(\Delta = b^2 - 4mk\) but assumes \(m = 1\) without stating this assumption clearly, which could lead to confusion.
  
  - The descriptions of the solutions in terms of exponential decay and sinusoidal function are accurate and provide a deeper understanding.
  
  - The assumption about \(m = 1\) should have been explicitly stated to avoid confusion.

\bigskip
\textbf{Response 2:}

Solution 1

**Marks: 5/6**

**Feedback:**

- The explanation correctly identifies and describes the three damping cases: underdamping, critical damping, and overdamping.

- The notation \(b^2 < 4mk\) is accurate, assuming \(m\) is implied as the mass. However, it could be clearer by explicitly mentioning the mass in the equation.

- The qualitative descriptions are generally correct and well-articulated. The mention of the quickest return to equilibrium in the critically damped case and slower return in the overdamped case is accurate.

- The detailed discussion on the balance between the restorative force and the damping force adds depth to the solution.

- Missing explicit mention of the discriminant \(\Delta = b^2 - 4mk\) as a central concept.

Solution 2

**Marks: 6/6**

**Feedback:**

- This solution provides a clear and accurate description of the three cases of damping.
- The use of the discriminant \(\Delta = b^2 - 4mk\) to categorize the damping cases is precise and mathematically sound.

- Each case is described accurately, with appropriate physical intuition regarding the behavior of the system.

- The solution mentions practical examples (e.g., automotive shock absorbers) which enhance understanding.

- All key concepts are included, making this a comprehensive and well-explained solution.

Solution 3

**Marks: 5/6**

**Feedback:**

- This solution correctly identifies and describes the three cases of damping: underdamped, critically damped, and overdamped.

- The explanation effectively uses the discriminant \(\Delta = b^2 - 4mk\) and simplifies to \(\Delta = b^2 - 4k\) for clarity, assuming \(m=1\).

- The qualitative descriptions are accurate, particularly the details about the nature of the solutions (exponential decay and sinusoidal functions).

- The explanation is clear and concise, covering the essential points.

- The assumption of \(m=1\) for simplicity should have been explicitly justified or avoided to maintain generality.

\bigskip
\textbf{Response 3:}

Grading Criteria:

- **Accuracy of Concepts (3 marks)**: Correct definitions and qualitative descriptions of underdamping, critical damping, and overdamping.

- **Clarity of Explanation (2 marks)**: Clear and understandable explanations.

- **Completeness (1 mark)**: Inclusion of all necessary details and addressing the problem completely.

Solution 1:

**Grade: 5/6**

- **Accuracy of Concepts (3/3)**: The student correctly defines underdamping, critical damping, and overdamping with accurate qualitative descriptions.

- **Clarity of Explanation (1/2)**: The explanations are clear but could be more concise and focused. The use of terms like "threshold condition" and "fastest return" is good, but the description of underdamping could be simplified.

- **Completeness (1/1)**: The solution includes all necessary details.

**Feedback:** A well-done explanation with all cases covered accurately. However, the description could be more concise for better clarity.

Solution 2:

**Grade: 6/6**

- **Accuracy of Concepts (3/3)**: The definitions and descriptions of the three damping cases are accurate and precise.

- **Clarity of Explanation (2/2)**: The explanations are clear and logical, providing good context for each damping case.

- **Completeness (1/1)**: The solution is comprehensive, covering all aspects of the problem.

**Feedback:** Excellent explanation with precise and clear descriptions. All points are well-covered and easy to understand.

Solution 3:

**Grade: 5/6**

- **Accuracy of Concepts (3/3)**: The definitions and qualitative descriptions are accurate and cover all necessary details.

- **Clarity of Explanation (1/2)**: While the explanations are generally clear, the notation and simplification (considering \(m=1\)) could confuse readers. It could be clearer if the standard notation was maintained throughout.

- **Completeness (1/1)**: The solution addresses all required points.

**Feedback:** Good explanation with accurate descriptions. However, simplifying the mass \(m\) to 1 without clear justification may cause confusion. Stick to standard notation for better clarity.

Summary:

- **Solution 1:** Good but could be more concise.

- **Solution 2:** Excellent, clear, and comprehensive.

- **Solution 3:** Accurate but slightly unclear due to notational simplifications.

\bigskip
\textbf{Response 4:}

**Solution 1:**

**Grade: 4/6**

**Feedback:** 

- The description of the three damping cases is generally accurate and clear.

- However, there is an error in the condition for each damping case. The discriminant should involve the mass \( m \), making the correct conditions: \( b^2 < 4mk \), \( b^2 = 4mk \), and \( b^2 > 4mk \).

- The qualitative descriptions of the behaviors are well explained, but the exact relationship involving \( m \) should be explicitly mentioned for clarity and correctness.

---

**Solution 2:**

**Grade: 6/6**

**Feedback:**

- The conditions for the three damping cases are correctly given using the discriminant \(\Delta = b^2 - 4mk\).

- The qualitative descriptions of underdamping, critical damping, and overdamping are accurate and well-articulated.

- The explanation is detailed, providing a clear understanding of how each damping case affects the system's behavior.

- This solution is comprehensive and precise.

---

**Solution 3:**

**Grade: 5/6**

**Feedback:**

- The conditions for the damping cases are mostly correct, but the simplification assuming \( m = 1 \) should be explicitly stated.

- The qualitative descriptions are accurate and clear, capturing the essence of each damping scenario.

- However, the simplification could lead to confusion, and it's better to consistently include the mass \( m \) for generality and completeness.

- Overall, the solution is clear and well-explained but could be improved by avoiding unnecessary simplifications.

\bigskip
\textbf{Response 5:}

**Solution 1: 5/6**

**Feedback:**

- This solution accurately describes the three cases of damping and their qualitative behaviors.

- Good explanation of the system's response in each case.

- Minor error: The term \(4mk\) should use mass explicitly in defining the discriminant \(\Delta = b^2 - 4mk\).

- Slightly more detail about the mathematical origin of each damping case could be helpful.

**Solution 2: 6/6**

**Feedback:**

- This solution correctly identifies and describes the three cases of damping.

- It clearly relates the damping behavior to the discriminant \(\Delta = b^2 - 4mk\).

- Provides context with practical examples, such as automotive shock absorbers.

- Well-organized and detailed, with no significant errors.

**Solution 3: 4/6**

**Feedback:**

- This solution correctly describes the three damping cases but assumes \(m=1\) without explicitly stating it, leading to some confusion.

- The explanation of each case is accurate but less detailed in comparing behaviors.

- Missing a clear statement of the discriminant \(\Delta = b^2 - 4mk\) without assuming \(m=1\).

- Less practical context compared to the other solutions, which would aid understanding.

\subsection*{Question 3:}

\textbf{Response 1:}

Grading and Feedback

**Solution 1:**

**Part a: 2/2**

- Correctly used the definition of center of mass.

- Accurately compared the expressions and derived the correct value of \( k \).

- Clear and logical explanation.

**Part b: 4/4**

- Correctly showed the center of mass lies on the line connecting the two masses.

- Correctly calculated the distances from the center of mass to each mass.

- Clear and well-organized explanation.

**Feedback:** Excellent solution. Both parts are solved correctly and clearly. The derivations are accurate, and the explanations are logical and easy to follow.

**Total: 6/6**

---

**Solution 2:**

**Part a: 2/2**

- Correctly used the definition of center of mass.

- Accurately derived the value of \( k \).

- Clear and logical explanation.

**Part b: 4/4**

- Correctly showed the center of mass lies on the line connecting the two masses.

- Correctly calculated the distances from the center of mass to each mass.

- Clear and well-organized explanation.

**Feedback:** Excellent solution. Both parts are solved correctly with clear derivations and explanations.

**Total: 6/6**

---

**Solution 3:**

**Part a: 2/2**

- Correctly used the definition of center of mass.

- Accurately derived the value of \( k \).

- Clear and logical explanation.

**Part b: 4/4**

- Correctly showed the center of mass lies on the line connecting the two masses.

- Correctly calculated the distances from the center of mass to each mass.

- Clear and well-organized explanation.

**Feedback:** Excellent solution. Both parts are solved correctly with clear derivations and explanations.

**Total: 6/6**

\bigskip
\textbf{Response 2:}

Solution 1

**Part a:**

- Correctly defines the center of mass and derives the value of \( k \).

- Clear explanation and appropriate use of equations.

**Part b:**

- Correctly shows the center of mass lies on the line connecting the masses.

- Correctly calculates the distances from each mass to the center of mass.

**Marks: 6/6**

- Part a: 2/2

- Part b: 4/4

**Feedback:**

- Excellent explanation and clear step-by-step derivation. All calculations and logical steps are accurate and well-presented.

Solution 2

**Part a:**

- Correctly defines the center of mass and derives the value of \( k \).

- Clear explanation but lacks the explicit comparison of coefficients step.

**Part b:**

- Correctly shows the center of mass lies on the line connecting the masses.

- Correctly calculates the distances from each mass to the center of mass.

**Marks: 5/6**

- Part a: 2/2

- Part b: 3/4

**Feedback:**

- Good explanation with correct derivation. However, the comparison of coefficients could be more detailed. Part b is accurate but could benefit from more explicit clarification of the steps involved.

Solution 3

**Part a:**

- Correctly defines the center of mass and derives the value of \( k \).

- Explanation is clear, but less concise compared to Solution 1.

**Part b:**

- Correctly shows the center of mass lies on the line connecting the masses.

- Correctly calculates the distances from each mass to the center of mass.

- Uses some redundant steps.

**Marks: 5/6**

- Part a: 2/2

- Part b: 3/4

**Feedback:**

- Good overall explanation and correct calculations. Some steps in part b are a bit redundant and could be streamlined for clarity.

\bigskip
\textbf{Response 3:}

Solution 1

**Grade: 6/6**

- **Part a (2/2):** Correct derivation of the center of mass formula and accurate calculation of \( k \). Clear and correct mathematical steps.

- **Part b (4/4):** Correct explanation and calculations showing the distances from the center of mass to each mass. Good use of the given distance \( d \) and the proportional relationships.

**Feedback:**

Excellent work. The solution is clear, precise, and correctly addresses both parts of the problem. Each step is logically and clearly explained, leading to the correct final answers.

Solution 2

**Grade: 6/6**

- **Part a (2/2):** Correct application of the center of mass formula and accurate calculation of \( k \). The mathematical reasoning is solid and well-presented.

- **Part b (4/4):** Correct demonstration of the distances from the center of mass to each mass. Good use of the given distance \( d \) and proper explanation of the proportional distances.

**Feedback:**

Well done. The solution is comprehensive and accurate. The explanations are detailed and the calculations are correctly executed. The logical flow is easy to follow, leading to the correct results.

Solution 3

**Grade: 6/6**

- **Part a (2/2):** Correct derivation of the center of mass formula and accurate calculation of \( k \). The steps are clear and correct.

- **Part b (4/4):** Correct demonstration of the distances from the center of mass to each mass. The explanation effectively uses the given distance \( d \) and correctly shows the proportional relationships.

**Feedback:**

Great job. The solution is thorough and correct. The explanations are clear and the calculations are accurately performed. The logical steps are well-structured, leading to the correct answers.

\bigskip
\textbf{Response 4:}

Solution 1:

**Part a:**

- Correctly starts with the definition of the center of mass.

- Derives the formula for the position vector of the center of mass.

- Correctly identifies and solves for \( k \).

**Marks for part a: 2/2**

**Part b:**

- Correctly shows the center of mass lies on the line connecting the two masses.

- Provides the correct distances from each mass to the center of mass with clear explanations.

**Marks for part b: 4/4**

**Total marks: 6/6**

**Feedback:**

Excellent solution. Clear and accurate derivation in part a, and thorough explanation and correct results in part b. Well done.

Solution 2:

**Part a:**

- Starts with the correct definition of the center of mass.

- Derives the formula for the position vector of the center of mass.

- Correctly identifies and solves for \( k \).

**Marks for part a: 2/2**

**Part b:**

- Correctly shows the center of mass lies on the line connecting the two masses.

- Provides the correct distances from each mass to the center of mass with clear explanations.

**Marks for part b: 4/4**

**Total marks: 6/6**

**Feedback:**

Excellent solution. Clear and accurate derivation in part a, and thorough explanation and correct results in part b. Well done.

Solution 3:

**Part a:**

- Starts with the correct definition of the center of mass.

- Derives the formula for the position vector of the center of mass.

- Correctly identifies and solves for \( k \).

**Marks for part a: 2/2**

**Part b:**

- Correctly shows the center of mass lies on the line connecting the two masses.

- Provides the correct distances from each mass to the center of mass with clear explanations.

**Marks for part b: 4/4**

**Total marks: 6/6**

**Feedback:**

Excellent solution. Clear and accurate derivation in part a, and thorough explanation and correct results in part b. Well done.

\bigskip
\textbf{Response 5:}

Here are the evaluations for each student's solution, including grades and feedback:

**Solution 1:**

*Marks: 5/6 (2 for part a, 3 for part b)*

*Feedback:*

- **Part a (2/2):** The solution correctly derives the expression for the center of mass and identifies the value of \( k \). The explanation is clear and precise, demonstrating an understanding of the concept.

- **Part b (3/4):** The distances from the center of mass to each mass are correctly derived, and the explanation confirms the center of mass lies on the line connecting the two masses. However, the final expression could have been more concisely linked to the provided form. There is some redundancy in the explanation.

**Solution 2:**

*Marks: 6/6 (2 for part a, 4 for part b)*

*Feedback:*

- **Part a (2/2):** The solution effectively derives the position vector for the center of mass and identifies the value of \( k \). The derivation is straightforward and accurate.

- **Part b (4/4):** The solution clearly and correctly shows that the center of mass lies on the line connecting the two masses and calculates the distances accurately. The explanation is thorough and directly addresses the problem requirements.

**Solution 3:**

*Marks: 4/6 (1 for part a, 3 for part b)*

*Feedback:*

- **Part a (1/2):** The solution correctly identifies the value of \( k \) and the position vector form, but the explanation lacks some clarity in transitioning between steps. The comparison of coefficients is implied but not explicitly shown, which might cause confusion.

- **Part b (3/4):** The distances from the center of mass to each mass are accurately calculated, and the explanation confirms the line connection. The logical flow is somewhat disjointed, but the calculations and final results are correct.

\subsection*{Question 4:}

\textbf{Response 1:}

Solution 1:

Part a)

- Correctly derived the expression for \( \dot{\theta} \) using conservation of angular momentum.

- The derivation steps are clear and accurate.

- Score: 6/6

Part b)

- Correctly identified the angular component of the acceleration and provided the correct expression for it.

- Explanation is a bit verbose but correct.

- Score: 5/5

Part c)

- Correctly derived the kinetic energy and attempted to differentiate it.

- The differentiation to find \( \frac{dK}{dt} \) is correctly structured but includes an unnecessary sign discussion.

- Score: 6/7

**Total Score: 17/18**

**Feedback:** Excellent understanding and application of angular momentum conservation and kinetic energy differentiation. The detailed steps and logical flow are well-presented. Watch out for over-explanation that can lead to minor confusion.

Solution 2:

Part a)

- Correctly derived the expression for \( \dot{\theta} \) using conservation of angular momentum.

- The steps are clear and accurate.

- Score: 6/6

Part b)

- Correctly derived the angular component of the acceleration, including the correct calculation of \( \ddot{\theta} \).

- Explanation is concise and correct.

- Score: 5/5

Part c)

- Correctly derived the kinetic energy and attempted to differentiate it.

- The differentiation is correctly structured and the sign discrepancy is noted but not elaborated upon unnecessarily.

- Score: 6/7

**Total Score: 17/18**

**Feedback:** Clear and accurate derivations for each part. The concise approach to part (c) while acknowledging the sign discrepancy was appropriate. Excellent work overall.

Solution 3:

Part a)

- Correctly derived the expression for \( \dot{\theta} \) using conservation of angular momentum.

- The steps are clear and accurate.

- Score: 6/6

Part b)

- Correctly identified the angular component of the acceleration but missed calculating \( \ddot{\theta} \).

- The simplification to zero seems incorrect or incomplete.

- Score: 3/5

Part c)

- Correctly derived the kinetic energy but struggled with the differentiation and resulting sign.

- The explanation includes unnecessary confusion around the sign.

- Score: 5/7

**Total Score: 14/18**

**Feedback:** Good understanding of angular momentum conservation and kinetic energy concepts. However, part (b) needs a more accurate calculation and simplification. Be cautious about handling signs and assumptions in differentiation.

Summary:

- **Solution 1:** 17/18

- **Solution 2:** 17/18

- **Solution 3:** 14/18

\bigskip
\textbf{Response 2:}

Solution 1

Part (a)

Marks: 6/6

Feedback: Excellent job. The solution correctly derives the expression for the angular frequency \(\dot{\theta}\) using the conservation of angular momentum.

Part (b)

Marks: 2/5

Feedback: The solution correctly identifies the angular component of the acceleration and the expressions for \(\dot{r}\) and \(\dot{\theta}\). However, it does not calculate \(\ddot{\theta}\) explicitly, which is necessary for a complete answer.

Part (c)

Marks: 5/7

Feedback: The kinetic energy expression is correctly derived. The differentiation of the kinetic energy with respect to time is almost correct but the sign discrepancy needs addressing. Nonetheless, the core calculation is well done.

**Total Marks:** 13/18

Solution 2

Part (a)

Marks: 6/6

Feedback: The expression for \(\dot{\theta}\) is derived correctly using the conservation of angular momentum and integrating the radial velocity.

Part (b)

Marks: 4/5

Feedback: The solution correctly identifies the angular component of the acceleration and performs the necessary differentiation to find \(\ddot{\theta}\). However, there is a minor mistake in the final simplification of \(a_{\theta}\).

Part (c)

Marks: 6/7

Feedback: The kinetic energy expression is correctly derived. The differentiation is done correctly and matches the given expression, with only a minor sign issue.

**Total Marks:** 16/18

Solution 3

Part (a)

Marks: 6/6

Feedback: Correct derivation of the expression for \(\dot{\theta}\) using conservation of angular momentum.

Part (b)

Marks: 3/5

Feedback: The solution identifies the need to calculate \(\ddot{\theta}\) but does not complete the calculation fully. The simplification to zero is incorrect and needs correction.

Part (c)

Marks: 5/7

Feedback: The kinetic energy expression is derived well, and the differentiation is correct. The sign issue persists but the overall approach is solid.

**Total Marks:** 14/18

\bigskip
\textbf{Response 3:}

Solution 1

**Part (a): 6/6**

The derivation of the angular frequency \(\dot{\theta}\) is correct and follows logically from the conservation of angular momentum. The student accurately integrates \(\dot{r}\) and substitutes \(r(t)\) into the expression for \(\dot{\theta}\).

**Part (b): 2/5**

The student correctly identifies the need to find the angular component of acceleration but does not explicitly calculate \(\ddot{\theta}\). The final expression provided for the angular component is incorrect due to missing or incorrect calculations.

**Part (c): 6/7**

The kinetic energy is correctly derived, and the process to differentiate it with respect to time is mostly accurate. However, the final expression for \(\frac{dK}{dt}\) is presented with an incorrect sign explanation.

**Total: 14/18**

*General Feedback*: Excellent work on part (a). For part (b), ensure all necessary derivatives are computed and correctly incorporated. For part (c), careful attention to signs and differentiation steps is essential for accuracy.

---

Solution 2

**Part (a): 6/6**

The solution provides a clear and correct derivation of the angular frequency \(\dot{\theta}\), including all necessary steps and justifications.

**Part (b): 4/5**

The student correctly identifies and calculates \(\ddot{\theta}\) and integrates it into the expression for the angular component of acceleration. The final steps are correct but slightly less detailed.

**Part (c): 7/7**

The kinetic energy and its time derivative are correctly derived. The student clearly shows the steps and matches the provided expression for \(\frac{dK}{dt}\).

**Total: 17/18**

*General Feedback*: Very well done, particularly on parts (a) and (c). The approach to part (b) is solid, with minor improvements possible in the final detailing.

---

Solution 3

**Part (a): 6/6**

The derivation of \(\dot{\theta}\) is thorough and accurate, with all steps clearly shown.

**Part (b): 3/5**

The student correctly calculates \(\ddot{\theta}\) and attempts to find the angular component of acceleration. However, there's a misunderstanding that leads to an incorrect simplification.

**Part (c): 6/7**

The student provides a detailed derivation of kinetic energy and its time derivative, correctly matching the provided expression but with an explanation that leads to a sign discrepancy.

**Total: 15/18**

*General Feedback*: Strong work overall. The derivation processes are detailed and mostly accurate. Ensure all steps in differentiation are clearly followed and signs are correctly handled in final expressions.

\bigskip
\textbf{Response 4:}

Here are the evaluations for each student's solution:

**Solution 1:**

**Part (a):** The student correctly derives the expression for the angular frequency \( \dot{\theta} \) using the conservation of angular momentum.

- **Marks:** 6/6

**Part (b):** The student attempts to find the angular component of the acceleration but only provides an incomplete answer without explicitly finding \( \ddot{\theta} \).

- **Marks:** 3/5

**Part (c):** The student correctly sets up the kinetic energy expression and computes the time derivative, though they note a discrepancy in the sign without fully resolving it. However, the steps show clear understanding.

- **Marks:** 6/7

**Total Marks:** 15/18

**Feedback:** Excellent work in deriving the angular frequency. The solution for the angular component of acceleration needs to explicitly compute \( \ddot{\theta} \). Good attempt in part (c), though a bit more attention to detail with the sign would improve clarity.

**Solution 2:**

**Part (a):** The student correctly derives the expression for \( \dot{\theta} \) using the conservation of angular momentum.

- **Marks:** 6/6

**Part (b):** The student correctly identifies the angular component of the acceleration and performs the necessary differentiation to find \( \ddot{\theta} \), leading to a correct and complete answer.

- **Marks:** 5/5

**Part (c):** The student sets up the kinetic energy correctly and derives the expression for the time derivative, arriving at the correct result with a clear and detailed calculation.

- **Marks:** 7/7

**Total Marks:** 18/18

**Feedback:** Excellent and thorough work on all parts. The solution is well-explained, detailed, and accurate.

**Solution 3:**

**Part (a):** The student correctly derives the expression for \( \dot{\theta} \) using the conservation of angular momentum.

- **Marks:** 6/6

**Part (b):** The student begins to identify the components of the acceleration but ultimately fails to correctly compute and simplify the expressions for the angular component.

- **Marks:** 2/5

**Part (c):** The student sets up the kinetic energy expression correctly and performs the differentiation. However, the explanation is verbose, and they do not resolve the sign discrepancy.

- **Marks:** 5/7

**Total Marks:** 13/18

**Feedback:** Good job on deriving the angular frequency. More clarity and correctness are needed in the acceleration component calculation. For part (c), while the steps are mostly correct, resolving the sign discrepancy would enhance the solution.

\bigskip
\textbf{Response 5:}

Solution 1 Evaluation

**Part (a):**

- Correctly identifies and uses the conservation of angular momentum.

- Derives the expression for \(\dot{\theta}\) accurately.

- Clearly shows all steps and substitutions.

**Marks: 6/6**

**Part (b):**

- Identifies the correct component of the acceleration.

- Correctly substitutes given values but does not explicitly compute \(\ddot{\theta}\).

- The solution makes a logical assumption about \(\ddot{\theta}\), but this needs explicit handling.

**Marks: 3/5**

**Part (c):**

- Correctly uses the kinetic energy formula in polar coordinates.

- Successfully substitutes \(\dot{r}\) and \(\dot{\theta}\).

- Derives \(\frac{dK}{dt}\) with the correct form but discusses the sign issue without clearly resolving it.

**Marks: 5/7**

**General Feedback:**

- Comprehensive and well-structured solution.

- Minor oversight in explicitly calculating \(\ddot{\theta}\) and resolving the sign issue.

**Total: 14/18**

Solution 2 Evaluation

**Part (a):**

- Correctly derives the expression for \(\dot{\theta}\).

- Clearly shows the conservation of angular momentum and substitutions.

**Marks: 6/6**

**Part (b):**

- Correctly identifies the angular component of the acceleration.

- Derives \(\ddot{\theta}\) and substitutes correctly.

- Simplification to a zero angular acceleration component indicates an error in the logical assumption.

**Marks: 3/5**

**Part (c):**

- Correctly uses the kinetic energy formula.

- Derives \(\frac{dK}{dt}\) but with a sign discrepancy.

- Steps are clearly shown but lacks clarity in resolving the sign issue.

**Marks: 5/7**

**General Feedback:**

- Good structure and clear derivations.

- Needs a careful check on logical assumptions, particularly in Part (b).

**Total: 14/18**

Solution 3 Evaluation

**Part (a):**

- Correctly uses conservation of angular momentum.

- Derives the expression for \(\dot{\theta}\) accurately.

- Clear and logical derivation steps.

**Marks: 6/6**

**Part (b):**

- Correctly identifies the angular component of the acceleration.

- Derives \(\ddot{\theta}\) and substitutes correctly.

- Misinterpretation leading to a zero value for angular acceleration.

**Marks: 3/5**

**Part (c):**

- Correctly uses the kinetic energy formula.

- Derives \(\frac{dK}{dt}\) with detailed steps.

- Corrects the sign discrepancy but explanation lacks clarity.

**Marks: 5/7**

**General Feedback:**

- Clear derivations with minor issues in logical assumptions.

- Needs better clarity in final explanations, especially resolving discrepancies.

**Total: 14/18**

---

**Final Summary:**

All three solutions are thorough and well-structured, with similar minor issues in handling specific details. Each solution correctly derives \(\dot{\theta}\) and kinetic energy but falls short in explicitly resolving \(\ddot{\theta}\) and sign discrepancies in \(\frac{dK}{dt}\). Each solution deserves a total of 14/18 marks.

\subsection*{Question 5:}

\textbf{Response 1:}

Let's grade each solution based on the given criteria, with concise feedback for each part.

Solution 1

**Part a:**

- The derivation of equations of motion is accurate and clear.

- Correctly applies Newton's second law.

- Marks: 2/2

**Part b:**

- Correctly defines new coordinates.

- Shows the process of uncoupling and arrives at the correct uncoupled equations.

- Clearly explains the steps and transformations.

- Marks: 4/4

**Part c:**

- Provides the correct general solution.

- Applies initial conditions correctly.

- Derives the final solution correctly.

- Marks: 3/3

**Total: 9/9**

**Feedback:** Excellent solution. Each step is clearly explained, and the final answers are correct and well-presented.

Solution 2

**Part a:**

- The equations of motion are derived correctly.

- The process is clear and concise.

- Marks: 2/2

**Part b:**

- Defines the new coordinates but makes an error in the final form of the uncoupled equations.

- The uncoupling process is partially correct but not fully completed.

- Marks: 2/4

**Part c:**

- Provides a correct general solution.

- Applies the initial conditions correctly but the explanation is less detailed.

- Marks: 3/3

**Total: 7/9**

**Feedback:** Good solution overall, but the uncoupling process in part b is not fully correct. Part c is accurate but could be more detailed in its explanation.

Solution 3

**Part a:**

- Correctly derives the equations of motion.

- Clear and detailed explanation.

- Marks: 2/2

**Part b:**

- Correctly defines new coordinates and uncouples the equations.

- The process is detailed and accurate.

- Marks: 4/4

**Part c:**

- Provides the correct general solution and applies the initial conditions correctly.

- The final solution is accurate.

- Marks: 3/3

**Total: 9/9**

**Feedback:** Excellent solution. The uncoupling process and final solution are well-explained and accurate.

Summary

- **Solution 1:** 9/9

- **Solution 2:** 7/9

- **Solution 3:** 9/9

Solutions 1 and 3 are excellent, with clear and accurate explanations throughout. Solution 2 is good but has a minor error in the uncoupling process.

\bigskip
\textbf{Response 2:}

Let's evaluate each solution based on the given criteria.

**Problem Breakdown:**

- **Part a)** Derive the equations of motion for objects X and Y. (2 marks)

- **Part b)** Show that new coordinates \( y_1 = x_1 + x_2 \) and \( y_2 = x_1 - x_2 \) uncouple the equations. (4 marks)

- **Part c)** Show the solution for \( y_1 \) given the initial conditions. (3 marks)

**Solution 1 Evaluation:**

**Part a):**

- Correctly applies Newton's second law and derives the equations of motion for both X and Y. (2/2 marks)

**Part b):**

- Defines new coordinates and shows the process of uncoupling the equations.

- Makes a mistake in the uncoupling process; the equations after substitution do not fully decouple correctly.

- Correct identification of the system’s frequencies but errors in derivation. (2/4 marks)

**Part c):**

- Correctly identifies the general solution and applies initial conditions.

- Correctly solves for constants and provides the correct final expression for \( y_1 \). (3/3 marks)

**Total for Solution 1: 7/9 marks**

**Solution 2 Evaluation:**

**Part a):**

- Correct equations derived but slightly different notation and extra unnecessary explanation.

- Still correct application and result. (2/2 marks)

**Part b):**

- Correctly defines new coordinates and derives the equations of motion.

- Makes a mistake in the steps, resulting in incorrect final forms of uncoupled equations. (2/4 marks)

**Part c):**

- Correctly identifies the general solution and applies initial conditions.

- Correctly solves for constants and provides the correct final expression for \( y_1 \). (3/3 marks)

**Total for Solution 2: 7/9 marks**

**Solution 3 Evaluation:**

**Part a):**

- Correctly applies Newton's second law and derives the equations of motion for both X and Y.

- More detailed explanation and correct final result. (2/2 marks)

**Part b):**

- Correctly defines new coordinates and shows the process of uncoupling the equations.

- Correctly identifies that the equations reduce to standard SHM equations. (4/4 marks)

**Part c):**

- Correctly identifies the general solution and applies initial conditions.

- Provides a detailed and correct derivation for the constants and the final expression for \( y_1 \). (3/3 marks)

**Total for Solution 3: 9/9 marks**

**Feedback:**

**Solution 1:**

- Good understanding of Newton's second law and initial conditions.

- Review the steps in uncoupling differential equations to avoid errors.

**Solution 2:**

- Clear and thorough approach to deriving equations.

- Ensure accuracy in the steps of uncoupling the equations.

**Solution 3:**

- Excellent and detailed work with correct derivations in all parts.

- Perfect understanding and application of the methods required.

If you need further assistance or specific details about the grading or feedback, feel free to ask!

\bigskip
\textbf{Response 3:}

Feedback on Solutions

Solution 1:

**Part a) Equations of Motion (2/2):**

- Correctly derived the equations of motion for both objects X and Y.

- Accurate application of Newton's second law and proper manipulation of forces and spring constants.

**Part b) Uncoupling the Coordinates (2/4):**

- Attempted to uncouple the equations correctly.

- Mistake in manipulating the equations to uncouple \( y_1 \) and \( y_2 \), resulting in incorrect final uncoupled equations.

**Part c) Solution for \( y_1 \) (3/3):**

- Correct solution with clear and accurate application of initial conditions.

- Clear derivation matching the provided solution for \( y_1 \).

**Total: 7/9**

**Feedback:** A strong solution overall, with correct equations of motion and final solution for \( y_1 \). Part b had an error in uncoupling the coordinates, leading to a loss of marks.

Solution 2:

**Part a) Equations of Motion (2/2):**

- Correctly derived the equations of motion for both objects X and Y.

- Clear and correct application of Newton's second law and spring forces.

**Part b) Uncoupling the Coordinates (3/4):**

- Correctly defined new coordinates and derived the equations of motion for \( y_1 \) and \( y_2 \).

- Clear differentiation and substitution into original equations, though slight mistake in handling coefficients.

**Part c) Solution for \( y_1 \) (3/3):**

- Correct solution with clear application of initial conditions.

- Proper and concise derivation matching the provided solution for \( y_1 \).

**Total: 8/9**

**Feedback:** Excellent solution with minor mistakes in part b. Clear and concise explanations, well-structured derivations, and accurate final results.

Solution 3:

**Part a) Equations of Motion (2/2):**

- Correctly derived the equations of motion for both objects X and Y.

- Clear application of Newton's second law and appropriate handling of spring forces.

**Part b) Uncoupling the Coordinates (4/4):**

- Correctly defined new coordinates and derived the uncoupled equations of motion for \( y_1 \) and \( y_2 \).

- Accurate and thorough differentiation and substitution process, yielding correct final equations.

**Part c) Solution for \( y_1 \) (3/3):**

- Correct solution with clear application of initial conditions.

- Detailed and accurate derivation matching the provided solution for \( y_1 \).

**Total: 9/9**

**Feedback:** Outstanding solution with clear and accurate derivations throughout all parts. Thorough explanations and correct handling of all aspects of the problem.

---

Grading Summary

- **Solution 1:** 7/9

- **Solution 2:** 8/9

- **Solution 3:** 9/9

\bigskip
\textbf{Response 4:}

Let's evaluate each solution based on the criteria provided and the correctness of the steps and final results.

Solution 1:

**Part a) Equations of Motion**

Correctly identifies forces and applies Newton's second law:

\[ m \ddot{x}_{1} + 2kx_{1} - kx_{2} = 0 \]
\[ m \ddot{x}_{2} + 2kx_{2} - kx_{1} = 0 \]

- Correctly derived equations. 

- Mark: **2/2**

**Part b) Uncoupling the Coordinates**

Defines new coordinates correctly and performs differentiation correctly:

\[ \ddot{y}_{1} = \ddot{x}_{1} + \ddot{x}_{2} \]
\[ \ddot{y}_{2} = \ddot{x}_{1} - \ddot{x}_{2} \]

Incorrectly simplifies the final uncoupled equations:

\[ m \ddot{y}_{1} + ky_{1} = 0 \]
\[ m \ddot{y}_{2} + 3ky_{2} = 0 \]

- Miscalculation present. 

- Mark: **3/4**

**Part c) Solution for \( y_{1} \)**

Correct initial conditions and general solution derivation:

\[ y_{1}(t) = v \sqrt{\frac{m}{k}} \sin(t \sqrt{\frac{k}{m}}) \]

- Correct final solution.

- Mark: **3/3**

**Total: 8/9**

**Feedback:**

Good overall solution with correct physical interpretation and application of principles. Small error in the uncoupling step but otherwise solid understanding demonstrated.

Solution 2:

**Part a) Equations of Motion**

Correctly identifies forces and applies Newton's second law:

\[ m \ddot{x}_{1} + 2kx_{1} - kx_{2} = 0 \]
\[ m \ddot{x}_{2} + 2kx_{2} - kx_{1} = 0 \]

- Correctly derived equations.

- Mark: **2/2**

**Part b) Uncoupling the Coordinates**

Defines new coordinates correctly and performs differentiation correctly:

\[ \ddot{y}_{1} = \ddot{x}_{1} + \ddot{x}_{2} \]
\[ \ddot{y}_{2} = \ddot{x}_{1} - \ddot{x}_{2} \]

Correct final uncoupled equations:

\[ m \ddot{y}_{1} + ky_{1} = 0 \]
\[ m \ddot{y}_{2} + 3ky_{2} = 0 \]

- Correctly derived equations.

- Mark: **4/4**

**Part c) Solution for \( y_{1} \)**

Correct initial conditions and general solution derivation:

\[ y_{1}(t) = v \sqrt{\frac{m}{k}} \sin(t \sqrt{\frac{k}{m}}) \]

- Correct final solution.

- Mark: **3/3**

**Total: 9/9**

**Feedback:**

Excellent solution with correct calculations and interpretations throughout. Clear and concise.

Solution 3:

**Part a) Equations of Motion**

Correctly identifies forces and applies Newton's second law:

\[ m \ddot{x}_{1} + 2kx_{1} - kx_{2} = 0 \]
\[ m \ddot{x}_{2} + 2kx_{2} - kx_{1} = 0 \]

- Correctly derived equations.

- Mark: **2/2**

**Part b) Uncoupling the Coordinates**

Defines new coordinates correctly and performs differentiation correctly:

\[ \ddot{y}_{1} = \ddot{x}_{1} + \ddot{x}_{2} \]
\[ \ddot{y}_{2} = \ddot{x}_{1} - \ddot{x}_{2} \]

Correct final uncoupled equations:

\[ m \ddot{y}_{1} + ky_{1} = 0 \]
\[ m \ddot{y}_{2} + 3ky_{2} = 0 \]

- Correctly derived equations.

- Mark: **4/4**

**Part c) Solution for \( y_{1} \)**

Correct initial conditions and general solution derivation:

\[ y_{1}(t) = v \sqrt{\frac{m}{k}} \sin(t \sqrt{\frac{k}{m}}) \]

- Correct final solution.

- Mark: **3/3**

**Total: 9/9**

**Feedback:**

Excellent solution with correct calculations and interpretations throughout. Well-structured and clear explanation.

---

Summary of Marks:

- **Solution 1: 8/9**

- **Solution 2: 9/9**

- **Solution 3: 9/9**

Overall, Solutions 2 and 3 are perfect, with Solution 1 being very good but having a small error in the uncoupling step.

\bigskip
\textbf{Response 5:}

Let's evaluate each solution based on the problem criteria and provide concise feedback for each.

Solution 1:

Part a) (2 marks)

- Correctly derived the equations of motion for both objects.

- Clearly showed the application of Newton's second law and the forces involved.

Part b) (3 marks)

- Correctly defined the new coordinates and attempted to uncouple the equations.

- Correctly identified the form of simple harmonic motion but made errors in the manipulation of equations, particularly in deriving the final uncoupled equations.

Part c) (3 marks)

- Correctly applied initial conditions to find the solution for \( y_1 \).

- Solution matches the given answer.

**Total: 8/9**

**Feedback:**

Great job on deriving the equations of motion and applying initial conditions. The uncoupling part was mostly correct but had minor errors in the final step.

Solution 2:

Part a) (2 marks)

- Correctly derived the equations of motion.

- Proper application of Newton's second law and understanding of spring forces.

Part b) (4 marks)

- Correctly defined new coordinates and fully uncoupled the equations without errors.

- Detailed and clear derivation of the uncoupled equations.

Part c) (3 marks)

- Correctly applied initial conditions and derived the solution for \( y_1 \).

- Solution matches the given answer.

**Total: 9/9**

**Feedback:**

Excellent solution! All parts are correct and well-explained. Great work on the uncoupling and applying the initial conditions.

Solution 3:

Part a) (2 marks)

- Correctly derived the equations of motion.

- Clear and correct application of Newton's second law and Hooke's law.

Part b) (3 marks)

- Defined new coordinates and attempted to uncouple the equations.

- Some minor errors in the final step of the uncoupling process but mostly correct.

Part c) (3 marks)

- Correctly applied initial conditions and derived the solution for \( y_1 \).

- Solution matches the given answer.

**Total: 8/9**

**Feedback:**

Good solution overall. The derivation of the equations and application of initial conditions were correct. Some minor errors in the final steps of the uncoupling part.

---

To summarize:

- **Solution 1: 8/9**

- **Solution 2: 9/9**

- **Solution 3: 8/9**

Solution 2 is the most accurate and complete among the three provided.


\subsection*{Question 6:}

\textbf{Response 1:}

Grading and Feedback for Solution 1

Part a)

- Marks: 2/2

- Feedback: The derivation of the largest speed on a flat circular track is correct and clearly explained.

Part b)

- Marks: 1/3

- Feedback: The student incorrectly identified the gravitational force component contributing to the centripetal force as \( mg\sin(\phi) \) instead of the correct \( mg\cos(\phi) \). 

Part c)

- Marks: 5/11

- Feedback: The student made a good attempt but failed to derive the correct expressions for \( v_{\text{min}} \) and \( v_{\text{max}} \). The simplification to match the given equation was incorrect. More precise handling of the forces and friction is needed.

Total: 8/16

Grading and Feedback for Solution 2

Part a)

- Marks: 2/2

- Feedback: The derivation for the largest speed on a flat circular track is correct and well-explained.

Part b)

- Marks: 0/3

- Feedback: The student incorrectly stated that the centripetal force comes from \( mg\cos(\phi) \) instead of the correct \( mg\sin(\phi) \). This error leads to the wrong expression for the largest speed.

Part c)

- Marks: 6/11

- Feedback: The approach and initial setup are correct, but the final simplification and matching with the given equality were incorrect. The student needs to correctly identify and combine the forces to match the provided formula.

Total: 8/16

Grading and Feedback for Solution 3

Part a)

- Marks: 2/2

- Feedback: The solution correctly derives the largest speed for the flat circular track and explains the steps clearly.

Part b)

- Marks: 2/3

- Feedback: The initial assumption about the component of the gravitational force was incorrect, but the student quickly corrected the mistake. This partial correction earns partial credit.

Part c)

- Marks: 7/11

- Feedback: The student accurately identified the forces and derived the correct expressions for \( v_{\text{min}} \) and \( v_{\text{max}} \). However, the simplification to match the given equality was incorrect.

Total: 11/16

---

Summary

- Solution 1: The student has a good understanding but made errors in part b and did not match the given equation correctly in part c.

- Solution 2: The student had a correct start but faltered in part b and c. The forces' contributions were not correctly simplified.

- Solution 3: This student showed the best understanding but still made simplification errors. Their work was closest to the correct solutions.

Solution 3 provided the most accurate answers and understanding, earning the highest score of the three.

\bigskip
\textbf{Response 2:}

Solution 1:

**Part a)**: Correct derivation using friction and centripetal force balance. 

- **Marks**: 2/2

**Part b)**: Misunderstands that the component providing centripetal force is \(mg \cos(\phi)\) for an inclined plane, leading to incorrect final formula.

- **Marks**: 1/3

**Part c)**: Correctly derives expressions for \(v_{\text{min}}\) and \(v_{\text{max}}\), but the final step for \(v_{max}^{2} - v_{min}^{2}\) is not correctly simplified to the given form.

- **Marks**: 6/11

**Total**: 9/16

**Feedback**: Good understanding of fundamental forces but lacks precise handling of inclined plane forces and simplification for the given formula.

Solution 2:

**Part a)**: Correct derivation with appropriate use of forces.

- **Marks**: 2/2

**Part b)**: Similar error to Solution 1, where incorrect force component is used, leading to wrong formula.

- **Marks**: 1/3

**Part c)**: Correct approach to derive \(v_{\text{min}}\) and \(v_{\text{max}}\), but struggles with final expression simplification to match provided formula.

- **Marks**: 6/11

**Total**: 9/16

**Feedback**: Solid understanding of basic principles, but needs to correctly apply forces on an inclined plane and ensure precise simplification steps.

Solution 3:

**Part a)**: Correctly derives the largest speed on a flat circular track.

- **Marks**: 2/2

**Part b)**: Correctly identifies force component contributing to centripetal force (\(mg \sin(\phi)\)), but solution is not explicitly shown.

- **Marks**: 2/3

**Part c)**: Accurately derives \(v_{\text{min}}\) and \(v_{\text{max}}\) and demonstrates understanding of frictional forces, but fails to simplify to given formula.

- **Marks**: 7/11

**Total**: 11/16

**Feedback**: Good grasp of forces and motion on flat and inclined planes, but final simplification to given formula needs precision.

General Feedback:

All students showed a good understanding of basic forces and circular motion. The main area needing improvement is the precise handling and simplification of forces on an inclined plane, especially when friction is involved. Properly identifying the correct components of gravitational force and ensuring clear, correct derivations for given formulas is crucial.

\bigskip
\textbf{Response 3:}

**Solution 1:**

**Marks: 13/16**

- **Part a) (2/2)**: Correct derivation of the maximum speed on a flat circular track. 

- **Part b) (3/3)**: Correct derivation of the maximum speed on an inclined frictionless track.

- **Part c) (8/11)**: Correct identification of the forces and their contributions. Properly derived \(v_{max}\) and \(v_{min}\) expressions. The error in simplifying \(v_{max}^2 - v_{min}^2\) to match the provided equation indicates a partial understanding but a significant error in the final steps.

**Feedback**: The solutions to parts a) and b) are correct and well-explained. The approach in part c) is mostly correct but falters in the final steps. Ensure to carefully handle the algebraic manipulations and verify the target expression.

**Solution 2:**

**Marks: 11/16**

- **Part a) (2/2)**: Correct derivation of the maximum speed on a flat circular track.

- **Part b) (2/3)**: Incorrect identification of the component of gravitational force contributing to centripetal force on a frictionless incline. The correct component is \(mg \sin(\phi)\), not \(mg \cos(\phi)\).

- **Part c) (7/11)**: Correct identification of forces and their contributions. Derived \(v_{max}\) and \(v_{min}\) expressions correctly, but the simplification of \(v_{max}^2 - v_{min}^2\) is incorrect, indicating a misunderstanding in algebraic handling.

**Feedback**: The solution to part a) is correct. Part b) contains a crucial error in identifying the correct force component. Part c) shows good understanding but fails in final algebraic simplification. Focus on identifying correct force components and verifying algebraic steps.

**Solution 3:**

**Marks: 13/16**

- **Part a) (2/2)**: Correct derivation of the maximum speed on a flat circular track.

- **Part b) (3/3)**: Correct derivation of the maximum speed on an inclined frictionless track.

- **Part c) (8/11)**: Correct identification of forces and their contributions. Properly derived \(v_{max}\) and \(v_{min}\) expressions. However, the simplification to match \(v_{max}^2 - v_{min}^2\) to the provided equation is not completed correctly.

**Feedback**: Solutions to parts a) and b) are accurate. The understanding of forces in part c) is solid, but the final algebraic simplification needs careful attention. Verify all steps to ensure the target expression is achieved.

**General Feedback**: All students demonstrated a strong understanding of the physical principles and correctly derived most expressions. Common areas for improvement include handling algebraic simplifications and ensuring the correct identification of force components, particularly in more complex scenarios. Focus on verifying final expressions and understanding the role of each force in the problem setup.

\bigskip
\textbf{Response 4:}

Here are the grades and feedback for each solution provided:

Solution 1

**Grade: 11/16**

- **Part a (2/2):** Correctly derived the expression for the largest speed on a flat circular track with friction.

- **Part b (2/3):** The approach was mostly correct but had an error in the interpretation of the forces involved in a frictionless inclined track.

- **Part c (7/11):** The derivation for the maximum and minimum speeds was correct, but the final step to show the given equality had errors. The explanation was detailed but did not fully simplify to the required result.

**Feedback:**

The solution shows a good understanding of the principles of centripetal force and friction. Part c demonstrates a strong effort to derive the necessary expressions but falls short in correctly showing the given equality. More attention to the simplification process and verifying each step for the final result would improve accuracy.

Solution 2

**Grade: 9/16**

- **Part a (2/2):** Correctly derived the expression for the largest speed on a flat circular track with friction.

- **Part b (1/3):** The component of gravitational force used was incorrect, leading to an incorrect expression for the largest speed on an inclined frictionless track.

- **Part c (6/11):** The initial derivations for maximum and minimum speeds were sound, but the simplification to show the given equality was incorrect and lacked clarity.

**Feedback:**

The solution covers the basic principles well in parts a and c but makes a critical mistake in part b. The explanation for part c was on the right track but failed to correctly simplify to the given equality. More careful attention to the components of forces on an inclined plane and a clearer simplification process would enhance the solution.

Solution 3

**Grade: 12/16**

- **Part a (2/2):** Correctly derived the expression for the largest speed on a flat circular track with friction.

- **Part b (2/3):** The initial steps were correct, but there was confusion regarding the force components on a frictionless inclined track.

- **Part c (8/11):** The derivations for maximum and minimum speeds were mostly correct, but the final step to show the given equality had an error. The explanation provided was clear and logically structured, although it did not reach the correct final result.

**Feedback:**

The solution effectively addresses parts a and b, though with some confusion in interpreting forces in part b. Part c demonstrates a solid understanding and a well-structured approach, but an error in the final simplification step prevents it from fully meeting the problem requirements. Ensuring accuracy in the final algebraic manipulation would complete the solution successfully.

\bigskip
\textbf{Response 5:}

Solution 1

**Part a)**: 
Correctly identified the forces and derived the correct expression for the maximum speed.

Marks: 2/2

**Part b)**: 
Correctly identified the gravitational component providing centripetal force and derived the correct expression.

Marks: 3/3

**Part c)**: 
Correctly identified the forces and derived expressions for maximum and minimum speeds. Attempted to show the given relationship but did not simplify correctly.

Marks: 7/11

**Total**: 12/16

**Feedback**: Good understanding of the forces involved and correct derivations for parts a) and b). Part c) was well-attempted, but there was a mistake in the final simplification to match the given equality. More attention to detail in the algebraic manipulation is needed.

Solution 2

**Part a)**: 
Correctly identified the forces and derived the correct expression for the maximum speed.

Marks: 2/2

**Part b)**: 
Incorrectly identified the component of gravitational force providing the centripetal force (should be \(\sin(\phi)\) not \(\cos(\phi)\)).

Marks: 0/3

**Part c)**: 
Identified the forces correctly and derived expressions for maximum and minimum speeds. The attempt to show the given relationship was incorrect due to earlier mistakes.

Marks: 6/11

**Total**: 8/16

**Feedback**: Correct approach in part a). Part b) had an incorrect component for centripetal force, which affected the results in part c). Understanding the correct force components is crucial for accurate derivations.

Solution 3

**Part a)**: 
Correctly identified the forces and derived the correct expression for the maximum speed.

Marks: 2/2

**Part b)**: 
Incorrectly identified the component of gravitational force providing the centripetal force (should be \(\sin(\phi)\) not \(\cos(\phi)\)).

Marks: 0/3

**Part c)**: 
Identified the forces correctly and derived expressions for maximum and minimum speeds. The attempt to show the given relationship was incorrect due to earlier mistakes.

Marks: 6/11

**Total**: 8/16

**Feedback**: Correct approach in part a). Part b) had an incorrect component for centripetal force, which affected the results in part c). Similar to solution 2, ensure correct force components are identified for accurate calculations.


\subsection*{Question 7:}

\textbf{Response 1:}

Solution 1

**a) Energy-Momentum 4-Vector and Lorentz Invariance (3/4)**

- Correct definition of the energy-momentum 4-vector.

- Properly shows the scalar product and its invariance under Lorentz transformation.

- Minor errors in the scalar product derivation (e.g., incorrect final form and simplification steps).

**b) Four-Momenta of Decay Particles in Lab Frame (4/5)**

- Correctly identifies the rest frame four-momenta of massless particles.

- Uses Lorentz transformation to obtain the lab frame four-momenta.

- Small error in the notation and derivation for the final expressions.

**c) Wavelengths in Rest and Lab Frames (3/4)**
- Correct initial setup and equations.

- Some numerical results are inaccurate or not detailed clearly.

- Good application of Doppler shift concepts, though slightly off in final wavelength calculations.

**Total: 10/13**

- General feedback: Strong grasp of relativistic concepts with some minor errors in calculations and notation.

Solution 2

**a) Energy-Momentum 4-Vector and Lorentz Invariance (3/4)**

- Proper definition and explanation of the energy-momentum 4-vector.

- Correct use of scalar product and demonstration of Lorentz invariance.

- Minor issues with notation consistency and clarity in the final invariant mass result.

**b) Four-Momenta of Decay Particles in Lab Frame (4/5)**

- Correct approach to finding four-momenta in the rest frame.

- Proper use of Lorentz transformation to obtain lab frame expressions.

- Minor calculation errors and lack of detailed steps for clarity.

**c) Wavelengths in Rest and Lab Frames (2/4)**

- Correct method and equations for finding photon energy and wavelength.

- Numerical results are inconsistent and incorrect in both frames.

- Some key steps and explanations are missing.

**Total: 9/13**

- General feedback: Good understanding with solid initial setup; needs more accurate calculations and detailed steps.

Solution 3

**a) Energy-Momentum 4-Vector and Lorentz Invariance (4/4)**

- Clear and correct definition of the energy-momentum 4-vector.

- Accurate and thorough demonstration of the scalar product's invariance under Lorentz transformations.

**b) Four-Momenta of Decay Particles in Lab Frame (4/5)**

- Correct identification of rest frame four-momenta.

- Proper application of Lorentz transformation.

- Missing some steps and final explicit forms for the four-momenta in the lab frame.

**c) Wavelengths in Rest and Lab Frames (3/4)**
- Correct approach and method to find energies and wavelengths.

- Good calculations, though final numerical results have minor errors.

- Clear demonstration of relativistic Doppler effect.

**Total: 11/13**

- General feedback: Excellent understanding and application of concepts with minor calculation errors; well-explained steps.

\bigskip
\textbf{Response 2:}

Solution 1 Evaluation

Part a)

Score: 3/4

- **Correct approach**: Correctly defined the energy-momentum 4-vector and showed its scalar product.

- **Missing Details**: Minor algebraic steps and explicit statement on invariance could be clearer.

Part b)

Score: 4/5

- **Correct approach**: Properly identified the four-momenta in the rest frame and correctly applied Lorentz transformations.

- **Minor issues**: More clarity on why \( E = cp_0 \) simplifies the transformation would be beneficial.

Part c)

Score: 3/4

- **Correct approach**: Correctly identified energy distribution and transformations for wavelengths.

- **Minor issues**: Numerical calculation errors in wavelengths; correct method but execution mistakes.

**General Feedback**: Solid understanding of concepts and transformations. Pay attention to numerical details and algebraic clarity.

Solution 2 Evaluation

Part a)

Score: 4/4

- **Correct approach**: Correctly defined and calculated the scalar product showing its invariance.

- **Clear and detailed**: Provided sufficient algebraic details and clear explanation.

Part b)

Score: 4/5

- **Correct approach**: Properly identified four-momenta and applied Lorentz transformations.

- **Minor issues**: Some steps in the transformations could be expanded for clarity.

Part c)

Score: 2/4

- **Correct approach**: Correctly identified energy distribution.

- **Calculation errors**: Incorrect energy value for photons leading to incorrect wavelengths.

**General Feedback**: Well-structured answers and clear logical steps. Numerical precision and correct calculations are crucial.

Solution 3 Evaluation

Part a)

Score: 4/4

- **Correct approach**: Correctly defined the energy-momentum 4-vector and scalar product invariance.

- **Clear and detailed**: Provided detailed algebraic steps and clear explanations.

Part b)

Score: 3/5

- **Correct approach**: Identified four-momenta and Lorentz transformation method.

- **Unclear details**: The transformation process and resulting expressions are less clear and concise.

Part c)

Score: 2/4

- **Correct approach**: Correctly identified energy distribution.

- **Calculation errors**: Incorrect energy and wavelength calculations.

**General Feedback**: Good conceptual understanding and clear initial steps. Improve clarity in transformations and ensure numerical accuracy.

---

**Overall Comments**:

- **Solution 1**: Good understanding with minor algebraic and numerical errors.

- **Solution 2**: Clear and detailed, but numerical precision needs improvement.

- **Solution 3**: Conceptually strong, but clarity and numerical accuracy need attention.

\bigskip
\textbf{Response 3:}

Here are the grades and feedback for each solution:

Solution 1

Part a)

- **Marks: 4/4**

- **Feedback:** The explanation of the energy-momentum 4-vector and the scalar product invariance is thorough and correct. The steps are clear, and the mathematical derivation is accurate.

Part b)

- **Marks: 4/5**

- **Feedback:** The initial setup of the problem and the Lorentz transformation are well-explained. However, the final step should involve transforming the energies and momenta using the full 4-vector transformation, considering both the time and spatial components. The provided transformation for momentum in part b) is incomplete.

Part c)

- **Marks: 3/4**

- **Feedback:** The calculation of the wavelengths is mostly correct, but the final numerical results have minor inaccuracies. Additionally, there is a slight misinterpretation of the final energies of the photons in the lab frame; they should be recalculated carefully.

**Total: 11/13**

Solution 2

Part a)

- **Marks: 4/4**

- **Feedback:** This part is correctly addressed with a clear explanation and proper derivation showing the invariance of the scalar product under Lorentz transformations.

Part b)

- **Marks: 5/5**

- **Feedback:** The transformation of the 4-momenta of the decay products in the lab frame is correctly derived and explained. The solution effectively uses the Lorentz transformations and provides the right expressions.

Part c)

- **Marks: 3/4**

- **Feedback:** The calculations for the wavelengths are generally accurate, but there are discrepancies in the final numerical results, indicating possible calculation errors. The correct approach is followed, but double-checking the final values is necessary.

**Total: 12/13**

Solution 3

Part a)

- **Marks: 4/4**

- **Feedback:** This part is well-explained, with a clear presentation of the energy-momentum 4-vector and the demonstration of its invariance under Lorentz transformations.

Part b)

- **Marks: 4/5**

- **Feedback:** The setup and explanation are correct, but the solution lacks explicit application of the Lorentz transformation equations to the decay products' 4-momenta. More detailed calculation steps are needed to show the final expressions in the lab frame.

Part c)

- **Marks: 3/4**

- **Feedback:** The approach and method to calculate the wavelengths are correct, but the numerical values provided are incorrect. There seems to be an error in converting the energies and determining the corresponding wavelengths.

**Total: 11/13**

\bigskip
\textbf{Response 4:}

Solution 1

**Part a: 4/4 marks**

- The form of the energy-momentum 4-vector is correctly stated.

- The Lorentz invariance of the scalar product is shown clearly and correctly.

- The algebra is correct and the explanation is thorough.

**Part b: 3/5 marks**

- The initial setup for the four-momenta of the decay particles is correct.

- However, the Lorentz transformations applied to the four-momenta are not fully accurate.

- There are some calculation errors in transforming the momenta components.

**Part c: 4/4 marks**

- The calculation of the wavelengths in both frames is correct.

- The explanation of the relativistic Doppler effect is clear and well-presented.

- The numerical results are accurate.

**General feedback:**

- Strong understanding of relativistic concepts and transformations.

- Minor errors in the application of Lorentz transformations in part b.

- Overall, a clear and well-explained solution.

**Total: 11/13 marks**

---

Solution 2

**Part a: 4/4 marks**

- The form of the energy-momentum 4-vector is correctly stated.

- The invariance of the scalar product under Lorentz transformation is correctly shown.

- The steps are well-detailed and correct.

**Part b: 4/5 marks**

- The initial setup for the four-momenta in the rest frame is correct.

- The Lorentz transformations are mostly correct but have slight errors in the explanation.

- The final transformed four-momenta are not entirely accurate.

**Part c: 3/4 marks**

- The energy and wavelength calculations are mostly correct.

- There is a slight error in the final numerical results for the wavelengths.

- The explanation of the relativistic effects is clear.

**General feedback:**

- Good understanding of the theoretical aspects.

- Minor errors in the calculations, particularly in parts b and c.

- Clear presentation and logical progression.

**Total: 11/13 marks**

---

Solution 3

**Part a: 4/4 marks**

- The form of the energy-momentum 4-vector is correctly stated.

- The Lorentz invariance of the scalar product is shown correctly.

- The explanation and algebra are thorough and accurate.

**Part b: 4/5 marks**

- The initial setup for the four-momenta is correct.

- The Lorentz transformation application is mostly correct but lacks some detail.

- There are some minor inaccuracies in the final expressions for the lab frame.

**Part c: 3/4 marks**

- The approach to finding the wavelengths is correct.

- There are some minor errors in the energy transformation and wavelength calculation.

- The explanation of relativistic effects is clear.

**General feedback:**

- Strong grasp of relativistic concepts.

- Minor inaccuracies in parts b and c.

- Clear and logical explanation throughout the solution.

**Total: 11/13 marks**

---

All three solutions demonstrate a good understanding of relativistic physics concepts, with minor calculation errors and inaccuracies affecting parts b and c. Each solution is well-explained and logically structured.

\bigskip
\textbf{Response 5:}

Solution 1

**a) Energy-Momentum 4-Vector and Lorentz Invariance**

- Form of the 4-vector is correct.

- Correctly identifies scalar product.

- Shows invariance under Lorentz transformation correctly.

**Marks: 4/4**

**b) Four-Momenta of Decay Particles in Lab Frame**

- Correct identification of momenta in the rest frame.

- Correct application of Lorentz transformations.

- Clear and accurate derivation.

**Marks: 5/5**

**c) Wavelengths in Rest and Lab Frames**

- Correct energy calculation in the rest frame.

- Correct application of Doppler shift in the lab frame.

- Correct wavelength calculations and explanations.

**Marks: 4/4**

**Total: 13/13**

**Feedback:** Excellent solution. Each part is addressed thoroughly and accurately, with correct derivations and calculations throughout.

Solution 2

**a) Energy-Momentum 4-Vector and Lorentz Invariance**

- Form of the 4-vector is correct.

- Correctly identifies scalar product.

- Shows invariance under Lorentz transformation correctly.

**Marks: 4/4**

**b) Four-Momenta of Decay Particles in Lab Frame**

- Correct identification of momenta in the rest frame.

- Correct application of Lorentz transformations.

- Clear and accurate derivation.

**Marks: 5/5**

**c) Wavelengths in Rest and Lab Frames**

- Correct energy calculation in the rest frame.

- Miscalculation of wavelengths due to incorrect rest frame wavelength value.

- Correct method but errors in numerical results.

**Marks: 3/4**

**Total: 12/13**

**Feedback:** Strong solution with clear steps and mostly accurate calculations. Minor numerical errors in the final part.

Solution 3

**a) Energy-Momentum 4-Vector and Lorentz Invariance**

- Form of the 4-vector is correct.

- Correctly identifies scalar product.

- Shows invariance under Lorentz transformation correctly.

**Marks: 4/4**

**b) Four-Momenta of Decay Particles in Lab Frame**

- Correct identification of momenta in the rest frame.

- Uses Lorentz transformation correctly but lacks explicit lab frame derivations.

**Marks: 4/5**

**c) Wavelengths in Rest and Lab Frames**

- Correct energy calculation in the rest frame.

- Uses correct approach for lab frame but results are not explicitly shown.

- Some values seem off due to possible misinterpretation.

**Marks: 2/4**

**Total: 10/13**

**Feedback:** Good understanding and approach. Needs more explicit derivations and attention to numerical results in the final part.



\subsection*{Question 8:}

\textbf{Response 1:}

Let's grade each solution based on the criteria given, providing feedback for both part a) and part b):

Solution 1

**Part a) (out of 9):** 7/9

- The solution correctly identifies the Lagrangian formulation and sets up the kinetic and potential energies accurately.

- However, there are unnecessary steps, such as stating to use Python for simplification, which is not part of the theoretical solution.

- The derivation is mostly correct but is not as streamlined as it could be, which slightly detracts from clarity.

**Part b) (out of 1):** 1/1

- The small-angle approximation is well-executed and clearly connects to the classic pendulum equation.

**General Feedback:** A good effort with mostly correct derivations and clear logical steps. Simplifying the process and avoiding extraneous details would improve clarity.

**Total:** 8/10

Solution 2

**Part a) (out of 9):** 8/9

- This solution provides a clear and correct derivation of the Lagrangian and the subsequent equation of motion.

- It includes all necessary components and handles the algebra well, maintaining a focus on the problem requirements.

**Part b) (out of 1):** 1/1

- The small-angle approximation is clear and correct, reducing correctly to the classic pendulum equation.

**General Feedback:** This solution is very clear and logically structured, with correct derivations and appropriate simplifications.

**Total:** 9/10

Solution 3

**Part a) (out of 9):** 6/9

- The solution correctly identifies the kinetic and potential energies and sets up the Lagrangian.

- However, the derivation of the equation of motion is a bit convoluted and contains a few algebraic manipulations that could be clearer.

- The final equation is correct but not as neatly presented as it could be.

**Part b) (out of 1):** 1/1

- The small-angle approximation and reduction to the classic pendulum equation are handled well.

**General Feedback:** While the solution is fundamentally correct, the presentation and clarity could be improved, especially in the derivation steps.

**Total:** 7/10

---

**Summary of Grades:**

1. Solution 1: 8/10

2. Solution 2: 9/10

3. Solution 3: 7/10

\bigskip
\textbf{Response 2:}

Here are the grades and feedback for each solution:

Solution 1

Grade: 8/10

- **Part a (7/9 marks):** 

  - **Strengths:**
  
    - Thoroughly derived the Lagrangian and the equations of motion.
    
    - Correctly identified the kinetic and potential energy terms.
    
    - Good use of detailed explanations and calculations.
    
  - **Areas for Improvement:**
  
    - The solution is somewhat verbose and could benefit from a more streamlined approach.
    
    - The use of Python is mentioned but not clearly integrated into the solution.
    
- **Part b (1/1 mark):** 

  - Correctly simplified the equation for small angles and zero values of \( v \) and \( A \), demonstrating the classic pendulum problem.

Feedback:

- Well-structured and detailed solution. Consider providing more concise explanations to improve readability. Good job on clearly deriving the Lagrangian and equations of motion.

Solution 2

Grade: 7/10

- **Part a (6/9 marks):** 

  - **Strengths:**
  
    - Good understanding of the kinetic and potential energy components.
    
    - Correctly applied the Euler-Lagrange equation to derive the equation of motion.
    
  - **Areas for Improvement:**
  
    - The kinetic energy derivation is slightly simplified and does not fully account for the translational kinetic energy due to the support's motion.
    
    - Missing the detailed intermediate steps in the calculation of the Lagrangian derivatives.
    
- **Part b (1/1 mark):** 

  - Correctly derived the classic pendulum equation for small angles and zero values of \( v \) and \( A \).

Feedback:

- Good approach and understanding. Ensure all components of the kinetic energy are fully considered and provide more detailed intermediate steps in calculations.

Solution 3

Grade: 9/10

- **Part a (8/9 marks):** 

  - **Strengths:**
  
    - Detailed and accurate derivation of the Lagrangian.
    
    - Comprehensive explanation of the kinetic and potential energy terms.
    
    - Correct application of the Euler-Lagrange equation to find the equation of motion.
    
  - **Areas for Improvement:**
  
    - Slightly verbose; could benefit from more concise explanations.
    
- **Part b (1/1 mark):** 

  - Correctly simplified the equation for small angles and zero values of \( v \) and \( A \), showing the classic pendulum problem.

Feedback:

- Excellent and thorough solution. Consider simplifying explanations for improved clarity. Great job on correctly deriving and simplifying the equations of motion. 

Overall, each solution demonstrated a strong understanding of the problem and correctly applied Lagrangian mechanics to derive the equations of motion. Improvements can be made in the clarity and conciseness of explanations, as well as in the detailed consideration of kinetic energy components.

\bigskip
\textbf{Response 3:}

Let's proceed to grade each solution out of a total of 10 marks, with up to 9 marks for part (a) and 1 mark for part (b), along with providing concise feedback.

Solution 1

**Grade: 7/10**

**Feedback:**

- **Part (a):** 6/9
  
  - The approach is correct, starting with the Lagrangian and moving towards the equation of motion.
 
  - Good breakdown of kinetic and potential energy components.
 
  - Some steps are overly complicated, and partial derivatives of the Lagrangian could be simplified more directly.
  
  - Proper application of the Euler-Lagrange equation but final expression is overly complex.

- **Part (b):** 1/1
  
  - Correctly shows the limit case reducing to the classic pendulum equation.

Solution 2

**Grade: 8/10**

**Feedback:**

- **Part (a):** 7/9
  
  - Clear and concise derivation using the Lagrangian.
  
  - Correct identification of kinetic and potential energy.
  
  - Proper application of the Euler-Lagrange equation with accurate final equation of motion.
  
  - Some minor omissions in handling the dependence of kinetic energy on the time derivative of the support motion.

- **Part (b):** 1/1
  
  - Correctly shows the limit case reducing to the classic pendulum equation.

Solution 3

**Grade: 9/10**

**Feedback:**

- **Part (a):** 8/9
  
  - Clear and systematic approach to deriving the Lagrangian and equation of motion.
  
  - Properly identifies kinetic and potential energy components and simplifies correctly.
  
  - Correct application of the Euler-Lagrange equation with accurate final equation of motion.
  
  - Very minor detail missing in the interpretation of the final simplified form.

- **Part (b):** 1/1
 
  - Correctly shows the limit case reducing to the classic pendulum equation.

---

These evaluations are based on the clarity of the derivation, correctness of mathematical steps, and proper application of theoretical concepts in Lagrangian mechanics. Each solution provided a logical pathway to the final answer but varied in the level of detail and simplicity in the intermediate steps.

\bigskip
\textbf{Response 4:}

Let's grade each solution out of 10 marks, with up to 9 marks for part a) and 1 mark for part b), and provide concise feedback.

Solution 1:

Part a)

- Correctly identifies kinetic and potential energy terms. (2 marks)

- Derives the Lagrangian, correctly accounting for the support's movement. (3 marks)

- Uses the Euler-Lagrange equation correctly. (2 marks)

- Some algebraic errors and overly complicated derivation. (2 marks)
  
Part b)

- Correctly reduces to the classic pendulum problem for small angles. (1 mark)

**Total: 10/10**

**Feedback:** Excellent solution. Thorough and correct, but the derivation could be streamlined for clarity.

Solution 2:

Part a)

- Correctly identifies kinetic and potential energy terms. (2 marks)

- Derives the Lagrangian but misses some components of the kinetic energy due to support movement. (2 marks)

- Uses the Euler-Lagrange equation correctly. (2 marks)

- Simplified too early, missed some terms. (1 mark)
  
Part b)

- Correctly reduces to the classic pendulum problem for small angles. (1 mark)

**Total: 8/10**

**Feedback:** Good attempt. Misses some complexity in the kinetic energy term and simplifies too early.

Solution 3:

Part a)

- Correctly identifies kinetic and potential energy terms. (2 marks)

- Derives the Lagrangian, accounting for the support's movement. (3 marks)

- Uses the Euler-Lagrange equation correctly. (2 marks)

- Some algebraic errors in the final equation of motion. (1 mark)
  
Part b)

- Correctly reduces to the classic pendulum problem for small angles. (1 mark)

**Total: 9/10**

**Feedback:** Very good solution. Some minor algebraic errors but overall clear and correct approach.

**Summary:**

- **Solution 1:** 10/10

- **Solution 2:** 8/10

- **Solution 3:** 9/10

\bigskip
\textbf{Response 5:}

Let's evaluate each of the student solutions provided for the given physics problem, using the criteria of up to 9 marks for part a) and up to 1 mark for part b).

**Problem Recap:**

**a)** Starting from the expression of the Lagrangian, derive the equation of motion of the angle, \( \phi \), of the pendulum.

**b)** In the limit of \( v \), \(A\) approaching 0 and small angles (\( \phi \) approaching 0), show this reduces to the classic pendulum problem and equation of motion.

**Solution 1 Evaluation:**

**Part a):** The student begins with the correct approach by defining the kinetic and potential energy. They also correctly express the Lagrangian and use the Euler-Lagrange equation. However, the solution is overly verbose and has some algebraic inaccuracies. The final equation of motion derived has the correct form but could be simplified further. 

**Part b):** The student correctly applies the small angle approximation and shows the reduction to the classic pendulum equation.

**Marks:**

- Part a): 6/9

- Part b): 1/1

**Total: 7/10**

**General Feedback:**

- The approach and method are correct, but there are some algebraic manipulations and simplifications that could be improved. The solution is clear but can be more concise.

---

**Solution 2 Evaluation:**

**Part a):** The student simplifies the problem by splitting kinetic energy into tangential and radial components. They derive the equation of motion correctly but omit the influence of the support's movement on the kinetic energy, which is a significant aspect given the problem's conditions.

**Part b):** The reduction to the classic pendulum problem is correctly shown using small-angle approximations.

**Marks:**

- Part a): 5/9

- Part b): 1/1

**Total: 6/10**

**General Feedback:**

- The solution captures the core idea but misses some critical details regarding the moving support's influence. The simplification approach for kinetic energy is good but lacks completeness.

---

**Solution 3 Evaluation:**

**Part a):** This solution correctly formulates the Lagrangian and incorporates the moving support's influence accurately. The Euler-Lagrange equation is applied correctly, and the equation of motion derived is correct. This solution is more concise and correct in its algebraic manipulations compared to the others.

**Part b):** The student correctly reduces the equation of motion to the classic pendulum problem using small-angle approximations.

**Marks:**

- Part a): 8/9

- Part b): 1/1

**Total: 9/10**

**General Feedback:**

- This is a well-structured solution with accurate and concise algebraic steps. The student demonstrates a clear understanding of the problem and correctly derives the necessary equations.

---

**Summary of Marks:**

1. **Solution 1:** 7/10

2. **Solution 2:** 6/10

3. **Solution 3:** 9/10

Solution 3 is the most accurate and concise, correctly addressing all aspects of the problem.

\subsection*{Question 9:}

\textbf{Response 1:}

Here are the grades and feedback for each solution:

Solution 1

**Part a)**

- Correctly derived the torque equation for the disk and the relation between linear and angular acceleration.

- Errors in simplifying the equations, leading to incorrect expressions for \( \alpha \) and \( a \).

- Correctly identified the relationship between tension, mass, and gravity but made algebraic mistakes in deriving the final expressions.

**Grade**: 4/6

**Part b)**

- Correctly identified the numerical values for given parameters.

- Calculation steps are correct, but the results rely on the incorrect expressions from part a).

**Grade**: 1/1

**Overall Grade**: 5/7

**Feedback**: Good understanding of the fundamental principles and relationships. However, algebraic manipulation errors led to incorrect final expressions. Ensure careful simplification of equations to avoid mistakes.

Solution 2

**Part a)**

- Correctly used Newton's second law and the rotational equivalent.

- Properly related linear and angular acceleration.

- Correctly derived the expressions for \( \alpha \), \( a \), and \( T \).

**Grade**: 6/6

**Part b)**

- Provided correct calculations using the derived expressions.

- Clear and accurate final numerical values.

**Grade**: 1/1

**Overall Grade**: 7/7

**Feedback**: Excellent solution with a clear understanding of both the physics principles and the algebra involved. Well-structured and correct throughout.

Solution 3

**Part a)**

- Correctly derived the torque and force equations.

- Properly related linear and angular acceleration.

- Derivations are clear and correct, leading to the right expressions for \( \alpha \), \( a \), and \( T \).

**Grade**: 6/6

**Part b)**

- Correctly applied the derived expressions to calculate numerical values.

- Accurate and clear final values.

**Grade**: 1/1

**Overall Grade**: 7/7

**Feedback**: Excellent solution with thorough and correct derivations. Clear, logical progression from principles to final answers. Well done.

---

To summarize:

- **Solution 1**: 5/7

- **Solution 2**: 7/7

- **Solution 3**: 7/7

Both Solution 2 and Solution 3 are perfect, with Solution 1 being strong but needing more careful algebraic work.

\bigskip
\textbf{Response 2:}

Grading the Solutions

Solution 1:

- **Part a):**

  1. The expression for angular acceleration (\( \alpha \)): \[ \alpha = \frac{2T}{M R} \]
  
  2. The expression for tension (\( T \)): \[ T = \frac{mgM}{M + 2m} \]
  
  3. The expression for acceleration (\( a \)): \[ a = \frac{2mg}{M + 2m} \]

  Correct steps and logical flow with minor computational oversight in simplifying the acceleration expression. The relations and derivations are mostly correct.

  **Score for Part a: 5/6**

- **Part b):**

  Correct substitution and final numerical results:
  
  - Tension: 19.6 N
  
  - Angular acceleration: 16.33 rad/s²
  
  - Linear acceleration: 3.27 m/s²

  **Score for Part b: 1/1**

  **Total Score: 6/7**

Solution 2:

- **Part a):**

  1. Correct identification of forces and torques.
  
  2. Correct elimination of \( T \) and solving for \( a \):
     \[ a = \frac{mg}{m + \frac{1}{2} M} \]
  
  3. Correct derivation of \( \alpha \) and \( T \):
     \[ \alpha = \frac{mg}{R(m + \frac{1}{2} M)} \]
     \[ T = \frac{1}{2} M \frac{mg}{m + \frac{1}{2} M} \]

  Very well-explained and correct derivations.

  **Score for Part a: 6/6**

- **Part b):**

  Correct final numerical results:
  
  - Tension: 19.6 N
  
  - Angular acceleration: 16.33 rad/s²
  
  - Linear acceleration: 3.27 m/s²

  **Score for Part b: 1/1**

  **Total Score: 7/7**

Solution 3:

- **Part a):**

  1. Correct identification of forces and torques.
  
  2. Derivation of \( \alpha \) and \( T \):
     \[ \alpha = \frac{2T}{MR} \]
     \[ T = \frac{mg}{1 + \frac{2m}{M}} \]
     
  3. Derivation of \( a \):
     \[ a = \frac{2mg}{M + 2m} \]

  Correct derivations but \( \alpha \) should have been derived explicitly in the intermediate steps.

  **Score for Part a: 5/6**

- **Part b):**

  Correct final numerical results:
  
  - Tension: 19.6 N
  
  - Angular acceleration: 16.35 rad/s² (slightly rounded)
  
  - Linear acceleration: 3.27 m/s²

  **Score for Part b: 1/1**

  **Total Score: 6/7**

Summary of Scores:

- **Solution 1:** 6/7

- **Solution 2:** 7/7

- **Solution 3:** 6/7

**General Feedback:**

- Solution 1 and Solution 3 both showed clear understanding and good derivation steps but missed a point due to slight inaccuracies or minor computational errors.

- Solution 2 was the most comprehensive and accurate, hence scored full marks.

\bigskip
\textbf{Response 3:}

**Solution 1:**

*Marking:*

- Part a):

  - Step-by-step derivation of \(\alpha\) is clear and correct. (2/2)
  
  - Derivation for \(T\) using Newton's second law is correctly related to \(\alpha\) but has an algebraic mistake. (2/2)
  
  - Expression for \(a\) derived correctly but it could be clearer and more direct. (2/2)
  
- Part b):

  - Correct substitution and calculations for given values. (1/1)

*Total: 6/7*

*Feedback:* Very clear and organized solution. The algebraic steps are mostly correct. Pay attention to clarity and simplification in deriving \(a\).

**Solution 2:**

*Marking:*

- Part a):

  - Correct torque and rotational equations leading to \(\alpha\). (2/2)
  
  - Correct expression for \(T\). (2/2)
  
  - Correct final expression for \(a\). (2/2)
  
- Part b):

  - Substitutions are correct, but missing detailed step-by-step calculations for full marks. (1/1)

*Total: 7/7*

*Feedback:* Well done! Clear, correct, and concise derivation of expressions. Ensure to show intermediate calculation steps in part b for completeness.

**Solution 3:**

*Marking:*

- Part a):

  - Correctly derives \(\alpha\) but should provide more details. (2/2)
  
  - Correctly derives \(T\) and \(a\) but steps could be more organized and clear. (2/2)
  
  - Solution is clear but could benefit from more concise notation. (2/2)
  
- Part b):

  - Correct calculations for given values. (1/1)

*Total: 7/7*

*Feedback:* Excellent solution with correct derivations and final values. Pay attention to detailed explanations and clear notation.

**Summary:**

1. **Solution 1:** 6/7

2. **Solution 2:** 7/7

3. **Solution 3:** 7/7

All solutions are well done with minor areas for improvement in clarity and completeness.

\bigskip
\textbf{Response 4:}

Here is the evaluation of each solution based on the provided problem and the requested parts (a) and (b).

**Solution 1 Evaluation:**

- **Part a):**

  - Correctly identifies and applies the moment of inertia and the relationship between torque and angular acceleration.
  
  - Correctly relates the linear acceleration of the block to the angular acceleration of the disk.
  
  - Correctly derives the expressions for the tension in the rope and the acceleration of the block.
  
  - Minor errors in algebraic manipulations and clarity of the derivation process.
  
  - Marks: 5/6

- **Part b):**

  - Correctly substitutes the given values and performs calculations accurately.
  
  - Marks: 1/1

**Total: 6/7**

**Feedback:** Well-structured and correct solution, with minor algebraic errors. Good job on the calculations.

**Solution 2 Evaluation:**

- **Part a):**

  - Correctly applies Newton's second law for both the block and the disk.
  
  - Correctly relates linear and angular accelerations.
  
  - Derives the expressions for tension, linear acceleration, and angular acceleration correctly.
  
  - Clear and logical step-by-step derivation.
  
  - Marks: 6/6

- **Part b):**

  - Correctly substitutes the given values and performs calculations accurately.
  
  - Marks: 1/1

**Total: 7/7**

**Feedback:** Excellent solution with clear derivations and correct calculations. Perfect understanding and execution.

**Solution 3 Evaluation:**

- **Part a):**

  - Correctly identifies the torque and applies Newton's second law for rotation.
  
  - Correctly derives the expressions for tension and acceleration.
  
  - Slight error in the final expression for angular acceleration.
  
  - Marks: 5/6

- **Part b):**

  - Correctly substitutes the given values and performs calculations accurately.
  
  - Marks: 1/1

**Total: 6/7**

**Feedback:** Well-explained and structured solution with minor errors. Good accuracy in calculations.

Overall, Solution 2 is the best, with a perfect score due to its clarity, correctness, and thoroughness. Solutions 1 and 3 are also well done, with minor issues reducing their scores slightly.

\bigskip
\textbf{Response 5:}

Let's grade each solution based on the problem statement and the criteria provided.

Solution 1

**Part a)**

1. The solution provides a clear step-by-step derivation of the angular acceleration of the disk (\(\alpha\)).

2. The tension (\(T\)) is correctly derived using Newton's second law.

3. The acceleration of the block (\(a\)) is related to the angular acceleration and derived correctly.

4. The explanation is coherent and logical.

**Part b)**

1. Correct numerical values are calculated for \(\alpha\), \(T\), and \(a\).

**Marks:**

- Part a: 6/6

- Part b: 1/1

- **Total: 7/7**

**Feedback:**

Excellent job. The derivation is clear and correct, and the numerical values are accurate.

---

Solution 2

**Part a)**

1. The solution applies Newton's second law correctly for both the block and the disk.

2. The relationship between linear and angular acceleration is correctly identified.

3. The final expressions for \(\alpha\), \(T\), and \(a\) are correct and derived logically.

**Part b)**

1. Correct numerical values are calculated for \(\alpha\), \(T\), and \(a\).

**Marks:**

- Part a: 6/6

- Part b: 1/1

- **Total: 7/7**

**Feedback:**

Great work. The solution is thorough and accurate, with correct derivations and calculations.

---

Solution 3

**Part a)**

1. The solution provides a clear step-by-step derivation for \(\alpha\), \(T\), and \(a\).

2. The relationships between forces, torque, and acceleration are correctly identified.

3. The final expressions for \(\alpha\), \(T\), and \(a\) are correct.

**Part b)**

1. Correct numerical values are calculated for \(\alpha\), \(T\), and \(a\).

**Marks:**

- Part a: 6/6

- Part b: 1/1

- **Total: 7/7**

**Feedback:**

Excellent work. The derivation is detailed and correct, and the numerical results are accurate.

---

Summary

All three solutions provided correct derivations and numerical calculations. Each solution demonstrated a clear understanding of the problem and applied the relevant physical principles correctly. Therefore, all three solutions receive full marks.

\subsection*{Question 10:}

\textbf{Response 1:}

**Solution 1:**

**Part a)**

Marks: 6/6

Feedback: Excellent job in deriving and solving the differential equation. The explanation is clear, logical, and covers all necessary steps to arrive at the solution form provided in the problem statement. The transition to identifying the terminal velocity and incorporating it into the general solution is well done.

**Part b)**

Marks: 2/2

Feedback: The method for determining \(\underline{A}\) using the initial condition is correct and clearly explained. The steps are logical, and the final expression for \(\underline{A}\) is accurately derived.

---

**Solution 2:**

**Part a)**

Marks: 5/6

Feedback: Good derivation of the differential equation and explanation of the general solution. However, the approach to separating the components and focusing primarily on the vertical component could be confusing as the problem involves the entire vector \(\underline{v}\). The solution form and reasoning are correct, but some clarity is lost in separating components unnecessarily.

**Part b)**

Marks: 2/2

Feedback: Correct determination of \(\underline{A}\) using the initial condition. The explanation is clear and logically leads to the final expression for \(\underline{A}\).

---

**Solution 3:**

**Part a)**

Marks: 6/6

Feedback: Thorough and accurate derivation of the differential equation and its solution. The explanation is clear and logically progresses through the necessary steps. The verification of the solution by differentiation is well executed and shows a strong understanding of the problem.

**Part b)**

Marks: 2/2

Feedback: The initial condition is correctly applied to find \(\underline{A}\). The explanation is clear and the final expression for \(\underline{A}\) is accurately derived.

\bigskip
\textbf{Response 2:}

Solution 1 Evaluation

**Part a)**: (6/6)

- Correctly applies Newton's second law.

- Derives the differential equation \( m\frac{d\underline{v}}{dt} = -mg \hat{\underline{k}} - b\underline{v} \) accurately.

- Correctly identifies and solves the differential equation.

- Derives the solution \( \underline{v} = \underline{A} e^{-bt/m} - \frac{mg}{b} \hat{\underline{k}} \).

**Part b)**: (2/2)

- Correctly uses the initial condition to find \( \underline{A} \).
- The expression \( \underline{A} = \underline{u} + \frac{mg}{b} \hat{\underline{k}} \) is accurate.

**Feedback**: 

- Well-structured and thorough solution. 

- Clear and concise explanation of each step. 

**Total Marks**: 8/8

Solution 2 Evaluation

**Part a)**: (5/6)

- Correctly applies Newton's second law.

- Derives the differential equation correctly.

- The solution derivation is a bit verbose and slightly less direct, but the final result is correct.

- Splitting the components separately is not strictly necessary but still correct.

**Part b)**: (2/2)

- Uses the initial condition correctly to find \( \underline{A} \).

- The expression \( \underline{A} = \underline{u} + \frac{mg}{b} \hat{\underline{k}} \) is accurate.

**Feedback**: 

- Good understanding and correct derivation, although more concise derivation would be preferred.

- Correct and clear solution for part b).

**Total Marks**: 7/8

Solution 3 Evaluation

**Part a)**: (6/6)

- Correctly applies Newton's second law.

- Derives the differential equation \( m\frac{d\underline{v}}{dt} = -mg \hat{\underline{k}} - b\underline{v} \) accurately.

- Correctly solves the differential equation and verifies the solution.

- Derives the solution \( \underline{v} = \underline{A} e^{-bt/m} - \frac{mg}{b} \hat{\underline{k}} \).

**Part b)**: (2/2)

- Correctly uses the initial condition to find \( \underline{A} \).

- The expression \( \underline{A} = \underline{u} + \frac{mg}{b} \hat{\underline{k}} \) is accurate.

**Feedback**: 

- Well-structured and detailed solution.

- Each step is clearly explained and verified, ensuring correctness.

**Total Marks**: 8/8

\bigskip
\textbf{Response 3:}

Solution 1:

**Part a)** 

Marks: 5/6

**Feedback:**

- The student correctly applies Newton's second law and sets up the differential equation.

- The solution method is accurate, and the separation of variables is correctly handled.

- There is a minor point deduction for not clearly explaining the nature of the solution to the differential equation and assuming without explicitly verifying the form of the solution.

**Part b)** 

Marks: 2/2

**Feedback:**

- The expression for \( \underline{A} \) is derived correctly using the initial condition.

- The explanation is clear and follows logically from the given information.

Total Marks: 7/8

Solution 2:

**Part a)** 

Marks: 4/6

**Feedback:**

- The student correctly identifies and sets up the differential equation.

- The solution process is correct, but the explanation is a bit disjointed, and the separation of the components (vertical and horizontal) isn't necessary for the general form of the solution.

- The link between the general solution and the specific form given in the problem isn't as clearly made as it could be.

**Part b)** 

Marks: 2/2

**Feedback:**

- The initial condition is used correctly to find \( \underline{A} \).

- The explanation is clear and logically presented.

Total Marks: 6/8

Solution 3:

**Part a)** 

Marks: 6/6

**Feedback:**

- The student correctly applies Newton's second law and derives the differential equation.

- The solution method is accurate, and the verification of the solution is thorough.

- The explanation is clear, logical, and complete.

**Part b)** 

Marks: 2/2

**Feedback:**

- The expression for \( \underline{A} \) is derived correctly using the initial condition.

- The explanation is clear and follows logically from the given information.

Total Marks: 8/8

\bigskip
\textbf{Response 4:}

Solution 1

Part a)

Marks: 6/6

- Correct application of Newton's second law.

- Correct identification and setup of the forces.

- Correct differential equation formulation and solution.

- Correct final expression in the required form.

Feedback: Excellent work. The derivation is clear, and the solution is presented in a well-organized manner. The mathematical steps are accurate and thorough.

Part b)

Marks: 2/2

- Correct initial condition application.

- Correct determination of the constant vector \( \underline{A} \).

Feedback: The initial condition is correctly applied, and the final expression for \( \underline{A} \) is correct.

---

Solution 2

Part a)

Marks: 5/6

- Correct application of Newton's second law.

- Correct identification of forces and formulation of the differential equation.

- The steps to solve the differential equation are clear but lack some details in the separation of components.

Feedback: Good work overall. However, the solution would benefit from more explicit detail in solving the differential equation, particularly in treating the components separately.

Part b)

Marks: 2/2

- Correct initial condition application.

- Correct determination of the constant vector \( \underline{A} \).

Feedback: The expression for \( \underline{A} \) is accurate and well-justified using the initial conditions.

---

Solution 3

Part a)

Marks: 5/6

- Correct application of Newton's second law.

- Correct formulation of the differential equation.

- The solution approach is valid, but there is a minor inconsistency in verifying the solution.

Feedback: The derivation is mostly clear and correct, but the verification step could be more explicit. The mathematical steps leading to the solution are accurate.

Part b)

Marks: 2/2

- Correct initial condition application.

- Correct determination of the constant vector \( \underline{A} \).

Feedback: The initial condition is correctly applied, and the expression for \( \underline{A} \) is correctly derived.

---

Summary

- **Solution 1**: 8/8 - Excellent and comprehensive solution.

- **Solution 2**: 7/8 - Good solution, minor detail missing in component separation.

- **Solution 3**: 7/8 - Good solution, minor inconsistency in verification.

\bigskip
\textbf{Response 5:}

Solution 1

**Marks: 7/8 (5 for part a, 2 for part b)**

**Feedback:**

- Part a: Excellent derivation of the equation using Newton's second law. The solution is clear and logical, correctly identifying the force components and solving the differential equation. However, there is a slight omission in not explicitly showing the step-by-step integration process of the differential equation, which would make the solution more rigorous.

- Part b: Correctly identifies the constant vector \(\underline{A}\) using the initial condition. The explanation is clear and correctly applies the initial velocity.

Solution 2

**Marks: 6/8 (4 for part a, 2 for part b)**

**Feedback:**

- Part a: Good attempt at solving the differential equation, but the explanation is somewhat vague regarding the separation of components and does not clearly outline the process of solving the equation for both horizontal and vertical components. A more detailed integration process would improve clarity.

- Part b: Correct identification of \(\underline{A}\) using the initial condition. The explanation is clear and correctly applies the initial velocity.

Solution 3

**Marks: 8/8 (6 for part a, 2 for part b)**

**Feedback:**

- Part a: Excellent derivation of the equation using Newton's second law, with clear and detailed steps. The process of solving the differential equation is well-explained, and the solution is verified correctly.

- Part b: Correctly identifies the constant vector \(\underline{A}\) using the initial condition. The explanation is clear and concise, correctly applying the initial velocity.

\end{document}
