\documentclass[a4paper,11pt]{article}
\setlength\parindent{0pt}
\usepackage{amsmath, amssymb}
\usepackage[margin=1in,letterpaper]
{geometry} % decreases margins

\usepackage{setspace}
\setstretch{1.2}
\usepackage{spverbatim}

\title{GPT-4o Grading (No Mark Scheme)}

\begin{document}

\maketitle
This document compiles the GPT-4o responses when prompted to grade the solutions which have been generated. No markscheme is given to GPT-4o in the prompt. This is the grading for the Quantum mechanics (QM) questions.

\section{QM}

\subsection*{Question 1:}

\subsubsection*{Response 1:}

Solution 1 :

\textbf{Marks: 4/5}

\textbf{Feedback:}

\textbf{Part a) - 2/3}

\begin{itemize}
    \item The definition of the commutator is correct and well-explained.
    \item The explanation of compatibility and the implication on the commutator is accurate.
    \item However, the solution could provide a bit more detail about the physical interpretation of compatibility.
\end{itemize}

\textbf{Part b) - 2/2}

\begin{itemize}
    \item The explanation is clear and addresses the effect of non-commuting operators on the system state.
    \item It correctly discusses the uncertainty in the outcome of a subsequent measurement of$\hat{P}$.
\end{itemize}

\textbf{General feedback:}

\begin{itemize}
    \item Good understanding and clear explanations.
    \item Could enhance part a) by including more about the physical significance of having a common set of eigenstates.
\end{itemize}



Solution 2 :

\textbf{Marks: 5/5}

\textbf{Feedback:}

\textbf{Part a) - 3/3}

\begin{itemize}
    \item The definition and explanation are clear and concise.
    \item It includes a good explanation of what compatibility implies about the operators and their measurements.
\end{itemize}

\textbf{Part b) - 2/2}

\begin{itemize}
    \item Correctly explains the consequences of non-commuting operators on subsequent measurements.
    \item The discussion on the overlap between eigenstates adds depth to the explanation.
\end{itemize}

\textbf{General feedback:}

\begin{itemize}
    \item Excellent clarity and depth in both parts.
    \item Comprehensive explanation with a strong understanding of the concepts.
\end{itemize}


Solution 3 :

\textbf{Marks: 4/5}

\textbf{Feedback:}

\textbf{Part a) - 3/3}

\begin{itemize}
    \item The definition of the commutator and the explanation of compatibility are correct and well-written.
    \item It clearly explains the implications of compatibility on measurements and eigenstates.
\end{itemize}

\textbf{Part b) - 1/2}

\begin{itemize}
    \item The explanation is accurate but could use more detail about why the subsequent measurement of $\hat{P}$ can yield any of its eigenvalues.
    \item The discussion is slightly less detailed compared to other solutions.
\end{itemize}

\textbf{General feedback:}

\begin{itemize}
    \item Solid understanding and clear presentation.
    \item Part b) could benefit from a bit more elaboration on the measurement process and its consequences.
\end{itemize}



\subsubsection*{Response 2:}

Solution 1 :

\textbf{Marks: 5/5}

\textbf{Feedback:}

\textbf{Part a) - 3/3}

\begin{itemize}
    \item Clear and correct definition of the commutator.
    \item Accurate explanation of compatibility and its implications.
\end{itemize}

\textbf{Part b) - 2/2}

\begin{itemize}
    \item Correct explanation of the implications of non-compatibility.
    \item Well-articulated reasoning about the possible results of measuring $\hat{P}$ again.
\end{itemize}


Solution 2 :

\textbf{Marks: 5/5}

\textbf{Feedback:}

\textbf{Part a) - 3/3}

\begin{itemize}
    \item Correct definition of the commutator and compatibility.
    \item Thorough explanation of the implications, including common eigenstates and simultaneous measurements.
\end{itemize}

\textbf{Part b) - 2/2}

\begin{itemize}
    \item Correct and detailed explanation of the effects of non-compatibility.
    \item Clear explanation of the implications for subsequent measurements.
\end{itemize}



Solution 3 :

\textbf{Marks: 5/5}

\textbf{Feedback:}

\textbf{Part a) - 3/3}

\begin{itemize}
    \item Clear and correct definition of the commutator.
    \item Correct explanation of compatibility and its implications.
\end{itemize}

\textbf{Part b) - 2/2}

\begin{itemize}
    \item Accurate description of the effects of measuring incompatible operators.
    \item Well-explained reasoning about the possible results of measuring $\hat{P}$ again.
\end{itemize}


\textbf{General feedback:}
All three solutions provided accurate definitions and explanations for both parts of the problem. Each solution demonstrated a solid understanding of the concepts of commutators, compatible operators, and the implications of non-compatibility in quantum mechanics. The explanations were clear and detailed, addressing all necessary points thoroughly.




\subsubsection*{Response 3:}

Solution 1 :

\textbf{Marks: 5/5}

\textbf{Feedback:}

\textbf{Part a) - 3/3}

\begin{itemize}
    \item The definition of the commutator is clear and correct.
    \item The explanation of compatibility is accurate and well-explained.
    \item The connection between commutation and simultaneous measurability is clearly stated.
\end{itemize}

\textbf{Part b) - 2/2}

\begin{itemize}
    \item The student correctly explains the implications of non-compatibility.
    \item The explanation of the sequence of measurements and the resulting uncertainty is thorough and accurate.
\end{itemize}

\textbf{General feedback:}
Excellent explanation and clarity in both parts. The student has demonstrated a strong understanding of the concepts and their implications.



Solution 2 :

\textbf{Marks: 5/5}

\textbf{Feedback:}

\textbf{Part a) - 3/3}

\begin{itemize}
    \item The definition of the commutator is correct.
    \item Compatibility is well-defined, with a clear connection to the commutator being zero.
    \item The explanation of simultaneous measurements and common eigenstates is precise.
\end{itemize}

\textbf{Part b) - 2/2}

\begin{itemize}
    \item The student accurately describes the impact of measuring non-compatible operators.
    \item The explanation covers the collapse of the wave function and the resulting uncertainty in subsequent measurements.
\end{itemize}

\textbf{General feedback:}
Very good and precise answers with clear and concise explanations. The student has demonstrated a solid grasp of the subject matter.



Solution 3 :

\textbf{Marks: 5/5}

\textbf{Feedback:}

\textbf{Part a) - 3/3}

\begin{itemize}
    \item The definition of the commutator is accurate.
    \item Compatibility is clearly defined, with a good explanation of what it implies.
    \item The connection between commutation and simultaneous measurement is well-explained.
\end{itemize}

\textbf{Part b) - 2/2}

\begin{itemize}
    \item The student correctly explains the effect of non-compatibility.
    \item The description of measurement sequence and its impact on the system state is clear and accurate.
\end{itemize}

\textbf{General feedback:}
Clear and accurate answers with good explanations. The student has demonstrated a comprehensive understanding of the concepts involved.

\textbf{Summary :}
All three solutions are well-done, with each student correctly defining the commutator, explaining compatibility, and discussing the implications of non-compatible operators in a measurement sequence. Each solution demonstrates a strong understanding of the principles of quantum mechanics, earning full marks for both parts of the problem.


\subsubsection*{Response 4:}

Solution 1 :

\textbf{Marks: 4/5}

\textbf{Feedback:}

\textbf{Part a) - 3/3}

\begin{itemize}
    \item The definition of the commutator is correct.
    \item The explanation of compatibility and its implications is thorough and clear.
    \item The concept of common eigenstates and the ability to determine observables simultaneously is well articulated.
\end{itemize}

\textbf{Part b) - 1/2}

\begin{itemize}
    \item The explanation correctly identifies that the state changes after measuring $\hat{Q}$, leading to an uncertain outcome for the second measurement of $\hat{P}$.
    \item The answer lacks specific detail about the possible outcomes of measuring $\hat{P}$ again. It should mention the probabilities depending on the overlap between the eigenstates of $\hat{Q}$ and $\hat{P}$.
\end{itemize}

\textbf{General feedback:}
Excellent explanation of the commutator and compatibility. For part (b), provide more detail about the relationship between the eigenstates of $\hat{P}$ and $\hat{Q}$ to fully explain the uncertainty in the second measurement.



Solution 2 :

\textbf{Marks: 5/5}

\textbf{Feedback:}

\textbf{Part a) - 3/3}

\begin{itemize}
    \item The definition of the commutator and compatibility is clear and correct.
    \item The implications of compatibility, including the shared eigenstates and simultaneous measurements, are well explained.
\end{itemize}

\textbf{Part b) - 2/2}

\begin{itemize}
    \item The explanation correctly identifies the impact of the measurements on the system's state.
    \item The mention of the probabilities depending on the overlap between the eigenstates of $\hat{Q}$ and $\hat{P}$ provides a complete understanding of the possible results of the second measurement of $\hat{P}$.
\end{itemize}

\textbf{General feedback:}
Thorough and well-explained solutions for both parts. The detail about eigenstate overlap in part (b) is particularly good.



Solution 3 :

\textbf{Marks: 4/5}

\textbf{Feedback:}

\textbf{Part a) - 3/3}

\begin{itemize}
    \item The definition of the commutator and compatibility is correct and well presented.
    \item The implications of compatibility, including the common set of eigenstates and the ability to measure observables simultaneously, are well explained.
\end{itemize}

\textbf{Part b) - 1/2}

\begin{itemize}
    \item The explanation correctly identifies that measuring $\hat{Q}$ alters the state, leading to an uncertain outcome for the second measurement of $\hat{P}$.
    \item The answer is missing details about the specific possible outcomes of measuring $\hat{P}$ again, particularly the role of eigenstate overlap in determining the probabilities of different outcomes.
\end{itemize}

\textbf{General feedback:}
Solid explanation for part (a). For part (b), add details about the possible outcomes and their probabilities to fully explain the effect of non-compatibility on the measurement sequence.


\subsubsection*{Response 5:}

Solution 1 :

\textbf{Marks: 5/5}

\textbf{Feedback:}

\textbf{Part a) - 3/3}

\begin{itemize}
    \item Clear and accurate definition of the commutator.
    \item Correctly explains compatibility and its implications on the commutator and measurements.
    \item Nicely details the concept of common eigenstates and simultaneous measurement.
\end{itemize}

\textbf{Part b) - 2/2}

\begin{itemize}
    \item Correctly explains the implications of non-commuting operators on measurement outcomes.
    \item Clearly describes how the measurement of $\hat{Q}$ affects the state of the system and the subsequent measurement of $\hat{P}$.
    \item Adequately highlights the uncertainty introduced by the non-commutativity.
\end{itemize}

\textbf{General feedback:}
Comprehensive and accurate. The student has a strong understanding of the concepts and their implications in quantum mechanics.



Solution 2 :

\textbf{Marks: 5/5}

\textbf{Feedback:}

\textbf{Part a) - 3/3}

\begin{itemize}
    \item Correct definition of the commutator.
    \item Adequately explains compatibility and common eigenstates.
    \item Thoroughly covers the implications of simultaneous measurements and non-disturbance.
\end{itemize}

\textbf{Part b) - 2/2}

\begin{itemize}
    \item Correctly addresses the non-compatibility of operators and its effect on measurements.
    \item Describes the collapse of the system's state due to measurements and the resultant uncertainty.
    \item Highlights the dependency of subsequent measurements on the altered state post-measurement.
\end{itemize}

\textbf{General feedback:}
Clear and precise. The explanation covers all necessary points and shows a good grasp of quantum mechanical principles.


Solution 3 :

\textbf{Marks: 5/5}

\textbf{Feedback:}

\textbf{Part a) - 3/3}

\begin{itemize}
    \item Correctly defines the commutator.
    \item Accurately explains the meaning of compatibility and the implications for measurements and eigenstates.
    \item Details the ability to determine precise values simultaneously.
\end{itemize}

\textbf{Part b) - 2/2}

\begin{itemize}
    \item Correctly identifies the impact of non-commuting operators on measurement outcomes.
    \item Clearly explains the alteration of the system's state and the uncertainty in repeated measurements.
    \item Describes how the system does not return to the original state due to lack of common eigenstates.
\end{itemize}

\textbf{General feedback:}
Well-written and precise. The student provides a clear and detailed explanation of the concepts, demonstrating a good understanding of the subject matter.



\subsection*{Question 2:}

\subsubsection*{Response 1:}

Solution 1 :

\textbf{Marks: 4/4}

\textbf{Feedback:}

\textbf{Part a) - 2/2}

\begin{itemize}
    \item Excellent explanation. The student correctly identifies that the eigenvalues of a Hermitian operator are real and provides a thorough physical interpretation, including the context of observable physical quantities.
\end{itemize}

\textbf{Part b) - 2/2}

\begin{itemize}
    \item Clear and precise explanation of orthonormality. The student correctly describes orthogonality and normalization, providing appropriate mathematical expressions and explaining their significance in quantum mechanics.
\end{itemize}


Solution 2 :

\textbf{Marks: 4/4}

\textbf{Feedback:}

\textbf{Part a) - 2/2}

\begin{itemize}
    \item The explanation is accurate and clear. The student correctly states the reality of eigenvalues for Hermitian operators and provides a solid physical interpretation, emphasizing the role of these operators in quantum mechanics.
\end{itemize}

\textbf{Part b) - 2/2}

\begin{itemize}
    \item The explanation is concise and correct. The student clearly defines orthonormality, using the Kronecker delta to succinctly express the orthogonality and normalization conditions.
\end{itemize}



Solution 3 :

\textbf{Marks: 4/4}

\textbf{Feedback:}

\textbf{Part a) - 2/2}

\begin{itemize}
    \item Comprehensive and detailed explanation. The student correctly identifies the real nature of eigenvalues of Hermitian operators and provides a clear mathematical derivation and physical interpretation.
\end{itemize}

\textbf{Part b) - 2/2}

\begin{itemize}
    \item The explanation is detailed and accurate. The student correctly describes orthonormality and provides a thorough mathematical expression, explaining the significance of these properties in quantum mechanics.
\end{itemize}



\subsubsection*{Response 2:}

Solution 1 :

\textbf{Marks: 4/4}

\textbf{Feedback:}

\textbf{Part a) - 2/2}

\begin{itemize}
    \item The student correctly identifies that the eigenvalues of a Hermitian operator are real.
    \item The explanation includes both the mathematical reasoning and the physical interpretation, which is well-articulated and thorough.
\end{itemize}

\textbf{Part b) - 2/2}

\begin{itemize}
    \item The student accurately describes orthonormality, covering both orthogonality and normalization.
    \item The mathematical expressions are correct and clearly explained.
    \item The explanation of the significance in quantum mechanics is comprehensive and precise.
\end{itemize}

\textbf{General Feedback:}

\begin{itemize}
    \item Excellent work. Both the mathematical properties and physical interpretations are clearly and accurately presented. Your explanation of orthonormality is detailed and well-articulated.
\end{itemize}


Solution 2 :

\textbf{Marks: 4/4}

\textbf{Feedback:}

\textbf{Part a) - 2/2}

\begin{itemize}
    \item The student correctly states that the eigenvalues of a Hermitian operator are real and provides a clear mathematical justification.
    \item The physical interpretation is also correctly explained and tied to the real nature of observable quantities.
\end{itemize}

\textbf{Part b) - 2/2}

\begin{itemize}
    \item The student provides a clear and correct explanation of orthonormality, using the Kronecker delta to succinctly convey orthogonality and normalization.
    \item The significance in quantum mechanics is well-explained.
\end{itemize}

\textbf{General Feedback:}

\begin{itemize}
    \item Excellent response. Both the mathematical and physical aspects of Hermitian operators and orthonormality are well-explained and accurate.
\end{itemize}


Solution 3 :

\textbf{Marks: 4/4}

\textbf{Feedback:}

\textbf{Part a) - 2/2}

\begin{itemize}
    \item The student correctly identifies that the eigenvalues of a Hermitian operator are real and provides a sound mathematical proof.
    \item The physical interpretation is clear and appropriately explained.
\end{itemize}

\textbf{Part b) - 2/2}

\begin{itemize}
    \item The student accurately describes the orthonormality condition, using correct mathematical notation.
    \item The explanation of the importance in quantum mechanics is well-covered and clear.
\end{itemize}

\textbf{General Feedback:}

\begin{itemize}
    \item Very good job. Your explanations are precise and your use of mathematical proofs to support your points is strong. The physical significance is also well-articulated.
\end{itemize}

\textbf{Summary :}
All three solutions have demonstrated a thorough understanding of the mathematical properties and physical interpretations of Hermitian operators and their eigenvalues, as well as the orthonormality of their eigenfunctions. Each student has provided clear, accurate explanations and appropriate mathematical justifications. Well done!


\subsubsection*{Response 3:}

Solution 1 :

\textbf{Marks: 4/4}

\textbf{Feedback:}

\textbf{Part a) - 2/2}

\begin{itemize}
    \item Thorough and accurate explanation of the mathematical property and physical interpretation. Clear and well-articulated.
\end{itemize}

\textbf{Part b) - 2/2}

\begin{itemize}
    \item Provides a detailed and correct explanation of orthonormality. Uses appropriate mathematical notation and explains the significance clearly.
\end{itemize}

\textbf{General Feedback:}

\begin{itemize}
    \item Excellent work. Both parts are comprehensive and demonstrate a solid understanding of the concepts. The explanations are clear and detailed.
\end{itemize}


Solution 2 :

\textbf{Marks: 4/4}

\textbf{Feedback:}

\textbf{Part a) - 2/2}

\begin{itemize}
    \item Correctly identifies the mathematical property and physical interpretation. The explanation is precise and well-structured.
\end{itemize}

\textbf{Part b) - 2/2}

\begin{itemize}
    \item Correct explanation of orthonormality with clear mathematical notation. The importance of orthogonality and normalization is well-articulated.
\end{itemize}

\textbf{General Feedback:}

\begin{itemize}
    \item Well done. The answers are accurate and the explanations are clear and concise. The use of the Kronecker delta is appropriate and correctly applied.
\end{itemize}


Solution 3 :

\textbf{Marks: 4/4}

\textbf{Feedback:}

\textbf{Part a) - 2/2}

\begin{itemize}
    \item Accurate and detailed explanation of the mathematical property and physical interpretation. The use of mathematical notation is appropriate and correct.
\end{itemize}

\textbf{Part b) - 2/2}

\begin{itemize}
    \item Correct and thorough explanation of orthonormality. The explanation is clear and well-supported with appropriate mathematical notation.
\end{itemize}

\textbf{General Feedback:}

\begin{itemize}
    \item Excellent response. Both parts are comprehensive and well-explained. The use of mathematical notation is accurate and the explanations are clear and precise.
\end{itemize}

\textbf{Summary :}
Overall, all three solutions are excellent, demonstrating a strong understanding of the concepts related to Hermitian operators and their properties in quantum mechanics. Each solution effectively uses mathematical notation and provides clear, accurate explanations.


\subsubsection*{Response 4:}

Solution 1 :

\textbf{Marks: 4/4}

\textbf{Feedback:}

\textbf{Part a) - 2/2}

\begin{itemize}
    \item The explanation accurately covers the fact that the eigenvalues of Hermitian operators are real and ties this to the physical interpretation of measurements in quantum mechanics.
    \item It correctly identifies that these eigenvalues represent possible outcomes of measurements.
\end{itemize}

\textbf{Part b) - 2/2}

\begin{itemize}
    \item The student clearly explains both orthogonality and normalization.
    \item The mathematical expressions and the physical implications are correctly and thoroughly described.
\end{itemize}

\textbf{General Feedback:}

\begin{itemize}
    \item Excellent explanation of both the mathematical and physical aspects. The mathematical notation and interpretations are accurate and well-articulated.
\end{itemize}


Solution 2 :

\textbf{Marks: 4/4}

\textbf{Feedback:}

\textbf{Part a) - 2/2}

\begin{itemize}
    \item The explanation is accurate, describing the real nature of eigenvalues for Hermitian operators and the connection to observable physical quantities.
    \item It includes the mathematical reasoning and physical interpretation effectively.
\end{itemize}

\textbf{Part b) - 2/2}

\begin{itemize}
    \item The student correctly explains orthonormality using the Kronecker delta and its significance in forming a complete basis for quantum states.
    \item The explanation is concise and accurate, with appropriate mathematical notation.
\end{itemize}

\textbf{General Feedback:}

\begin{itemize}
    \item Well-done. The explanation is clear and concise, covering all necessary points with accurate mathematical and physical interpretations.
\end{itemize}


Solution 3 :

\textbf{Marks: 4/4}

\textbf{Feedback:}

\textbf{Part a) - 2/2}

\begin{itemize}
    \item The student provides a detailed explanation of why eigenvalues of Hermitian operators are real, including both mathematical and physical justifications.
    \item The reasoning is thorough and correct.
\end{itemize}

\textbf{Part b) - 2/2}

\begin{itemize}
    \item The explanation of orthonormality is precise and includes correct mathematical notation.
    \item The importance of orthonormality in quantum mechanics is well explained.
\end{itemize}

\textbf{General Feedback:}

\begin{itemize}
    \item Comprehensive and detailed explanation, with accurate use of mathematical reasoning and physical interpretation. Well-structured and clear.
\end{itemize}

\textbf{Summary :}
All three solutions are excellent, providing accurate and thorough explanations with correct use of mathematical notation and physical interpretations. Each solution has been awarded full marks for both parts.


\subsubsection*{Response 5:}

Solution 1 :

\textbf{Marks: 4/4}

\textbf{Feedback:}

\textbf{Part a) - 2/2}

\begin{itemize}
    \item The explanation is thorough and correctly explains that the eigenvalues of a Hermitian operator are real. The physical interpretation is clearly connected to the reality of measurable quantities.
\end{itemize}

\textbf{Part b) - 2/2}

\begin{itemize}
    \item The description of orthonormality is accurate, detailing both orthogonality and normalization with the correct mathematical expressions. The explanation links these properties to their significance in quantum mechanics.
\end{itemize}



Solution 2 :

\textbf{Marks: 4/4}

\textbf{Feedback:}

\textbf{Part a) - 2/2}

\begin{itemize}
    \item The answer correctly states that eigenvalues of Hermitian operators are real and explains the mathematical reasoning. The physical interpretation is also clearly mentioned.
\end{itemize}

\textbf{Part b) - 2/2}

\begin{itemize}
    \item The explanation of orthonormality is concise and accurate, using the Kronecker delta notation correctly. The significance in quantum mechanics is well articulated.
\end{itemize}



Solution 3 :

\textbf{Marks: 4/4}

\textbf{Feedback:}

\textbf{Part a) - 2/2}

\begin{itemize}
    \item The solution provides a correct and detailed explanation of why the eigenvalues of Hermitian operators are real. The physical interpretation is well addressed.
\end{itemize}

\textbf{Part b) - 2/2}

\begin{itemize}
    \item The explanation of orthonormality is accurate and clearly stated, with correct mathematical notation and a solid connection to its importance in quantum mechanics.
\end{itemize}


\textbf{Summary :}
All three solutions are well-written, accurate, and complete. They correctly address both the mathematical properties and the physical interpretations, and their explanations of orthonormality are clear and precise. Each solution earns full marks.


\subsection*{Question 3:}

\subsubsection*{Response 1:}

Solution 1 :

\textbf{Marks: 9/9}

\textbf{Feedback:}

\textbf{Part a) - 4/4}

\textbf{Part b) - 5/5}


\textbf{Feedback:}
Excellent solution. The student correctly applied the definition of Hermitian operators, used integration by parts effectively, and showed step-by-step that the operators $\hat{p}$ and $\hat{Q}$ are Hermitian. All steps are clearly explained, and the final conclusion is accurate.


Solution 2 :

\textbf{Marks: 7/9}

\textbf{Feedback:}

\textbf{Part a) - 3/4}


\textbf{Part b) - 4/5}


\textbf{Feedback:}
Good attempt overall. For part (a), the method is mostly correct, but the explanation is slightly less clear compared to Solution 1. The use of integration by parts is correct, but the final comparison of both sides could be more explicit. For part (b), the integration by parts approach is mentioned but not fully shown, which could lead to some confusion. However, the conclusion is correct.


Solution 3 :

\textbf{Marks: 8/9}

\textbf{Feedback:}

\textbf{Part a) - 4/4}


\textbf{Part b) - 4/5}


\textbf{Feedback:}
Well done. Part (a) is correctly and clearly worked through, matching Solution 1 in clarity and correctness. For part (b), the explanation is correct but skips some intermediate steps in the integration by parts, which could make it less clear to follow. The final conclusion is correct, and the student demonstrates a good understanding of the concept.


\textbf{Summary :}
\begin{itemize}
    \item Solution 1 is exemplary, demonstrating a thorough and clear understanding of the problem and its solution.
    \item Solution 2 is good but could benefit from more detailed explanations and clearer comparisons.
    \item Solution 3 is also very good, but the omission of some steps in part (b) slightly reduces clarity.
\end{itemize}



\subsubsection*{Response 2:}

Solution 1 :

\textbf{Marks: 9/9}

\textbf{Feedback:}

\textbf{Part a) - 4/4}

\begin{itemize}
    \item Correctly applies the definition of Hermitian operator.
    \item Integration by parts is performed accurately, showing clear understanding.
    \item Boundary terms are correctly identified and used.
\end{itemize}

\textbf{Part b) - 5/5}

\begin{itemize}
    \item Correctly applies the definition to the second derivative operator.
    \item Uses integration by parts twice appropriately.
    \item Clearly shows that boundary terms vanish and both sides are equivalent.
\end{itemize}

\textbf{Feedback:}
Excellent work. The solution is thorough and demonstrates a clear understanding of the Hermitian property through proper use of integration by parts and boundary conditions.


Solution 2 :

\textbf{Marks: 7/9}

\textbf{Feedback:}

\textbf{Part a) - 3/4}

\begin{itemize}
    \item Correctly applies the definition of Hermitian operator.
    \item The steps are correct, but the integration by parts is not shown explicitly.
    \item Lacks a bit of clarity in the final matching of LHS and RHS.
\end{itemize}

\textbf{Part b) - 4/5}

\begin{itemize}
    \item Correctly identifies the operator and starts with the correct integral.
    \item Mentioned integration by parts twice, but the explicit steps are missing.
    \item Shows the boundary terms vanish but not in full detail.
\end{itemize}

\textbf{Feedback:}
Good understanding of the concepts, but the explanation could benefit from more detailed steps, especially in the integration by parts.


Solution 3 :

\textbf{Marks: 8/9}

\textbf{Feedback:}

\textbf{Part a) - 4/4}

\begin{itemize}
    \item Correctly applies the definition of Hermitian operator.
    \item Integration by parts is performed accurately, showing clear understanding.
    \item Boundary terms are correctly identified and used.
\end{itemize}

\textbf{Part b) - 4/5}

\begin{itemize}
    \item Correctly applies the definition to the second derivative operator.
    \item Uses integration by parts appropriately but skips intermediate steps.
    \item Shows that boundary terms vanish and both sides are equivalent.
\end{itemize}

\textbf{General Feedback:}
Well-done overall. The solution is correct but could be improved by including more intermediate steps in part b) for clarity.

\textbf{Summary :}
Solution 1 is perfect and demonstrates an excellent understanding of the problem with detailed and clear steps. Solution 2 is good but could use more detail in the explanations. Solution 3 is also very good but needs to include intermediate steps for a more comprehensive answer.



\subsubsection*{Response 3:}

Solution 1 :

\textbf{Marks: 8/9}

\textbf{Feedback:}

\textbf{Part a) - 4/4}

\begin{itemize}
    \item Correctly applies the definition of Hermitian operator.
    \item Shows clear steps and calculations using integration by parts.
    \item Correctly concludes that$\hat{p} = -i \hbar \frac{d}{dx}$ is Hermitian.
\end{itemize}

\textbf{Part b) - 4/5}

\begin{itemize}
    \item Correctly applies the definition and follows appropriate steps.
    \item Uses integration by parts twice and appropriately handles boundary terms.
    \item Minor errors or omissions in the final simplification step.
\end{itemize}

\textbf{Feedback:}

\begin{itemize}
    \item Solution is well-structured and logically presented.
    \item Integration by parts is correctly used.
    \item Conclusion and verification are correct.
    \item Minor clarity improvement needed in the final steps of part b.
\end{itemize}


Solution 2 :

\textbf{Marks: 7/9}

\textbf{Feedback:}

\textbf{Part a) - 4/4}

\begin{itemize}
    \item Clearly uses the definition and follows correct steps.
    \item Integration by parts is appropriately applied.
    \item Correctly identifies $\hat{p}$ as Hermitian with proper boundary conditions.
\end{itemize}

\textbf{Part b) - 3/5}

\begin{itemize}
    \item Applies the definition correctly.
    \item Good attempt at using integration by parts, but lacks explicit detailed steps.
    \item Conclusion is correct but missing detailed verification of steps and simplifications.
\end{itemize}

\textbf{Feedback:}

\begin{itemize}
    \item Clear explanation for part a.
    \item Part b needs more detailed steps and explicit verification.
    \item Overall, logical approach but needs more thoroughness in part b.
\end{itemize}


Solution 3 :

\textbf{Marks: 7/9}

\textbf{Feedback:}

\textbf{Part a) - 4/4}

\begin{itemize}
    \item Correctly uses the definition and shows clear steps.
    \item Integration by parts is appropriately handled.
    \item Correct conclusion about $\hat{p}$.
\end{itemize}

\textbf{Part b) - 3/5}

\begin{itemize}
    \item Correctly identifies the operator to check.
    \item Good conceptual understanding but lacks detailed verification.
    \item Conclusion is correct but the solution is somewhat verbose and missing explicit integration steps.
\end{itemize}

\textbf{Feedback:}

\begin{itemize}
    \item Clear and logical explanation for part a.
    \item Part b is conceptually correct but needs more detailed calculations.
    \item Simplifications should be shown more explicitly.
\end{itemize}


\textbf{Summary :}
Overall, each student has a good grasp of the concepts but could benefit from more detailed explanations and verifications in part b.



\subsubsection*{Response 4:}

Solution 1 :

\textbf{Marks: 7/9}

\textbf{Feedback:}

\textbf{Part a) - 3/4}

\begin{itemize}
    \item Correctly sets up the problem and uses the integration by parts.
    \item Misses a few intermediate steps which could clarify the process.
    \item Final conclusion is correct but the presentation could be more streamlined.
\end{itemize}

\textbf{Part b) - 4/5}

\begin{itemize}
    \item Correctly identifies the need for integration by parts and applies it twice.
    \item Clearly shows boundary terms vanish and provides a correct conclusion.
    \item Explanation is detailed but slightly verbose.
\end{itemize}

\textbf{General Feedback:}

\begin{itemize}
    \item The solution is mostly correct and well-structured.
    \item Some steps, particularly in part a, could be more explicit for clarity.
    \item Reducing verbosity while maintaining completeness would improve readability.
\end{itemize}


Solution 2 :

\textbf{Marks: 7/9}

\textbf{Feedback:}

\textbf{Part a) - 3/4}

\begin{itemize}
    \item Correctly applies the definition of Hermitian operator.
    \item Correctly uses integration by parts but lacks some intermediate steps for clarity.
    \item Final conclusion is correct but could be more concise.
\end{itemize}

\textbf{Part b) - 4/5}

\begin{itemize}
    \item Correctly applies integration by parts and explains the process well.
    \item Properly concludes the Hermitian nature of the operator.
    \item Slightly verbose, but detailed and correct.
\end{itemize}

\textbf{General Feedback:}

\begin{itemize}
    \item Correct overall and fairly detailed.
    \item Providing more intermediate steps in part a would help in clarity.
    \item Reducing verbosity would make the solution more concise.
\end{itemize}


Solution 3 :

\textbf{Marks: 7/9}

\textbf{Feedback:}

\textbf{Part a) - 3/4}

\begin{itemize}
    \item Correctly sets up the problem and applies the definition.
    \item Uses integration by parts but skips some steps which could help in understanding.
    \item Conclusion is correct but lacks intermediate steps for clarity.
\end{itemize}

\textbf{Part b) - 4/5}

\begin{itemize}
    \item Correctly uses integration by parts twice and explains the steps.
    \item Conclusion is correct and the boundary terms are well handled.
    \item Some intermediate steps skipped, but the logic is clear.
\end{itemize}

\textbf{General Feedback:}

\begin{itemize}
    \item Solution is correct but skipping intermediate steps reduces clarity.
    \item Providing all steps would help in understanding the process.
    \item Overall, well-structured but could benefit from more detail in steps.
\end{itemize}

\textbf{Summary :}
All three solutions demonstrate a solid understanding of the Hermitian operators and correctly apply the definition to the given problems. The main areas for improvement are in providing more intermediate steps for clarity and reducing verbosity to make the explanations more concise. All solutions scored 7/9, indicating good but not perfect answers.



\subsubsection*{Response 5:}

Solution 1 :

\textbf{Marks: 8/9}

\textbf{Feedback:}

\textbf{Part a) - 3/4}

\begin{itemize}
    \item Correctly applies the definition of a Hermitian operator and uses integration by parts effectively. However, there is a minor sign error in the final step.
\end{itemize}

\textbf{Part b) - 5/5}

\begin{itemize}
    \item Correctly applies the definition and the integration by parts method. The explanation is clear and thorough, leading to the correct conclusion.
\end{itemize}

\textbf{General Feedback:}

\begin{itemize}
    \item A well-detailed solution with correct application of the mathematical techniques required. Minor sign error in part a but overall strong understanding.
\end{itemize}


Solution 2 :

\textbf{Marks: 6/9}

\textbf{Feedback:}

\textbf{Part a) - 2/4}

\begin{itemize}
    \item Correctly sets up the problem and uses integration by parts. However, it misses the explicit step where the boundary term vanishes, leading to a less clear argument.
\end{itemize}

\textbf{Part b) - 4/5}

\begin{itemize}
    \item Provides a correct approach but is somewhat repetitive and lacks detailed intermediate steps, especially the explicit use of integration by parts.
\end{itemize}

\textbf{General Feedback:}

\begin{itemize}
    \item Good understanding of the concepts but lacks some details and clarity, especially in the integration by parts steps. Explicitly showing the boundary term going to zero would strengthen the argument.
\end{itemize}


Solution 3 :

\textbf{Marks: 8/9}

\textbf{Feedback:}

\textbf{Part a) - 4/4}

\begin{itemize}
    \item Clearly and correctly sets up the problem, uses integration by parts properly, and shows the boundary terms vanish. Well-explained steps.
\end{itemize}

\textbf{Part b) - 4/5}

\begin{itemize}
    \item Correctly applies the definition and uses integration by parts. The explanation is clear, but skipping intermediate steps reduces clarity.
\end{itemize}

\textbf{General Feedback:}

\begin{itemize}
    \item Clear and correct approach with strong understanding demonstrated. Including all intermediate steps would improve clarity, especially in part b.
\end{itemize}


%27/7/24 JI
\subsection*{Question 4:}

\subsubsection*{Response 1:}

Solution 1 :

\textbf{Marks: 17/20}

\textbf{Feedback:}

\textbf{Part a) - (2/3): Correctly identifies the commutator is zero but the explanation is somewhat verbose and lacks clarity in the final simplification.}

\textbf{Part b) - (3/3): Correctly identifies the commutator is zero with a clear explanation.}

\textbf{Part c) - (3/4): Identifies the non-zero term correctly but misses details in the intermediate steps.}

\textbf{Part d) - (4/4): Provides a clear and correct derivation.}

\textbf{Part e) - (5/6): The explanation is correct but could be more concise regarding why each term commutes.}


\textbf{Feedback:}
Good understanding of the problem with mostly clear derivations. Be concise and detailed in explanations to avoid confusion. Intermediate steps, especially for non-zero commutators, could be clearer.


Solution 2 :

\textbf{Marks: 15/20}

\textbf{Feedback:}

\textbf{Part a) - (2/3): Correctly identifies the commutator is zero but lacks explanation of intermediate steps.}

\textbf{Part b) -  (3/3): Correctly identifies the commutator is zero with a concise explanation.}

\textbf{Part c) - (3/4): Correctly identifies the non-zero term but misses out on explaining the intermediate commutation relations.}

\textbf{Part d) - (3/4): Correctly identifies the non-zero commutator but lacks a detailed explanation.}

\textbf{Part e) - (4/6): Correctly identifies the commutator as zero but the explanation could be more detailed and clear.}


\textbf{Feedback:}
Solid grasp of commutator relations but explanations need more detail to fully show understanding. Ensure intermediate steps are thoroughly explained, especially for non-zero commutators.


Solution 3 :

\textbf{Marks: 18/20}

\textbf{Feedback:}

\textbf{Part a) - (3/3): Correctly identifies the commutator is zero with a clear explanation.}

\textbf{Part b) - (3/3): Correctly identifies the commutator is zero with a concise explanation.}

\textbf{Part c) - (4/4): Correctly identifies and explains the non-zero commutator.}

\textbf{Part d) - (4/4): Provides a clear and correct derivation with detailed explanation.}

\textbf{Part e) - (4/6): Correctly identifies the commutator as zero but could provide more detail on the reasoning for each component.}


\textbf{Feedback:}
Excellent understanding and clear, detailed explanations. The solution demonstrates a strong grasp of the material. More detail on the reasoning behind why each term commutes in part e) would improve the explanation.


\subsubsection*{Response 2:}

Solution 1 :

\textbf{Marks: 17/20}

\textbf{Feedback:}

\textbf{Part a) - 3/3 (Correct derivation)}

\textbf{Part b) - 3/3 (Correct derivation)}

\textbf{Part c) - 3/4 (Correct final result, but insufficient intermediate steps)}

\textbf{Part d) - 3/4 (Correct final result, but insufficient intermediate steps)}

\textbf{Part e) - 5/6 (Correct logic, but lacks clarity in intermediate steps)}


\textbf{General Feedback:}
This solution is well-structured and mostly correct. The main issue lies in the lack of detailed intermediate steps for parts c) and d). More explicit calculations would improve clarity and completeness. The final results are correct, indicating a good understanding of the principles involved.


Solution 2 :

\textbf{Marks: 18/20}

\textbf{Feedback:}

\textbf{Part a) - 3/3 (Correct derivation)}

\textbf{Part b) - 3/3 (Correct derivation)}

\textbf{Part c) - 4/4 (Correct and clear derivation)}

\textbf{Part d) - 3/4 (Correct final result, but insufficient explanation for the steps)}

\textbf{Part e) - 5/6 (Correct logic, but lacks detailed explanation)}


\textbf{General Feedback:}
This solution demonstrates a strong understanding of the commutator relations. The explanations are clear for most parts, but part d) would benefit from more detailed steps to improve understanding. The derivation in part c) is particularly well done, showing the student's grasp of the commutation relations.

Solution 3 :

\textbf{Marks: 19/20}

\textbf{Feedback:}

\textbf{Part a) -  3/3 (Correct derivation)}

\textbf{Part b) - 3/3 (Correct derivation)}

\textbf{Part c) - 4/4 (Correct and clear derivation)}

\textbf{Part d) - 4/4 (Correct and clear derivation)}

\textbf{Part e) - 5/6 (Correct result, but missing detailed explanation)}


\textbf{General Feedback:}
This solution is comprehensive and well-explained, especially in parts c) and d), where the steps are clearly outlined. Part e) would benefit from a more detailed explanation, but overall, this solution shows a solid understanding of the commutation relations and their derivations.

\textbf{Summary :}
Each solution displays a good understanding of the commutator relations, with Solution 3 providing the most detailed and clear explanations overall.


\subsubsection*{Response 3:}

Solution 1 :

\textbf{Marks: 18/20}

\textbf{Feedback:}

\textbf{Part a) - 3/3 Correctly identifies the commutator and justifies why it is zero.}

\textbf{Part b) - 3/3 Correctly identifies the commutator and justifies why it is zero.}

\textbf{Part c) - 3/4 Correctly derives the commutator but misses a detailed step in explaining the non-zero term.}

\textbf{Part d) - 4/4 Correctly derives the commutator and provides a clear explanation.}

\textbf{Part e) - 5/6 Correct approach and result, but explanation could be clearer regarding why each term commutes.}


\textbf{General Feedback:}
Very good solution overall. Detailed steps are clear, but some explanations could be more explicit, particularly in parts (c) and (e).


Solution 2 :

\textbf{Marks: 17/20}

\textbf{Feedback:}

\textbf{Part a) - 3/3 Correctly identifies the commutator and justifies why it is zero.}

\textbf{Part b) - 3/3 Correctly identifies the commutator and justifies why it is zero.}

\textbf{Part c) - 3/4 Correct result but missing a more detailed explanation of the intermediate steps.
}

\textbf{Part d) - 3/4 Correct result but explanation lacks detail in intermediate steps.}

\textbf{Part e) - 5/6 Correct approach and result, but explanation could be clearer on why each term commutes.}


\textbf{General Feedback:}
Good understanding and correct results, but intermediate steps and detailed explanations are missing in parts (c), (d), and (e).


Solution 3 :

\textbf{Marks: 19/20}

\textbf{Feedback:}

\textbf{Part a) - 3/3 Correctly identifies the commutator and justifies why it is zero.
}

\textbf{Part b) - 3/3 Correctly identifies the commutator and justifies why it is zero.}

\textbf{Part c) - 3/4 Correct result but lacks some detail in the explanation.}

\textbf{Part d) - 4/4 Correctly derives the commutator and provides a clear explanation.}

\textbf{Part e) - 6/6 Correct approach, detailed explanation, and clear reasoning why each term commutes.}


\textbf{General Feedback:}
Excellent work overall. Clear explanations and correct results, with minor detail missing in part (c).


\textbf{General Feedback for all solutions:}
\begin{itemize}
    \item All solutions demonstrate a strong understanding of the problem and correct application of commutator properties.
    \item Detailed step-by-step derivations are crucial for full marks, especially in parts where the commutator is not immediately intuitive.
    \item Keep improving clarity in explanations, especially in intermediate steps, to ensure all logical steps are thoroughly communicated.
\end{itemize}



\subsubsection*{Response 4:}

Solution 1 :

\textbf{Marks: 17/20}

\textbf{Feedback:}

\textbf{Part a) - (3/3): Correct derivation using the commutation properties of operators.}

\textbf{Part b) - (3/3): Correct explanation and result using similar logic as part (a).}

\textbf{Part c) - (3/4): The derivation correctly identifies the non-zero term, but there’s a slight clarity issue in the explanation.}

\textbf{Part d) - (3/4): The derivation is correct but lacks detail in showing how the commutator leads to the final result.}

\textbf{Part e) - (5/6): Correctly identifies the commutation of $\hat{L_x}$ with $\hat{P^2}$, but the explanation could be clearer on the individual components.}


\textbf{Feedback:}
A strong solution with accurate results. Minor improvements in clarity and detail would make the solution more robust.


Solution 2 :

\textbf{Marks: 15/20}

\textbf{Feedback:}

\textbf{Part a) - (3/3):  Correct application of the commutation properties.}

\textbf{Part b) -  (3/3): Correct explanation and result.}

\textbf{Part c) - (2/4): Correct method, but a mistake in simplifying the commutator.}

\textbf{Part d) - (2/4): Correct approach, but a similar mistake as in (c) when simplifying.}

\textbf{Part e) - (5/6): Correct result but lacks sufficient detail in the explanation.}


\textbf{Feedback:}
Good understanding of the concepts, but minor errors in simplifying commutators affected the final results. More detailed explanations would strengthen the solution.


Solution 3 :

\textbf{Marks: 18/20}

\textbf{Feedback:}

\textbf{Part a) - (3/3): Correct approach and result.}

\textbf{Part b) - (3/3): Correct method and result.}

\textbf{Part c) - (3/4): Correct identification of the commutator term, but lacks clarity.}

\textbf{Part d) - (3/4): Correct approach, but the final explanation could be clearer.}

\textbf{Part e) - (6/6): Correct identification of the commutator with a good explanation.}


\textbf{Feedback:}
Well-detailed and mostly accurate solutions. Some parts could benefit from clearer explanations, but overall a strong understanding is demonstrated.


\subsubsection*{Response 5:}

Solution 1 :

\textbf{Marks: 20/20}

\textbf{Feedback:}

\textbf{Part a) - (3/3):  Correctly identified that $\hat{X}$ commutes with $\hat{Y},\hat{Z}, \hat{P_y}$ and $\hat{P_z}$.}

\textbf{Part b) - (3/3): Correctly noted that $\hat{P_x}$ commutes with $\hat{Y},\hat{Z}, \hat{P_y}$ and $\hat{P_z}$.}

\textbf{Part c) - (4/4): Properly derived using commutation relations and included necessary steps.}

\textbf{Part d) - (4/4): Clearly derived the relation by using commutation properties.}

\textbf{Part e) - (6/6): Thoroughly explained how the operator commutes with each term of $\hat{P^2}$}


\textbf{Feedback:}
Excellent solution with clear and precise steps. Each part is derived correctly with proper justification.


Solution 2 :

\textbf{Marks: 19/20}

\textbf{Feedback:}

\textbf{Part a) - (3/3): Correctly identified the commutation.}

\textbf{Part b) -  (3/3): Properly noted the commutation relation.}

\textbf{Part c) - (3/4): Derivation is mostly correct but lacks detail in explaining the non-zero term explicitly.}

\textbf{Part d) - (4/4): Correctly derived the relation.}

\textbf{Part e) - (6/6): Clearly explained the commutation with each term.}


\textbf{Feedback:}
Well-done solution with clear logic. Adding a bit more detail in part c) would make it perfect.



Solution 3 :

\textbf{Marks: 19/20}

\textbf{Feedback:}

\textbf{Part a) - (3/3): Correctly noted the commutation.}

\textbf{Part b) - (3/3): Properly derived.}

\textbf{Part c) - (4/4): Clearly explained the commutation and non-zero term.}

\textbf{Part d) - (3/4): Mostly correct but lacks detail in deriving the non-zero term.}

\textbf{Part e) - (6/6): Properly explained the commutation with each term.}


\textbf{Feedback:}
Very good solution with correct steps. More detailed explanations in parts c) and d) would enhance clarity.



\subsection*{Question 5:}

\subsubsection*{Response 1 :}

Solution 1:


\textbf{ Marks : 6/7}

\textbf{Feedback :}

\textbf{Part a: 3/4}

\begin{itemize}
    \item The student correctly identifies the perturbation and sets up the integral for the first-order correction. The use of trigonometric identities is appropriate. However, the detailed steps to show that the integral evaluates to zero are not provided, which would have been useful to show complete understanding.
\end{itemize}

\textbf{Part b: 3/3}
\begin{itemize}
    \item The calculation is correct and the reasoning is clearly explained. The final expression for the energy correction is accurate.
\end{itemize}



Solution 2:


\textbf{ Marks : 6/7}

\textbf{Feedback :}

\textbf{Part a: 4/4}

\begin{itemize}
    \item The solution is well-presented with correct setup and simplification steps. The integral evaluation is detailed and leads to the correct conclusion.
\end{itemize}

\textbf{Part b: 2/3}
\begin{itemize}
    \item The integral setup is correct, but the final expression should simplify to $2\alpha sin^2(\frac{n\pi}{2}$), which is not shown explicitly. The explanation needs to be more detailed in showing how the delta function affects the eigenvalues.
\end{itemize}



Solution 3:


\textbf{ Marks : 7/7}

\textbf{Feedback :}

\textbf{Part a: 4/4}

\begin{itemize}
    \item This solution is clear and thorough, with a correct approach to the integral and use of trigonometric identities. The explanation for the integral simplifying to zero is complete.
\end{itemize}

\textbf{Part b: 3/3}
\begin{itemize}
    \item The solution correctly applies the delta function and arrives at the correct expression for the energy correction. The step-by-step reasoning is well-explained and leads to the correct final expression.
\end{itemize}

\textbf{General feedback :}

\begin{itemize}
    \item This solution demonstrates a clear understanding of the theory and applies it well. However, there is an error in part a, resulting in the incorrect conclusion for the first-order correction. Part b is accurately solved
\end{itemize}


\subsubsection*{Response 2 :}


Solution 1:


\textbf{ Marks : 6/7}

\textbf{Feedback :}

\textbf{Part a: 3/3}

\begin{itemize}
    \item Correctly identified the formula for first-order correction and used the ground state eigenfunction.
    \item The setup of the integral was mostly correct, but there was some unnecessary complexity and a slight confusion in trigonometric identities.
    \item Correctly identified that the correction is zero, but the final simplification was not shown clearly.
\end{itemize}

\textbf{General feedback :}
Good understanding of the process, but clarity in the final simplification and integral evaluation would improve the solution.


\textbf{Part b: 3/3}
\begin{itemize}
    \item Correctly identified the formula for first-order correction.
    \item Correctly handled the Dirac delta function and provided the correct expression for the correction.
\end{itemize}

\textbf{General feedback :}
Well done on handling the Dirac delta function and deriving the correct expression.



Solution 2:

\textbf{ Marks : 7/7}

\textbf{Feedback :}

\textbf{Part a: 4/4}

\begin{itemize}
    \item Correctly identified the formula and the integral setup for the first-order correction.
    \item Simplified and evaluated the integral correctly, demonstrating that the correction is zero.
\end{itemize}

\textbf{General Feedback :}
Clear and correct solution. Well done on the use of trigonometric identities and integration.


\textbf{Part b: 3/3}
\begin{itemize}
    \item Correctly identified the formula for first-order correction.
    \item Handled the Dirac delta function correctly and derived the correct expression.
\end{itemize}

\textbf{General feedback :}
Excellent work on both parts. Clear and correct throughout.



Solution 3:


\textbf{ Marks : 7/7}

\textbf{Feedback :}

\textbf{Part a: 4/4}

\begin{itemize}
    \item Correctly identified the formula and integral setup for the first-order correction.
    \item Used trigonometric identities correctly and showed the integral evaluation clearly, demonstrating that the correction is zero.
\end{itemize}

\textbf{General Feedback :}
Clear and precise solution. Excellent work on the integral simplification and evaluation.

\textbf{Part b: 3/3}
\begin{itemize}
    \item Correctly identified the formula for first-order correction.
    \item Handled the Dirac delta function correctly and derived the correct expression.
\end{itemize}

\textbf{General feedback :}
Clear and correct throughout. Well done on handling the Dirac delta function and deriving the correct expression.



\subsubsection*{Response 3 :}


Solution 1:


\textbf{ Marks : 5/7}

\textbf{Feedback :}

\textbf{Part a: 2/4}

\begin{itemize}
    \item The student correctly identifies the need to use first-order perturbation theory and sets up the integral correctly. However, the evaluation of the integral is incomplete. The trigonometric identities are introduced but not applied correctly, leading to a missing conclusion of the integral which should show that it equals zero explicitly.
\end{itemize}

\textbf{Part b: 3/3}
\begin{itemize}
    \item The student correctly applies the perturbation theory for a delta function perturbation and simplifies the integral properly. The use of the Dirac delta function is accurate, leading to the correct expression for the first-order energy correction.
\end{itemize}



Solution 2:


\textbf{ Marks : 6/7}

\textbf{Feedback :}

\textbf{Part a: 3/4}

\begin{itemize}
    \item This solution is clear and well-organized. The student correctly uses the first-order perturbation theory and sets up the integral accurately. They recognize that the integral evaluates to zero, but more detailed steps or a clear concluding statement showing the zero result explicitly would have been better.
\end{itemize}

\textbf{Part b: 3/3}
\begin{itemize}
    \item The student accurately handles the delta function perturbation, using the properties of the Dirac delta function correctly. The derived expression for the first-order correction to the energy eigenvalues is correct and well-explained.
\end{itemize}



Solution 3:


\textbf{ Marks : 6/7}

\textbf{Feedback :}

\textbf{Part a: 3/4}

\begin{itemize}
    \item This solution is thorough and demonstrates a solid understanding of the perturbation theory. The setup and simplification are correct, but the integral's zero result should be explicitly shown. The explanation is good, though slightly more detailed than necessary for a concise solution.
\end{itemize}

\textbf{Part b: 3/3}
\begin{itemize}
    \item The student effectively handles the delta function perturbation. The correct use of the Dirac delta function and the derivation of the first-order energy correction expression are clear and correct.
\end{itemize}

\textbf{Summary :}
All three solutions demonstrate a good understanding of the quantum mechanics perturbation theory. The main difference in the scoring for part a) comes from the explicitness and completeness of the integration steps. All students performed well in part b), correctly deriving the first-order correction due to the delta function perturbation.



\subsubsection*{Response 4 :}

Solution 1:


\textbf{ Marks : 4/7}

\textbf{Feedback :}

\textbf{Part a: 2/4}

\begin{itemize}
    \item Explanation and setup of the integral are correct. (2/4)
    \item Correctly identifies that the integral evaluates to zero, but the simplification process contains unnecessary steps and does not clearly lead to the conclusion. (1/4)
\end{itemize}

\textbf{Part b: 2/3}
\begin{itemize}
    \item Explanation and setup are correct and clear. (2/3)
    \item Correctly applies the delta function property and simplifies the expression accurately. (1/3)
\end{itemize}

\textbf{General feedback :}
Good understanding of the problem and correct final results, but some steps are overcomplicated, especially in part a). Streamlining the solution and clarifying the intermediate steps would improve the answer.


Solution 2:

\textbf{ Marks : 4/7}

\textbf{Feedback :}

\textbf{Part a: 2/4}

\begin{itemize}
    \item Correct approach and setup for the integral. (2/4)
    \item Correctly concludes the integral is zero, but the integration process could be more detailed and clarified. (1/4)
\end{itemize}

\textbf{Part b: 2/3}
\begin{itemize}
    \item Correctly uses the delta function property and sets up the integral properly. (2/3)
    \item Accurate simplification and final expression. (1/3)
\end{itemize}

\textbf{General feedback :}
The solution is generally correct and concise. However, the explanation of intermediate steps, especially in part a), needs more clarity and detail to demonstrate understanding fully.



Solution 3:


\textbf{ Marks : 4/7}

\textbf{Feedback :}

\textbf{Part a: 2/4}

\begin{itemize}
    \item Correct approach and use of perturbation theory. (2/4)
    \item Correctly concludes the integral is zero but lacks detailed explanation of the trigonometric simplifications. (1/4)
\end{itemize}

\textbf{Part b: 2/3}
\begin{itemize}
    \item Proper use of the delta function property and clear integral setup. (2/3)
    \item Correct final expression but could benefit from a clearer derivation process. (1/3)
\end{itemize}

\textbf{General feedback :}
The solution is generally accurate but needs more detailed explanation and clarity in the integration and simplification steps, particularly in part a). Adding more detailed steps would make the solution more comprehensible.


\subsubsection*{Response 5 :}

Solution 1:


\textbf{ Marks : 6/7}

\textbf{Feedback :}

\textbf{Part a: 3/4}

\begin{itemize}
    \item Correctly identifies the formula for first-order correction.
    \item Sets up the integral correctly but overcomplicates the simplification with unnecessary trigonometric identities.
    \item Correct conclusion that the first-order correction is zero.
\end{itemize}

\textbf{Part b: 3/3}
\begin{itemize}
    \item Correctly applies the formula for first-order correction.
    \item Correctly identifies the value at $x = \frac{L}{2}$ and simplifies it to the right expression.
\end{itemize}

\textbf{General feedback :}
Good understanding and application of perturbation theory. However, the solution could be more concise by avoiding unnecessary intermediate steps in part a).


Solution 2:


\textbf{ Marks : 7/7}

\textbf{Feedback :}

\textbf{Part a: 4/4}

\begin{itemize}
    \item Correct identification of the formula for first-order correction.
    \item Sets up and simplifies the integral correctly, concluding that the first-order correction is zero.
    \item Less overcomplicated than Solution 1.
\end{itemize}

\textbf{Part b: 3/3}
\begin{itemize}
    \item Correct application of the formula for first-order correction.
    \item Correctly uses the Dirac delta function properties and simplifies to the right expression.
\end{itemize}

\textbf{General feedback :}
Clear and accurate solution, demonstrating a solid understanding of perturbation theory and concise mathematical manipulation.



Solution 3:


\textbf{ Marks : 6/7}

\textbf{Feedback :}

\textbf{Part a: 3/4}

\begin{itemize}
    \item Correctly identifies the formula for first-order correction.
    \item Correct setup of the integral but includes unnecessary trigonometric identity applications.
    \item Correct conclusion that the first-order correction is zero.
\end{itemize}

\textbf{Part b: 3/3}
\begin{itemize}
    \item Correct application of the formula for first-order correction.
    \item Proper use of the Dirac delta function properties and simplifies to the correct expression.
\end{itemize}

\textbf{General feedback :}
Good overall understanding and application of perturbation theory. However, part a) includes some unnecessary steps that could be simplified.

\textbf{Summary :}
Overall, Solutions 2 provided the clearest and most concise application of perturbation theory, earning a full score. Solutions 1 and 3 demonstrated good understanding but included some unnecessary complexity, costing them a point each in part a).



\subsection*{Question 6:}

\subsubsection*{Response 1:}

Solution 1 :

\textbf{Marks : 12/17}

\textbf{Part a: 2/4}
\begin{itemize}
    \item Correctly starts by expanding $\hat{J^2}$.
    \item Expresses the cross terms in terms of ladder operators but does not simplify them correctly.
\end{itemize}

\textbf{Part b : 3/4}
\begin{itemize}
    \item Correct explanation about eigenvalues and eigenstates.
    \item Calculation of eigenvalue is correct but lacks the explicit demonstration of why the terms $\hat{L_+}\hat{S_-}$ and $\hat{L_-}\hat{S_+}$ are zero.
\end{itemize}

\textbf{Part c : 7/9}
\begin{itemize}
    \item Correct basis states and appropriate vector representation.
    \item Correct calculation of $\hat{L^2}$ and the construction of the matrix.
    \item The transition matrix for $\hat{S_+}\hat{L_z}$ lacks explicit computation for each basis ket.
    \item Correct final matrix construction but without detailed intermediate steps.
\end{itemize}

\textbf{Feedback:}
Good understanding of the concepts but needs more detail in intermediate steps, especially in parts a) and c).


Solution 2 :

\textbf{Marks : 13/17}

\textbf{Part a: 4/4}
\begin{itemize}
    \item Correctly expands $\hat{J^2}$ and simplifies using ladder operators.
\end{itemize}

\textbf{Part b : 3/4}
\begin{itemize}
    \item Correctly explains eigenvalues and eigenstates but the detailed demonstration of why the ladder operator terms vanish is lacking.
\end{itemize}

\textbf{Part c : 6/9}
\begin{itemize}
    \item Correct basis states and appropriate vector representation.
    \item Lacks intermediate steps for calculating the matrix elements of $\hat{S_+}\hat{L_z}$.
    \item The final matrix construction is incomplete and lacks clear transitions.
\end{itemize}

\textbf{General feedback:}
Clear understanding but more detailed intermediate calculations are needed, especially in part c).


Solution 3 :

\textbf{Marks : 11/17}

\textbf{Part a: 2/4}
\begin{itemize}
    \item Correctly starts with the expansion but does not complete the simplification using ladder operators properly.
\end{itemize}

\textbf{Part b : 3/4}
\begin{itemize}
    \item Good explanation of eigenstates and eigenvalues but lacks explicit verification for the ladder operator terms.
\end{itemize}

\textbf{Part c : 6/9}
\begin{itemize}
    \item Correctly identifies basis states and vector representation.
    \item Provides a structure for the matrix but lacks detailed steps and complete transitions.
\end{itemize}

\textbf{General feedback:}
Good overall approach but needs more complete and detailed calculations, especially for parts a) and c).


\subsubsection*{Response 2:}

Solution 1 :

\textbf{Marks : 17/17}

\textbf{Part a: 4/4}
\begin{itemize}
    \item Correct initial expression for$\hat{J^2}$ : 1 mark
    \item Proper use of ladder operator identities: 2 marks
    \item Correct final expression: 1 mark
\end{itemize}

\textbf{Part b : 4/4}
\begin{itemize}
    \item Correctly applies operator expression: 2 marks
    \item Correct eigenvalue calculation: 2 marks
\end{itemize}

\textbf{Part c : 9/9}
\begin{itemize}
    \item Identifies 6 basis states: 1 mark
    \item Correct vector representation for each basis ket: 1 mark
    \item Correct matrix representation of $\hat{L^2}$ : 2 marks
    \item Correctly calculates $\hat{S_+}\hat{L_z}$ and matrix representation: 4 marks
    \item Constructs matrix representation of $\hat{O}$: 1 mark
\end{itemize}

\textbf{General feedback:}
Excellent solution. The student demonstrates a strong understanding of quantum mechanics and matrix representation, with thorough and accurate steps.


Solution 2 :

\textbf{Marks : 13/17}

\textbf{Part a: 3/4}
\begin{itemize}
    \item Correct initial expression for $\hat{J^2}$: 1 mark
    \item Proper use of ladder operator identities, though slightly unclear: 1 mark
    \item Correct final expression: 1 mark   
\end{itemize}

\textbf{Part b : 2/4}
\begin{itemize}
    \item Attempts to apply operator expression but lacks clarity: 1 mark
    \item Correct eigenvalue calculation but not well-explained: 1 mark
\end{itemize}

\textbf{Part c : 8/9}
\begin{itemize}
    \item Identifies 6 basis states: 1 mark
    \item Correct vector representation for each basis ket: 1 mark
    \item Correct matrix representation of $\hat{L^2}$: 2 marks
    \item Correct calculation for $\hat{S_+}\hat{L_z}$, but matrix representation unclear: 3 marks
    \item Constructs matrix representation of $\hat{O}$ but with some confusion: 1 mark
\end{itemize}

\textbf{General feedback:}
Good effort with accurate calculations but lacking in clarity and thorough explanation. Needs to ensure each step is clearly justified.


Solution 3 :

\textbf{Marks : 14/17}

\textbf{Part a: 4/4}
\begin{itemize}
    \item Correct initial expression for $\hat{J^2}$: 1 mark
    \item Proper use of ladder operator identities: 2 marks
    \item Correct final expression: 1 mark
\end{itemize}

\textbf{Part b : 3/4}
\begin{itemize}
    \item Applies operator expression correctly but briefly: 2 marks
    \item Correct eigenvalue calculation but not well-explained: 1 mark
\end{itemize}

\textbf{Part c : 7/9}
\begin{itemize}
    \item Identifies 6 basis states: 1 mark
    \item Correct vector representation for each basis ket: 1 mark
    \item Correct matrix representation of $\hat{L^2}$: 2 marks
    \item Miscalculation in $\hat{S_+}\hat{L_z}$ leading to incorrect matrix elements: 2 marks
    \item Constructs matrix representation of $\hat{O}$ but with some errors: 1 mark
\end{itemize}

\textbf{General feedback:}
Solid understanding and correct approach for the most part but errors in detailed calculations and matrix construction. Needs to double-check work for accuracy.




\subsubsection*{Response 3:}

Solution 1 :

\textbf{Marks : 13/17}

\textbf{Part a: 3/4}
\begin{itemize}
    \item Correctly identified the initial squaring of $\hat{J}$ and expanded the dot product.
    \item Properly used the ladder operators for $\hat{L_x}, \hat{L_y}, \hat{S_x}$ and $\hat{S_y}$.
    \item Simplified to the final expression correctly but could have explicitly shown more intermediate steps.
\end{itemize}

\textbf{Feedback :}
Good application of ladder operators and the dot product expansion. More intermediate steps would provide better clarity.


\textbf{Part b : 4/4}
\begin{itemize}
    \item Properly identified the action of $\hat{J_2}$ on the given state.
    \item Clearly explained why the ladder operators' contributions are zero.
    \item Correctly derived the eigenvalue.
\end{itemize}

\textbf{Feedback:}
Excellent step-by-step explanation and correct final answer.


\textbf{Part c : 6/9}
\begin{itemize}
    \item Correctly identified the basis states and vector representation.
    \item Properly calculated $\hat{L^2}$ acting on basis kets.
    \item Partial calculation of $\hat{S_+}\hat{L_z}$ but missing some explicit matrix elements and transitions.
\end{itemize}

\textbf{Feedback:}
Good identification of the basis and correct $\hat{L^2}$ matrix. Complete calculations for $\hat{S_+}\hat{L_z}$ are needed for full credit.


Solution 2 :

\textbf{Marks : 15/17}

\textbf{Part a: 4/4}
\begin{itemize}
    \item Correctly identified the initial squaring of $\hat{J}$ and expanded the dot product.
    \item Used ladder operators and correctly simplified the expression.
    \item Clear and accurate steps leading to the final expression.
\end{itemize}

\textbf{Feedback :}
Comprehensive and accurate solution. Well-explained and correct final answer.


\textbf{Part b : 4/4}
\begin{itemize}
    \item Properly derived the action of $\hat{J^2}$ on the given state.
    \item Correctly identified why certain operators' contributions are zero.
    \item Clear calculation of the eigenvalue.
\end{itemize}

\textbf{Feedback :}
Clear and precise explanation. Correct final answer.


\textbf{Part c : 7/9}
\begin{itemize}
    \item Correctly identified the basis states.
    \item Proper calculation of $\hat{L^2}$ acting on the basis kets.
    \item Partial calculation for $\hat{S_+}\hat{L_z}$ with some explicit transitions but missing a couple of elements for completeness.
\end{itemize}

\textbf{Feedback:}
Good identification and calculation. Include all transitions for completeness.



Solution 3 :

\textbf{Marks : 12/17}

\textbf{Part a: 3/4}
\begin{itemize}
    \item Correctly identified the initial squaring of $\hat{J}$ and expanded the dot product.
    \item Used ladder operators but missed some simplifications in intermediate steps.
    \item Correct final answer but with less clarity in steps.
\end{itemize}

\textbf{Feedback :}
Good use of ladder operators. More detailed steps would improve clarity.


\textbf{Part b : 3/4}
\begin{itemize}
    \item Correctly identified the action of $\hat{J^2}$ on the given state.
    \item Explained why certain operators' contributions are zero but missed some intermediate steps.
    \item Correct final eigenvalue but less detailed explanation.
\end{itemize}

\textbf{Feedback :}
Correct identification and calculation. More detailed steps are needed for full clarity.


\textbf{Part c : 6/9}
\begin{itemize}
    \item Correctly identified the basis states.
    \item Proper calculation of $\hat{L^2}$ acting on the basis kets.
    \item Partial calculation for $\hat{S_+}\hat{L_z}$ but missing some explicit transitions and elements.
\end{itemize}

\textbf{Feedback:}
Good identification and partial calculation. Complete the transitions and elements for full credit.


\subsubsection*{Response 4:}

Solution 1 :

\textbf{Marks : 17/17}

\textbf{Part a: 4/4}
\begin{itemize}
    \item Correct identification of the terms involved and use of vector operators: 1 mark
    \item Proper substitution of $\hat{L_x}$ and $\hat{L_y}$ expressions: 2 marks
    \item Correct final answer: 1 mark
\end{itemize}


\textbf{Part b : 4/4}
\begin{itemize}
    \item Correct application of the operator expression and noting states composing the tensor product: 2 marks
    \item Correct calculation of eigenvalues and showing it is an eigenstate of $\hat{J^2}$: 2 marks
\end{itemize}


\textbf{Part c : 9/9}
\begin{itemize}
    \item Correct identification of basis states: 1 mark
    \item Correct vector representation of each basis ket: 1 mark
    \item Calculation of $\hat{L^2}$ acting on basis kets: 2 marks
    \item Calculation of $\hat{S_+}\hat{L_z}$ acting on basis kets: 4 marks
    \item Construction of the matrix representation of $\hat{O}$: 1 mark
\end{itemize}

\textbf{Feedback:}
Excellent solution. Clear and methodical approach, demonstrating thorough understanding of the problem and accurate calculations.


Solution 2 :

\textbf{Marks : 11/17}

\textbf{Part a: 3/4}
\begin{itemize}
    \item Correct identification of the terms involved: 1 mark
    \item Missing explicit substitution of $\hat{L_x}$ and $\hat{L_y}$ expressions: 1 mark
    \item Correct final answer: 1 mark
\end{itemize}


\textbf{Part b : 3/4}
\begin{itemize}
    \item Correct application of operator expression: 2 marks
    \item Missing detailed calculation of eigenvalues and proof of eigenstate: 1 mark
\end{itemize}


\textbf{Part c : 5/9}
\begin{itemize}
    \item Correct identification of basis states: 1 mark
    \item Missing explicit vector representation of basis kets: 0 marks
    \item Missing detailed calculation of $\hat{L^2}$ acting on basis kets: 1 mark
    \item Correct but insufficient calculation of $\hat{S_+}\hat{L_z}$ transitions: 2 marks
    \item Partial matrix representation of $\hat{O}$: 1 mark
\end{itemize}

\textbf{Feedback:}
Good attempt, but some steps and explicit calculations are missing, leading to an incomplete solution. More detailed work needed, especially in Part c.



Solution 3 :

\textbf{Marks : 12/17}

\textbf{Part a: 4/4}
\begin{itemize}
    \item Correct identification of the terms involved: 1 mark
    \item Proper substitution of $\hat{L_x}$ and $\hat{L_y}$ expressions: 2 marks
    \item Correct final answer: 1 mark
\end{itemize}



\textbf{Part b : 3/4}
\begin{itemize}
    \item Correct application of operator expression: 2 marks
    \item Missing detailed calculation of eigenvalues and proof of eigenstate: 1 mark
\end{itemize}



\textbf{Part c : 5/9}
\begin{itemize}
    \item Correct identification of basis states: 1 mark
    \item Missing explicit vector representation of basis kets: 0 marks
    \item Missing detailed calculation of $\hat{L^2}$ acting on basis kets: 1 mark
    \item Correct calculation of $\hat{S_+}\hat{L_z}$ transitions: 2 marks
    \item Partial matrix representation of $\hat{O}$: 1 mark
\end{itemize}

\textbf{Feedback:}
Good approach, but missing some detailed steps and explicit calculations, particularly in Part c. Ensure all steps are fully demonstrated for complete understanding.


\subsubsection*{Response 5:}

Solution 1 :

\textbf{Marks : 17/17}

\textbf{Part a: 4/4}
\begin{itemize}
    \item Correctly starts by expanding $\hat{J^2}$.
    \item Uses appropriate expressions for  $\hat{L_x}, \hat{L_y}, \hat{S_x}, \hat{S_y}$ with correct transformations.
    \item Completes derivation with final expression.
\end{itemize}


\textbf{Part b : 4/4}
\begin{itemize}
    \item Correctly applies $\hat{J^2}$ to $|l, -l ; s, -s>$ and handles ladder operators.
    \item Successfully calculates eigenvalues using eigenstate properties.
\end{itemize}


\textbf{Part c : 9/9}
\begin{itemize}
    \item Identifies correct basis kets and appropriate vector representations.
    \item Correctly calculates the action of $\hat{L^2}$ on basis kets.
    \item Calculates $\hat{S_+}\hat{L_z}$ acting on basis kets.
    \item Combines contributions to form the final matrix representation.
\end{itemize}

\textbf{General Feedback:}
Excellent, detailed, and accurate derivations. Clear application of quantum mechanics principles.


Solution 2 :

\textbf{Marks : 13/17}

\textbf{Part a: 3/4}
\begin{itemize}
    \item Correctly expands $\hat{J^2}$ and uses appropriate vector operator identities.
    \item Adequate use of ladder operators, but misses explicit detailed steps in deriving the final form.
\end{itemize}


\textbf{Part b : 3/4}
\begin{itemize}
    \item Provides a correct outline for eigenvalue derivation but lacks explicit steps.
    \item General explanation of eigenvalue derivation is good but lacks specific calculations.
\end{itemize}


\textbf{Part c : 7/9}
\begin{itemize}
    \item Correctly identifies basis kets.
    \item Starts calculating contributions correctly but misses explicit steps.
    \item General explanation is good but lacks specific matrix construction.
\end{itemize}

\textbf{General Feedback:}
Good understanding shown but lacks detailed derivations and explicit steps in parts b and c.



Solution 3 :

\textbf{Marks : 10/17}

\textbf{Part a: 3/4}
\begin{itemize}
    \item Correctly expands $\hat{J^2}$ and uses appropriate vector operator identities.
    \item Misses detailed steps in deriving final expression.
\end{itemize}



\textbf{Part b : 2/4}
\begin{itemize}
    \item Provides correct eigenvalue outline but lacks detailed steps and explicit calculations.
\end{itemize}



\textbf{Part c : 5/9}
\begin{itemize}
    \item Correctly identifies basis kets and their representations.
    \item Provides matrix elements but with some incorrect and missing details.
    \item Misses explicit calculation steps for $\hat{S_+}\hat{L_z}$.
\end{itemize}

\textbf{General Feedback:}
Basic understanding is shown but lacks detailed steps and explicit calculations in parts b and c.





\subsection*{Question 7 : }

\subsubsection*{Response 1 :}

Solution 1 :

\textbf{Marks : 8/8}

\textbf{Feedback :}

\textbf{Part a: 3/3}
\begin{itemize}
    \item Correct identification and application of the time-dependent Schrödinger equation.
    \item Derivation of the time-dependent wavefunction is clear and accurate.
    \item Correctly shows that the probability density is independent of space and time.
\end{itemize}

\textbf{Part b : 5/5}
\begin{itemize}
    \item Correctly calculates the particle flux.
    \item Explanation and physical interpretation of the result are accurate.
\end{itemize}

\textbf{Feedback :}
Excellent work. You have clearly explained both the time-dependent solution and the particle flux, with accurate calculations and solid physical interpretation.


Solution 2 :

\textbf{Marks : 8/8}

\textbf{Feedback :}

\textbf{Part a: 3/3}
\begin{itemize}
    \item Correct identification and application of the time-dependent Schrödinger equation.
    \item Derivation of the time-dependent wavefunction is clear and accurate.
    \item Correctly shows that the probability density is independent of space and time.
\end{itemize}

\textbf{Part b : 5/5}

\begin{itemize}
    \item Correctly calculates the particle flux.
    \item Explanation and physical interpretation of the result are clear and accurate.
\end{itemize}

\textbf{Feedback :}
Excellent solution. Both the derivations and explanations are clear and correct.


Solution 3 :

\textbf{Marks : 8/8}

\textbf{Feedback :}

\textbf{Part a: 3/3}

\begin{itemize}
    \item Correct identification and application of the time-dependent Schrödinger equation.
    \item Derivation of the time-dependent wavefunction is accurate.
    \item Correctly shows that the probability density is independent of space and time.
\end{itemize}

\textbf{Part b : 5/5}

\begin{itemize}
    \item Correctly calculates the particle flux.
    \item Explanation and physical interpretation of the result are clear and accurate.
\end{itemize}

\textbf{Feedback :}
Excellent job. You have clearly explained both the time-dependent solution and the particle flux with precise calculations and interpretation.

\textbf{Summary :}
All three solutions have demonstrated a clear understanding of the problem, correctly deriving the time-dependent wavefunction and the particle flux, with accurate and concise explanations. Each solution has earned full marks for both parts.



\subsubsection*{Response 2 :}

Solution 1 :

\textbf{Marks : 8/8}

\textbf{Feedback :}

\textbf{Part a: 3/3}
\begin{itemize}
    \item Correctly identifies the form of the time-dependent wavefunction.
    \item Properly uses the relationship $\omega = \frac{\hbar k^2}{2m}$.
    \item Shows that $|\Psi (x,t)|^2$ is constant, demonstrating independence of space and time.
\end{itemize}

\textbf{Part b : 5/5}

\begin{itemize}
    \item Accurately calculates the derivatives and substitutes them into the flux formula.
    \item Provides a clear and correct physical interpretation of the flux in terms of particle velocity.
\end{itemize}

\textbf{Feedback :}
Excellent solution with thorough and correct steps. Clearly demonstrates understanding of both the time-dependent wavefunction and particle flux.


Solution 2 :

\textbf{Marks : 8/8}

\textbf{Feedback :}

\textbf{Part a: 3/3}
\begin{itemize}
    \item Correctly derives the time-dependent wavefunction.
    \item Demonstrates that the probability density is independent of space and time.
\end{itemize}

\textbf{Part b : 5/5}

\begin{itemize}
    \item Correctly calculates the particle flux and provides the correct interpretation of the result.
    \item Includes all necessary steps and clearly explains the physical meaning of the flux.
\end{itemize}

\textbf{Feedback :}
A well-done solution that correctly addresses both parts of the problem with clear and accurate explanations.


Solution 3 :

\textbf{Marks : 6/8}

\textbf{Feedback :}

\textbf{Part a: 2/3}
\begin{itemize}
    \item Correctly identifies the time-dependent wavefunction, but makes a minor notation error in combining exponential terms.
    \item Demonstrates that the probability density is constant, which is correct.
\end{itemize}

\textbf{Part b : 4/5}

\begin{itemize}
    \item Correctly calculates the particle flux, but makes a minor simplification error in the sign and intermediate steps.
    \item Provides an accurate physical interpretation of the flux.
\end{itemize}

\textbf{Feedback :}
Good overall solution. Minor notation and simplification errors detract slightly from the clarity and correctness.



\subsubsection*{Response 3 :}

Solution 1 :

\textbf{Marks : 7/8}

\textbf{Feedback :}

\textbf{Part a: 3/3}
\begin{itemize}
    \item Correctly found the time-dependent solution $|\Psi (x,t)$ using the given wavefunction and potential.
    \item Properly demonstrated that the probability density $|\Psi (x,t)|^2$ is independent of both space and time.
\end{itemize}

\textbf{Part b : 4/5}

\begin{itemize}
    \item Correctly calculated the particle flux $\Gamma$.
    \item Provided a clear physical interpretation relating the flux to the velocity of the particles.
    \item Minor errors in steps or notational clarity could be improved.
\end{itemize}

\textbf{General Feedback :}
Excellent solution. Clear and accurate derivations with minor areas for improvement in clarity and detail in some steps.


Solution 2 :

\textbf{Marks : 8/8}

\textbf{Feedback :}

\textbf{Part a: 3/3}
\begin{itemize}
    \item Correctly found the time-dependent solution $|\Psi (x,t)$ using the given wavefunction and potential.
    \item Correctly demonstrated that the probability density $|\Psi (x,t)|^2$ is independent of both space and time.
\end{itemize}

\textbf{Part b : 5/5}

\begin{itemize}
    \item Correctly calculated the particle flux $\Gamma$.
    \item Provided a good physical interpretation relating the flux to the velocity of the particles.
    \item Some steps in the derivation could be more concise or clearer.
\end{itemize}

\textbf{General Feedback :}
Very good solution with clear and accurate derivations. A bit more detail or clarity in certain steps would make it excellent.


Solution 3 :

\textbf{Marks : 7/8}

\textbf{Feedback :}

\textbf{Part a: 3/3}
\begin{itemize}
    \item Correctly found the time-dependent solution $|\Psi (x,t)$ using the given wavefunction and potential.
    \item Correctly demonstrated that the probability density $|\Psi (x,t)|^2$ is independent of both space and time.
\end{itemize}

\textbf{Part b : 4/5}

\begin{itemize}
    \item Correctly calculated the particle flux $\Gamma$.
    \item Provided a good physical interpretation relating the flux to the velocity of the particles.
    \item Derivation is clear but some steps could be streamlined or clarified further.
\end{itemize}

\textbf{General Feedback :}
Solid solution with clear and accurate work. Further streamlining and clarity in derivation steps would enhance the overall quality.

\textbf{Summary :}
\begin{itemize}
    \item All three solutions effectively solved part (a) and demonstrated the independence of the probability density from space and time.
    \item For part (b), all solutions correctly derived the particle flux and provided physical interpretations, but they could be improved in clarity and conciseness in their steps.
    \item Each solution received full marks for part (a) and minor deductions for part (b), reflecting minor issues in derivation or presentation.
\end{itemize}


\subsubsection*{Response 4 :}

Solution 1 :

\textbf{Marks : 8/8}

\textbf{Feedback :}

\textbf{Part a: 3/3}
\begin{itemize}
    \item Excellent explanation and correct derivation of the time-dependent solution. Clear demonstration of the probability density being independent of space and time.
\end{itemize}

\textbf{Part b : 5/5}

\begin{itemize}
    \item The flux calculation is accurate and well-explained. The physical interpretation is correctly linked to the velocity of the particles.
\end{itemize}


Solution 2 :

\textbf{Marks : 8/8}

\textbf{Feedback :}

\textbf{Part a: 3/3}
\begin{itemize}
    \item Comprehensive explanation and correct time-dependent solution. Clearly shows that the probability density is constant.
\end{itemize}

\textbf{Part b : 5/5}

\begin{itemize}
    \item Correct calculation of the particle flux and an accurate physical interpretation of the result. The steps are clearly laid out.
\end{itemize}



Solution 3 :

\textbf{Marks : 7/8}

\textbf{Feedback :}

\textbf{Part a: 2/3}
\begin{itemize}
    \item The solution is mostly correct, but there is a slight error in the notation at the end. The time-dependent part should include a negative sign in the exponent. However, the key concept is correct.
\end{itemize}

\textbf{Part b : 5/5}

\begin{itemize}
    \item The flux calculation is correct and the physical interpretation is well-explained. Despite the minor notation error in part a), the flux derivation is handled correctly.
\end{itemize}


\subsubsection*{Response 5 :}

Solution 1 :

\textbf{Marks : 8/8}

\textbf{Feedback :}

\textbf{Part a: 3/3}
\begin{itemize}
    \item Correctly derived the time-dependent wavefunction $\Psi (x,t)$.
    \item Demonstrated that the probability density is constant and independent of both space and time.
\end{itemize}

\textbf{Part b : 5/5}

\begin{itemize}
    \item Correctly evaluated the particle flux $\Gamma$.
    \item Provided a clear physical interpretation relating the flux to the velocity of the particles.
\end{itemize}

\textbf{Feedback :}
Excellent solution. The explanations are clear, and all steps are logically followed and well presented. Both the mathematical derivations and physical interpretations are correct and complete.


Solution 2 :

\textbf{Marks : 8/8}

\textbf{Feedback :}

\textbf{Part a: 3/3}
\begin{itemize}
    \item Correctly derived the time-dependent wavefunction $\Psi (x,t)$.
    \item Demonstrated that the probability density is constant and independent of both space and time.
\end{itemize}

\textbf{Part b : 5/5}

\begin{itemize}
    \item Correctly evaluated the particle flux $\Gamma$.
    \item Provided a clear physical interpretation relating the flux to the velocity of the particles.
\end{itemize}

\textbf{Feedback :}
Excellent work. The solution is well-structured and comprehensive. The calculations and interpretations are correct and clearly explained.


Solution 3 :

\textbf{Marks : 7/8}

\textbf{Feedback :}

\textbf{Part a: 3/3}
\begin{itemize}
    \item Correctly derived the time-dependent wavefunction $\Psi (x,t)$.
    \item Demonstrated that the probability density is constant and independent of both space and time.
\end{itemize}

\textbf{Part b : 4/5}

\begin{itemize}
    \item Correctly evaluated the particle flux $\Gamma$.
    \item The explanation for the physical interpretation is slightly less detailed than in the other solutions, but still correct.
\end{itemize}

\textbf{Feedback :}
Very good solution. The mathematical derivations are correct and well presented. The physical interpretation could be slightly more detailed to match the completeness of the other solutions, but overall, it is a strong response.

\textbf{Overall comments:}
All three solutions show a strong understanding of the problem and correctly derive both the time-dependent wavefunction and the particle flux. The differences in marks are minimal and mostly related to the depth of explanation in the physical interpretation of the results.





\subsection*{Question 8:}


\subsubsection*{Response 1 :}

Solution 1 :

\textbf{Marks : 9/9}

\textbf{Feedback :}

\textbf{Part a : 5/5}

\begin{itemize}
    \item The student has correctly followed the steps for differentiating the expectation value and applied the Schrödinger equation appropriately. The derivation is clear, well-organized, and leads to the correct result.
\end{itemize}

\textbf{Part b : 4/4}

\begin{itemize}
    \item The student correctly identifies the Hamiltonian and calculates the commutator accurately. The use of the canonical commutation relation is precise, leading to the correct result. The explanation is thorough and well-structured.
\end{itemize}

\textbf{General feedback :}
A well-structured solution with clear logical steps. Including more detailed intermediate steps in part b) would improve the solution further.


Solution 2 :

\textbf{Marks : 7/9}

\textbf{Feedback :}

\textbf{Part a : 4/5}

\begin{itemize}
    \item The steps are clear and mostly correct. However, the explanation could be more detailed, particularly in step 3, where the commutator is identified. A bit more emphasis on the substitution of the Schrödinger equation would improve clarity.
\end{itemize}

\textbf{Part b : 3/4}

\begin{itemize}
    \item The student correctly starts with the Hamiltonian and proceeds with the commutator. However, the explanation of why the potential term does not contribute is a bit brief. The final commutator evaluation is correct, but the derivation could be clearer.
\end{itemize}

\textbf{General feedback :}
A good attempt with all necessary steps included. More concise explanations and explicit intermediate steps would improve the clarity of the solution.



Solution 3 :

\textbf{Marks : 7/9}

\textbf{Feedback :}

\textbf{Part a : 4/5}

\begin{itemize}
    \item The steps are correct, but the explanation lacks detail in some parts, particularly in step 3. The derivation reaches the correct conclusion, but additional clarity in the intermediate steps would enhance understanding.
\end{itemize}

\textbf{Part b : 3/4}

\begin{itemize}
    \item The student correctly calculates the commutator and uses the result from part a) effectively. However, the explanation of the commutation relation and how it simplifies could be clearer. The final result is correct, but the derivation is slightly confusing.
\end{itemize}



\subsubsection*{Response 2 :}

Solution 1 :

\textbf{Marks : 9/9}

\textbf{Feedback :}

\textbf{Part a : 5/5}

\begin{itemize}
    \item The solution correctly derives the required relationship using clear and logical steps.
    \item The explanation is thorough, and each step is justified with proper application of the Schrödinger equation and differentiation rules.
\end{itemize}

\textbf{Part b : 4/4}

\begin{itemize}
    \item The commutation relations are evaluated correctly.
    \item The solution uses the canonical commutation relations and derives the expected result efficiently.
    \item The explanation is clear and logically progresses to the final result.
\end{itemize}

\textbf{General feedback :}
Excellent solution. The derivations are complete, correct, and well-explained. The student shows a strong understanding of quantum mechanics principles.


Solution 2 :

\textbf{Marks : 7/9}

\textbf{Feedback :}

\textbf{Part a : 4/5}

\begin{itemize}
    \item The solution follows the correct steps and arrives at the correct result.
    \item However, it lacks a detailed justification of some intermediate steps and assumes familiarity with the process without explicitly stating all calculations.
\end{itemize}

\textbf{Part b : 3/4}

\begin{itemize}
    \item The commutator evaluation is mostly correct but skips detailed explanations of certain steps, such as the detailed commutation involving $\hat{P}$.
    \item The solution arrives at the correct result but lacks thorough justification.
\end{itemize}

\textbf{General feedback :}
Good solution. The student demonstrates an understanding of the concepts but should provide more detailed intermediate steps to ensure clarity.



Solution 3 :

\textbf{Marks : 7/9}

\textbf{Feedback :}

\textbf{Part a : 4/5}

\begin{itemize}
    \item The solution correctly derives the required result, and the steps are clear.
    \item However, some explanations could be more detailed, particularly the application of the Schrödinger equation to justify the steps.
\end{itemize}

\textbf{Part b : 3/4}

\begin{itemize}
    \item The solution correctly uses the commutator relations but is a bit brief on explaining the intermediate steps, especially in evaluating the commutation.
    \item The final result is correct, but the solution would benefit from a more detailed justification of the steps leading to it.
\end{itemize}

\textbf{General feedback :}
Good solution. The student shows an understanding of the key principles but should aim to provide more detailed explanations of intermediate steps to ensure full clarity.



\subsubsection*{Response 3 :}

Solution 1 :

\textbf{Marks : 9/9}

\textbf{Feedback :}

\textbf{Part a : 5/5}

\begin{itemize}
    \item Correctly started with the definition of expectation value.
    \item Applied the product rule correctly.
    \item Substituted the Schrödinger equation and its conjugate accurately.
    \item Properly identified and used the commutator relation.
\end{itemize}

\textbf{Feedback :}
The solution is clear, logical, and demonstrates a solid understanding of the required steps and their justification.


\textbf{Part b : 4/4}

\begin{itemize}
    \item Correctly wrote the Hamiltonian for the given system.
    \item Correctly evaluated the commutator $[\hat{X}, \hat{H}]$ and identified the commutation relations.
    \item Used the result from part a) appropriately to derive the final expression.
\end{itemize}

\textbf{Feedback :}
The solution is thorough and shows a strong grasp of the commutator properties and their implications.


Solution 2 :

\textbf{Marks : 7/9}

\textbf{Feedback :}

\textbf{Part a : 4/5}

\begin{itemize}
    \item Correctly started with the definition of expectation value.
    \item Applied the product rule correctly.
    \item Substituted the Schrödinger equation and its conjugate accurately.
    \item Properly identified and used the commutator relation.
\end{itemize}

\textbf{Feedback :}
The solution is correct but lacks some detailed explanations compared to Solution 1.

\textbf{Part b : 3/4}

\begin{itemize}
    \item Correctly wrote the Hamiltonian for the given system.
    \item Evaluated the commutator $[\hat{X}, \hat{H}]$ but was less clear in the intermediate steps.
    \item Used the result from part a) but the explanation could be more detailed.
\end{itemize}

\textbf{Feedback :}
While correct, the explanation is slightly less detailed and clear than Solution 1.



Solution 3 :

\textbf{Marks : 6/9}

\textbf{Feedback :}

\textbf{Part a : 4/5}

\begin{itemize}
    \item Correctly started with the definition of expectation value.
    \item Applied the product rule correctly.
    \item Substituted the Schrödinger equation and its conjugate accurately.
    \item Properly identified and used the commutator relation.
\end{itemize}

\textbf{Feedback :}
The solution is correct but lacks some detailed explanations compared to Solution 1.


\textbf{Part b : 2/4}

\begin{itemize}
    \item Correctly wrote the Hamiltonian for the given system.
    \item Evaluated the commutator $[\hat{X}, \hat{H}]$ but had a minor mistake in the steps involving commutators and algebra.
    \item Used the result from part a) but with a slightly less rigorous approach.
\end{itemize}

\textbf{Feedback :}
The explanation had some algebraic errors and was less rigorous in deriving the commutator result.



\subsubsection*{Response 4 :}

Solution 1 :

\textbf{Marks : 8/9}

\textbf{Feedback :}

\textbf{Part a : 5/5}

\begin{itemize}
    \item The solution is clear, logically structured, and correctly applies the Schrödinger equation and the product rule to derive the result. The steps are detailed and easy to follow, leading to the correct conclusion.
\end{itemize}


\textbf{Part b : 3/4}

\begin{itemize}
    \item The commutator calculation is correct, and the solution correctly applies the result from part a) to derive the desired relationship. However, a more detailed explanation of why $[\hat{X}, V(x)]$ would have made the solution more robust.
\end{itemize}


Solution 2 :

\textbf{Marks : 9/9}

\textbf{Feedback :}

\textbf{Part a : 5/5}

\begin{itemize}
    \item This solution is also clear and logically structured, correctly applying the necessary equations and steps to derive the result. The explanation is concise yet complete, making it easy to understand.
\end{itemize}


\textbf{Part b : 4/4}

\begin{itemize}
    \item The solution effectively calculates the commutator and correctly uses it to derive the required result. The explanation is thorough and well-articulated, clearly showing the steps and logic involved.
\end{itemize}




Solution 3 :

\textbf{Marks : 9/9}

\textbf{Feedback :}

\textbf{Part a : 5/5}

\begin{itemize}
    \item The solution is clear and methodical, providing a step-by-step derivation that correctly applies the Schrödinger equation and the product rule. The logical flow is easy to follow and leads to the correct result.
\end{itemize}


\textbf{Part b : 4/4}

\begin{itemize}
    \item This solution correctly calculates the commutator and uses it to derive the desired relationship. The explanation is detailed and accurate, covering all necessary points and providing a thorough understanding of the process.
\end{itemize}



\subsubsection*{Response 5 :}

Solution 1 :

\textbf{Marks : 8/9}

\textbf{Feedback :}

\textbf{Part a : 5/5}

\begin{itemize}
    \item The solution clearly and correctly shows each step, starting from the definition of the expectation value, applying the product rule, and substituting the Schrödinger equation.
    \item The final result is obtained through a well-structured derivation.
\end{itemize}


\textbf{Part b : 3/4}

\begin{itemize}
    \item The commutator $[\hat{X}, \hat{H}]$ is evaluated correctly, showing good use of the properties of operators.
    \item However, more detail could have been provided on the evaluation of the commutator, especially the step involving $[\hat{X}, \hat{P^2}]$.
\end{itemize}

\textbf{General Feedback :}
The solution is thorough and well-explained for part a), while part b) correctly applies the relevant principles but lacks some intermediate steps.


Solution 2 :

\textbf{Marks : 8/9}

\textbf{Feedback :}

\textbf{Part a : 4/5}

\begin{itemize}
    \item The derivation is mostly correct and clear, but the explanation is a bit concise, especially in the step combining terms under a single integral.
    \item A minor loss of marks for not explicitly showing the substitution of the Schrödinger equation into the differentiated expectation value.
\end{itemize}


\textbf{Part b : 4/4}

\begin{itemize}
    \item The commutator $[\hat{X}, \hat{H}]$ is evaluated correctly, and the relation $m\frac{d}{dt} <\hat{X}> = <\hat{P}> $ is clearly derived.
    \item The solution provides sufficient detail to follow the steps logically.
\end{itemize}

\textbf{General Feedback :}
Part a) could benefit from a more detailed intermediate step explanation. Part b) is well-done and shows a solid understanding of operator commutation relations.



Solution 3 :

\textbf{Marks : 8/9}

\textbf{Feedback :}

\textbf{Part a : 4/5}

\begin{itemize}
    \item The steps are logically laid out and lead to the correct result.
    \item However, the explanation could be slightly more detailed in the step involving the application of the Schrödinger equation.
\end{itemize}


\textbf{Part b : 4/4}

\begin{itemize}
    \item The commutator $[\hat{X}, \hat{H}]$ is evaluated correctly and the subsequent steps are well-explained.
    \item The derivation successfully shows the required relationship with clear and correct application of the operator properties.
\end{itemize}

\textbf{General Feedback :}
Part a) is almost perfect but could use a bit more detail in the middle steps. Part b) is correctly and clearly derived.

\textbf{Summary :}
All solutions are well-structured and demonstrate a solid understanding of the principles involved in the problem. They mainly differ in the level of detail provided in their explanations, particularly in part a).



\subsection*{Question 9 :}

\subsubsection*{Response 1 :}

Solution 1 : 

\textbf{Marks : 17/17}

\textbf{Feedback :}

\textbf{Part a: 5/5}
\begin{itemize}
    \item Explanation is clear and defines $k_1$ and $k_2$ correctly.
\end{itemize}


\textbf{Part b: 5/5}
\begin{itemize}
    \item Boundary conditions are applied correctly and solutions for $B$ and $C$ are accurate.
\end{itemize}

\textbf{Part c: 7/7}
\begin{itemize}
    \item Flux calculations are detailed and correct, leading to the correct expressions for $T$ and $R$.
\end{itemize}


\textbf{Feedback :}
Excellent solution. Each step is clearly explained, and the calculations are accurate. Good use of boundary conditions and detailed flux calculations.


Solution 2 :

\textbf{Marks : 16/17}

\textbf{Feedback :}

\textbf{Part a: 5/5}
\begin{itemize}
    \item Correct solutions and definitions for $k_1$ and $k_2$.
\end{itemize}


\textbf{Part b: 5/5}
\begin{itemize}
    \item Correct application of boundary conditions and accurate solutions for $B$ and $C$.
\end{itemize}

\textbf{Part c: 6/7}
\begin{itemize}
    \item Correct flux calculations, but the reflected flux calculation has an incorrect sign. Correct final expressions for $T$ and $R$.
\end{itemize}


\textbf{Feedback :}
Well-explained and mostly accurate solution. Pay attention to the signs in flux calculations, but overall a very good understanding and execution.


Solution 3 :


\textbf{Marks : 16/17}

\textbf{Feedback :}

\textbf{Part a: 5/5}
\begin{itemize}
    \item Correct solutions and definitions for $k_1$ and $k_2$.
\end{itemize}


\textbf{Part b: 5/5}
\begin{itemize}
    \item Correct application of boundary conditions and accurate solutions for $B$ and $C$.
\end{itemize}

\textbf{Part c: 6/7}
\begin{itemize}
    \item Flux calculations are generally correct, but there is a slight inconsistency in presenting the reflected flux. Correct final expressions for $T$ and $R$.
\end{itemize}


\textbf{General feedback :}
A well-structured and accurate solution. Ensure consistency in the details of flux calculations. Good job on deriving the final expressions for transmission and reflection probabilities.

\subsubsection*{Response 2 :}

Solution 1 : 

\textbf{Marks : 16/17}

\textbf{Feedback :}

\textbf{Part a: 5/5}
\begin{itemize}
    \item Correctly identifies and defines $k_1$ and $k_2$.
    \item Provides the correct general solutions for both regions.
\end{itemize}


\textbf{Part b: 5/5}
\begin{itemize}
    \item Applies boundary conditions correctly.
    \item Derives the correct expressions for $B$ and $C$.
\end{itemize}

\textbf{Part c: 6/7}
\begin{itemize}
    \item Correctly calculates the incident, reflected, and transmitted flux.
    \item Provides correct final expressions for $T$ and $R$.
    \item Minor notational inconsistency (should clearly show squared terms for $B$ in $\Gamma_r$).
\end{itemize}


\textbf{Feedback :}
Excellent solution overall. The steps are detailed and clear. Ensure consistency in notation for squared terms in flux calculations.


Solution 2 :

\textbf{Marks : 17/17}

\textbf{Feedback :}

\textbf{Part a: 5/5}
\begin{itemize}
    \item Correctly identifies and defines $k_1$ and $k_2$.
    \item Provides the correct general solutions for both regions.
\end{itemize}


\textbf{Part b: 5/5}
\begin{itemize}
    \item Applies boundary conditions correctly.
    \item Derives the correct expressions for $B$ and $C$.
\end{itemize}

\textbf{Part c: 7/7}
\begin{itemize}
    \item Correctly calculates the incident, reflected, and transmitted flux.
    \item Provides correct final expressions for $T$ and $R$.
    \item Clear and thorough explanation of each step.
\end{itemize}


\textbf{Feedback :}
Excellent work. Clear, accurate, and thorough explanation. All steps and calculations are well-presented.


Solution 3 :

\textbf{Marks : 17/17}

\textbf{Feedback :}

\textbf{Part a: 5/5}
\begin{itemize}
    \item Correctly identifies and defines $k_1$ and $k_2$.
    \item Provides the correct general solutions for both regions.
\end{itemize}


\textbf{Part b: 5/5}
\begin{itemize}
    \item Applies boundary conditions correctly.
    \item Derives the correct expressions for $B$ and $C$.
\end{itemize}

\textbf{Part c: 7/7}
\begin{itemize}
    \item Correctly calculates the incident, reflected, and transmitted flux.
    \item Provides correct final expressions for $T$ and $R$.
    \item Detailed and accurate derivations.
\end{itemize}



\textbf{Feedback :}
Excellent solution with clear steps and accurate results. The detailed explanation and correct calculations demonstrate a strong understanding of the problem.


\subsubsection*{Response 3 :}


Solution 1 : 

\textbf{Marks : 17/17}

\textbf{Feedback :}

\textbf{Part a: 5/5}
\begin{itemize}
    \item The solution correctly identifies the Schrödinger equations for both regions and provides the correct general solutions.
    \item $k_1$ and $k_2$ are accurately defined.
\end{itemize}


\textbf{Part b: 5/5}
\begin{itemize}
    \item The boundary conditions are correctly applied.
    \item The values of $B$ and $C$ are accurately derived.
\end{itemize}

\textbf{Part c: 7/7}
\begin{itemize}
    \item The calculation of the incident, reflected, and transmitted flux is thorough and correct.
    \item The derivation of $T$ and $R$ is precise and matches the required expressions.
\end{itemize}


\textbf{General feedback :}
Excellent solution. Each part is well-explained, and the derivations are clear and accurate. Great work!


Solution 2 :

\textbf{Marks : 16/17}

\textbf{Feedback :}

\textbf{Part a: 5/5}
\begin{itemize}
    \item The solution correctly addresses the Schrödinger equations for both regions.
    \item The definitions of $k_1$ and $k_2$ are accurate.
\end{itemize}


\textbf{Part b: 5/5}
\begin{itemize}
    \item The boundary conditions are appropriately applied, and the values of $B$ and $C$ are correctly derived.
\end{itemize}

\textbf{Part c: 6/7}
\begin{itemize}
    \item The incident, reflected, and transmitted flux calculations are correct.
    \item The derivation of $T$ and $R$ is precise and accurate.
\end{itemize}


\textbf{General feedback :}
Excellent work. The explanations are clear, and the calculations are correct. Your solution demonstrates a strong understanding of the problem.


Solution 3 :


\textbf{Marks : 17/17}

\textbf{Feedback :}

\textbf{Part a: 5/5}
\begin{itemize}
    \item The solution provides correct Schrödinger equations for both regions.
    \item $k_1$ and $k_2$ are correctly defined.
\end{itemize}


\textbf{Part b: 5/5}
\begin{itemize}
    \item The boundary conditions are accurately applied.
    \item The derivations of $B$ and $C$ are correct.
\end{itemize}

\textbf{Part c: 7/7}
\begin{itemize}
    \item The calculations for incident, reflected, and transmitted flux are correct.
    \item The derivation of $T$ and $R$ is accurate and matches the required expressions.
\end{itemize}

\textbf{General feedback :}
Excellent job. The explanations and calculations are thorough and correct. You have clearly understood the problem and the necessary steps to solve it.


\subsubsection*{Response 4 :}


Solution 1 : 

\textbf{Marks : 17/17}

\textbf{Feedback :}

\textbf{Part a: 5/5}
\begin{itemize}
    \item Correctly identifies and solves the Schrödinger equation for both regions.
    \item Defines $k_1$ and $k_2$ accurately.
    \item Provides the correct general solutions for both regions.
\end{itemize}


\textbf{Part b: 5/5}
\begin{itemize}
    \item Correctly applies boundary conditions at $x = 0$
    \item Derives the expressions for $B$ and $C$ accurately.
\end{itemize}

\textbf{Part c: 7/7}
\begin{itemize}
    \item Correctly calculates the incident, reflected, and transmitted flux.
    \item Accurately derives the expressions for $T$ and $R$.
\end{itemize}


\textbf{General feedback :}
Excellent solution with accurate and complete derivations. All steps are clearly explained and correct.


Solution 2 :

\textbf{Marks : 17/17}

\textbf{Feedback :}

\textbf{Part a: 5/5}
\begin{itemize}
    \item Correctly identifies and solves the Schrödinger equation for both regions.
    \item Defines $k_1$ and $k_2$ accurately.
    \item Provides the correct general solutions for both regions.
\end{itemize}


\textbf{Part b: 5/5}
\begin{itemize}
    \item Correctly applies boundary conditions at $x = 0$
    \item Derives the expressions for $B$ and $C$ accurately.
\end{itemize}

\textbf{Part c: 7/7}
\begin{itemize}
    \item Correctly calculates the incident, reflected, and transmitted flux.
    \item Accurately derives the expressions for $T$ and $R$.
\end{itemize}


\textbf{General feedback :}
Very well-structured and accurate solution. All required steps and calculations are presented clearly and correctly.


Solution 3 :


\textbf{Marks : 17/17}

\textbf{Feedback :}

\textbf{Part a: 5/5}
\begin{itemize}
    \item Correctly identifies and solves the Schrödinger equation for both regions.
    \item Defines $k_1$ and $k_2$ accurately.
    \item Provides the correct general solutions for both regions.
\end{itemize}


\textbf{Part b: 5/5}
\begin{itemize}
    \item Correctly applies boundary conditions at $x = 0$
    \item Derives the expressions for $B$ and $C$ accurately.
\end{itemize}

\textbf{Part c: 7/7}
\begin{itemize}
    \item Correctly calculates the incident, reflected, and transmitted flux.
    \item Accurately derives the expressions for $T$ and $R$.
\end{itemize}

\textbf{General Fedback :}
Comprehensive and accurate solution. All parts are addressed with clear and correct derivations.


\subsubsection*{Response 5 :}


Solution 1 : 

\textbf{Marks : 17/17}

\textbf{Feedback :}

\textbf{Part a: 5/5}
\begin{itemize}
    \item The student correctly derived the general solutions and defined $k_1$ and $k_2$ properly.
\end{itemize}


\textbf{Part b: 5/5}
\begin{itemize}
    \item The boundary conditions were applied correctly, and the expressions for $B$ and $C$ were derived accurately.
\end{itemize}

\textbf{Part c: 7/7}
\begin{itemize}
    \item The student correctly calculated the fluxes and derived the expressions for $T$ and $R$ accurately.
\end{itemize}


\textbf{General feedback :}
Excellent work! Your solution is thorough and correct in all parts. Great job deriving the fluxes and final expressions for transmission and reflection probabilities.


Solution 2 :

\textbf{Marks : 17/17}

\textbf{Feedback :}

\textbf{Part a: 5/5}
\begin{itemize}
    \item The student correctly solved for the general solutions in both regions and defined $k_1$ and $k_2$.
\end{itemize}


\textbf{Part b: 5/5}
\begin{itemize}
    \item The boundary conditions were applied correctly, and the student successfully derived the expressions for $B$ and $C$.
\end{itemize}

\textbf{Part c: 7/7}
\begin{itemize}
    \item The flux calculations and derivation of $T$ and $R$ are accurate and well-explained.
\end{itemize}


\textbf{General feedback :}
Excellent work! Your step-by-step approach is clear and correct. Well done on the flux calculations and final expressions.



Solution 3 :


\textbf{Marks : 16/17}

\textbf{Feedback :}

\textbf{Part a: 5/5}
\begin{itemize}
    \item The general solutions and definitions of $k_1$ and $k_2$ are correct.
\end{itemize}


\textbf{Part b: 5/5}
\begin{itemize}
    \item The boundary conditions are applied correctly, and the derivation of $B$ and $C$ is accurate.
\end{itemize}

\textbf{Part c: 7/7}
\begin{itemize}
    \item The flux calculations are correct, but the explanation is less detailed than the other solutions. The derivation of $T$ and $R$ is correct, but providing more detailed steps would improve clarity.
\end{itemize}

\textbf{General feedback :}
Good work! Your solution is mostly correct, but adding more detailed explanations for the flux calculations would enhance your answer. Overall, a strong effort.




\subsection*{Question 10 :}

\subsubsection*{Response 1 :}

Solution 1 :

\textbf{Marks : 10/16}

\textbf{Feedback :}

\textbf{Part a: 4/5}
\begin{itemize}
    \item The solution correctly normalized the state and calculated the expectation value of the Hamiltonian, although it stopped short of providing the final simplified expression.
\end{itemize}

\textbf{Feedback :}
Good work on normalization and the initial calculation of the expectation value. Make sure to simplify the final expression for clarity.


\textbf{Part b: 2/2}
\begin{itemize}
    \item The perturbation was correctly expressed in terms of the raising and lowering operators.
\end{itemize}

\textbf{Feedback :}
Correctly transformed the perturbation term. Well done.


\textbf{Part c: 4/9}
\begin{itemize}
    \item The explanation acknowledges the variational principle but lacks detailed derivation. The expression for $b$ was not derived, only discussed in a general sense.
\end{itemize}

\textbf{Feedback :}
You need to show the detailed steps of the variational principle application and the specific calculation that leads to the expression for $b$.


Solution 2 :

\textbf{Marks : 10/16}

\textbf{Feedback :}

\textbf{Part a: 3/5}
\begin{itemize}
    \item The normalization process and the calculation for the expectation value of the Hamiltonian contained errors in normalization steps.
\end{itemize}

\textbf{Feedback :}
Ensure accurate handling of normalization conditions and simplify the final expression of the expectation value for clarity.


\textbf{Part b: 2/2}
\begin{itemize}
    \item The perturbation was correctly rewritten in terms of the raising and lowering operators.
\end{itemize}

\textbf{Feedback :}
Good job on this part. Clear and accurate.


\textbf{Part c: 5/9}
\begin{itemize}
    \item The general approach was correct, but the specific derivation of the given expression for $b$ was missing.
\end{itemize}

\textbf{Feedback :}
A solid understanding of the variational principle is demonstrated, but the detailed derivation for $b$ is necessary.


Solution 3 :

\textbf{Marks : 7/16}

\textbf{Feedback :}

\textbf{Part a: 2/5}
\begin{itemize}
    \item The normalization and expectation value calculation were unclear and contained mistakes in the interpretation of normalization.
\end{itemize}

\textbf{Feedback :}
Review the normalization process and ensure that all steps are logically consistent and calculations are accurate.


\textbf{Part b: 2/2}
\begin{itemize}
    \item The perturbation was correctly rewritten using the raising and lowering operators.
\end{itemize}

\textbf{Feedback :}
This part was well done. Clear and correct transformation.


\textbf{Part c: 3/9}
\begin{itemize}
    \item The solution did not proceed to the detailed derivation of the expression for $b$ and was unclear on how to apply the variational principle specifically.
\end{itemize}

\textbf{Feedback :}
Demonstrate the detailed application of the variational principle to show how the expression for $b$ is derived.

\textbf{Overall feedback :}
\begin{itemize}
    \item Pay careful attention to the steps required in normalization and expectation value calculations.
    \item Ensure you provide detailed derivations when asked to show specific results, particularly when using principles like the variational method.
    \item Simplify final expressions where possible for clarity and correctness.
\end{itemize}



\subsubsection*{Response 2 :}

Solution 1 :

\textbf{Marks : 8/16}

\textbf{Feedback :}

\textbf{Part a: 2/5}
\begin{itemize}
    \item The normalization step is correct but the expectation value calculation is incomplete and incorrect in the energy expression.
\end{itemize}

\textbf{Part b: 2/2}
\begin{itemize}
    \item Correctly rewrites the perturbation in terms of the raising and lowering operators.
\end{itemize}

\textbf{Part c: 4/9}
\begin{itemize}
    \item Provides a general description of the variational principle but lacks the detailed derivation required for the given result.
\end{itemize}

\textbf{Feedback :}
Good attempt at normalizing the state and rewriting the perturbation, but the expectation value of the Hamiltonian needs to be calculated more carefully. Part c) requires a more detailed step-by-step derivation to show the minimization explicitly.


Solution 2 :

\textbf{Marks : 10/16}

\textbf{Feedback :}

\textbf{Part a: 3/5}
\begin{itemize}
    \item Correct normalization and the beginning of the expectation value calculation, but makes an error in the final expression.
\end{itemize}

\textbf{Part b: 2/2}
\begin{itemize}
    \item Correctly rewrites the perturbation in terms of the raising and lowering operators.
\end{itemize}

\textbf{Part c: 5/9}
\begin{itemize}
    \item Provides a general description of the variational principle and indicates the correct approach, but does not explicitly derive the given formula for $b$.
\end{itemize}

\textbf{Feedback :}
Good job on normalization and perturbation rewriting. The variational principle explanation is clear but needs more detail in the mathematical derivation to arrive at the final expression for $b$.


Solution 3 :

\textbf{Marks : 7/16}

\textbf{Feedback :}

\textbf{Part a: 2/5}
\begin{itemize}
    \item Attempts normalization and expectation value calculation, but contains errors and lacks clarity.
\end{itemize}

\textbf{Part b: 2/2}
\begin{itemize}
    \item Correctly rewrites the perturbation in terms of the raising and lowering operators.
\end{itemize}

\textbf{Part c: 3/9}
\begin{itemize}
    \item Attempts to explain the variational principle, but the explanation is incomplete and lacks the necessary mathematical rigor to derive the given result.
\end{itemize}

\textbf{Feedback :}
Good start on normalization and perturbation rewriting. The explanation of the variational principle and the minimization process needs to be more thorough and mathematically detailed to fully address the problem.


\subsubsection*{Response 3 :}

Solution 1 :

\textbf{Marks : 12/16}

\textbf{Feedback :}

\textbf{Part a: 4/5}
\begin{itemize}
    \item Normalization is correctly addressed, showing the process and arriving at the correct normalized state.
    \item Expectation value calculation is mostly correct, but there is a mistake in the final energy expectation value.
\end{itemize}

\textbf{Part b: 2/2}
\begin{itemize}
    \item Correctly expresses the perturbation in terms of raising and lowering operators.
\end{itemize}

\textbf{Part c: 6/9}
\begin{itemize}
    \item Explains the variational principle and the general process.
    \item Lacks detailed steps and proper mathematical derivation to reach the given expression for $b$.
\end{itemize}

\textbf{Feedback :}
\begin{itemize}
    \item Good understanding of normalization and the variational principle.
    \item Correct approach in expressing the perturbation.
    \item Lacking detailed derivation in part c).
\end{itemize}


Solution 2 :

\textbf{Marks : 10/16}

\textbf{Feedback :}

\textbf{Part a: 3/5}
\begin{itemize}
    \item Correct approach to normalization but makes a mistake in concluding $|b|^2 = 1$.
    \item Correctly calculates the expectation value but makes an error in final value.
\end{itemize}

\textbf{Part b: 2/2}
\begin{itemize}
    \item Correctly expresses the perturbation in terms of raising and lowering operators.
\end{itemize}

\textbf{Part c: 5/9}
\begin{itemize}
    \item Explains the variational principle.
    \item Lacks detailed steps and proper mathematical derivation to reach the given expression for $b$.
\end{itemize}

\textbf{Feedback :}
\begin{itemize}
    \item Good understanding of the principles.
    \item Needs accuracy in normalization and expectation value calculation.
    \item More detailed derivation in part c) would improve the solution.
\end{itemize}



Solution 3 :

\textbf{Marks : 9/16}

\textbf{Feedback :}

\textbf{Part a: 3/5}
\begin{itemize}
    \item Correct approach to normalization but has incorrect reasoning about $|b|^2 = 1$.
    \item Correctly calculates the expectation value with some mistakes in final value.
\end{itemize}

\textbf{Part b: 2/2}
\begin{itemize}
    \item Correctly expresses the perturbation in terms of raising and lowering operators.
\end{itemize}

\textbf{Part c: 4/9}
\begin{itemize}
    \item Attempts to explain the variational principle.
    \item Explanation is more descriptive and lacks detailed derivation to reach the given expression for $b$.
\end{itemize}

\textbf{General feedback :}
\begin{itemize}
    \item Good understanding of normalization and expectation value calculation.
    \item Accurate perturbation expression.
    \item Descriptive but lacks mathematical rigor in variational principle derivation.
\end{itemize}


\subsubsection*{Response 4 :}

Solution 1 :

\textbf{Marks : 13/16}

\textbf{Feedback :}

\textbf{Part a: 5/5}
\begin{itemize}
    \item Correctly identifies the need to normalize the state.
    \item Correctly uses orthogonality and normalization conditions.
    \item Correctly normalizes the state.
    \item Correct approach to calculate the expectation value.
    \item Correctly substitutes the eigenvalues and calculates the expectation value.
\end{itemize}

\textbf{Part b: 2/2}
\begin{itemize}
    \item Correctly identifies the perturbation form.
    \item Correctly expresses $\hat{x}$ in terms of raising and lowering operators.
\end{itemize}

\textbf{Part c: 6/9}
\begin{itemize}
    \item Correctly states the variational principle.
    \item Describes the correct general approach.
    \item However, does not complete the detailed derivation to show the expression for $b$.
\end{itemize}

\textbf{General Feedback :}
Excellent work on parts a) and b). For part c), the explanation was good, but more detailed steps in the derivation were needed to fully demonstrate the minimization of energy and the resulting expression for $b$.


Solution 2 :

\textbf{Marks : 12/16}

\textbf{Feedback :}

\textbf{Part a: 4/5}
\begin{itemize}
    \item Correctly identifies the need to normalize the state.
    \item Correctly uses orthogonality and normalization conditions.
    \item Incorrect conclusion about $|b|^2 = 1$; the normalization is actually already satisfied.
    \item Correct approach but final numerical result of $2\hbar \omega$ is incorrect; expectation should depend on $b$.
\end{itemize}

\textbf{Part b: 2/2}
\begin{itemize}
    \item Correctly identifies the perturbation form.
    \item Correctly expresses $\hat{x}$ in terms of raising and lowering operators.
\end{itemize}

\textbf{Part c: 6/9}
\begin{itemize}
    \item Correctly states the variational principle.
    \item Describes the correct general approach.
    \item However, does not complete the detailed derivation to show the expression for $b$.
\end{itemize}

\textbf{General Feedback :}
Good work on normalization and perturbation parts. The expectation value calculation had a minor error, and the detailed steps for the variational principle were missing.



Solution 3 :

\textbf{Marks : 11/16}

\textbf{Feedback :}

\textbf{Part a: 3/5}
\begin{itemize}
    \item Correctly identifies the need to normalize the state.
    \item Correctly uses orthogonality and normalization conditions.
    \item Some confusion and incorrect steps in handling the normalization condition.
    \item Correct approach to calculate the expectation value.
    \item Correctly substitutes the eigenvalues and calculates the expectation value but needs clarity.
\end{itemize}

\textbf{Part b: 2/2}
\begin{itemize}
    \item Correctly identifies the perturbation form.
    \item Correctly expresses $\hat{x}$ in terms of raising and lowering operators.
\end{itemize}

\textbf{Part c: 6/9}
\begin{itemize}
    \item Correctly states the variational principle.
    \item EDescribes the correct general approach.
    \item However, does not complete the detailed derivation to show the expression for $b$.
\end{itemize}

\textbf{General feedback :}
The normalization part had some errors and confusion. The approach to parts b) and c) was generally good, but more detailed steps and clarity were needed, especially for the variational principle.




\subsubsection*{Response 5 :}

Solution 1 :

\textbf{Marks : 9/16}

\textbf{Feedback :}

\textbf{Part a: 3/5}
\begin{itemize}
    \item Normalization: This normalization calculation is incorrect; it should be $1 + |b|^2 = N^2$, where $N$ is the normalization constant.
    \item Expectation Value of the Hamiltonian: This part is mostly correct, but the previous error affects it.
\end{itemize}

\textbf{Part b: 2/2}
\begin{itemize}
    \item Correctly rewrites the perturbation in terms of raising and lowering operators.
\end{itemize}

\textbf{Part c: 4/9}
\begin{itemize}
    \item Discusses the variational principle and outlines the process, but lacks detailed steps to derive the specific form of $b$.
\end{itemize}

\textbf{Feedback :}
Good attempt, but there are normalization errors in part a and an incomplete derivation in part c. Clearer steps in deriving the final expression for $b$ are needed.


Solution 2 :

\textbf{Marks : 7/16}

\textbf{Feedback :}

\textbf{Part a: 2/5}
\begin{itemize}
    \item Normalization: Incorrect normalization; it should be $1 + |b|^2 = N^2$
    \item Expectation Value of the Hamiltonian: Some errors in substituting the value of $|b|^2$.
\end{itemize}

\textbf{Part b: 2/2}
\begin{itemize}
    \item Correctly rewrites the perturbation in terms of raising and lowering operators.
\end{itemize}

\textbf{Part c: 3/9}
\begin{itemize}
    \item Mentions the variational principle but does not show detailed steps to derive $b$.
\end{itemize}

\textbf{Feedback :}
The normalization and expectation value calculations need careful attention. Detailed steps in part c are missing, affecting the overall solution.


Solution 3 :

\textbf{Marks : 9/16}

\textbf{Feedback :}

\textbf{Part a: 3/5}
\begin{itemize}
    \item Normalization: Correctly outlines the normalization condition but has errors in simplification.
    \item Expectation Value of the Hamiltonian: Correctly outlines the Hamiltonian but fails to finalize the calculation accurately.
\end{itemize}

\textbf{Part b: 2/2}
\begin{itemize}
    \item Correctly rewrites the perturbation in terms of raising and lowering operators.
\end{itemize}

\textbf{Part c: 4/9}
\begin{itemize}
    \item Mentions the variational principle and acknowledges the complexity but lacks a detailed derivation of $b$.
\end{itemize}

\textbf{Feedback :}
Good conceptual understanding but lacks detailed calculations and final expressions. Ensure all steps are thoroughly detailed, especially in complex derivations.

\textbf{Summary :}
All solutions show a good understanding of the problem but need more detailed and accurate calculations. Especially for part c, ensure to derive the final expressions step by step. Pay careful attention to normalization and expectation value calculations in part a. Keep practicing detailed step-by-step derivations to improve accuracy and clarity in your solutions.


\end{document}