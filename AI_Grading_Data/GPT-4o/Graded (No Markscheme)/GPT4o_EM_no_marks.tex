\documentclass[a4paper,11pt]{article}
\setlength\parindent{0pt}
\usepackage{amsmath, amssymb}
\usepackage[margin=1in,letterpaper]
{geometry} % decreases margins

\usepackage{setspace}
\setstretch{1.2}
\usepackage{spverbatim}

\title{GPT-4o Grading (No Mark Scheme)}

\begin{document}

\maketitle
This document compiles the GPT-4o responses when prompted to grade the solutions which have been generated. No markscheme is given to GPT-4o in the prompt. This is the grading for the Electromagnetic Theory (EM) questions. 

\section{EM}
\subsection*{Question 1:}

\textbf{Response 1:}

Here is the grading and feedback for each solution:

Solution 1

**Grade: 8/8**

**Feedback:**

- **Strengths:** This solution is comprehensive and accurate. It clearly explains the three types of magnetic materials: ferromagnetic, paramagnetic, and diamagnetic, providing specific examples for each. The explanation of how each type of material responds to an external magnetic field is detailed and correctly describes the underlying physics, including the alignment of magnetic domains, the partial alignment of magnetic moments, and the induction of an opposing magnetic moment. The role of electron configuration and spin is well-articulated.

- **Improvements:** No significant improvements needed. This solution is thorough and well-structured.

Solution 2

**Grade: 7/8**

**Feedback:**

- **Strengths:** This solution accurately identifies the three types of magnetic materials and provides a clear explanation of their behavior in an external magnetic field. The examples given are appropriate, and the descriptions of the physical mechanisms (such as alignment of magnetic domains and atomic magnetic moments) are correct. The reference to Lenz's Law in the context of diamagnetic materials is correctly mentioned.

- **Improvements:** The explanation of the disruption of alignment in paramagnetic materials due to thermal motion is a good addition but could be expanded slightly for clarity. Also, while examples for diamagnetic materials are correct, including bismuth as in Solution 1 would enhance completeness.

Solution 3

**Grade: 6/8**

**Feedback:**

- **Strengths:** This solution correctly identifies the three types of magnetic materials and provides a basic explanation of their properties and behaviors. The descriptions of how each type of material responds to an external magnetic field are generally accurate. The mention of the role of unpaired electrons and Lenz's Law is correct.

- **Improvements:** The explanation for paramagnetic materials could be more detailed, particularly regarding why their magnetic moments only slightly align with the external field. The explanation for diamagnetic materials is a bit brief and could benefit from a more detailed discussion of the induced magnetic field opposing the external field. The use of "atomic dipoles" could be clarified as "magnetic moments" to be more precise.

\bigskip
\textbf{Response 2:}

Solution 1: 

**Marks: 8/8**

**Feedback:**
This is an excellent answer. The student accurately identifies and describes ferromagnetic, paramagnetic, and diamagnetic materials. They provide clear explanations of the mechanisms behind each type's response to a magnetic field, including the role of magnetic domains and electron alignment. The examples given for each type are correct and relevant. The explanation of the atomic structure's impact on magnetic response is detailed and accurate.

Solution 2: 

**Marks: 7/8**

**Feedback:**
This is a very good answer. The student correctly identifies the three types of magnetic materials and provides clear explanations for each. The descriptions of ferromagnetic and paramagnetic materials are thorough and include correct examples. However, the explanation for diamagnetic materials could be more detailed regarding the specific mechanisms, such as the role of electron pairing and induced magnetic moments. The overall response is slightly less comprehensive than Solution 1 but still accurate and informative.

Solution 3: 

**Marks: 7/8**

**Feedback:**

This is a very good answer as well. The student correctly identifies the three types of magnetic materials and provides clear explanations. The examples given for each type are correct. The explanation for ferromagnetic materials is particularly strong. However, the explanation for diamagnetic materials, while accurate, could benefit from more detail on the specific atomic mechanisms. The overall response is clear and accurate but lacks some of the depth seen in Solution 1.

\bigskip
\textbf{Response 3:}

Solution 1

**Grade: 8/8**

**Feedback:**

- Comprehensive and detailed explanation.

- Correct identification and explanation of ferromagnetic, paramagnetic, and diamagnetic materials.

- Accurate description of the atomic structure and electron arrangement contributing to their magnetic properties.

- Examples provided are accurate and relevant.

- Explanation of the underlying physics, such as magnetic domains and Lenz's Law, is precise and clear.

Solution 2

**Grade: 7/8**

**Feedback:**

- Correct identification and explanation of ferromagnetic, paramagnetic, and diamagnetic materials.

- Good examples and accurate descriptions of each type of material.

- Clear explanation of the role of magnetic domains and atomic magnetic moments.

- Slightly less detail about the atomic structure and electron behavior compared to Solution 1, particularly in explaining the electron spin and domain alignment.

Solution 3

**Grade: 6/8**

**Feedback:**

- Correct identification and explanation of ferromagnetic, paramagnetic, and diamagnetic materials.

- Good examples provided for each type.

- Adequate explanation of the response of these materials near a bar magnet.

- Less detailed about the specific atomic structures and electron configurations compared to Solution 1.

- Could improve by providing more detailed physical explanations, such as the behavior of magnetic domains and the role of thermal motion in paramagnetic materials.

\bigskip
\textbf{Response 4:}

Here are the grades and feedback for each solution:

Solution 1:

**Grade: 8/8**

**Feedback:**

- This solution is comprehensive and detailed. 

- It accurately identifies the three types of magnetic materials and provides clear explanations for their responses to an external magnetic field.

- The discussion of magnetic domains, electron alignment, and the role of Lenz's Law in diamagnetic materials is precise and informative.

- Excellent use of examples and correct scientific terminology throughout.

Solution 2:

**Grade: 7/8**

**Feedback:**

- This solution is well-structured and mostly correct.

- It correctly identifies the three types of magnetic materials and explains their responses to an external magnetic field.

- The explanation of ferromagnetic and paramagnetic materials is clear, but the effect of thermal motion on paramagnetic materials could be explained more explicitly.

- While the discussion of diamagnetic materials and Lenz's Law is accurate, it could benefit from more detail on how the induced magnetic moment works.

- Overall, a strong answer with minor areas for improvement.

Solution 3:

**Grade: 6/8**

**Feedback:**

- This solution correctly identifies the three types of magnetic materials and provides basic explanations for their behavior near a magnetic field.

- The explanation for ferromagnetic materials is clear and accurate.

- The description of paramagnetic materials could be more detailed, particularly regarding the reason for their weak response.

- The discussion of diamagnetic materials correctly mentions Lenz's Law but could include more detail on how the induced magnetic field opposes the external field.

- While generally correct, the answer lacks some of the depth and precision seen in the other solutions.

\bigskip
\textbf{Response 5:}

Solution 1:

**Grade: 8/8**

**Feedback:**

- Comprehensive and accurate explanations for each type of magnetic material.

- Clear and detailed descriptions of ferromagnetic, paramagnetic, and diamagnetic materials, including the role of magnetic domains and electron alignment.

- Proper examples given for each type of material.

- Well-organized and thorough in explaining the atomic structure's influence on magnetic behavior.

Solution 2:

**Grade: 7/8**

**Feedback:**

- Good explanation of ferromagnetic, paramagnetic, and diamagnetic materials, though slightly less detailed compared to Solution 1.

- Correct examples for each type of material.

- Explanation of thermal motion affecting paramagnetic materials is a useful addition.

- Slightly less emphasis on the role of electronic structure and magnetic domains compared to Solution 1.

Solution 3:

**Grade: 6/8**

**Feedback:**

- Clear and correct definitions of ferromagnetic, paramagnetic, and diamagnetic materials.

- Examples are appropriate and accurate.

- The explanation could benefit from more detail regarding the atomic structure and magnetic domains.

- Some minor repetition and less focus on the differences in the mechanisms behind each type of material's response to magnetic fields.

\subsection*{Question 2:}

\textbf{Response 1:}

**Solution 1:**

**Marks: 5/5**

**Feedback:**

This solution is thorough and clearly articulated. It accurately defines free space and provides a detailed explanation of the given wave equation. The interpretation of the equation in terms of wave propagation is well done, and the derivation of the speed of light is precise, including the correct formula and numerical calculation. Excellent work.

**Solution 2:**

**Marks: 4/5**

**Feedback:**

This solution is strong but not as detailed as Solution 1. The definition of free space is correct, and the explanation of the wave equation and its physical interpretation is clear. The comparison to the standard wave equation is accurate. However, the derivation of the speed of light, while correct, could be more detailed, especially in explaining the analogy more explicitly. Overall, a solid answer.

**Solution 3:**

**Marks: 4/5**

**Feedback:**

This solution is also well-written and provides a correct definition of free space and a good explanation of the wave equation. The physical interpretation is clear, and the derivation of the speed of light is accurate. However, similar to Solution 2, it could benefit from a bit more detail in the analogy with the standard wave equation. Nonetheless, it is a good and accurate response.

\bigskip
\textbf{Response 2:}

Solution 1

**Grade: 5/5**

**Feedback:**

- The definition of "free space" is accurate and well-articulated.

- The explanation of the wave equation is clear, providing insight into the physical significance of both spatial and temporal second derivatives.

- The connection between the given wave equation and the standard form of a wave equation is correctly made.

- The derivation of the speed of light from the constants \(\epsilon_{0}\) and \(\mu_{0}\) is correct, and the calculation is correctly performed.

- Overall, this solution is comprehensive and well-explained.

Solution 2

**Grade: 4/5**

**Feedback:**

- The definition of "free space" is accurate but could be slightly more detailed in terms of its characteristics.

- The explanation of the wave equation is correct, though a bit less detailed compared to Solution 1.

- The comparison to the standard wave equation is appropriate, but the algebraic manipulation could be clearer.

- The speed of light derivation is correct, and the value is accurately given.

- This solution is solid but could benefit from a bit more elaboration on the physical interpretation.

Solution 3

**Grade: 4/5**

**Feedback:**

- The definition of "free space" is accurate and includes the key points about the idealized medium and lack of matter or charges.

- The explanation of the wave equation is correct, but the physical interpretation could be more detailed.

- The comparison to the standard wave equation is appropriate, and the derivation of the speed of light is correctly done.

- The speed of light value is accurately calculated.

- This solution is good but would be improved with a more thorough explanation of the physical significance of the wave equation.

\bigskip
\textbf{Response 3:}

**Solution 1:**

Grade: 5/5

Feedback:

- **Definition of Free Space:** Clearly defined free space as a vacuum devoid of material substance, mentioning its characterization by \(\epsilon_0\) and \(\mu_0\).

- **Wave Equation Explanation:** Accurately described the wave equation, including the significance of the Laplace operator and second time derivative.

- **Physical Interpretation:** Provided a clear explanation of the physical interpretation of the wave equation in terms of electromagnetic wave propagation.

- **Derivation of Speed of Light:** Correctly derived the speed of light from the wave equation and used the correct values for \(\epsilon_0\) and \(\mu_0\) to calculate \( c \).

Solution 2:**

Grade: 4/5

Feedback:

- **Definition of Free Space:** Provided a good definition, noting it is a vacuum without matter or charges.

- **Wave Equation Explanation:** Explained the wave equation and the Laplacian operator accurately.

- **Physical Interpretation:** Briefly interpreted the physical meaning but could have been more detailed about the relationship between electric and magnetic fields.

- **Derivation of Speed of Light:** Correctly derived the speed of light from the wave equation and stated its value accurately. However, the explanation of wave propagation could have been clearer.

Solution 3:**

Grade: 4/5

Feedback:

- **Definition of Free Space:** Defined free space as a perfect vacuum and mentioned the absence of matter or charges.

- **Wave Equation Explanation:** Correctly explained the wave equation and the role of \(\epsilon_0\) and \(\mu_0\).

- **Physical Interpretation:** Described the propagation of electric fields and the relation between spatial and temporal variations well but lacked some depth in explaining the link to magnetic fields.

- **Derivation of Speed of Light:** Correctly derived the speed of light and provided the correct value. The explanation was concise and clear but could have included more on the relationship between electric and magnetic components of the waves.

\bigskip
\textbf{Response 4:}

Solution 1: 

**Marks: 5/5**

**Feedback:**

- The student provides a clear and accurate definition of free space in classical electromagnetism.

- The explanation of the wave equation is thorough, correctly identifying and explaining the roles of the Laplace operator and the time derivatives.

- The comparison to the general form of the wave equation is precise, leading to a correct derivation of the speed of light.

- The numerical calculation for the speed of light is correctly performed and presented.

- Overall, the solution is detailed and demonstrates a strong understanding of the topic.

Solution 2:

**Marks: 4/5**

**Feedback:**

- The definition of free space is accurate but slightly less detailed than Solution 1.

- The explanation of the wave equation is good, although the interpretation of the equation's physical significance could be expanded slightly.

- The derivation of the speed of light is correct, but there is a minor mistake in presenting the comparison to the standard wave equation. The student writes \(\frac{\partial^2 \underline{E}}{\partial t^{2}} = v^2 \nabla^{2}\underline{E}\), which should be the other way around.

- The speed of light value is correctly calculated.

- Overall, the solution is strong but could benefit from a bit more precision and detail.

Solution 3:

**Marks: 4/5**

**Feedback:**

- The definition of free space is accurate and well-stated.

- The explanation of the wave equation and its physical interpretation is solid.

- The student correctly compares the wave equation to the general form and accurately derives the speed of light.

- The speed of light value is correctly stated.

- However, the explanation, while correct, is slightly less detailed than Solution 1, particularly in describing the propagation mechanism.

- Overall, the solution is good but could use more elaboration in some areas.

\bigskip
\textbf{Response 5:}

Grading and Feedback

Solution 1

**Grade: 5/5**

**Feedback:**

- Clear and accurate definition of free space in the context of classical electromagnetism.

- Thorough explanation of the wave equation for the electric field and its components.

- Correct physical interpretation of the equation, emphasizing wave propagation.

- Correct and detailed derivation of the speed of light, including the calculation using \(\epsilon_0\) and \(\mu_0\).

- The explanation is well-structured and covers all aspects of the problem comprehensively.

Solution 2

**Grade: 4/5**

**Feedback:**

- Accurate definition of free space and correct identification of the relevant constants (\(\epsilon_0\) and \(\mu_0\)).

- Good explanation of the wave equation and its components.

- Correct physical interpretation of the wave equation.

- Correct derivation of the speed of light using the equation, but lacks the detailed step-by-step calculation and the values of \(\epsilon_0\) and \(\mu_0\).

- Overall, the explanation is clear but could benefit from including the actual numerical values and calculation to be more complete.

Solution 3

**Grade: 4/5**

**Feedback:**

- Correct definition of free space and its characteristics.

- Accurate description of the wave equation and its components.

- Good physical interpretation of how the wave equation describes electromagnetic wave propagation.

- Correct derivation of the speed of light, but the explanation would be improved by explicitly stating the values of \(\epsilon_0\) and \(\mu_0\) and showing the calculation steps.

- Overall, the solution is correct and clear, but slightly less detailed in the calculation of \(c\) compared to Solution 1.

\subsection*{Question 3:}

\textbf{Response 1:}

Grading and Feedback

**Solution 1:**

- Correct identification of the problem and appropriate use of Coulomb’s law.

- Correct setup of the integral and appropriate integration steps.

- Minor arithmetic errors and misinterpretation of constants.

- Clearly identified the limit for \(d \gg L\) and gave a correct physical interpretation.

- Incorrect final formula: the provided constant \(0.141047395886939\) is incorrect and unnecessary in this form.

**Marks: 11/14**

**Feedback:**

- Good approach and integration, but verify the constants and intermediate steps more carefully.

- Ensure the final formula is simplified correctly.

---

**Solution 2:**

- Correct identification and setup of the integral using Coulomb’s law.

- Correct integration process and explanation of how the components cancel out.

- Properly handled the limit \(d \gg L\) and provided a physical interpretation.

- Similar mistake with the constant factor \(0.141047395886939\).

**Marks: 12/14**

**Feedback:**

- Solid understanding and good application of concepts.

- Minor issue with the final numerical constant. Simplify the final expression correctly.

---

**Solution 3:**

- Correct setup and use of Coulomb’s law.

- Good explanation of integration and handling of vector components.

- Correct final formula but with an incorrect prefactor \(0.564 \sqrt{\pi}\).

- Correct handling of the limit \(d \gg L\) and physical interpretation.

**Marks: 12/14**

**Feedback:**

- Excellent work with detailed explanation and correct methodology.

- Double-check the constants and final simplification for accuracy.

---

Summary

- **Solution 1:** 11/14

- **Solution 2:** 12/14

- **Solution 3:** 12/14

All solutions demonstrated a strong understanding of the principles and methods needed to solve the problem but had minor issues with the numerical constants in their final expressions. Each student showed good problem-solving skills and clear reasoning.

\bigskip
\textbf{Response 2:}

**Solution 1: 10/14**

**Feedback:**

- The method used to find the electric field by integrating over the charge distribution is correct.

- Proper use of Coulomb's law and breakdown of the components of the electric field.

- Correct identification of the symmetry leading to cancellation of horizontal components.

- The final expression for the electric field, however, is not correctly simplified, and there are numerical errors in the constants.

- The limit \( d \gg L \) is correctly approached, but the constant 0.141047395886939 is unnecessary and confusing.

- The physical interpretation is correct but lacks clarity in explaining the reduction to a point charge behavior.

**Solution 2: 11/14**

**Feedback:**

- Similar correct approach using Coulomb's law and integration.

- Proper identification of differential charge elements and their contributions.

- Correctly identifies the horizontal components canceling out.

- The integral is set up correctly, and the limit \( d \gg L \) is handled well.

- The solution, however, uses the same constant as in Solution 1, which is unnecessary.

- Good explanation of the physical interpretation, with a clear conclusion that the electric field resembles that of a point charge at large distances.

**Solution 3: 12/14**

**Feedback:**

- Correct use of Coulomb's law and integration method.

- Proper breakdown of the components and integration limits.

- Accurate calculation of the vertical component and proper handling of the integral.

- The limit \( d \gg L \) is approached correctly, and the simplification is valid.

- The use of Coulomb's constant \( k_e \) instead of \( \frac{1}{4\pi\epsilon_0} \) is noted and correctly handled.

- Good explanation and physical interpretation of the result, correctly identifying the electric field reducing to that of a point charge at large distances.

- The factor 0.564 is derived and explained correctly, making the solution more precise.

**Grading Summary:**

- Solution 1: 10/14

- Solution 2: 11/14

- Solution 3: 12/14

\bigskip
\textbf{Response 3:}

Grading and Feedback

Solution 1:

**Marks: 10/14**

**Feedback:**

- The solution correctly sets up the problem using Coulomb's law and integrates over the line charge.

- The horizontal component cancellation is well-explained.

- The mathematical steps leading to the integral are correct.

- The final expression contains numerical constants that appear to be evaluated incorrectly.

- The limit for \( d \gg L \) is interpreted correctly, showing the reduction to a point charge behavior.

- However, the integral result is not simplified correctly, which affects the clarity of the solution.

Solution 2:

**Marks: 10/14**

**Feedback:**

- The setup and integration process are correct, similarly to Solution 1.

- The horizontal components' cancellation and vertical component derivation are clearly explained.

- The approach to the limit \( d \gg L \) is correct, and the interpretation aligns with the physical intuition of a point charge at large distances.

- The solution contains some errors in the numerical factor, which appear to be miscalculated similarly to Solution 1.

- The final expression and interpretation are clear and concise.

Solution 3:

**Marks: 11/14**

**Feedback:**

- The solution correctly follows Coulomb's law and sets up the problem well.

- The derivation of the vertical component and integration are properly conducted.

- The constants are correctly identified, and the proportionality factor is explained.

- The limit \( d \gg L \) and the interpretation as a point charge are correctly presented.

- The numerical factors and expressions are more precise than in Solutions 1 and 2, providing a clearer final result.

- The factor \( 0.564 \sqrt{\pi} \) is a result of integration and is presented clearly, but there should be more detail on how this factor arises.

General Comments:

- All three solutions demonstrate a strong understanding of the electric field due to a line charge and correctly approach the problem.

- Numerical factors should be checked for accuracy and clarity in the final expression.

- The limit \( d \gg L \) is well-understood and correctly interpreted in all solutions, aligning with the physical intuition of a point charge at large distances.

- Clearer simplification and presentation of the final expressions would enhance the quality of the solutions.

Overall, the students show a good grasp of the concepts, but need to be careful with numerical calculations and final expression clarity.

\bigskip
\textbf{Response 4:}

I will now review the three solutions to the given problem, provide marks out of 14 for each, and offer general feedback.

**Solution 1:**

Marks: 9/14

Feedback:

1. **Correct Use of Coulomb's Law**: The solution starts correctly by setting up the problem using Coulomb's law and integrating over the length of the line charge. 

2. **Symmetry Consideration**: The explanation correctly notes that only the vertical component of the electric field remains due to symmetry.

3. **Integration Setup**: The integral setup is mostly correct, though the limits and specifics could be more clearly detailed.

4. **Integral Evaluation**: The solution includes an evaluated integral, but the final expression provided for the electric field contains constants (like 0.141047395886939) that are not standard or explained, making it unclear how these values were obtained.

5. **Limit Consideration**: The solution correctly addresses the \( d \gg L \) limit but the derived expression's constants are again unclear and not standard.

6. **Physical Interpretation**: The interpretation in terms of Coulomb's law and the behavior of the electric field as a point charge is correct but could be more concise.

Improvements:

- Ensure that constants and steps in the integration are standard and clearly explained.

- Provide a clearer path from the integral setup to the final simplified expression.

**Solution 2:**

Marks: 11/14

Feedback:

1. **Correct Use of Coulomb's Law**: The solution correctly uses Coulomb's law and sets up the integration correctly.

2. **Symmetry and Components**: The explanation effectively covers the cancellation of horizontal components and focuses on the vertical component.

3. **Integration Setup and Details**: The setup and details for the integral are clear, though the exact evaluation steps could be more detailed.

4. **Final Expression and Limit**: The final expression and its limit as \( d \gg L \) are correctly derived, and the constants are more standard.

5. **Physical Interpretation**: The physical interpretation is accurate, describing how the electric field behaves like that of a point charge at large distances.

Improvements:

- Provide a more detailed evaluation of the integral.

- Explain any constants or factors in the final expression more thoroughly.

**Solution 3:**

Marks: 12/14

Feedback:

1. **Correct Use of Coulomb's Law**: The solution sets up the problem correctly using Coulomb's law and explains the steps clearly.

2. **Symmetry and Components**: It correctly addresses the symmetry and focuses on the vertical component.

3. **Integration Setup and Steps**: The integration is set up clearly, and the evaluation steps are detailed, though the result includes a non-standard constant (0.564).

4. **Final Expression and Limit**: The final expression is derived correctly, and the limit as \( d \gg L \) is handled properly.

5. **Physical Interpretation**: The physical interpretation is accurate and clearly explained.

Improvements:

- Clarify the origin and meaning of the constant (0.564) in the context of the integral evaluation.

- Ensure that all steps leading to the final expression are detailed and standard.

---

Overall, Solution 3 is the most comprehensive and clear, though all solutions could benefit from more detailed explanations of constants and integral evaluations.

\bigskip
\textbf{Response 5:}

General Feedback:

Solution 1

1. **Correct Approach**: The student correctly identifies the need to use Coulomb’s law and integrate over the length of the line charge.

2. **Integration Setup**: The integration setup is appropriate, but the solution should show the integral evaluation more explicitly.

3. **Final Expression**: The final expression for the electric field is incorrect and seems overly complex.

4. **Limit Analysis**: The limit as \( d \gg L \) is correctly analyzed, but the final simplified expression has errors.

**Score: 10/14**

Solution 2

1. **Correct Approach**: Similar to Solution 1, the student correctly sets up the problem using Coulomb’s law.

2. **Integration Setup**: The integral setup is clear, and the simplification steps are properly explained.

3. **Final Expression**: The expression for the electric field is closer to the expected form but contains some inconsistencies.

4. **Limit Analysis**: The limit analysis is correct, and the physical interpretation is well-articulated.

**Score: 12/14**

Solution 3

1. **Correct Approach**: The approach is again correct, utilizing Coulomb’s law and integrating over the line charge.

2. **Integration Setup**: The integral setup and evaluation steps are less detailed compared to the other solutions.

3. **Final Expression**: The final expression contains a numerical factor that seems incorrect.

4. **Limit Analysis**: The limit analysis is correct, but the physical interpretation is less clearly explained compared to Solution 2.

**Score: 11/14**

Problem Solution:

To determine the electric field at a distance \( d \) above one end of a line segment of length \( L \) with a constant line charge density \( \lambda \), follow these steps:

1. **Setup**: 
    - Consider an infinitesimal element \( dq = \lambda dx \) at a distance \( x \) from the end of the line segment.
    - The distance from the element to the point \( P \) is \( r = \sqrt{d^2 + x^2} \).
    - The vertical component of the electric field \( dE \) due to \( dq \) is:
      \[
      dE_y = \frac{1}{4\pi\epsilon_0} \frac{dq}{r^2} \frac{d}{r} = \frac{1}{4\pi\epsilon_0} \frac{\lambda dx}{(d^2 + x^2)} \frac{d}{\sqrt{d^2 + x^2}}
      \]
      \[
      dE_y = \frac{1}{4\pi\epsilon_0} \frac{\lambda d\, dx}{(d^2 + x^2)^{3/2}}
      \]

2. **Integration**: Integrate from \( x = 0 \) to \( x = L \) to find the total electric field:
    \[
    E_y = \int_0^L \frac{1}{4\pi\epsilon_0} \frac{\lambda d\, dx}{(d^2 + x^2)^{3/2}}
    \]
    \[
    E_y = \frac{\lambda d}{4\pi\epsilon_0} \int_0^L \frac{dx}{(d^2 + x^2)^{3/2}}
    \]
    Using the integral result:
    \[
    \int_0^L \frac{dx}{(d^2 + x^2)^{3/2}} = \frac{L}{d\sqrt{d^2 + L^2}}
    \]
    Therefore:
    \[
    E_y = \frac{\lambda d}{4\pi\epsilon_0} \cdot \frac{L}{d (d^2 + L^2)^{1/2}} = \frac{\lambda L}{4\pi\epsilon_0 (d^2 + L^2)^{1/2}}
    \]

3. **Limit Analysis**: In the limit \( d \gg L \):
    \[
    (d^2 + L^2)^{1/2} \approx d \quad \text{(since \( L^2 \) becomes negligible)}
    \]
    \[
    E_y \approx \frac{\lambda L}{4\pi\epsilon_0 d}
    \]
    This result suggests that at large distances, the electric field appears as if it originates from a point charge \( Q = \lambda L \).

Revised Scores:

1. **Solution 1**: 10/14

2. **Solution 2**: 12/14

3. **Solution 3**: 11/14

\subsection*{Question 4:}

\textbf{Response 1:}

Let's evaluate the solutions based on the given physics problem and the associated grading criteria.

Problem:

"A square loop of wire (sides of length \( L \)) lies a fixed distance \( d \) from a very long straight wire, which carries a constant current \( I \) directed downwards (as seen in the figure).

a) Find the flux of \( \underline{B} \) through the loop.

b) If the loop is moved directly away from the wire to the right, at speed \( v \), what emf is generated? In what direction (clockwise or counterclockwise) does the current flow?

c) What happens if the loop only moves downwards at speed \( v \)?"

Solution 1:

a) Flux of \( \underline{B} \) through the loop:

Correct method and equations:
\[
\Phi = \frac{\mu_0 I L}{2 \pi} \ln\left(\frac{d+L}{d}\right)
\]

Marks: 5/5

b) EMF generated when the loop is moved to the right:

Correct application of Faraday's Law, but the final calculation is incorrect. 
\[
\mathcal{E} = -\frac{\mu_0 I L^2 v}{2 \pi d(d+L)}
\]
Lenz's law correctly predicts counterclockwise current.

Marks: 4/5

c) Moving downwards at speed \( v \):

Correct reasoning: No change in flux, no emf induced.

Marks: 1/1

**Total: 10/11**

**Feedback: Well-reasoned and accurate flux calculation, minor calculation error in emf formula derivation.**

Solution 2:

a) Flux of \( \underline{B} \) through the loop:

Correct integral setup, but final answer provided without explicit calculation:
\[
\Phi_B = \frac{\mu_0 I L}{2 \pi} \ln\left(\frac{d+L}{d}\right)
\]

Marks: 4/5

b) EMF generated when the loop is moved to the right:

Incorrect differentiation and conclusion. Flux does depend on \( d \).

Marks: 2/5

c) Moving downwards at speed \( v \):

Correct reasoning: No change in flux, no emf induced.

Marks: 1/1

**Total: 7/11**

**Feedback: Correct approach for flux calculation, but emf calculation contains a major conceptual error.**

Solution 3:

a) Flux of \( \underline{B} \) through the loop:

Correct method and equations:
\[
\Phi = \frac{\mu_0 I L}{2 \pi} \ln\left(\frac{d+L}{d}\right)
\]

Marks: 5/5

b) EMF generated when the loop is moved to the right:

Correct differentiation:
\[
\mathcal{E} = -\frac{d}{dt}\left(\frac{\mu_0 I L}{2 \pi} \ln\left(\frac{d+vt+L}{d+vt}\right)\right)
\]
Resulted in:
\[
\mathcal{E} = -\frac{\mu_0 I L v}{2 \pi d (d+L)}
\]
Correct application of Lenz's law: counterclockwise.

Marks: 5/5

c) Moving downwards at speed \( v \):

Correct reasoning: No change in flux, no emf induced.

Marks: 1/1

**Total: 11/11**

**Feedback: Accurate and complete calculations throughout, correctly applied Faraday's and Lenz's laws.**

Summary of Marks:

- **Solution 1:** 10/11

- **Solution 2:** 7/11

- **Solution 3:** 11/11

Solution 3 provided the most accurate and complete responses across all parts of the problem.

\bigskip
\textbf{Response 2:}


I will now grade each solution based on the problem statement and provided criteria, which includes 5 marks for part (a), 5 marks for part (b), and 1 mark for part (c). 

Solution 1:

**Part a:**

- Correct use of Ampère's law to find \( B \).

- Correct setup of the integral for magnetic flux.

- Correct integration and final expression for the flux.

**Score for part a: 5/5**

**Part b:**

- Correct application of Faraday's law of induction.

- Correct identification of the variable \( d \) changing over time.

- Incorrect final expression for the emf; however, the approach was on the right track.

- Correct determination of the direction of the current (counterclockwise).

**Score for part b: 4/5**

**Part c:**

- Correctly identified that no emf is induced if the loop moves downwards.

**Score for part c: 1/1**

**Total Score: 10/11**

**General Feedback:** Excellent understanding of magnetic flux and Faraday's law. The minor mistake in the final emf calculation should be carefully checked.

---

Solution 2:

**Part a:**

- Correct use of Ampère's law.

- Correct setup for the flux integral.

- Incorrect result stating that flux is independent of \( d \); this is conceptually incorrect as it affects the integration limits.

**Score for part a: 3/5**

**Part b:**

- Correct conceptual approach using Faraday's law.

- Miscalculation in differentiation leading to zero emf, which is incorrect since the flux does depend on \( d \).

- Correct reasoning for the direction of the current (counterclockwise).

**Score for part b: 2/5**

**Part c:**

- Correct identification that no emf is induced if the loop moves downwards.

**Score for part c: 1/1**

**Total Score: 6/11**

**General Feedback:** Good conceptual understanding but significant calculation errors. Ensure accurate integration and differentiation steps.

---

Solution 3:

**Part a:**

- Correct use of Ampère's law.

- Correct setup and evaluation of the integral for flux.

- Correct final expression for the flux.

**Score for part a: 5/5**

**Part b:**

- Correct application of Faraday's law.

- Correct identification of the changing \( d \).

- Correct final expression for emf and proper explanation using Lenz's law.

- Correct determination of the direction of the current (counterclockwise).

**Score for part b: 5/5**

**Part c:**

- Correct identification that no emf is induced if the loop moves downwards.

**Score for part c: 1/1**

**Total Score: 11/11**

**General Feedback:** Excellent solution with correct calculations and conceptual understanding. Well done!

---

Summary:

- **Solution 1:** 10/11

- **Solution 2:** 6/11

- **Solution 3:** 11/11

\bigskip 
\textbf{Response 3:}

Great, I see you’ve uploaded an image to help with the problem. Here is the grading for each solution based on the specified criteria:

Solution 1

Part a) (Out of 5)

- The approach to find the magnetic field and the integration over the loop area are correct.

- Correctly identifies the need to integrate from \(d\) to \(d+L\).

- Uses proper limits and the natural logarithm function.

**Marks: 5/5**

Part b) (Out of 5)

- Correctly uses Faraday's law to find the emf.

- Correctly identifies the direction of current using Lenz's law.

- The differentiation step seems unnecessarily complicated and might have mistakes.

**Marks: 4/5**

Part c) (Out of 1)

- Correctly states that no emf is induced if the loop moves downward since the distance \(d\) doesn't change.

**Marks: 1/1**

**Total: 10/11**

**Feedback: Very good solution overall. Part b) is mostly correct but could be simplified.**

Solution 2

Part a) (Out of 5)

- Correct approach and use of integration.

- Correctly sets up the integral and computes it properly.

- Correctly identifies the limits and the natural logarithm function.

- However, the final expression for \(\Phi_B\) is simplified incorrectly, leading to an incorrect result.

**Marks: 4/5**

Part b) (Out of 5)

- Uses Faraday's law correctly.

- Misunderstands the dependence of \(\Phi_B\) on \(d\), leading to an incorrect conclusion about the emf being zero.

**Marks: 2/5**

Part c) (Out of 1)

- Correctly identifies no emf is induced if the loop moves downward.

**Marks: 1/1**

**Total: 7/11**

**Feedback: Good attempt, but the incorrect final expressions in part a) and the misunderstanding in part b) affected the correctness of the solutions.**

Solution 3

Part a) (Out of 5)

- Correctly sets up the magnetic flux calculation.

- Uses proper integration limits and natural logarithm function.

- However, the final expression for \(\Phi\) is simplified incorrectly, leading to an incorrect result.

**Marks: 4/5**

Part b) (Out of 5)

- Correctly uses Faraday's law.

- The differentiation step is done correctly.

- Correctly identifies the direction of current.

- However, the final expression for the emf is incorrect due to an incorrect final expression for \(\Phi\).

**Marks: 3/5**

Part c) (Out of 1)

- Correctly states that no emf is induced if the loop moves downward.

**Marks: 1/1**

**Total: 8/11**

**Feedback: Solid understanding, but errors in simplification led to incorrect final results in parts a) and b).**

Summary

- **Solution 1:** 10/11

- **Solution 2:** 7/11

- **Solution 3:** 8/11

Solution 1 demonstrates a strong understanding and correct approach, with only a minor issue in part b). Solution 2 and Solution 3 have errors in simplification that affected their final results.

\bigskip 
\textbf{Response 4:}

Here is the grading and feedback for each solution provided by the students:

Solution 1:

**a) Flux Calculation:**

- Correct approach to integrate the magnetic field to find the flux.

- Correctly set up the integral and performed the integration.

- Correct final expression: \(\Phi = \frac{\mu_0 I L}{2 \pi} \ln\left(\frac{d+L}{d}\right)\).

**Marks:** 5/5

**b) EMF Calculation:**

- Correct approach using Faraday's law.

- Error in the differentiation process.

- Incorrect final expression for the induced emf.

- Correct explanation of Lenz's law and direction of current.

**Marks:** 3/5

**c) Downward Movement:**

- Correct explanation that there is no change in flux, and hence no emf is induced.

**Marks:** 1/1

**Total Marks:** 9/11

**Feedback:** The flux calculation is well done and correct. However, there was an error in the differentiation process for the emf calculation. The understanding of Lenz's law and direction of current is clear.

Solution 2:

**a) Flux Calculation:**

- Correct expression for the magnetic field.

- Set up the integral correctly.

- Error in final flux expression \(\Phi_B = \frac{I L \mu_0 (L + d)}{2 \pi} - \frac{I L d \mu_0}{2 \pi} = \frac{I L^2 \mu_0}{2 \pi}\), which is incorrect as it should depend on \(d\) and \(L\).

**Marks:** 2/5

**b) EMF Calculation:**

- Correct use of Faraday's law.

- Incorrect differentiation and resulting in emf being zero.

- Incorrect final conclusion that emf is zero.

- Correct explanation of Lenz's law.

**Marks:** 1/5

**c) Downward Movement:**

- Correct explanation that no emf is induced due to no change in flux.

**Marks:** 1/1

**Total Marks:** 4/11

**Feedback:** The flux calculation contains an error, leading to an incorrect emf calculation. There is a misunderstanding regarding the dependency of flux on \(d\). The explanation of why no emf is induced during downward movement is accurate.

Solution 3:

**a) Flux Calculation:**

- Correct setup of integral and proper use of the magnetic field equation.

- Correct final expression for the flux: \(\Phi = \frac{L I \mu_0}{2\pi} \left( \log(L + d) - \log(d) \right)\).

**Marks:** 5/5

**b) EMF Calculation:**

- Correct application of Faraday's law and correct differentiation process.

- Correct final expression for the emf: \(\mathcal{E} = \frac{I L^2 \mu_0}{2\pi d (L + d)}\).

- Correct explanation of Lenz's law and the direction of the current.

**Marks:** 5/5

**c) Downward Movement:**

- Correct explanation that there is no change in flux, and hence no emf is induced.

**Marks:** 1/1

**Total Marks:** 11/11

**Feedback:** Excellent work. The flux calculation, emf determination, and explanation of the current direction are all correct. The explanation for the lack of emf during downward movement is clear and concise.

---

**Summary:**

- **Solution 1:** 9/11 - Strong in flux calculation and understanding of current direction, but with a differentiation error in emf calculation.

- **Solution 2:** 4/11 - Mistakes in flux calculation leading to incorrect emf conclusions.

- **Solution 3:** 11/11 - Perfectly accurate and thorough in all parts.

\bigskip 
\textbf{Response 5:}

Great, you've uploaded the image for the problem. Let's proceed with grading the solutions based on the given criteria.

Solution 1

Part (a)

- Correctly identifies the magnetic field due to a long straight wire using Ampère's law.

- Properly sets up the integral to find the flux through the loop.

- Correct integration and application of limits.

- Correctly includes the cos(theta) factor and explains why it is 1.

- Correctly simplifies the integral to the final form.

**Marks: 5/5**

Part (b)

- Correctly identifies that the emf is given by the rate of change of flux.

- Correctly identifies that \( d \) increases with time and uses \( d = vt \).

- Proper differentiation with respect to time.

- Correctly uses Lenz's law to determine the direction of the current.

**Marks: 5/5**

Part (c)

- Correctly identifies that moving the loop downwards does not change the flux.

- Correctly concludes that no emf is induced.

**Marks: 1/1**

**Total: 11/11**

**Feedback:** Excellent work. All parts are correct and well-explained. Your solution is detailed and accurate.

Solution 2

Part (a)

- Correctly identifies the magnetic field due to a long straight wire using Ampère's law.

- Sets up the integral correctly to find the flux through the loop.

- However, the final flux expression given is incorrect. The flux depends on \( d \) and the integral needs proper limits.

- Error in the flux calculation: \(\Phi_B = \frac{I L^2 \mu_0}{2 \pi}\) is incorrect.

**Marks: 3/5**

Part (b)

- Correctly identifies Faraday's law and the concept of induced emf.

- Incorrect calculation in part (a) leads to incorrect conclusion that emf is zero.

- Incorrect conclusion about the emf being zero when moving to the right.

**Marks: 2/5**

Part (c)

- Correctly identifies that moving the loop downwards does not change the flux.

- Correctly concludes that no emf is induced.

**Marks: 1/1**

**Total: 6/11**

**Feedback:** Good attempt, but there is an error in the flux calculation for part (a) that leads to incorrect conclusions in part (b). Part (c) is correct.

Solution 3

Part (a)

- Correctly identifies the magnetic field due to a long straight wire using Ampère's law.

- Sets up the integral correctly to find the flux through the loop.

- Final expression for the flux is correct.

- Proper explanation and setup.

**Marks: 5/5**

Part (b)

- Correctly identifies Faraday's law and the concept of induced emf.

- Correctly identifies \( d = vt \).

- Proper differentiation and correct calculation of emf.

- Correctly uses Lenz's law to determine the direction of the current.

**Marks: 5/5**

Part (c)

- Correctly identifies that moving the loop downwards does not change the flux.

- Correctly concludes that no emf is induced.

**Marks: 1/1**

**Total: 11/11**

**Feedback:** Excellent solution. All parts are correct and well-explained. Your explanations and calculations are clear and accurate.

---

Overall, solutions 1 and 3 are perfect, while solution 2 has a significant error in part (a) that affects part (b).


\subsection*{Question 5:}

\textbf{Response 1:}

I see you have uploaded an image file. Let me include it in the context of the problem and solutions. Here's the updated problem statement incorporating the image:

---

**Problem:**

"Consider the circuit diagram seen in the figure below.

![Circuit Diagram](sandbox:/mnt/data/EMq5figure.PNG)

1. Calculate the current seen by the ammeter.
2. Calculate the energy delivered by the \(12\, \text{V}\) battery in \(4\) seconds."

---

Let's proceed to grade each solution based on this context.

---

**Solution 1:**

**Part a) Calculation of current seen by the ammeter**

- **Step-by-step process (3/8 marks):** The student identified circuit elements and described the analysis steps in detail. However, they did not complete the necessary calculations to find the equivalent resistance and did not derive the final current.

- **Calculation and Final Answer (0/8 marks):** The student didn't finish the calculations correctly and thus didn't arrive at the right answer.

**Part b) Calculation of energy delivered by the \(12\, \text{V}\) battery in \(4\) seconds**

- **Correct method and steps (2/3 marks):** The student correctly identified the formulas to use and demonstrated a clear understanding of the energy calculation.

- **Correct calculation and final answer (0/3 marks):** However, the power value and subsequent energy calculation are incorrect because the incorrect current was used.

**Total for Solution 1: 5/11 marks**

**Feedback:**
- The methodology is sound but needs completion and accuracy in the calculations.
- Ensure that intermediate steps are correctly computed to avoid errors in final answers.

---

**Solution 2:**

**Part a) Calculation of current seen by the ammeter**

- **Step-by-step process (4/8 marks):** The student correctly started the analysis but incorrectly combined the resistances and overlooked the impact of the second voltage source.

- **Calculation and Final Answer (1/8 marks):** The student reached a numerical answer but used incorrect reasoning and calculations.

**Part b) Calculation of energy delivered by the \(12\, \text{V}\) battery in \(4\) seconds**

- **Correct method and steps (2/3 marks):** The student demonstrated the correct use of power and energy formulas.

- **Correct calculation and final answer (1/3 marks):** The calculation was correct based on their previous answer, though it was based on an incorrect current value.

**Total for Solution 2: 8/11 marks**

**Feedback:**

- Your approach is mostly correct, but careful attention to the circuit's components and their interactions is necessary.

- Be precise in calculating equivalent resistances and the influence of all voltage sources.

---

**Solution 3:**

**Part a) Calculation of current seen by the ammeter**

- **Step-by-step process (6/8 marks):** The student provided a thorough analysis but made a mistake in simplifying the circuit.

- **Calculation and Final Answer (2/8 marks):** The student arrived at an incorrect numerical answer due to a miscalculation.

**Part b) Calculation of energy delivered by the \(12\, \text{V}\) battery in \(4\) seconds**

- **Correct method and steps (2/3 marks):** The energy calculation method is sound.

- **Correct calculation and final answer (1/3 marks):** The final energy delivered calculation used an incorrect current value.

**Total for Solution 3: 11/11 marks**

**Feedback:**

- Your analysis is clear and methodologically sound, but ensure accurate equivalent resistance calculations.

- Double-check steps to prevent small errors that can lead to incorrect final answers.

---

Given these evaluations, the total scores are as follows:

- **Solution 1:** 5/11

- **Solution 2:** 8/11

- **Solution 3:** 7/11

\bigskip 
\textbf{Response 2:}

Let's analyze each solution provided by the students, and grade them accordingly:

**Solution 1:**

This solution provides a detailed breakdown of the steps needed to solve both parts of the problem. However, there are some inaccuracies in the calculations and assumptions. Let's go through it step by step.

Part a) Calculate the current seen by the ammeter:

1. **Identifying the circuit elements**: The identification is correct.

2. **Analyze the circuit**: The description is somewhat correct, but the solution misidentifies some connections.

3. **Calculate equivalent resistance**: The student doesn't complete the calculation for the equivalent resistance correctly.

The calculation for the current has mistakes, leading to an incorrect final answer.

Part b) Calculate the energy delivered by the \(12\,V\) battery in \(4\) seconds:

1. **Use the power formula**: The formula application is correct.

2. **Calculate energy**: The formula application is correct, but since the current in part a is wrong, the final energy value is also incorrect.

**Marks:**

- Part a: 3/8 (identification and analysis are partially correct, but the final calculation is incorrect)

- Part b: 1/3 (correct approach but wrong value due to error in part a)

**Total: 4/11**

**Feedback:**

Good approach and clear steps, but the execution of calculations needs to be improved, especially in finding the correct equivalent resistance and current.

---

**Solution 2:**

This solution also provides a clear step-by-step approach to solving the problem. Let's examine the steps and calculations.

Part a) Calculate the current seen by the ammeter:

1. **Total resistance calculation**: The solution correctly identifies the need to calculate the series and parallel resistances but makes an error in the equivalent resistance of the series combination with the \(2V\) cell.

2. **Using Ohm's law**: The application of Ohm's law is correct, but since the resistance calculation is wrong, the current is incorrect.

### Part b) Calculate the energy delivered by the \(12\,V\) battery in \(4\) seconds:

1. **Use the power formula**: Correct.

2. **Calculate energy**: Correct, but based on an incorrect current.

**Marks:**

- Part a: 5/8 (correct approach and clearer identification, but some errors in calculation)

- Part b: 2/3 (correct approach, but incorrect final value due to error in part a)

**Total: 7/11**

**Feedback:**

Clear steps and logical progression, but there are errors in the equivalent resistance calculations which lead to incorrect final answers.

---

**Solution 3:**

This solution follows a systematic approach similar to the others. Let's see how it handles the calculations.

Part a) Calculate the current seen by the ammeter:

1. **Calculate equivalent resistance**: The solution explains the process but makes an error in calculating the equivalent resistance and the voltage drops.

2. **Ohm's law application**: The final current calculation is incorrect due to earlier mistakes.

Part b) Calculate the energy delivered by the \(12\,V\) battery in \(4\) seconds:

1. **Use the power formula**: Correct.

2. **Calculate energy**: Correct, but based on an incorrect current.

**Marks:**

- Part a: 4/8 (good approach but incorrect calculations)

- Part b: 1/3 (correct approach, but incorrect value)

**Total: 5/11**

**Feedback:**

Logical steps and good attempt at solving the problem, but needs more accuracy in resistance and current calculations.

---

**Summary of Grades:**

- **Solution 1:** 4/11

- **Solution 2:** 7/11

- **Solution 3:** 5/11

The main issue in all solutions is the incorrect calculation of the equivalent resistance, which affects the final current and energy calculations. Improved attention to detail in the calculation steps is needed to achieve correct results.

\bigskip 
\textbf{Response 3:}

I've reviewed the problem and each of the three solutions provided by the students. Here are the grades and feedback for each solution.

Solution 1

**Part a) Current Calculation:**

- The approach begins by identifying circuit elements and analyzing the circuit configuration.

- The calculation steps described are correct up to identifying the series and parallel combinations.

- However, the final calculation steps and result of 18A are incorrect due to a misunderstanding of the circuit configuration.

- A significant error in assuming the current without solving for the correct equivalent resistance and voltage distribution.

**Part b) Energy Calculation:**

- Correct approach to calculating power using \( P = VI \).

- However, this is based on the incorrect current value found in part a.

- Therefore, the final energy calculation is incorrect.

**Marks:**

- Part a: 3/8 (Good start, but the final calculations and understanding of the circuit configuration are incorrect.)

- Part b: 1/3 (Correct approach, but incorrect result due to the error in part a.)

**Total: 4/11**

**Feedback:** You identified the circuit elements and started correctly, but the final calculations and understanding of the circuit were incorrect. Make sure to check the equivalent resistance and voltage distribution in the circuit thoroughly.

Solution 2

**Part a) Current Calculation:**

- Correctly identifies the need to calculate the total resistance.

- Correct series combination calculation.

- Incorrectly calculates the voltage drop due to misunderstanding the voltage distribution.

- Correctly applies the parallel resistance formula but arrives at an incorrect equivalent resistance.

- Final current calculation of 10A is incorrect.

**Part b) Energy Calculation:**

- Correctly uses the power formula \( P = VI \).

- Calculation based on incorrect current value, leading to an incorrect energy result.

**Marks:**

- Part a: 5/8 (Accurate steps and approach, but some critical calculation errors.)

- Part b: 2/3 (Correct method, but final answer is incorrect due to error in part a.)

**Total: 7/11**

**Feedback:** Good attempt with the right approach. Ensure to carefully verify each step, especially the voltage and resistance calculations, to avoid errors.

Solution 3

**Part a) Current Calculation:**
- Correctly identifies the need to calculate equivalent resistance.

- Provides detailed calculation steps for series and parallel combinations.

- Assumes incorrect voltage drop across certain resistors, leading to a calculation error.

- Incorrectly concludes with a current of 18A.

**Part b) Energy Calculation:**

- Correct approach using \( P = VI \) and energy formula \( E = Pt \).

- Incorrect due to the error in part a.

**Marks:**

- Part a: 4/8 (Detailed steps, but critical errors in voltage assumptions and final calculation.)

- Part b: 1/3 (Correct method, but based on incorrect current value.)

**Total: 5/11**

**Feedback:** Detailed work and good understanding of the concepts. However, verify voltage assumptions and double-check calculations to ensure accuracy.

Correct Solution

To provide a correct solution:

**Part a) Current Calculation:**

1. Combine the \(1 \Omega\) resistors in series: \(1 \Omega + 1 \Omega = 2 \Omega\).

2. The \(2 \Omega\) resistor and \(2V\) battery combination: Effective voltage drop calculation needed.

3. Find the total equivalent resistance.

4. Apply Ohm's law with the \(12V\) battery.

**Part b) Energy Calculation:**

1. Use \( P = VI \) with the correct current.

2. Calculate \( E = Pt \).

Given the errors, let's provide the correct solution:

1. Equivalent resistance of the circuit 
considering the \(2 \Omega\) and \(2V\) battery.

2. Calculate the correct current.

3. Use \( P = VI \) to find energy delivered.

By applying these steps accurately, one would get the correct values.

\bigskip 
\textbf{Response 4:}

Let's start by grading each solution based on the provided problem and the corresponding answers.

Solution 1

**Part a:**

- Identifies the circuit elements but does not accurately calculate the equivalent resistance and misinterprets the configuration of resistors and batteries.

- Assumes an incorrect internal resistance for the 2V battery and simplifies without proper calculations.

- Incorrectly concludes the total current as 18A.

Marks: 3/8

**Part b:**

- Correctly uses power and energy formulas but uses an incorrect current value from part a.

- Incorrect calculation of power and energy.

Marks: 1/3

**Total Marks: 4/11**

**Feedback:**

The solution shows a good understanding of the basic principles, but there are significant errors in the calculations, especially in handling the resistor combinations and the battery configurations.

---

Solution 2

**Part a:**

- Properly identifies the parallel and series resistor combinations.

- Calculates the total equivalent resistance and current correctly.

- Uses correct steps and Ohm's law accurately to determine the current.

Marks: 8/8

**Part b:**

- Correctly uses the power formula and calculates energy delivered by the battery accurately.

Marks: 3/3

**Total Marks: 11/11**

**Feedback:**

This solution accurately and thoroughly addresses both parts of the problem with correct calculations and clear logical steps.

---

Solution 3

**Part a:**

- Identifies the parallel and series resistances but incorrectly interprets the voltage drops and configurations.

- Incorrect calculation for the equivalent resistance and total current, leading to an incorrect current of 18A.

Marks: 3/8

**Part b:**

- Correctly uses power and energy formulas but relies on the incorrect current value from part a.

- Incorrect calculation of power and energy.

Marks: 1/3

**Total Marks: 4/11**

**Feedback:**
While the solution demonstrates an understanding of the principles involved, errors in the interpretation of the circuit configuration lead to incorrect calculations.

---

Summary

- Solution 1: 4/11

- Solution 2: 11/11

- Solution 3: 4/11

Solution 2 is the most accurate and complete, correctly solving both parts of the problem. Solutions 1 and 3 have similar errors in the calculation of the equivalent resistance and current, affecting their accuracy.

\bigskip 
\textbf{Response 5:}

Grading of Solutions

Solution 1:

**Part a)**:

- Correct identification of circuit elements and approach.

- Good explanation of series and parallel resistor combinations.

- Misses calculation steps for equivalent resistance in parallel and total resistance.

- Incorrectly assumes the current through the ammeter.

**Marks for Part a: 5/8**

**Part b)**:

- Correct use of power and energy formulas.

- Incorrect total current value leads to incorrect power and energy calculations.

**Marks for Part b: 1/3**

**Total Marks: 6/11**

**Feedback**: Good initial steps in identifying the circuit elements and approach. However, it lacks detail in the equivalent resistance calculation and has incorrect final current and energy values.

Solution 2:

**Part a)**:

- Good identification of series and parallel resistor combinations.

- Correct calculation of total resistance.

- Correct use of Ohm's law to find total current.

**Marks for Part a: 8/8**

**Part b)**:

- Correct use of power and energy formulas.

- Correct calculation of power and energy with the given time.

**Marks for Part b: 3/3**

**Total Marks: 11/11**

**Feedback**: Excellent solution with correct analysis and calculations. Clearly explained steps and accurate results.

Solution 3:

**Part a)**:

- Good initial steps and identification of circuit elements.

- Some misinterpretations in calculating voltage drops and equivalent resistance.

- Incorrect use of the current division rule.

- Incorrect final current value.

**Marks for Part a: 4/8**

**Part b)**:

- Correct use of power and energy formulas.

- Incorrect total current value leads to incorrect power and energy calculations.

**Marks for Part b: 1/3**

**Total Marks: 5/11**

**Feedback**: Decent initial steps in analyzing the circuit but incorrect interpretations of voltage drops and equivalent resistance calculations. The final current and energy values are incorrect.

Summary of Solutions:

- **Solution 1**: 6/11

- **Solution 2**: 11/11

- **Solution 3**: 5/11


\subsection*{Question 6:}

\textbf{Response 1:}

Grading and Feedback

Solution 1:

- **Part a:** 3/3

  - Correctly identified the relationship between the electric field and potential difference: \( E = \frac{V}{d} \).
  
  - Calculated the electric field correctly as \( 800 \, \text{N/C} \).

- **Part b:** 2/2

  - Correctly used the electric field to find the force: \( F = qE \).
  
  - Correctly calculated the acceleration using Newton's second law: \( a = \frac{F}{m} \).

**Total:** 5/5

**Feedback:** Excellent solution. All steps are clear and calculations are accurate. Well done on applying the principles correctly and arriving at the correct numerical answers.

Solution 2:

- **Part a:** 3/3

  - Correctly identified the relationship between the electric field and potential difference: \( E = \frac{V}{d} \).
  
  - Calculated the electric field correctly as \( 800 \, \text{N/C} \).

- **Part b:** 2/2

  - Correctly used the electric field to find the force: \( F = qE \).
  
  - Correctly calculated the acceleration using Newton's second law: \( a = \frac{F}{m} \).

**Total:** 5/5

**Feedback:** Excellent work. Clear and logical presentation of each step leading to accurate calculations. The use of correct physical principles and consistency in units is commendable.

Solution 3:

- **Part a:** 3/3

  - Correctly identified the relationship between the electric field and potential difference: \( E = \frac{V}{d} \).
  
  - Calculated the electric field correctly as \( 800 \, \text{N/C} \).

- **Part b:** 2/2

  - Correctly used the electric field to find the force: \( F = qE \).
  
  - Correctly calculated the acceleration using Newton's second law: \( a = \frac{F}{m} \).

**Total:** 5/5

**Feedback:** Excellent solution. All calculations are correct and the steps are well-explained. Good job on converting units appropriately and applying the physical laws accurately.

\bigskip
\textbf{Response 2:}

Here are the grades and feedback for each student's solution:

**Solution 1:**

*Part a) Determine the Electric Field*

Grade: 3/3

Feedback: The solution correctly identifies the relationship between potential difference and electric field, and uses \( E = \frac{V}{d} \) accurately. The calculated value of \( E = 800 \, \text{N/C} \) is correct.

*Part b) Determine the Acceleration*

Grade: 2/2

Feedback: The solution correctly applies \( F = qE \) and Newton's second law \( a = \frac{F}{m} \). The final acceleration \( a = 34.78 \, \text{m/s}^2 \) is correct.

**Total: 5/5**

**Solution 2:**

*Part a) Determine the Electric Field*

Grade: 3/3

Feedback: The solution correctly explains the concept of the electric field due to parallel plates and uses \( E = \frac{V}{d} \) to calculate \( E = 800 \, \text{N/C} \) correctly.

*Part b) Determine the Acceleration*

Grade: 2/2

Feedback: The solution accurately applies the correct formulas and correctly calculates the acceleration \( a = 34.78 \, \text{m/s}^2 \).

**Total: 5/5**

**Solution 3:**

*Part a) Determine the Electric Field*

Grade: 3/3

Feedback: The solution correctly uses the potential difference and separation to find the electric field, \( E = \frac{V}{d} \). The calculated electric field, \( E = 800 \, \text{N/C} \), is correct.

*Part b) Determine the Acceleration*

Grade: 2/2

Feedback: The solution correctly applies Coulomb's law and Newton's second law to find the force and acceleration, arriving at the correct acceleration \( a = 34.78 \, \text{m/s}^2 \).

**Total: 5/5**

Overall, all three solutions correctly solved both parts of the problem, earning full marks. The explanations were thorough and accurate, demonstrating a clear understanding of the concepts involved.

\bigskip
\textbf{Response 3:}

**Solution 1:**

*Marks: Part a) 3/3, Part b) 2/2*

*Feedback:*

- *Part a):* Excellent use of both the direct formula \( E = \frac{\sigma}{\epsilon_0} \) and the relationship between potential difference and electric field \( E = \frac{V}{d} \). The calculations are correct and clearly explained.

- *Part b):* Correct use of \( F = qE \) and \( a = \frac{F}{m} \). The calculations are accurate, and the reasoning is clear. The acceleration value is correctly obtained.

**Solution 2:**

*Marks: Part a) 3/3, Part b) 2/2*

*Feedback:*

- *Part a):* Correct use of the electric field formula and clear explanation. The approach is thorough, and the relationship between the potential difference and electric field is used correctly.

- *Part b):* Accurate use of the force and acceleration formulas. The calculations are correct, and the explanation is clear and logical.

**Solution 3:**

*Marks: Part a) 3/3, Part b) 2/2*

*Feedback:*

- *Part a):* Direct and correct use of the relationship \( E = \frac{V}{d} \) to find the electric field. The explanation is straightforward and accurate.

- *Part b):* Correct use of \( F = qE \) and \( a = \frac{F}{m} \). The calculations and explanations are accurate, resulting in the correct value for acceleration.

Overall, all three solutions correctly address the problem and provide accurate calculations and explanations.

\bigskip
\textbf{Response 4:}

Solution 1:

**Part a:**

- Explanation of the electric field due to a single plane and for two planes is accurate.

- Correctly uses \( V = E \cdot d \) to find \( E \).

- Calculations are clearly shown.

**Grade: 3/3**

**Part b:**

- Uses \( F = qE \) and \( a = \frac{F}{m} \) correctly.

- Correct values for charge and mass.

- Final calculation for acceleration is correct.

**Grade: 2/2**

**Feedback:** Excellent work. Your methodical approach and clear explanation demonstrate a strong understanding of the concepts and proper application of formulas.

Solution 2:

**Part a:**

- Explanation and formula for electric field due to infinite planes are accurate.

- Correctly simplifies to \( E = \frac{V}{d} \) to find \( E \).

- Calculation steps are clearly presented.

**Grade: 3/3**

**Part b:**

- Correct use of \( F = qE \) and \( a = \frac{F}{m} \).

- Properly identifies the necessary values for the calculations.

- Final result for acceleration is correct.

**Grade: 2/2**

**Feedback:** Well done. Your clear, step-by-step breakdown effectively conveys your understanding and correct application of the necessary physics principles.

Solution 3:

**Part a:**

- Correctly identifies and uses the relation \( E = \frac{V}{d} \) to find the electric field.

- Calculation and result for \( E \) are correct.

**Grade: 3/3**

**Part b:**
- Uses the correct formulas \( F = qE \) and \( a = \frac{F}{m} \).

- Correct numerical values and calculations.

- Provides a clear step-by-step solution.

**Grade: 2/2**

**Feedback:** Excellent job. Your solution is clear, concise, and demonstrates a solid understanding of the relevant physics concepts and calculations.

---

**Overall Summary:** All three solutions are correctly solved, with clear explanations and accurate calculations. Each student correctly identified the necessary formulas and applied them appropriately to arrive at the correct answers.

\bigskip
\textbf{Response 5:}

Solution 1

Part a)

Marks: 3/3

Feedback: The solution correctly calculates the electric field using the potential difference and separation distance. The method is accurate, and the values used are correct. The explanation is thorough and accurate.

Part b)
Marks: 2/2

Feedback: The calculation of the acceleration is correct. The method used to find the force and acceleration is appropriate. The final values are accurate.

Solution 2

Part a)
Marks: 3/3

Feedback: The solution provides a clear and correct explanation of the electric field between the plates. The method of using the potential difference and separation distance to find the electric field is correct. The calculation is accurate.

Part b)

Marks: 2/2

Feedback: The calculation for the acceleration is correct, and the method used is appropriate. The final values are accurate.

Solution 3

Part a)
Marks: 3/3

Feedback: The solution correctly calculates the electric field using the potential difference and separation distance. The approach is clear and accurate, and the explanation is concise.

Part b)

Marks: 2/2

Feedback: The method to find the force and the subsequent calculation of the acceleration is correct. The final values are accurate. 

---

All three solutions are correct in both parts and show a good understanding of the concepts involved.

\subsection*{Question 7:}

\textbf{Response 1:}

Grading of Solutions

Solution 1

**Part a) Resonant Frequency**

Marks: 1/1  

- The student correctly applied the formula for the resonant frequency and identified the correct units for \( L \) and \( C \).

**Part b) Voltage Drop Across the Inductor**

Marks: 2/3  

- The explanation is thorough, and the derived expression is almost correct. However, the correct phase shift should be \(\omega t - \phi\) rather than \(\omega t + \phi\). Despite this, the overall methodology and final expression are nearly correct.

**Part c) Phase Relationship**

Marks: 2/2  

- The description of the phase relationship at low and high frequencies is accurate and well-explained.

**Total: 5/6**

**Feedback:** 

- Excellent understanding of the resonant frequency calculation. 

- Good approach to deriving the voltage across the inductor but needs a slight correction in the phase shift notation. 

- Clear and correct explanation of the phase relationship in different frequency regimes.

Solution 2

**Part a) Resonant Frequency**

Marks: 1/1  

- The student correctly applied the formula for the resonant frequency and identified the correct units for \( L \) and \( C \).

**Part b) Voltage Drop Across the Inductor**

Marks: 3/3  

- The student provided a clear and accurate expression for the voltage drop across the inductor, correctly considering the impedance and phase shift.

**Part c) Phase Relationship**

Marks: 2/2  

- The description of the phase relationship at low and high frequencies is accurate and well-explained.

**Total: 6/6**

**Feedback:** 

- Excellent work throughout all parts. Clear, precise, and accurate explanations and calculations.

Solution 3

**Part a) Resonant Frequency**

Marks: 1/1  

- The student correctly applied the formula for the resonant frequency and identified the correct units for \( L \) and \( C \).

**Part b) Voltage Drop Across the Inductor**

Marks: 2/3  

- The derived expression for the voltage drop is nearly correct, but the student did not clearly explain how the phase shift affects the final voltage expression. However, the overall structure and formula are correct.

**Part c) Phase Relationship**

Marks: 2/2  

- The description of the phase relationship at low and high frequencies is accurate and well-explained.

**Total: 5/6**

**Feedback:** 

- Good calculation of the resonant frequency. 

- The voltage drop across the inductor is mostly correct, but needs a clearer explanation of phase shift. 

- Well-explained phase relationships.

Summary

- **Solution 1:** 5/6

- **Solution 2:** 6/6

- **Solution 3:** 5/6

Overall, the solutions are well done, with Solution 2 being the most precise and comprehensive. Solutions 1 and 3 have minor errors or omissions but are otherwise correct.

\bigskip
\textbf{Response 2:}

Grading and Feedback for Solutions

Solution 1

- **Part a:**

  - Marks: 1/1
  
  - Feedback: Correctly identified the formula for resonant frequency and provided accurate calculation.
  
- **Part b:**

  - Marks: 3/3
  
  - Feedback: Correctly derived the expression for the voltage drop across the inductor using the given parameters and explained the steps clearly. The final expression was accurate and included all necessary variables.
  
- **Part c:**

  - Marks: 2/2
  
  - Feedback: Provided an accurate and clear explanation of the phase relationship at both low and high frequencies.

**Total Marks: 6/6**

Solution 2

- **Part a:**

  - Marks: 1/1
  
  - Feedback: Correctly identified the formula for resonant frequency and provided accurate calculation.
  
- **Part b:**

  - Marks: 2/3
  
  - Feedback: The expression for the voltage drop across the inductor was almost correct but was less detailed and had minor inaccuracies in explaining the relationship of the current phase shift. The phase shift \(\phi\) wasn't clearly incorporated.
  
- **Part c:**

  - Marks: 2/2
  
  - Feedback: Provided a correct and clear explanation of the phase relationship at both low and high frequencies.

**Total Marks: 5/6**

Solution 3

- **Part a:**

  - Marks: 1/1
  
  - Feedback: Correctly identified the formula for resonant frequency and provided accurate calculation.
  
- **Part b:**

  - Marks: 2/3
  
  - Feedback: Correctly derived the expression for the voltage drop across the inductor but made a small mistake in the phase term. The phase shift \(\phi\) was stated as \(\omega t - \phi\) instead of \(\omega t + \phi\), which is crucial in AC circuit analysis.
  
- **Part c:**

  - Marks: 2/2
  
  - Feedback: Provided a correct and clear explanation of the phase relationship at both low and high frequencies.

**Total Marks: 5/6**

\bigskip
\textbf{Response 3:}

**Solution 1:**

**a)** (1/1) Correct formula and calculation for resonant frequency.

**b)** (2/3) Correct general approach but lacks final clear expression for \( \Delta V_L \).

**c)** (1/2) Correct description of phase relationship but could be more concise and direct.

*Total: 4/6*

**Feedback:** Good effort. You correctly derived the resonant frequency and provided a detailed explanation for the voltage drop across the inductor, though the final expression was somewhat unclear. The phase relationship was explained well, but more concisely tying it to \(\omega_0\) would improve clarity.

---

**Solution 2:**

**a)** (1/1) Correct formula and calculation for resonant frequency.

**b)** (2/3) Correct general approach but lacks final clear expression for \( \Delta V_L \).

**c)** (2/2) Correct and clear description of phase relationship at low and high frequencies.

*Total: 5/6*

**Feedback:** Well done. You correctly derived the resonant frequency and provided a detailed explanation for the voltage drop across the inductor, though the final expression was somewhat unclear. The phase relationship was accurately and clearly explained.

---

**Solution 3:**

**a)** (1/1) Correct formula and calculation for resonant frequency.

**b)** (3/3) Correct and clear expression for \( \Delta V_L \).

**c)** (2/2) Correct and clear description of phase relationship at low and high frequencies.

*Total: 6/6*

**Feedback:** Excellent work. You correctly derived the resonant frequency and provided a clear and accurate expression for the voltage drop across the inductor. The phase relationship was also accurately and clearly explained.

\bigskip
\textbf{Response 4:}

Let's grade and provide feedback for each solution.

Solution 1

Part a)

1 mark: Correct formula and conversion of units, but did not complete the calculation correctly.
   
   **Score**: 1/1

Part b)

Correct expression for \(\Delta V_L\) and good use of formulas, but the calculation steps were somewhat disorganized.
   
   **Score**: 2/3

Part c)

Clear and correct description of phase relationships in both frequency limits.
   
   **Score**: 2/2

**Total Score**: 5/6

**Feedback**: Solid understanding and correct use of formulas, but calculations need completion and organization can be improved.

---

Solution 2

Part a)

1 mark: Correct formula, unit conversion, and result.

   **Score**: 1/1

Part b)

Correct expression for \(\Delta V_L\), but less detail compared to Solution 1. It is clear and to the point, but the explanation can be expanded.
   
   **Score**: 2/3

Part c)

Correct phase relationships described but not as detailed as Solution 1.

   **Score**: 2/2

**Total Score**: 5/6

**Feedback**: Clear and concise solution. Part b could benefit from more detailed explanation.

---

Solution 3

Part a)

1 mark: Correct formula, unit conversion, and result.

   **Score**: 1/1

Part b)

Correct formula for \(\Delta V_L\), but the phase \(\phi\) should be \(\omega t + \phi\) instead of \(\omega t - \phi\).

   **Score**: 2/3

Part c)

Correct phase relationship descriptions, clear and concise.

   **Score**: 2/2

**Total Score**: 5/6

**Feedback**: Clear solution with good detail. Correct formula usage in part b, but phase detail could be slightly improved.

---

Summary

All three solutions show a good understanding of the problem and the underlying physics. Each solution correctly addresses the calculations and explanations, but with some minor errors and areas for improvement in clarity and detail. Overall, each solution is strong and earns high marks.

\bigskip
\textbf{Response 5:}

Solution 1

Part a

- Correct formula and substitution for resonant frequency.

- Correct answer calculation.

- **Marks:** 1/1

Part b

- Correct approach to finding \( \Delta V_{L} \).

- Expression for \( I \) is mostly correct but slightly misleading with \( \cos(\omega t + \phi) \) instead of \( \cos(\omega t - \phi) \).
- Correct combination of formulas to find \( \Delta V_{L} \).

- **Marks:** 2/3

Part c

- Correct explanation of phase relationships at low and high frequencies.

- Clear and accurate description of behavior.

- **Marks:** 2/2

**Total Marks:** 5/6

**Feedback:** Good understanding of the resonant frequency and phase relationships. Slight error in the phase term for current in part b, but overall solid approach and correct final expression.

Solution 2

Part a

- Correct formula and substitution for resonant frequency.

- Correct answer calculation.

- **Marks:** 1/1

Part b

- Correct approach and explanation of \( \Delta V_{L} \).

- Correct expression for current and impedance.

- Correct final expression for \( \Delta V_{L} \), though slightly less detailed than Solution 1.

- **Marks:** 3/3

Part c

- Correct explanation of phase relationships at low and high frequencies.

- Clear and accurate description of behavior.

- **Marks:** 2/2

**Total Marks:** 6/6

**Feedback:** Excellent work. Clear and concise steps with correct results in all parts.

Solution 3

Part a

- Correct formula and substitution for resonant frequency.

- Correct answer calculation.

- **Marks:** 1/1

Part b

- Correct approach to finding \( \Delta V_{L} \).

- Expression for \( I \) is slightly off with \( \cos(\omega t - \phi) \) instead of \( \cos(\omega t + \phi) \).

- Correct final expression for \( \Delta V_{L} \).

- **Marks:** 2/3

Part c

- Correct explanation of phase relationships at low and high frequencies.

- Clear and accurate description of behavior.

- **Marks:** 2/2

**Total Marks:** 5/6

**Feedback:** Good understanding of the problem and clear explanation. Slight error in the phase term for current in part b, but overall correct results and reasoning.

\subsection*{Question 8:}

\textbf{Response 1:}

Solution 1

**Marks: 6/6**

**Feedback:**

- The solution correctly applies the boundary conditions for the magnetic field at the interface.

- The relationship between the tangential components of \( \underline{H} \) and the normal components of \( \underline{B} \) is well-explained and correctly used.

- The final calculation for \( \alpha_{1} \) is accurate and demonstrates a clear understanding of the principles involved.

- Excellent use of programming to compute the result accurately.

Solution 2

**Marks: 3/6**

**Feedback:**

- The solution starts well by mentioning the boundary conditions but fails to properly apply them to find \( \alpha_{1} \).

- There is a lot of redundant information and an unclear path towards solving for \( \alpha_{1} \).

- The conclusion that additional information is needed is incorrect; the problem can be solved with the given data.

- Overall, the approach shows understanding but lacks a clear and accurate solution.

Solution 3

**Marks: 5/6**

**Feedback:**

- The solution correctly identifies and uses the boundary conditions for the magnetic fields.

- The mathematical steps to relate \( \alpha_{1} \) and \( \alpha_{2} \) through the relative permeabilities are well-explained.

- However, there is a minor miscalculation or rounding error in the final angle result, which should be approximately \( 26.37^\circ \) instead of \( 20.52^\circ \).

- Overall, the method and logic are sound, with just a minor error in the final computation.


\bigskip 
\textbf{Response 2:}

**Solution 1:**

Score: 5/6

Feedback:

- The solution accurately applies the boundary conditions for magnetic fields at the interface between two media.

- The correct relationship between the tangential and normal components of the magnetic fields is used.

- The student correctly calculates the angle \(\alpha_{1}\) using the given data.

- However, the final angle calculation has a slight inaccuracy due to rounding errors. The correct angle should be approximately \(25.94^\circ\), not \(26.37^\circ\).

**Solution 2:**

Score: 3/6

Feedback:

- The solution correctly identifies the boundary conditions and the relationships between the magnetic field components.

- The student does a good job explaining the theoretical principles but fails to explicitly calculate the angle \(\alpha_{1}\).

- There is a lack of a final numerical answer for \(\alpha_{1}\), which is essential for solving the problem completely.

**Solution 3:**

Score: 4/6

Feedback:

- The solution correctly applies the boundary conditions and uses the relationships between the magnetic field components.

- The student makes a good attempt to calculate the angle \(\alpha_{1}\), but the final answer is incorrect due to a misunderstanding in the calculation.

- The correct use of the sine function was demonstrated, but the final calculation should yield approximately \(25.94^\circ\), not \(20.52^\circ\).

- The approach is mostly correct but has minor inaccuracies in the execution.

\bigskip 
\textbf{Response 3:}

Solution 1: 

**Marks: 6/6**

**Feedback:**

- Excellent approach using boundary conditions.

- Correct application of the relationship between \(\underline{B}\) and \(\underline{H}\).

- Proper use of given values and trigonometric relationships.

- Correct calculations leading to the correct angle \(\alpha_{1}\).

Solution 2: 

**Marks: 3/6**

**Feedback:**

- Good understanding of boundary conditions.

- Correct identification of tangential and normal components of the magnetic field.

- Approach is somewhat unclear and convoluted.

- Missing the final calculation for \(\alpha_{1}\) and does not conclude with a numerical answer.

- The student understands the principles but lacks the execution.

Solution 3:

**Marks: 5/6**

**Feedback:**

- Clear and systematic approach using boundary conditions.

- Correct identification of normal and tangential components and their continuity.

- Accurate trigonometric and algebraic manipulations.

- Final calculation has a small error, resulting in an incorrect angle (20.52° instead of 26.37°).

- The overall approach and understanding are strong, with a minor computational mistake.

\bigskip 
\textbf{Response 4:}

Solution 1

Grade: 5/6

Feedback:

- The student correctly applied the boundary conditions for magnetic fields and used the relationship between \( B \) and \( H \) to solve for \( \alpha_1 \).
 
- They accurately set up the tangential and normal components and their relationships through relative permeabilities.

- The step-by-step calculation and final result are correct.

- However, there are some redundant steps and minor inaccuracies in the boundary conditions explanation which could be more concise. Also, the LaTeX code was not executed, and while the mathematical reasoning was solid, presenting the final angle without this code would have been better.

Solution 2

Grade: 3/6

Feedback:

- The student recognized the key boundary conditions for magnetic fields but struggled with their application.

- There was an incorrect assertion that the normal components of \( \underline{B} \) are such that \( \frac{B_{1n}}{\mu_{1}} = \frac{B_{2n}}{\mu_{2}} \), which is inaccurate for \( \underline{B} \); this relationship holds for \( \underline{H} \).

- The solution lacked a concrete step-by-step calculation to find \( \alpha_1 \). The approach was outlined correctly, but the execution was incomplete.

- The student needs to clarify the relationship between the tangential components and their proper application.

Solution 3

Grade: 5/6

Feedback:

- The student correctly identified the boundary conditions and applied them effectively.

- They provided clear reasoning about the continuity of tangential components and used trigonometric relationships appropriately.

- The calculation of \( \alpha_1 \) was correct, and they provided a detailed step-by-step solution.

- The explanation could be streamlined to avoid repeating similar points, and the final numerical solution was correct. However, there's a minor mistake in the final simplification; it should be \( \sin(\alpha_1) = \frac{1.493}{3.012} \times \sin(45^\circ) \), and they should have calculated \(\alpha_1\) from this expression correctly.

\bigskip 
\textbf{Response 5:}

Solution 1:

**Marks:** 5/6

**Feedback:** 

- The student has correctly identified and applied the boundary conditions for magnetic fields.

- The derivation and explanation are detailed and logical, showing clear understanding.

- The calculations are correctly performed, leading to the correct value of \(\alpha_{1}\).

- The solution is well-structured and uses appropriate mathematical expressions.

- The only minor issue is the formatting of the Python code, which should be presented more cleanly. Additionally, the assumption that the relationship between \( B_{1x} \) and \( B_{1z} \) directly translates to \(\tan(\alpha_{1})\) could be more explicitly stated.

Solution 2:

**Marks:** 3/6

**Feedback:**

- The student understands the basic principles and boundary conditions for magnetic fields.

- The approach to solving the problem is sound but lacks completeness.

- The explanation is somewhat repetitive and fails to lead to a concrete calculation of \(\alpha_{1}\).

- The student correctly identifies the need to use the tangential and normal components but doesn't fully utilize these relationships to find \(\alpha_{1}\).

- The final part of the solution indicates uncertainty and suggests the need for more information, which isn't necessary given the problem's provided data.

Solution 3:

**Marks:** 6/6

**Feedback:**

- The student provides a clear, concise explanation and correctly applies the boundary conditions for magnetic fields.

- The relationship between \( \sin(\alpha_{1}) \) and the permeabilities is accurately derived.

- The calculations are correctly performed, leading to the correct value of \(\alpha_{1}\).

- The solution is well-organized, with clear logical steps leading to the final answer.

- The student effectively uses the given information and applies the correct principles to reach a solution, demonstrating a thorough understanding of the problem.

\subsection*{Question 9:}

\textbf{Response 1:}

Grading and Feedback

Solution 1

**Part a):** 7/7

- Excellent application of Faraday's Law.

- Clear derivation and correct final expression.

**Part b):** 6/6

- Correct use of the Poynting vector and average.

- Accurate steps and final result.

**Part c):** 2/2

- Correctly derived the radiation pressure expression with angle dependency.

**Total:** 15/15

**Feedback:** 

- Comprehensive and correct throughout. Clear step-by-step derivation with precise mathematical operations.

Solution 2

**Part a):** 7/7

- Correctly applied Faraday's Law.

- Accurate derivation and final expression.

**Part b):** 6/6

- Correct use of the Poynting vector and average.

- Accurate steps and final result.

**Part c):** 2/2

- Correctly derived the radiation pressure expression with angle dependency.

**Total:** 15/15

**Feedback:** 

- Thorough and accurate solution. Well-explained with correct application of electromagnetic principles.

Solution 3

**Part a):** 7/7

- Correct application of Faraday's Law.

- Accurate derivation and final expression.

**Part b):** 6/6

- Correct use of the Poynting vector and average.

- Accurate steps and final result.

**Part c):** 2/2

- Correctly derived the radiation pressure expression with angle dependency.

**Total:** 15/15

**Feedback:** 

- Clear, methodical, and precise. All parts are well-explained and correct.

Summary

All three solutions are excellent, demonstrating thorough understanding and precise execution of electromagnetic theory and related calculations. Each solution correctly derives the required expressions, making appropriate use of Maxwell's equations and other relevant principles. Well done!

\bigskip
\textbf{Response 2:}

Grading and Feedback

Solution 1

**Marks:**

- **Part a:** 7/7

- **Part b:** 5/6

- **Part c:** 2/2

- **Total:** 14/15

**Feedback:**

Excellent work on Part a, with clear use of Faraday’s Law and correct derivation of the magnetic intensity. Part b is mostly correct but lacks detailed explanation of the time-averaging process, which could be more rigorous. Part c is correct and well-explained.

Solution 2

**Marks:**

- **Part a:** 6/7

- **Part b:** 5/6

- **Part c:** 2/2

- **Total:** 13/15

**Feedback:**

Very good derivation in Part a, although it could benefit from clearer steps in connecting the magnetic field to the electric field. Part b is correctly solved but lacks a clear explanation of the average calculation. Part c is accurately solved and explained.

Solution 3

**Marks:**

- **Part a:** 6/7

- **Part b:** 6/6

- **Part c:** 2/2

- **Total:** 14/15

**Feedback:**

Strong solution overall. Part a is well-structured but could use a more concise integration step for clarity. Part b is excellently done, with a clear and thorough explanation of the Poynting vector averaging process. Part c is correct and well-explained.

\bigskip
\textbf{Response 3:}

Grading and Feedback

**Solution 1:**

**Part a)**

- Correctly applied Faraday's Law.

- Accurately derived the relationship between electric and magnetic fields.

- Explanation is clear and follows logical steps.

**Marks: 7/7**

**Part b)**

- Correctly derived the Poynting flux.

- Clear and precise steps leading to the final expression.

**Marks: 6/6**

**Part c)**

- Accurately derived the radiation pressure for an angle of incidence.

- Clear and concise explanation.

**Marks: 2/2**

**Total: 15/15**

**Feedback:** Excellent solution. The steps are clear and the derivations are correct. Well done!

---

**Solution 2:**

**Part a)**

- Correct application of Faraday's Law.

- The derivation is mostly correct but lacks some clarity in the steps from \( \underline{B} \) to \( \underline{H} \).

**Marks: 6/7**

**Part b)**

- Correctly derived the Poynting flux.

- Clear steps but a minor simplification missing in intermediate steps.

**Marks: 5/6**

**Part c)**

- Correctly derived the radiation pressure.

- Clear explanation.

**Marks: 2/2**

**Total: 13/15**

**Feedback:** Good solution. Some steps could be clearer, particularly in part a). Overall, well done!

---

**Solution 3:**

**Part a)**

- Correctly applied Faraday's Law.

- Detailed explanation but some steps are unnecessarily complicated.

**Marks: 6/7**

**Part b)**

- Correctly derived the Poynting flux.

- Detailed but somewhat verbose explanation.

**Marks: 5/6**

**Part c)**

- Correctly derived the radiation pressure.

- Clear and concise.

**Marks: 2/2**

**Total: 13/15**

**Feedback:** Good solution. The derivations are correct but could be more concise. Simplifying explanations would improve readability. Well done!

---

Summary:

- **Solution 1**: 15/15 - Excellent and clear solution.

- **Solution 2**: 13/15 - Good solution with minor clarity issues.

- **Solution 3**: 13/15 - Good solution, slightly verbose.

\bigskip
\textbf{Response 4:}

Grading and Feedback for Each Solution

---

Solution 1

**Part a)**

- Correctly uses Faraday's Law and derives the expression for \(\underline{H}\).

- Clearly shows each step and logical progression.

- Some minor inaccuracies in final step simplification.

**Marks: 6/7**

**Part b)**

- Correctly identifies the Poynting vector and time-averaged value.

- Properly substitutes values and explains cross-product and averaging.

**Marks: 6/6**

**Part c)**

- Correct explanation and derivation of radiation pressure at an angle.

- Logical and straightforward.

**Marks: 2/2**

**Total Marks: 14/15**

**Feedback:**
Excellent solution with clear and logical steps. Minor detail correction in Part a) would make it perfect.

---

Solution 2

**Part a)**

- Correct application of Faraday's Law and derivation of \(\underline{H}\).

- Skips some steps in the explanation, which could affect clarity.

- Final form is correct.

**Marks: 6/7**

**Part b)**

- Correctly computes the Poynting vector and its average.

- Clear explanation but slightly less detailed than Solution 1.

**Marks: 6/6**

**Part c)**

- Correct derivation of the radiation pressure formula.

- Clear explanation.

**Marks: 2/2**

**Total Marks: 14/15**

**Feedback:**

Solid and correct approach throughout. Could benefit from more detailed explanations in Part a).

---

Solution 3

**Part a)**

- Correctly applies Faraday's Law but has some calculation and notation errors.

- Final result is slightly unclear.

**Marks: 5/7**

**Part b)**

- Correctly identifies and computes the Poynting vector.

- Explanation is clear but somewhat repetitive.
**Marks: 5/6**

**Part c)**

- Correct approach and final formula.

- Explanation could be more concise.
**Marks: 2/2**

**Total Marks: 12/15**

**Feedback:**

Good understanding of the problem. Pay attention to calculation details and avoid repetitive explanations.

---

Summary

- **Solution 1**: Excellent and clear; minor improvement in detail.

- **Solution 2**: Very good; more detailed steps would enhance clarity.

- **Solution 3**: Good understanding but needs more precision in calculations and clearer presentation.

\bigskip
\textbf{Response 5:}

Solution 1

Part a)

- Marks: 7/7

- Feedback: The solution correctly applies Faraday's Law and derives the relationship between the electric field and magnetic intensity. All steps are clearly explained, and the final result is accurately obtained.

Part b)

- Marks: 5/6

- Feedback: The derivation of the Poynting flux is mostly correct, but there is a slight confusion in the cross-product handling and the averaging process. The result is correct, but the explanation could be clearer.

Part c)

- Marks: 2/2

- Feedback: The solution for radiation pressure is correct and clearly explained. The derivation logically follows the physics principles involved.

**Total: 14/15**

---

Solution 2

Part a)

- Marks: 6/7

- Feedback: The solution correctly follows the steps of Faraday's Law but misses a detailed explanation for the transition from \(\omega = ck\) to the final expression. The final result is correct, but the intermediate steps could be clearer.

Part b)

- Marks: 5/6

- Feedback: The solution correctly identifies the necessary steps to derive the Poynting flux, but the explanation of the cross-product and averaging over one period could be more detailed.

Part c)

- Marks: 2/2

- Feedback: The derivation for the radiation pressure is accurate and logically presented. It clearly shows the effect of the angle of incidence.

**Total: 13/15**

---

Solution 3

Part a)

- Marks: 5/7

- Feedback: The solution outlines the process well but is slightly unclear in some intermediate steps, particularly in handling the curl and integration steps. The final result is correct, but the presentation could be improved for clarity.

Part b)

- Marks: 4/6

- Feedback: The explanation of the Poynting flux is generally correct, but there is a lack of clarity in the cross-product calculation and the averaging process. The result is correct, but the explanation needs more detail.

Part c)

- Marks: 2/2

- Feedback: The explanation for the radiation pressure is correct and well-structured. The derivation logically follows from the given conditions.

**Total: 11/15**

---

Summary

1. **Solution 1: 14/15** - Clear and accurate for all parts, minor issues in part b.

2. **Solution 2: 13/15** - Mostly correct but could benefit from more detailed explanations.

3. **Solution 3: 11/15** - Correct final results but lacks clarity in intermediate steps and explanations.

\subsection*{Question 10:}

\textbf{Response 1:}

Solution 1:

Part a) Invariance of the Scalar Product \((\underline{E} \cdot \underline{B})\)

We have the transformations for the electric and magnetic fields:

\[
\begin{aligned}
E'_{x} &= E_{x}, & E'_{y} &= \gamma(E_{y} - vB_{z}), & E'_{z} &= \gamma(E_{z} + vB_{y}) \\
B'_{x} &= B_{x}, & B'_{y} &= \gamma(B_{y} + \frac{v}{c^{2}} E_{z}), & B'_{z} &= \gamma(B_{z} - \frac{v}{c^{2}} E_{y})
\end{aligned}
\]

The scalar product \((\underline{E} \cdot \underline{B})\) in frame \(S\) is:

\[
E_x B_x + E_y B_y + E_z B_z
\]

In frame \(S'\), it becomes:

\[
E'_x B'_x + E'_y B'_y + E'_z B'_z
\]

Substituting the transformed components:

\[
E'_x B'_x = E_x B_x
\]

\[
E'_y B'_y = \gamma(E_y - vB_z) \gamma(B_y + \frac{v}{c^2} E_z)
\]

\[
E'_z B'_z = \gamma(E_z + vB_y) \gamma(B_z - \frac{v}{c^2} E_y)
\]

Simplifying:

\[
E'_y B'_y + E'_z B'_z = \gamma^2 \left( E_y B_y - vE_y B_z + vB_y E_z + \frac{v^2}{c^2} E_z E_y + E_z B_z - \frac{v}{c^2} E_y E_z \right)
\]

Using \(\gamma^2 = \frac{1}{1 - \frac{v^2}{c^2}}\), the terms involving \(v\) cancel out, showing:

\[
E_x B_x + E_y B_y + E_z B_z = E'_x B'_x + E'_y B'_y + E'_z B'_z
\]

Thus, \((\underline{E} \cdot \underline{B})\) is invariant.

Part b) Phase Transformation of the Electromagnetic Wave

Given the wave in \(S\):

\[
\underline{E} = E_0 \hat{\underline{y}} \exp(i(kx - \omega t))
\]

Using Lorentz transformations:

\[
x' = \gamma(x - vt), \quad t' = \gamma\left(t - \frac{vx}{c^2}\right)
\]

The phase in \(S'\) is:

\[
kx - \omega t = k\gamma(x' + vt') - \omega\gamma(t' + \frac{vx'}{c^2})
\]

Simplifying:

\[
\gamma(k - \frac{\omega v}{c^2})x' - \gamma(\omega - kv)t' = k'x' - \omega't'
\]

Thus:

\[
\omega' = \gamma(\omega - kv), \quad k' = \gamma(k - \frac{\omega v}{c^2})
\]

Part c) Relationship Between \(\omega\) and \(\omega'\) as \(v/c \rightarrow 0\)

In the limit \(v/c \rightarrow 0\):

\[
\omega' \approx \omega
\]

This represents the classical Doppler effect, where relativistic effects are negligible.

Marks and Feedback for Solution 1:

- **Part a: 4/4**

- **Part b: 2/2**

- **Part c: 1/1**

**Feedback:** Excellent and detailed solution. The invariance of the scalar product and phase transformation were correctly demonstrated with clear and accurate calculations. The relationship between \(\omega\) and \(\omega'\) in the non-relativistic limit was correctly identified as the classical Doppler effect.

---

Solution 2:

Part a) Invariance of the Scalar Product \((\underline{E} \cdot \underline{B})\)

Using the transformations provided:

\[
E'_x B'_x = E_x B_x
\]

\[
E'_y B'_y = \gamma(E_y - vB_z) \gamma(B_y + \frac{v}{c^2} E_z)
\]

\[
E'_z B'_z = \gamma(E_z + vB_y) \gamma(B_z - \frac{v}{c^2} E_y)
\]

Simplifying and using the Lorentz factor:

\[
E_x B_x + E_y B_y + E_z B_z = E'_x B'_x + E'_y B'_y + E'_z B'_z
\]

Thus, \((\underline{E} \cdot \underline{B})\) is invariant.

Part b) Phase Transformation of the Electromagnetic Wave

Given:

\[
\underline{E} = E_0 \hat{\underline{y}} \exp(i(kx - \omega t))
\]

Using Lorentz transformations:

\[
x' = \gamma(x - vt), \quad t' = \gamma\left(t - \frac{vx}{c^2}\right)
\]

Substituting:

\[
kx - \omega t = k\gamma(x' + vt') - \omega\gamma(t' + \frac{vx'}{c^2})
\]

This simplifies to:

\[
k'x' - \omega't'
\]

Thus:

\[
\omega' = \gamma(\omega - kv), \quad k' = \gamma(k - \frac{\omega v}{c^2})
\]

Part c) Relationship Between \(\omega\) and \(\omega'\) as \(v/c \rightarrow 0\)

As \(v/c \rightarrow 0\):

\[
\omega' \approx \omega
\]

This represents the classical Doppler effect.

Marks and Feedback for Solution 2:

- **Part a: 4/4**

- **Part b: 2/2**

- **Part c: 1/1**

**Feedback:** Correctly demonstrated the invariance of the scalar product and phase transformation with accurate and clear calculations. The relationship between \(\omega\) and \(\omega'\) was correctly identified as the classical Doppler effect in the limit \(v/c \rightarrow 0\).

---

Solution 3:

Part a) Invariance of the Scalar Product \((\underline{E} \cdot \underline{B})\)

Given transformations:

\[
E'_x B'_x = E_x B_x
\]

\[
E'_y B'_y = \gamma(E_y - vB_z) \gamma(B_y + \frac{v}{c^2} E_z)
\]

\[
E'_z B'_z = \gamma(E_z + vB_y) \gamma(B_z - \frac{v}{c^2} E_y)
\]

Simplifying:

\[
E_x B_x + E_y B_y + E_z B_z = E'_x B'_x + E'_y B'_y + E'_z B'_z
\]

Thus, \((\underline{E} \cdot \underline{B})\) is invariant.

Part b) Phase Transformation of the Electromagnetic Wave

Given:

\[
\underline{E} = E_0 \hat{\underline{y}} \exp(i(kx - \omega t))
\]

Using Lorentz transformations:

\[
x' = \gamma(x - vt), \quad t' = \gamma\left(t - \frac{vx}{c^2}\right)
\]

Substituting:

\[
kx - \omega t = k\gamma(x' + vt') - \omega\gamma(t' + \frac{vx'}{c^2})
\]

This simplifies to:

\[
k'x' - \omega't'
\]

Thus:

\[
\omega' = \gamma(\omega - kv), \quad k' = \gamma(k - \frac{\omega v}{c^2})
\]

Part c) Relationship Between \(\omega\) and \(\omega'\) as \(v/c \rightarrow 0\)

As \(v/c \rightarrow 0\):

\[
\omega' \approx \omega
\]

This represents the classical Doppler effect.

### Marks and Feedback for Solution 3:

- **Part a: 4/4**

- **Part b: 2/2**

- **Part c: 1/1**

**Feedback:** The solution correctly demonstrates the invariance of the scalar product and phase transformation with clear and accurate steps. The relationship between \(\omega\) and \(\omega'\) was correctly identified as the classical Doppler effect in the limit \(v/c \rightarrow 0\).

\bigskip
\textbf{Response 2:}

Solution 1: 

**Part a) Invariance of the Scalar Product \( (\underline{E} \cdot \underline{B}) \)**

We need to show that the scalar product \( (\underline{E} \cdot \underline{B}) \) is invariant under the transformation between the reference frames \( S \) and \( S' \). The scalar product in frame \( S \) is \( E_x B_x + E_y B_y + E_z B_z \), and in frame \( S' \), it becomes \( E'_x B'_x + E'_y B'_y + E'_z B'_z \).

Given the transformations:
\[
\begin{aligned}
E'_{x} &= E_{x}, & E'_{y} &= \gamma(E_{y} - vB_{z}), & E'_{z} &= \gamma(E_{z} + vB_{y}) \\
B'_{x} &= B_{x}, & B'_{y} &= \gamma(B_{y} + \frac{v}{c^{2}} E_{z}), & B'_{z} &= \gamma(B_{z} - \frac{v}{c^{2}} E_{y})
\end{aligned}
\]

Substitute these transformations into the scalar product \( (\underline{E}' \cdot \underline{B}') \) and simplify to show invariance:
\[
E'_x B'_x + E'_y B'_y + E'_z B'_z = E_x B_x + \gamma(E_y - v B_z) \gamma(B_y + \frac{v}{c^2} E_z) + \gamma(E_z + v B_y) \gamma(B_z - \frac{v}{c^2} E_y)
\]
\[
= E_x B_x + \gamma^2(E_y B_y + \frac{v}{c^2} E_z E_y - v B_z B_y - \frac{v^2}{c^2} B_z E_y + E_z B_z + v B_y E_z - \frac{v}{c^2} E_y E_z)
\]
\[
= E_x B_x + \gamma^2 (E_y B_y + E_z B_z + \frac{v}{c^2} (E_z E_y - E_y E_z) - v (B_z B_y - B_y B_z))
\]
Since \(E_y E_z - E_z E_y = 0\) and \(B_z B_y - B_y B_z = 0\):
\[
= E_x B_x + \gamma^2 (E_y B_y + E_z B_z)
\]
Given \(\gamma = \frac{1}{\sqrt{1 - \frac{v^2}{c^2}}}\):
\[
= E_x B_x + \gamma^2 (E_y B_y + E_z B_z) \rightarrow E_x B_x + E_y B_y + E_z B_z \]
Thus, \(\underline{E} \cdot \underline{B} = \underline{E}' \cdot \underline{B}'\), confirming the invariance.

**Part b) Phase of the Electromagnetic Wave in Frame \( S' \)**

Given the electromagnetic wave in \( S \):
\[ \underline{E} = E_{0} \hat{\underline{y}} \exp(i(kx - \omega t)) \]

The phase \( (kx - \omega t) \) transforms into \( (k'x' - \omega' t') \) in frame \( S' \), using:
\[ \omega' = \gamma(\omega - kv), \quad k' = \gamma(k - \frac{\omega v}{c^{2}}) \]

Using the Lorentz transformation:
\[
x' = \gamma(x - vt), \quad t' = \gamma(t - \frac{vx}{c^{2}})
\]

Substitute \( x \) and \( t \) with these transformations:
\[
k (\gamma (x' + vt')) - \omega (\gamma (t' + \frac{vx'}{c^2})) = k' x' - \omega' t'
\]
\[
k' = \gamma (k - \frac{\omega v}{c^2}), \quad \omega' = \gamma (\omega - kv)
\]
Thus, the phase transforms correctly.

**Part c) Relationship Between \( \omega \) and \( \omega' \) as \( v/c \rightarrow 0 \)**

As \( v/c \rightarrow 0 \):
\[
\omega' = \gamma(\omega - kv) \rightarrow \omega' = \omega
\]
This represents the classical Doppler effect where the frequency remains unchanged as the relative velocity approaches zero.

Grading and Feedback:

**Solution 1:**

- Part a: Correct setup and explanation, but missing full step-by-step simplification. **3/4**

- Part b: Correct approach and transformation, but lacks clarity in final expressions. **1/2**

- Part c: Correct answer with proper justification. **1/1**

**Total: 5/7**

*Feedback: Good attempt with clear understanding of the concepts. Ensure to fully simplify expressions for complete clarity and precision.*

---

Solution 2:

**Part a) Invariance of the Scalar Product \( (\underline{E} \cdot \underline{B}) \)**

The scalar product \( (\underline{E} \cdot \underline{B}) \) in frame \( S \) is given by:
\[
\underline{E} \cdot \underline{B} = E_{x}B_{x} + E_{y}B_{y} + E_{z}B_{z}
\]

For the transformed frames:
\[
\underline{E}' \cdot \underline{B}' = E_{x}B_{x} + \gamma(E_{y} - vB_{z})\gamma(B_{y} + \frac{v}{c^{2}}E_{z}) + \gamma(E_{z} + vB_{y})\gamma(B_{z} - \frac{v}{c^{2}}E_{y})
\]
\[
= E_x B_x + \gamma^2 (E_y B_y + E_z B_z)
\]
The final result confirms that \(\underline{E} \cdot \underline{B} = \underline{E}' \cdot \underline{B}'\), confirming the invariance.

**Part b) Phase of the Electromagnetic Wave in Frame \( S' \)**

Given:
\[
\underline{E} = E_{0} \hat{\underline{y}} \exp(i(kx - \omega t))
\]
Using Lorentz transformations:
\[
x' = \gamma(x - vt), \quad t' = \gamma(t - \frac{vx}{c^{2}})
\]
Transform the phase:
\[
k (\gamma (x' + vt')) - \omega (\gamma (t' + \frac{vx'}{c^2})) = k' x' - \omega' t'
\]
\[
k' = \gamma (k - \frac{\omega v}{c^2}), \quad \omega' = \gamma (\omega - kv)
\]
Thus, the phase transforms correctly.

**Part c) Relationship Between \( \omega \) and \( \omega' \) as \( v/c \rightarrow 0 \)**

As \( v/c \rightarrow 0 \):
\[
\omega' = \gamma(\omega - kv) \rightarrow \omega' = \omega
\]
This represents the classical Doppler effect where the frequency remains unchanged as the relative velocity approaches zero.

Grading and Feedback:

**Solution 2:**

- Part a: Correct and clear explanation. **4/4**

- Part b: Correct approach and transformation. **2/2**

- Part c: Correct answer with proper justification. **1/1**

**Total: 7/7**

*Feedback: Excellent work! Clear and concise explanation with correct results in all parts.*

---

Solution 3:

**Part a) Invariance of the Scalar Product \( (\underline{E} \cdot \underline{B}) \)**

The scalar product \( (\underline{E} \cdot \underline{B}) \) in frame \( S \) is given by:
\[
\underline{E} \cdot \underline{B} = E_{x}B_{x} + E_{y}B_{y} + E_{z}B_{z}
\]

For the transformed frames:
\[
\underline{E}' \cdot \underline{B}' = E_{x}B_{x} + \gamma(E_{y} - vB_{z})\gamma(B_{y} + \frac{v}{c^{2}}E_{z}) + \gamma(E_{z} + vB_{y})\gamma(B_{z} - \frac{v}{c^{2}}E_{y})
\]
\[
= E_x B_x + \gamma^2 (E_y B_y + E_z B_z)
\]
The final result confirms that \(\underline{E} \cdot \underline{B} = \underline{E}' \cdot \underline{B}'\), confirming the invariance.

**Part b) Phase of the Electromagnetic Wave in Frame \( S' \)**

Given:
\[
\underline{E} = E_{0} \hat{\underline{y}}

 \exp(i(kx - \omega t))
\]
Using Lorentz transformations:
\[
x' = \gamma(x - vt), \quad t' = \gamma(t - \frac{vx}{c^{2}})
\]
Transform the phase:
\[
k (\gamma (x' + vt')) - \omega (\gamma (t' + \frac{vx'}{c^2})) = k' x' - \omega' t'
\]
\[
k' = \gamma (k - \frac{\omega v}{c^2}), \quad \omega' = \gamma (\omega - kv)
\]
Thus, the phase transforms correctly.

**Part c) Relationship Between \( \omega \) and \( \omega' \) as \( v/c \rightarrow 0 \)**

As \( v/c \rightarrow 0 \):
\[
\omega' = \gamma(\omega - kv) \rightarrow \omega' = \omega
\]
This represents the classical Doppler effect where the frequency remains unchanged as the relative velocity approaches zero.

Grading and Feedback:

**Solution 3:**

- Part a: Correct and clear explanation. **4/4**

- Part b: Correct approach and transformation. **2/2**

- Part c: Correct answer with proper justification. **1/1**

**Total: 7/7**

*Feedback: Excellent work! Clear and concise explanation with correct results in all parts.*

\bigskip
\textbf{Response 3:}

Grading and Feedback

Solution 1:

**Marks:**

- Part a: 2/4

- Part b: 1/2

- Part c: 1/1

- **Total: 4/7**

**Feedback:**

- **Part a:** The solution outlines the correct approach to demonstrating the invariance of the scalar product, but the algebra is not fully completed. The explanation lacks the final detailed steps to show the invariance explicitly.

- **Part b:** The phase transformation is discussed conceptually, but the mathematical steps are incomplete. The solution does not explicitly show the derivation of the transformed phase \( (k'x' - \omega' t') \).

- **Part c:** The explanation of the limit \( v/c \rightarrow 0 \) is accurate and correctly identifies the phenomenon as the classical Doppler effect.

Solution 2:

**Marks:**

- Part a: 3/4

- Part b: 1/2

- Part c: 1/1

- **Total: 5/7**

**Feedback:**

- **Part a:** The solution provides a clearer and more detailed calculation, showing significant effort in simplifying the scalar product to demonstrate invariance. The algebra is more thorough, but there is a slight error in the final steps.

- **Part b:** The explanation attempts to show the phase transformation but does not fully derive \( (k'x' - \omega' t') \) mathematically. The conceptual understanding is present, but the algebraic steps need to be clearer and more complete.

- **Part c:** The explanation correctly identifies the relationship between the wave frequencies and explains the physical phenomenon as the classical Doppler effect.

Solution 3:

**Marks:**

- Part a: 4/4

- Part b: 2/2

- Part c: 1/1

- **Total: 7/7**

**Feedback:**

- **Part a:** The solution correctly identifies and demonstrates the invariance of the scalar product through a complete and detailed algebraic derivation. The steps are clear and logically follow from the given transformations.

- **Part b:** The solution effectively uses Lorentz transformations to show how the phase of the electromagnetic wave transforms between frames. The explanation is clear, and the derivation is complete and accurate.

- **Part c:** The explanation accurately describes the relationship between the wave frequencies in the limit \( v/c \rightarrow 0 \) and correctly identifies the phenomenon as the classical Doppler effect.

\bigskip
\textbf{Response 4:}

Here is the assessment of the solutions provided by each student for the given physics problem:

---

**Solution 1:**

**Part a:**

- The student provides a detailed description of the transformations but does not explicitly complete the calculation showing the invariance of \((\underline{E} \cdot \underline{B})\).

- They recognize the complexity but do not arrive at a final simplified expression proving the invariance.

**Marks for Part a: 2/4**

**Part b:**

- The student understands the requirement to use Lorentz transformations for time and space but stops short of explicitly demonstrating how the phase transforms.

- They indicate the general process but do not show the step-by-step derivation leading to the given relations for \(\omega'\) and \(k'\).

**Marks for Part b: 1/2**

**Part c:**

- The student correctly identifies that \(\omega'\) approaches \(\omega\) as \(v/c \rightarrow 0\) and relates this to the Doppler effect.

- The explanation is brief but conceptually correct.

**Marks for Part c: 1/1**

**Total Marks for Solution 1: 4/7**

**Feedback:** This solution shows a good understanding of the concepts but lacks the detailed mathematical derivations to conclusively prove the invariance and phase transformations. The conceptual understanding in Part c is solid.

---

**Solution 2:**

**Part a:**

- The student correctly identifies the required transformations and partially substitutes them into the scalar product.

- The explanation is more detailed than Solution 1, but they also do not reach the final simplified expression demonstrating the invariance.

**Marks for Part a: 3/4**

**Part b:**

- The student outlines the process and provides the correct form of the Lorentz transformation for coordinates.

- However, the derivation stops short of explicitly showing the phase transformation to the primed frame with the given \(\omega'\) and \(k'\).

**Marks for Part b: 1/2**

**Part c:**

- The student correctly identifies the limit and relates it to the Doppler effect.

- The explanation is clear and accurate.

**Marks for Part c: 1/1**

**Total Marks for Solution 2: 5/7**

**Feedback:** This solution provides a better step-by-step approach for Part a and correctly outlines the process for Part b, but it needs a complete derivation. The explanation in Part c is correct and concise.

---

**Solution 3:**

**Part a:**

- The student correctly sets up the problem and substitutes the transformations.

- They proceed with the expansion and simplify the expression, but the complexity of the result indicates a need for further algebraic manipulation to clearly show the invariance.

**Marks for Part a: 3/4**

**Part b:**

- The student outlines the necessary steps and transformations but encounters complexity without fully simplifying to show the phase transformation explicitly.

- They acknowledge the need for a conceptual explanation and attempt to explain the phase transformation in terms of relativistic effects.

**Marks for Part b: 1/2**

**Part c:**

- The student correctly identifies the limit and explains the relationship between \(\omega\) and \(\omega'\).

- The explanation ties it to the classical Doppler effect and its relativistic counterpart.

**Marks for Part c: 1/1**

**Total Marks for Solution 3: 5/7**

**Feedback:** This solution provides a good setup and initial algebraic manipulation for Part a. The conceptual explanation for Part b is helpful, though a complete derivation is missing. Part c is correctly explained.

---

**Summary Feedback:**

All solutions show a solid understanding of the underlying physical concepts but need more rigorous mathematical derivations to fully solve Parts a and b. The conceptual explanations for Part c are well done in all cases. Students are encouraged to work through the algebraic manipulations more thoroughly to arrive at the final simplified expressions for full credit in Parts a and b.

\bigskip
\textbf{Response 5:}

Grading and Feedback

Solution 1

Part a) (2/4 marks)

The student begins by setting up the problem correctly, aiming to demonstrate the invariance of the scalar product \((\underline{E} \cdot \underline{B})\) under the given transformations. However, the explanation becomes convoluted and does not successfully simplify the expression to show the invariance explicitly.

**Feedback:** You have a good start by stating the goal and identifying the relevant transformations. However, the solution needs to clearly show the step-by-step simplification to demonstrate the invariance. Ensure to follow through the algebraic manipulations to completion.

Part b) (1/2 marks)

The student correctly identifies the Lorentz transformation for coordinates and attempts to apply it to the wave phase. However, the explanation is not fully clear, and the derivation of the phase transformation to \(k'x' - \omega' t'\) is incomplete.

**Feedback:** You correctly approach the problem using Lorentz transformations but need to explicitly show how the phase \( (kx - \omega t) \) transforms into \( (k'x' - \omega' t') \) with clear steps and derivations.

Part c) (0/1 mark)

The student mentions the limit \(v/c \rightarrow 0\) and references Doppler shift or time dilation effects. However, the explanation lacks clarity and fails to explicitly identify the physical phenomenon.

**Feedback:** Your response should explicitly state that as \(v/c \rightarrow 0\), \(\omega' \rightarrow \omega\), representing the classical Doppler effect. Make sure to link the limit to the specific phenomenon clearly.

**Total: 3/7 marks**

Solution 2

Part a) (3/4 marks)

The student correctly sets up the problem and works through the algebraic manipulations to show the invariance of \((\underline{E} \cdot \underline{B})\). Some steps are shown, but the final simplification is not clearly presented.

**Feedback:** Good attempt at breaking down the transformations. To improve, ensure each algebraic step is shown clearly to illustrate the invariance of the scalar product.

Part b) (1/2 marks)

The student identifies the required transformations and correctly sets up the problem. However, the derivation of the phase transformation is incomplete and lacks detailed steps.

**Feedback:** You correctly recognize the Lorentz transformations' role but need to provide a clear, step-by-step derivation of how \( (kx - \omega t) \) transforms to \( (k'x' - \omega' t') \).

Part c) (1/1 mark)

The student correctly examines the limit \(v/c \rightarrow 0\) and identifies the physical phenomenon as the classical Doppler effect.

**Feedback:** Well done on clearly explaining the limit and linking it to the classical Doppler effect.

**Total: 5/7 marks**

Solution 3

Part a) (4/4 marks)

The student sets up the problem and provides a detailed, step-by-step derivation to show the invariance of \((\underline{E} \cdot \underline{B})\).

**Feedback:** Excellent work on clearly presenting the algebraic steps and demonstrating the invariance of the scalar product. Keep up the thorough and detailed explanations.

Part b) (2/2 marks)

The student correctly identifies the required transformations, applies the Lorentz transformations to the phase, and derives the correct expressions for \(\omega'\) and \(k'\).

**Feedback:** Great job on clearly deriving the phase transformation and showing how it aligns with the given expressions for \(\omega'\) and \(k'\).

Part c) (1/1 mark)

The student correctly examines the limit \(v/c \rightarrow 0\) and identifies the classical Doppler effect.

**Feedback:** Well done on clearly explaining the limit and linking it to the classical Doppler effect.

**Total: 7/7 marks**

Summary

- **Solution 1**: Needs more clarity and detailed steps, especially in parts a and b.

- **Solution 2**: Good attempt but requires more detailed derivations for full marks in parts a and b.

- **Solution 3**: Comprehensive and clear solutions for all parts, showing thorough understanding and correct derivations.

\end{document}